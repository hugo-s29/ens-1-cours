\documentclass[./main]{subfiles}

\begin{document}
  \chapter{Le calcul propositionnel.}

  Le \textit{calcul propositionnel}, c'est la "grammaire" de la logique.
  %Je trouve ce commentaire un peu bizarre, la grammaire n'est pas complète puisqu'il n'y a pas les quantificateurs et il y a une composante sémantique au calcul propositionnel, alors que la grammaire est une notion purement syntaxique.
  Dans ce chapitre, on s'intéressera à
  \begin{enumerate}
    \item la construction des formules ($\triangleright$ la syntaxe) ;
    \item la sémantique et les théorèmes de compacité ($\triangleright$ la compacité sémantique).
  \end{enumerate}

  \section{Syntaxe.}

  \begin{defn}
    Un \textit{langage}, ou \textit{alphabet}, est un ensemble d'éléments (fini ou pas).
    Ses éléments sont appelés \textit{lettres}, et les suites finies de lettres sont appelées \textit{mots}.
  \end{defn}

  \begin{rmk}
    On choisit l'alphabet :
    \begin{itemize}
      \item $\mathcal{P} = \{x_0, x_1, \ldots\}$ des variables propositionnelles ;
      \item un ensemble de \textit{connecteurs} ou \textit{symboles logiques}, défini par $\{\lnot, \lor, \land, \to, \leftrightarrow\}$, il n'y a pas $\exists$ et $\forall$ pour l'instant.
      \item les parenthèses $\{\verb|(|, \verb|)|\}$.
    \end{itemize}


    Les formules logiques sont des mots. On les fabrique avec des briques de base (les variables) et des opérations de construction : si $F_1$ et $F_2$ sont deux formules, alors $\lnot F$,  $\verb|(| F_1 \lor F_2 \verb|)|$, $\verb|(| F_1 \land F_2 \verb|)|$, $\verb|(| F_1 \to F_2 \verb|)|$ et $\verb|(| F_1 \leftrightarrow F_2 \verb|)|$ aussi.
  \end{rmk}

  On donne deux définitions des formules et on pourra montrer qu'elles sont équivalentes.

  \begin{defn}["par le haut", "mathématique"]
    L'ensemble~$\mathcal{F}$ des formules du calcul propositionnel construit sur $\mathcal{P}$ est le plus petit ensemble contenant $\mathcal{P}$ et stable par les opérations de construction.
  \end{defn}

  \begin{defn}["par le bas", "informatique"]
    L'ensemble $\mathcal{F}$ des formules logiques du calcul propositionnel sur $\mathcal{P}$ est défini par 
    \begin{itemize}
      \item $\mathcal{F}_0 = \mathcal{P}$ ;
      \item $\mathcal{F}_{n+1} = \mathcal{F}_n \cup \mleft\{\, 
          \begin{array}{c}
            \lnot F_1\\
            \verb|(| F_1 \lor F_2 \verb|)|\\
            \verb|(| F_1 \land F_2 \verb|)|\\
            \verb|(| F_1 \to F_2 \verb|)|\\
            \verb|(| F_1 \leftrightarrow F_2 \verb|)|
          \end{array}
        \;\middle|\; F_1, F_2 \in \mathcal{F}_n \,\mright\}  $
    \end{itemize}
    puis on pose $\mathcal{F} = \bigcup_{n \in \mathds{N}} \mathcal{F}_n$.
  \end{defn}

  \begin{thm}[Lecture unique]
    Toute formule $G \in \mathcal{F}$ vérifie une et une seule de ces propriétés :
    \begin{itemize}
      \item $G \in \mathcal{P}$ ;
      \item il existe $F \in \mathcal{F}$ telle que $G = \lnot F$ ;
      \item il existe $F_1, F_2 \in \mathcal{F}$ telle que $G = \verb|(| F_1 \lor F_2 \verb|)|$ ;
      \item il existe $F_1, F_2 \in \mathcal{F}$ telle que $G = \verb|(| F_1 \land F_2 \verb|)|$ ;
      \item il existe $F_1, F_2 \in \mathcal{F}$ telle que $G = \verb|(| F_1 \to F_2 \verb|)|$ ;
      \item il existe $F_1, F_2 \in \mathcal{F}$ telle que $G = \verb|(| F_1 \leftrightarrow F_2 \verb|)|$ ;
    \end{itemize}
    et ces sous-formules sont uniques.
  \end{thm}
  \begin{prv}
    Montrons l'existence. 
    Posons \[\mathcal{G} = \mathcal{P} \cup \mleft\{\, 
          \begin{array}{c}
            \lnot F_1\\
            \verb|(| F_1 \lor F_2 \verb|)|\\
            \verb|(| F_1 \land F_2 \verb|)|\\
            \verb|(| F_1 \to F_2 \verb|)|\\
            \verb|(| F_1 \leftrightarrow F_2 \verb|)|
          \end{array}
        \;\middle|\; F_1, F_2 \in \mathcal{F} \,\mright\}\]
    Comme $\mathcal{F}$ est stable par les opérations de construction des formules, on a $\mathcal{G} \subseteq \mathcal{F}$.
    Montrons alors que $\mathcal{G}$ est stable par ces mêmes opérations. On a déjà $\mathcal{P} \subset \mathcal{G}$.
    Par ailleurs, pour tout $F \in \mathcal{G}$, on a $F \in \mathcal{F}$, d'où $\lnot F \in \mathcal{G}$.
    Soient ensuite $F_1, F_2 \in \mathcal{G}$ et $\sigma \in \{\lor, \land, \to, \leftrightarrow\}$.
    Comme $F_1$ et $F_2$ sont des formules, on obtient $\verb|(| F_1\ \sigma\ F_2 \verb|)| \in \mathcal{G}$.
    En conclusion, $\mathcal{G}$ est stable par les opérations de construction des formules, donc contient $\mathcal{F}$.
    
    La partie difficile de la preuve d'unicité est le cas des opérateurs binaires.
    Il faut par exemple montrer que pour toutes formules $F_1, F_2, G_1$ et $G_2$, les formules $\verb|(| F_1 \lor F_2 \verb|)|$ et $\verb|(| G_1 \land G_2 \verb |)|$ sont différentes.
    Pour cela, on montre qu'un préfixe propre d'une formule n'est pas une formule.
    On commence par obtenir des contraintes sur les parenthèses qu'un préfixe d'une formule peut avoir.
    Montrons déjà que toute formule contient autant de parenthèses ouvrantes que de parenthèses fermantes.
    Notons pour tout $F \in \mathcal{F}$, $o(F)$ le nombre de parenthèses ouvrantes de $F$ et $f(F)$ son nombre de parenthèses fermantes et montrons par récurrence simple, pour tout $n \in \mathds{N}$, la propriété $$\mathcal{P}(n) = \forall F \in \mathcal{F}_n, o(F) = f(F).$$
    \begin{itemize}
      \item Les éléments de $\mathcal{F}_0 = \mathcal{P}$ ne contiennent ni parenthèse ouvrante ni parenthèse fermante, d'où $\mathcal{P}(0)$.
      \item Soit $n \in \mathds{N}$ tel que $\mathcal{P}(n)$ est vraie.
        Soit $F \in \mathcal{F}_{n+1}$.
        \begin{itemize}
          \item Si $F \in \mathcal{F}_n$, alors $o(F) = f(F)$ d'après $\mathcal{P}(n)$.
          \item Supposons qu'on dispose de $G \in \mathcal{F}_n$ tel que $F = \lnot G$.
            Alors $\mathcal{P}(n)$ donne l'égalité $o(G) = f(G)$, d'où l'on tire $o(F) = o(G) = f(G) = f(F)$.
          \item Supposons qu'on ait $F_1, F_2 \in \mathcal{F}_n$ et $\sigma \in \{\lor, \land, \to, \leftrightarrow\}$ tels que $F = \verb|(| F_1\ \sigma\ F_2 \verb|)|$.
            $\mathcal{P}(n)$ s'écrit, pour $F_1$ et $F_2$, $o(F_1) = f(F_1)$ et $o(F_2) = f(F_2)$.
            Dès lors, on calcule $o(F) = o(F_1) + o(F_2) + 2 = f(F_1) + f(F_2) + 2 = f(F)$.
        \end{itemize}
        Dans tous les cas, on a $o(F) = f(F)$, d'où $\mathcal{P}(n+1)$.
    \end{itemize}

    Discutons maintenant des préfixes d'une formule.
    Notons, pour deux mots $u$ et $v$, $u < v$ lorsque $u$ est un préfixe propre de $v$ et montrons par récurrence simple, pour tout $n \in \mathds{N}$, la propriété $$\mathcal{P}(n) = \forall F \in \mathcal{F}_n, \forall G < F, o(G) = f(G) = 0 \lor f(G) < o(G).$$
    \begin{itemize}
      \item Les éléments de $\mathcal{F}_0 = \mathcal{P}$ n'ont pas de préfixe propre, d'où $\mathcal{P}(0)$.
      \item Soit $n \in \mathds{N}$ tel que $\mathcal{P}(n)$ est vrai.
        Soient $F \in \mathcal{F}_{n+1}$ et $G$ un préfixe propre de $F$.
        \begin{itemize}
          \item Si $F \in \mathcal{F}_n$, alors $o(G) = f(G) = 0$ ou $f(G) < o(G)$ d'après $\mathcal{P}(n)$
          \item Supposons qu'on ait $F_1 \in \mathcal{F}_n$ tel que $F = \lnot F_1$.
            Si $G$ est vide, alors $o(G) = f(G) = 0$.
            Sinon, $G$ s'écrit $\lnot G_1$ avec $G_1$ un préfixe strict de $F_1$.
            L'hypothèse $\mathcal{P}(n)$ donne $o(G_1) = f(G_1) = 0$ ou $f(G_1) < o(G_1)$.
            Or, on a $o(G) = o(G_1)$ et $f(G) = f(G_1)$, d'où $o(G) = f(G) = 0$ ou $f(G) < o(G)$.
          \item Supposons qu'on ait $F_1, F_2 \in \mathcal{F}_n$ et $\sigma \in \{\lor, \land, \to, \leftrightarrow\}$ tels que $F = \verb|(| F_1\ \sigma\ F_2 \verb|)|$.
            On distingue les cas selon la longueur de $G$.
            \begin{itemize}
              \item Si $G$ est le mot vide, alors $o(G) = f(G) = 0$
              \item Supposons que $G = \verb|(| G_1$ avec $G_1$ un préfixe strict de $F_1$.
                L'hypothèse $\mathcal{P}(n)$ appliquée à $F_1$ et $G_1$ donne $o(G_1) = f(G_1) = 0$ ou $f(G_1) < o(G_1)$.
                Dans les deux cas, on a $f(G_1) \le o(G_1)$, d'où $f(G) = f(G_1) < o(G_1) + 1 = o(G)$.
              \item Supposons que $G = \verb|(| F_1$. Alors $o(F_1) = f(F_1)$, donc $f(G) = f(F_1) = o(G_1) < o(G_1) + 1 = o(G)$.
              \item Supposons que $G = \verb|(| F_1 \sigma G_2$ avec $G_2$ un préfixe strict de $F_2$.
                $\mathcal{P}(n)$ donne $o(G_2) = f(G_2) = 0$ ou $f(G_2) < o(G_2)$.
                Dans les deux cas, on obtient $f(G_2) \le o(G_2)$.
                De plus, on a $o(F_1) = f(F_1)$, donc
                \[\begin{array}{rcl}
                  f(G)&=&f(F_1) + f(G_2)\\
                  &\le&o(F_1) + o(G_2)\\
                  &<&o(F_1) + o(G_2) + 1\\
                &=&o(G).\end{array}\]
            \end{itemize}
        \end{itemize}
        Dans tous les cas, on conclut bien $o(G) = f(G) = 0$ ou $f(G) < o(G)$.
        D'où $\mathcal{P}(n+1)$.
    \end{itemize}

    Montrons maintenant par récurrence simple, pour tout $n \in \mathds{N}$, la propriété $$\mathcal{P}(n) = \forall F \in \mathcal{F}_n, \forall G < F, G \notin \mathcal{F}.$$
    \begin{itemize}
      \item Une formule de $\mathcal{F}_0 = \mathcal{P}$ n'a pas de préfixe propre, d'où $\mathcal{P}(0)$.
      \item Soit $n \in \mathds{N}$ tel que $\mathcal{P}(n)$ est vrai.
        Soient $F \in \mathcal{F}_{n+1}$ et $G$ un préfixe propre de $F$.
        \begin{itemize}
          \item Si $F \in \mathcal{F}_n$, alors $G$ n'est pas une formule d'après $\mathcal{P}(n)$
          \item Supposons qu'on dispose de $F_1 \in \mathcal{F}_n$ tel que $F = \lnot F_1$.
            Supposons que $G$ est une formule.
            Comme une formule n'est pas vide, $G$ s'écrit $\lnot G_1$ avec $G_1$ un préfixe strict de $F_1$.
            Avec l'existence dans le théorème de lecture unique vue plus haut, $G_1$ est une formule, ce qui contredit $\mathcal{P}(n)$ pour $F_1$ et $G_1$.
          \item Supposons qu'on ait $F_1, F_2 \in \mathcal{F}_n$ et $\sigma \in \{\lor, \land, \to, \leftrightarrow\}$ tels que $F = \verb|(| F_1\ \sigma\ F_2 \verb|)|$.
            Supposons que $G$ est une formule.
            Comme une formule n'est pas vide, $G$ commence par une parenthèse ouvrante.
            En particulier, $o(G) \ne 0$.
            Or, $G$ est un préfixe propre de $F$, donc $f(G) < o(G)$.
            Ceci est impossible puisque, $G$ étant une formule, on a aussi $f(G) = o(G)$.
        \end{itemize}
        Dans tous les cas, $G$ n'est pas une formule, d'où $\mathcal{P}(n+1)$.
    \end{itemize}

    Nous sommes prêts à montrer que les propriétés du théorème de lecture unique sont mutuellement exclusives.
    Les deux premières le sont et excluent les autres en considérant la première lettre des formules qu'elles décrivent.
    Plus précisément, les formules satisfaisant la première propriété commencent par une variable, celles satisfaisant la deuxième commencent par une négation et les autres commencent par une parenthèse.
    Pour les autres cas, considérons $F_1, F_2, G_1, G_2 \in \mathcal{F}$ et $\sigma, \sigma' \in \{\lor, \land, \to, \leftrightarrow\}$ tels que $\verb|(| F_1\ \sigma\ F_2 \verb|)| = \verb|(| G_1\ \sigma'\ G_2 \verb|)|$.
    Il s'agit de montrer que $\sigma = \sigma'$.
    Quitte à échanger les rôles de $F_1$ et $G_1$, de $F_2$ et $G_2$ et de $\sigma$ et $\sigma'$, on peut supposer que $\verb|(| F_1$ est un préfixe de $\verb|(| G_1$.
    Alors $F_1$ est un préfixe de $G_1$.
    Or $F_1$ et $G_1$ sont des formules, donc $F_1$ n'est pas un préfixe propre de $G_1$, ou encore $F_1 = G_1$.
    Par conséquent, $\verb|(| F_1\ \sigma\ F_2 \verb|)| = \verb|(| F_1\ \sigma'\ G_2 \verb|)|$, d'où $\sigma = \sigma'$.
    On conclut bien l'unicité.
  \end{prv}

  % Le théorème précédent n'est pas suffisant pour dire que ceci est un corollaire, il faudrait aussi que dans chaque règle, les sous-formules obtenues soient uniques, mais c'est insupportable à écrire proprement.
  \begin{crlr}
    Il y a une bijection entre les formules et les arbres dont
    \begin{itemize}
      \item les feuilles sont étiquetées par des variables ;
      \item les nœuds internes sont étiquetés par des connecteurs ;
      \item ceux étiquetés par $\lnot$ ont un fils, les autres deux.
    \end{itemize}
  \end{crlr}

  \begin{exm}
    La formule $((\lnot x_0 \lor x_1) \to ((x_0 \land x_2) \leftrightarrow x_3))$ correspond à l'arbre 
    \begin{figure}[H]
      \centering
      \begin{tikzpicture}
        \node {$\to$}
        child[sibling distance=120] { node {$\lor$}
          child[sibling distance=60] { node {$\lnot$} child { node {$x_0$} } }
          child[sibling distance=60] { node {$x_1$} }
        }
        child[sibling distance=120] { node {$\leftrightarrow$}
          child[sibling distance=60] {
            node {$\land$}
            child { node {$x_0$} }
            child { node {$x_2$} }
          }
          child[sibling distance=60] { node {$x_3$} }
        }
        ;
      \end{tikzpicture}
    \end{figure}
  \end{exm}

  \section{Sémantique.}
  
  \begin{lem}
    Soit $\nu$ une fonction de $\mathcal{P}$ dans $\{0,1\}$ appelée \textit{valuation}.
    Alors $\nu$ s'étend de manière unique en une fonction $\bar{\nu}$ de $\mathcal{F}$ dans $\{0,1\}$ telle que 
    \begin{itemize}
      \item sur $\mathcal{P}$, $\nu = \bar{\nu}$ ;
      \item si $F, G \in \mathcal{F}$ sont des formules alors 
        \begin{itemize}
          \item $\bar{\nu}(\lnot F) = 1 - \bar{\nu}(F)$ ;
          \item $\bar{\nu}(F \lor G) = \max(\bar{\nu}(F), \bar{\nu}(G))$ ;
          \item $\bar{\nu}(F \land G) = \bar{\nu}(F) \times \bar{\nu}(G)$ ;
          \item $\bar{\nu}(F \to G) = 1$ ssi $\bar{\nu}(F) \le \bar{\nu}(G)$ ;
          \item $\bar{\nu}(F \leftrightarrow G) = 1$ ssi $\bar{\nu}(F) = \bar{\nu}(G)$.
        \end{itemize}
    \end{itemize}
    Par abus de notations, on notera $\nu$ pour $\bar{\nu}$ par la suite.
  \end{lem}
  \begin{prv}[Idée de preuve]
    \begin{description}
      \item[Existence.]
        On définit en utilisant le lemme de lecture unique, et par induction sur $\mathcal{F}$ :
        \begin{itemize}
          \item $\bar{\nu}$ est définie sur $\mathcal{F}_0 = \mathcal{P}$ ;
          \item si $\bar{\nu}$ est définie sur $\mathcal{F}_n$ alors pour $F \in \mathcal{F}_{n+1}$, on a la disjonction de cas
            \begin{itemize}
              \item si $F = \lnot G$ avec $G \in \mathcal{F}_n$, et on définit $\bar{\nu}(F) = 1 - \bar{\nu}(F_1)$ ;
              \item \textit{etc} pour les autres cas.
            \end{itemize}
        \end{itemize}
      \item[Unicité.]
        On peut montrer que si $\bar{\nu}$ et $\bar{\nu}'$ sont deux extensions de $\nu$, alors $\bar{\nu}$ et $\bar{\nu}'$ coïncident sur $\mathcal{F}_n$, pour tout $n \in \mathds{N}$, ce que l'on prouve par récurrence sur $n$.
    \end{description}
  \end{prv}

  \begin{exm}[Table de vérité]
    Pour la formule \[F = \verb|((|x_1\to x_2\verb|)| \to \verb|(|x_2 \to x_1\verb|))|, \] on construit la table 
    \begin{table}[H]
      \centering
      \begin{tabular}{r|cccc}
        $x_1$ & 0 & 0 & 1 & 1\\ \hline
        $x_2$ & 0 & 1 & 0 & 1\\ \hline
        $x_1 \to x_2$ & 1 & 1 & 0 & 1\\ \hline
        $x_2 \to x_1$ & 1 & 0 & 1 & 1\\ \hline
        $F$ & 1 & 0 & 1 & 1
      \end{tabular}
    \end{table}
  \end{exm}

  \begin{defn}
    \begin{itemize}
      \item Une formule $F$ est dite \textit{satisfaite par une valuation $\nu$} si  $\nu(F) = 1$.
      \item Une \textit{tautologie} est une formule satisfaite pour toutes les valuations.
      \item Un ensemble $\mathcal{E}$ de formules est \textit{satisfiable} s'il existe une valuation qui satisfait toutes les formules de $\mathcal{E}$.
      \item Un ensemble $\mathcal{E}$ de formules est \textit{finiment satisfiable} si tout sous-ensemble fini de $\mathcal{E}$ est satisfiable.
      \item Une formule $F$ est \textit{conséquence sémantique} d'un ensemble de formules $\mathcal{E}$ si toute valuation qui satisfait $\mathcal{E}$ satisfait $F$.
      \item Un ensemble de formules $\mathcal{E}$ est \textit{contradictoire} s'il n'est pas satisfiable.
      \item Un ensemble de formules $\mathcal{E}$ est \textit{finiment contradictoire} s'il existe un sous-ensemble fini contradictoire de $\mathcal{E}$.
    \end{itemize}
  \end{defn}

  \begin{thm}[compacité du calcul propositionnel]
    On donne trois énoncés équivalents (équivalence des trois énoncés laissée en exercice) du théorème de compacité du calcul propositionnel.

    \begin{description}
      \item[Version 1.] Un ensemble de formules $\mathcal{E}$ est satisfiable si et seulement s'il est finiment satisfiable.
      \item[Version 2.] Un ensemble de formules $\mathcal{E}$ est contradictoire si et seulement s'il est finiment contradictoire.
      \item[Version 3.] Pour tout ensemble $\mathcal{E}$ de formules du calcul propositionnel, et toute formule $F$, $F$ est conséquence sémantique de $\mathcal{E}$ si et seulement si $F$ est conséquence sémantique d'un sous-ensemble fini de $\mathcal{E}$.
    \end{description}

    \label{chap1-comp}
  \end{thm}
  \begin{prv}
    Supposons que $\mathcal{P}$ soit dénombrable. Le cas non-dénombrable sera traité après.

    \begin{description}
      \item["$\implies$".]
        Si $\nu$ satisfait toute formule de $\mathcal{E}$, alors tout sous-ensemble fini de $\mathcal{E}$ est satisfiable en utilisant la valuation $\nu$.
      \item["$\impliedby$".]
        Supposons $\mathcal{E}$ finiment satisfiable. Comme $\mathcal{P}$ est supposé dénombrable, notons $\mathcal{P} = \{x_1, x_2, \ldots\}$.
        % Ah ! Espèce de mathématicien qui compte à partir de 1 !
        On cherche une valuation $\nu$ qui satisfait toute formule de $\mathcal{E}$, et on va définir la suite des $\varepsilon_n = \nu(x_n)$ par récurrence.
    \end{description}

    \begin{enumerate}
      \item On construit par récurrence une suite $(\varepsilon_n)_{n \ge 1}$ qui satisfait, pour tout $n \in \mathds{N}$,
        \begin{quote}
          "pour toute partie finie $B$ de $\mathcal{E}$, il existe une valuation $\lambda$ satisfaisant  $B$ et telle que pour tout~$i \in \llbracket 1, n\rrbracket$, $\lambda(x_i) = \varepsilon_i$".
        \end{quote}
        On note cette propriété $\mathcal{R}_n$.
        \begin{itemize}
          \item Pour $n = 0$, la propriété $\mathcal{R}_0$
            \begin{quote}
              "pour toute partie finie $B$ de $\mathcal{E}$, il existe une valuation $\lambda$ satisfaisant  $B$",
            \end{quote}
            est vraie car $\mathcal{E}$ est supposé finiment satisfiable.
          \item Soit $n \in \mathds{N}$. Supposons que $\varepsilon_1,\ldots,\varepsilon_n$ soient construits et qu'ils satisfont $\mathcal{R}_n$.
            On a la disjonction de cas suivante :
            \begin{itemize}
              \item Si pour toute partie finie $B$ de $\mathcal{E}$, il existe une valuation $\lambda$ telle que $\forall i \in \llbracket 1, n\rrbracket$, $\lambda(x_i) = \varepsilon_i$ et que  $\lambda(x_{i+1}) = 0$, alors posons $\varepsilon_{n+1} := 0$ et on a immédiatement $\mathcal{R}_{n+1}$.
              \item S'il existe une partie $A$ de $\mathcal{E}$, telle que $(\star)$ toute valuation $\lambda$ satisfaisant $A$ et telle que $\lambda(x_i) = \varepsilon_i$ pour tout  $i \in \llbracket 1,n\rrbracket$ vérifie $\lambda(x_{n+1}) = 1$.
                On pose donc $\varepsilon_{n+1} := 1$.

                Montrons la propriété $\mathcal{R}_{n+1}$.
                Considérons une partie finie $B$ de $\mathcal{E}$.
                Par $\mathcal{R}_n$, il existe une valuation $\eta$ qui satisfait $A \cup B$, sous-ensemble fini de $\mathcal{E}$, et qui vérifie $\forall i \in \llbracket 1,n\rrbracket$, $\eta(x_i) = \varepsilon_i$.
                En particulier $\eta$ satisfait $A$, ce qui implique par $(\star)$, que $\eta(x_{n+1}) = 1 = \varepsilon_{n+1}$.
                Et comme $\eta$ satisfait $B$, on vérifie bien $\mathcal{R}_{n+1}$.
            \end{itemize}
        \end{itemize}

      \item Posons $\nu : x_i \mapsto \varepsilon_i$.
        Montrons que $\nu$ satisfait $\mathcal{E}$.
        Soit $F \in \mathcal{E}$.
        On sait que $F$ est un mot (fini), donc contient un ensemble fini de variables.
        Soit $n$ maximal tel que $x_n$ est une variable de $F$, de sorte que l'ensemble des variables de $F$ soit inclus dans $\{x_0, \ldots, x_n\}$.
        D'après $\mathcal{R}_n$, il existe une valuation $\eta$ qui satisfait $F$ et telle que $\eta(x_0) = \varepsilon_0, \ldots, \eta(x_n) = \varepsilon_n$.
        En particulier, $\nu$ et $\eta$ coïncident sur les variables de $F$.
        Donc (lemme simple), elles coïncident sur toutes les formules qui n'utilisent que ces variables.
        Donc, $\eta(F) = 1$, et on en conclut que $\eta$ satisfait $\mathcal{E}$.
    \end{enumerate}
  \end{prv}

  Dans le cas non-dénombrable, on utilise le \textit{lemme de Zorn}, un équivalent de l'\textit{axiome du choix}.

  \begin{defn}
    Un ensemble ordonné $(X, \mathcal{R})$ est inductif si pour tout sous-ensemble $Y$ de $X$ totalement ordonné par $\mathcal{R}$ (\textit{i.e.} une chaîne) admet un majorant dans $X$.
  \end{defn}

  \begin{rmk}
    On considère ici un majorant et non un plus grand élément (un maximum).
  \end{rmk}

  \begin{exm}
    \begin{enumerate}
      \item $(\mathcal{P}(X), \subseteq )$ est inductif, le majorant d'une chaîne étant l'union des parties de la chaîne.
      \item $(\mathds{R}, \le)$ n'est pas inductif car $\mathds{R}$ n'a pas de majorant dans $\mathds{R}$.
    \end{enumerate}
  \end{exm}

  \begin{lem}[Lemme de Zorn]
    Tout ensemble ordonné inductif non-vide admet un élément maximal.
  \end{lem}

  \begin{rmk}
    Un élément maximal n'est pas nécessairement le plus grand.
  \end{rmk}

  \vspace{1.5em}

  \begin{prv}[Cas non-dénombrable pour le théorème~\ref{chap1-comp}]
    Soit $\mathcal{E}$ un ensemble de formules finiment satisfiable, et $\mathcal{P}$ un ensemble de variables.
    On note $\mathcal{V}$ l'ensemble des valuations partielles prolongeables pour toute partie finie $\mathcal{C}$ de $\mathcal{E}$ en une valuation satisfaisant $\mathcal{C}$.
    C'est-à-dire :

    {
      \footnotesize
    \[
    \mathcal{V} := \mleft\{\,\varphi \in \bigcup_{X \subseteq \mathcal{P}} \{0,1\}^X \;\middle|\; \forall \mathcal{C} \in \wp_\mathrm{f}(\mathcal{E}), \exists \delta \in \{0,1\}^\mathcal{P}, 
      \begin{array}{l}
        \delta_{|\operatorname{dom}\varphi} = \varphi\\
        \forall F \in \mathcal{C}, \delta(F) = 1
      \end{array}
    \,\mright\} 
    .\] 
    }

    L'ensemble $\mathcal{V}$ est non-vide car il contient l'application vide de $\{0,1\}^\emptyset$ car $\mathcal{E}$ est finiment satisfiable.
    On définit une relation d'ordre~$\preceq$ sur $\mathcal{V}$ par : \[
      \varphi \preceq \psi \quad \text{ ssi } \quad \psi \text{ prolonge } \varphi
    .\]
    Montrons que $(\mathcal{V}, \preceq)$ est inductif.
    Soit $\mathcal{C}$ une chaîne de $\mathcal{V}$, construisons un majorant de $\mathcal{C}$.
    Soit $\lambda$ la valuation partielle définie sur $\operatorname{dom}\lambda = \bigcup_{\varphi \in \mathcal{C}} \operatorname{dom} \varphi$, par : 
    si $x_i \in \operatorname{dom} \lambda$ alors il existe $\varphi \in \mathcal{C}$ tel que $x_i \in \operatorname{dom} \varphi$ et on pose $\lambda(x_i) = \varphi(x_i)$.

    La valuation $\lambda$ est définie de manière unique,  \textit{i.e.} ne dépend pas du choix de $\varphi$. En effet, si $\varphi \in \mathcal{C}$ et $\psi \in \mathcal{C}$, avec $x_i \in \operatorname{dom} \varphi \cap \operatorname{dom} \psi$, alors on a $\varphi \preceq \psi$ ou $\psi \preceq \varphi$, donc $\varphi(x_i) = \psi(x_i)$.

    Notons que $\lambda$ est un majorant de $\mathcal{C}$.
    Montrons que $\lambda \in \mathcal{V}$.
    Soit $B$ une partie finie de $\mathcal{E}$. On cherche $\mu$ qui prolonge $\lambda$ et satisfait $B$.
    L'ensemble de formules $B$ est fini, donc utilise un ensemble fini de variables, dont l'intersection avec $\operatorname{dom}(\lambda)$ est un sous-ensemble fini $\{x_{i_1}, \ldots, x_{i_n}\}$.
    Il existe $\varphi_1, \ldots, \varphi_n$ dans $\mathcal{C}$ telle que $x_{i_1} \in \operatorname{dom} \varphi_1, \ldots, x_{i_n} \in \operatorname{dom} \varphi_n$.
    Comme $\mathcal{C}$ est une chaîne, $\varphi_0 = \max_{i \in \llbracket 1,n\rrbracket} \varphi_i$ est bien défini et on a $\varphi_0 \in \mathcal{C}$.
    Dès lors, $\varphi_0 \in \mathcal{V}$ est prolongeable en $\psi_0$ qui satisfait $B$.
    On a de plus $x_{i_1}, \ldots, x_{i_n} \in \operatorname{dom}\varphi_0$.
    On définit :
    \begin{align*}
      \mu: \mathcal{P} &\longrightarrow \{0,1\}  \\
      x \in \operatorname{dom} \lambda &\longmapsto \lambda(x) \\
      x \in \operatorname{var} B &\longmapsto \psi_0(x) \\
      \text{sinon} &\longmapsto 0
    .\end{align*}

    Comme $\lambda$ et $\psi_0$ coïncident sur $\operatorname{dom} \lambda \cap \operatorname{var} B$, $\mu$ est bien définie.
    De plus, comme $\mu$ et $\psi_0$ coïncident sur $\operatorname{var} B$, $\mu$ satisfait $B$.
    Donc $\lambda \in \mathcal{V}$ d'où l'on tire que $\mathcal{V}$ est inductif.

    D'après le lemme de Zorn, $\mathcal{V}$ admet un élément maximal $\varphi$.
    Pour montrer le théorème, il suffit de montrer que $\operatorname{dom} \gamma = \mathcal{P}$.
    % Techniquement non, mais il ne reste plus qu'à pousser les cartes un fois qu'on a ça.
    Si $\operatorname{dom} \gamma \neq \mathcal{P}$, soit $x \not\in \operatorname{dom} \gamma$.
    Montrons que $\gamma$ n'est pas maximal en définissant $\gamma' \in \mathcal{V}$ qui vérifie $\gamma \prec \gamma'$.
    On prend $\operatorname{dom}\gamma' = \operatorname{dom} \gamma \cup \{x\}$.
    \begin{itemize}
      \item Si, pour toute partie finie $B$ de $\mathcal{E}$, il existe une valuation $\delta$ qui prolonge $\gamma$ et qui satisfait $B$ telle que $\delta(x) = 0$, alors on pose $\gamma'(x) = 0$.
      \item Sinon, on pose $\gamma'(x) = 1$.
    \end{itemize}
    Montrons que $\gamma' \in \mathcal{V}$.
    Soit $B$ un ensemble fini de formules de $\mathcal{E}$.
    \begin{itemize}
      \item Si $\gamma'(x) = 0$ alors il existe une valuation $\delta$ prolongeant $\gamma$ et satisfaisant $B$ telle que $\delta(x) = 0$, et donc  $\delta$ prolonge $\gamma'$.
      \item Si $\gamma'(x) = 1$ alors il existe une partie finie $B_0$ de $\mathcal{E}$ telle que toute valuation $\delta$ prolongeant $\gamma$ et satisfaisant $B_0$ vérifie que $\delta(x) = 1$.
        On choisit une valuation qui prolonge $\gamma$ et satisfait $B \cup B_0$ ; elle prolonge $\gamma'$.
    \end{itemize}
    On a donc une contradiction avec la maximalité de $\gamma$. On en conclut que $\operatorname{dom} \gamma = \mathcal{P}$, ce qui termine la preuve du théorème.
  \end{prv}
\end{document}
