\documentclass[./main]{subfiles}

\begin{document}
  \chapter{Anneaux.}

  \section{Généralités.}

  \begin{defn}
    Un \textit{anneau} est un ensemble $A$ contenant $0,1$ et munis de deux lois internes notées $+$ et $\cdot$ telles que 
    \begin{itemize}
      \item $(A, +)$ est un groupe abélien de neutre $0$ ;
      \item $\forall  a, b, c \in A, \quad a (b c) = (a b) c$ ;
      \item $\forall a \in A, \quad a \cdot 1 = 1 \cdot a = a$ ;
      \item $\forall a, b, c \in A, \quad a (b + c) = a b + a c$ et $(b + c) a = ba + ca$.
    \end{itemize}
  \end{defn}

  \begin{rmk}
    De cette définition, on peut en déduire certaines règles de calculs :
    \begin{itemize}
      \item $\forall a \in A, \quad 0 \cdot a = 0$ ;
      \item $\forall a,b \in A, \quad (-a) b = - ab$ ;
      \item $\forall  a, b \in A, \quad (-a)(-b) = ab$ ;
      \item $\forall a, b \in A, \quad (a+b)^n = \sum_{k=0}^n {n \choose k} a^k b^{n-k}$ quand $A$ est \textit{commutatif}.
    \end{itemize}
  \end{rmk}

  \begin{defn}
    On dit que $A$ est \textit{commutatif} si pour tous $a, b \in A$, on a $ab = ba$.
  \end{defn}

  \begin{exm}
    Les ensembles $\mathds{Z}, \mathds{R}, \mathds{C}, \mathds{Q}$ sont des anneaux commutatifs.
    L'ensemble  \[
      \mathds{Z}[\sqrt{3}] := \mleft\{\,n + m \sqrt{3} \;\middle|\; n,m \in \mathds{Z}\,\mright\} \subseteq \mathds{R}
    \] est un anneau : en effet, \[
    \small (n + m \sqrt{3})(n' + m' \sqrt{3}) = n n' + (n m' + n'm) \sqrt{3}  + 3 m m' \in \mathds{Z}[\sqrt{3}]
    .\] 
    Mais, $\mathds{Z}[\sqrt[3]{3}]$ \textit{\textbf{n'est pas}} un anneau.

    De même, l'ensemble $\mathds{Q}[\sqrt{3}] \subseteq \mathds{R}$ est un anneau, c'est même un corps car il est stable par passage à l'inverse.
  \end{exm}

  \begin{defn}
    Soit $A$ un anneau.
    Une partie $B \subseteq A$ est appelée un \textit{sous-anneau} si $B$ contient $1$ et si la restriction des lois de $A$ à~$B$ lui confère une structure d'anneau.

    Autrement dit, si $B$ contient $1$ et est stable par somme et produit, c'est un sous-anneau de $A$.
  \end{defn}

  \begin{exm}
    L'ensemble $\mathds{Z}[i] \subseteq \mathds{C}$  est un sous-anneau de $\mathds{C}$. De même que pour $\mathds{Z}[\sqrt{3}]$, on vérifie facilement qu'il est stable par produit.
  \end{exm}

  \begin{defn}
    Soient $A \subseteq B$ des anneaux et $E$ une partie de~$B$.
    On note $A[E]$ l'anneau \textit{engendré par $E$ sur $A$} le plus petit sous-anneau de $B$ contenant $A$ et $E$. C'est l'intersection des sous-anneaux de $B$ contenant $A$ et $E$.
  \end{defn}

  \begin{rmk}
    Si $B$ est commutatif, alors $A[E]$ est l'ensemble des sommes finis de monômes de la forme $a e_1^{n_1} \cdots e_s^{n_s}$ avec $a \in A$, $e_i \in E$ et $n_i \in \mathds{N}$.
  \end{rmk}

  \begin{exm}
    Soit $G$ un groupe.

    On note $\mathds{C}[G] := \mathds{C}^{|G|} = \bigoplus_{g \in G} \mathds{C} \cdot \langle g \rangle$.
    Ses éléments sont de la forme $\sum_{g \in G} a_g \langle g\rangle$ avec $a_g \in \mathds{C}$.
    On définit $\langle g \rangle \cdot \langle h \rangle = \langle g h \rangle$ et puis~$\langle g \rangle a = a \langle g \rangle$, pour tout  $a \in \mathds{C}$.
    On définit alors le produit sur~$\mathds{C}[G]$ par :
    \begin{align*}
    \Big(\sum_{g \in G} a_g \langle g \rangle\Big) \cdot 
    \Big(\sum_{h \in G} b_h \langle g \rangle\Big)
    &= \sum_{g \in G} \sum_{h \in H} a_g b_h \langle gh \rangle\\
    &= \sum_{\ell \in G} \Big(\sum_{gh = \ell} a_g b_h\Big) \langle \ell \rangle \\
    &= \sum_{\ell \in G} \Big( \sum_{g \in G} a_g b_{g^{-1} \ell}\Big) \langle \ell \rangle\\
    .\end{align*}
  \end{exm}

  \begin{defn}
    Soit $A$ un anneau.
    On appelle  \textit{$A$-module} un ensemble $M$ muni d'une loi interne $+$ et d'une loi externe $A \times M \to M, (a, m) \mapsto a m$ telle que $(M, +)$ est un groupe abélien et que, pour tous  $a, b \in A$ et $m, n \in M$, on a :
    \begin{itemize}
      \item $a(bm) = (ab)m$ ;
      \item  $1_A m = m$ ;
      \item  $0_A m = m$ ;
      \item $(a+b) m = am + bm$ ;
      \item $(-b) m = -(bm)$ ;
      \item  $a(m+n) = am + an$.
    \end{itemize}
  \end{defn}

  \begin{rmk}
    Les groupes abéliens correspondent exactement aux $\mathds{Z}$-modules.
    On peut ainsi définir généralement les $A$-modules comme une donné d'un morphisme \[
    f : A \to \mathrm{End}(M)
    .\]
  \end{rmk}

  \begin{defn}
    Un \textit{morphisme d'anneau} $f : A \to B$ est une application telle que, pour tous $a, b \in A$, on a :
    \begin{itemize}
      \item $f(a + b) = f(a) + f(b)$ ;
      \item  $f(ab) = f(a) \: f(b)$ ;
      \item  $f(1_A) = 1_B$.
    \end{itemize}
  \end{defn}

  \begin{exm}
    On considère \begin{align*}
      \varphi: G &\longrightarrow \mathds{C}[G] \\
      g &\longmapsto \langle g \rangle
    .\end{align*}
    Alors, $\varphi(G)$ est un sous-groupe de  $(\mathds{C}[G], \cdot)$ isomorphe à $G$.
    Les représentations de $G$ correspondent exactement aux $\mathds{C}[G]$-modules. En effet, la donnée d'un morphisme de groupes $G \to \mathrm{Aut}_\mathds{C}(V)$ est équivalente à la donnée d'un morphisme d'anneau $\mathds{C}[G] \to \mathrm{End}_\mathds{C}(V)$.
  \end{exm}

  \textbf{\textsl{À partir de maintenant, tous les anneaux considérés sont commutatifs.}}

  \begin{defn}
    Soit $A$ un anneau et $a \in A$.
    \begin{enumerate}
      \item On dit que $a$ est \textit{nilpotent} s'il existe $n \in \mathds{N}^*$ tel que $a^n = 0$.
      \item On dit que $a$ est une \textit{racine de l'unité} s'il existe $n \in \mathds{N}^*$ tel que $a^n = 1$.
      \item On dit que $a$ est \textit{idempotent} si $a^2 = a$.
      \item On dit que $a \neq 0$ est un \textit{diviseur de zéro} s'il existe $b \neq 0$ tel que $a b = 0$.
      \item On dit que $a$ est \textit{inversible} s'il existe $b \in A$ tel que $ab = 1$. On notera $A^\times$ l'ensemble des éléments inversibles. L'ensemble $(A^\times, \cdot)$ forme un groupe.
      \item On dit que $A$ est un \textit{corps} si $A^\times = A \setminus \{0\}$.
      \item On dit que $A$ est \textit{intègre} si $A$ ne contient pas de diviseurs de zéro.
    \end{enumerate}
  \end{defn}

  \begin{rmk}
    \begin{itemize}
      \item On verra que $A$ est intègre si et seulement si $A$ est un sous-anneau d'un corps.
      \item Un sous-anneau d'un anneau intègre est intègre.
    \end{itemize}
  \end{rmk}

  \begin{lem}
    Soit $A$ un anneau intègre. Soient $a, b, c \in A$ avec $a \neq 0$. Alors, si $a b = a c$ on a  $b = c$, \textit{i.e.} on peut simplifier par $a$.
  \end{lem}
  \begin{prv}
    On a $ab - ac = 0$ donc $a(b-c) = 0$. Alors $b -c = 0$ car $A$ est intègre et $a \neq 0$. D'où, $b = c$.
  \end{prv}

  \section{Divisibilité.}

  \begin{defn}
    Soit $A$ un anneau. On dit que $a$ \textit{divise} $b$ et on note $a  \mid b$ s'il existe $c \in A$ tel que $b = a c$.
  \end{defn}

  \begin{rmk}
    Cette relation dépend de $A$. En effet, on peut avoir $A \subseteq B$ et $a,b \in A$ tels que $a  \mid b$ dans $B$ mais $a \nmid b$ dans $A$. Par contre, si $a  \mid b$ dans $A$ alors $a  \mid b$ dans $B$.
  \end{rmk}

  \begin{prop}
    Soient $A$ un anneau et $a, b, c \in A$.
    \begin{enumerate}
      \item On a $a  \mid a$.
      \item Si $a  \mid b$ et $b  \mid c$ alors $a  \mid c$.
      \item Si $a  \mid b$ et $a  \mid c$ alors $a  \mid \alpha b + \beta c$ pour $\alpha, \beta \in A$.
      \item Si $c a  \mid cb$ avec $c \neq 0$ et $A$ intègre alors $a  \mid b$. Autrement dit, on peut simplifier par $c$.
      \item Si $c \in A^\times$ alors $c  \mid a$ (car $a = a c^{-1} c$).
      \item On a $a  \mid 0$ (car $0 = a \cdot 0$).
      \item Si $a  \mid b$ et $b  \mid a$ et $a$ n'est pas un diviseur de zéro, alors $a = x b$ avec $x \in A^\times$.
      \item Pour tout $x \in A^\times$ on a équivalence :
        \[
        a  \mid b \iff a  \mid x b \iff x a  \mid b
        .\]
    \end{enumerate}
  \end{prop}

  \begin{rmk}
    La divisibilité se comporte mieux dans les \textit{modules inversibles}.
  \end{rmk}

  \begin{rmk}
    On a la chaîne d'inclusions :
    \begin{gather*}
      \text{Anneaux}\\
      \vertical \supseteq \\
      \text{Anneaux commutatifs} \\
      \vertical \supseteq \\
      \text{Anneaux commutatifs intègres} \\
      \vertical \supseteq \\
      \text{Anneaux intègres noethériens} \\
      \vertical \supseteq \\
      \text{Anneaux factoriels} \\
      \vertical \supseteq \\
      \text{Anneaux principaux} \\
      \vertical \supseteq \\
      \text{Anneaux euclidiens} \\
      \vertical \supseteq \\
      \text{Corps}
    \end{gather*}
  \end{rmk}

\end{document}
