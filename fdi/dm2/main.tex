\documentclass{../../notes}

\title{DM n°2 -- FDI}
\author{Hugo \scshape Salou}

\tikzset{
  initial text=,
  node distance=2cm,
  auto,
  every state/.style={draw=deepblue,thick},
  every path/.style={draw=deepblue},
}

\begin{document}
  \maketitle
  \chapter*{Exercice 1.}

  On pose, pour tout $n \in \mathds{N}$, le mot $v_n = 1^n 2$.
  Montrons que les mots de la suite $(v_n)_{n \in \mathds{N}}$ sont deux-à-deux non-équivalents pour la relation de congruence $\equiv_L$.

  Considérons les mots $v_n = 1^n 2$ et $v_m = 1^m 2$ pour $n \neq m$.
  Sans perdre en généralité, supposons $n < m$.
  Soit $z$ le mot $2^{m-1}$.
  Remarquons que l'on a  $v_m z = 1^m 2^m \in L$ mais que $v_n z = 1^n 2^m \not\in L$.

  Le théorème de \textsc{Myhill-Nerodes} démontre ainsi que l'automate minimal reconnaissant $L$ a un nombre infini d'états.
  On en conclut que $L$ n'est pas rationnel.

  \chapter*{Exercice 2.}

  On applique l'algorithme de Moore en calculant les classes d'équivalences des relations $\equiv_i$ pour tout $i \in \mathds{N}$ jusqu'à sa stationnarité :
  \begin{itemize}
    \item \textsl{\color{deepblue} Étape $0$.} les classes sont $\{1,2,6\}$ et $\{3,4,5\}$ ;\label{ex2-step0}
    \item \textsl{\color{deepblue} Étape $1$.} les classes sont $\{1,2\}$, $\{6\}$ et $\{3,4,5\}$ ;\label{ex2-step1}
    \item \textsl{\color{deepblue} Étape $2$.} les classes sont $\{1\}$, $\{2\}$, $\{6\}$ et $\{3,4,5\}$ ;\label{ex2-step2}
    \item \textsl{\color{deepblue} Étape $3$.} les classes sont $\{1\}$, $\{2\}$, $\{6\}$ et $\{3,4,5\}$.\label{ex2-step3}
  \end{itemize}
  On s'arrête à l'étape~\hyperref[ex2-step3]{3} car on a atteint un point fixe (${\equiv_{\hyperref[ex2-step2]{2}}} = {\equiv_{\hyperref[ex2-step3]{3}}}$).

  On en déduit l'automate minimal : chaque état représente une classe d'équivalence pour la congruence de \textsc{Nerode}.

  \begin{figure}[H]
    \centering
    \begin{tikzpicture}
      \node[state,initial] (q1) {1};
      \node[state,right of=q1] (q2) {2};
      \node[state,right of=q2,gray] (q6) {6};
      \node[state,above of=q2,accepting,rounded rectangle] (q3) {\phantom{--}$3,4,5$\phantom{--}};

      \draw[->]
        (q1) edge node{$a$} (q2)
        (q1) edge node{$b$} (q3)
        (q2) edge node{$b$} (q3)
        (q3) edge[loop above] node{$a$} (q3)
        ;

      \draw[->,gray]
        (q3) edge[gray] node[swap]{$b$} (q6)
        (q2) edge[gray] node{$a$} (q6)
        (q6) edge[gray,loop right] node{$a,b$} (q6)
        ;
        
    \end{tikzpicture}
    \caption{Automate minimal équivalent}
  \end{figure}
\end{document}
