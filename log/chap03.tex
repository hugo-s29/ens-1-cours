\documentclass[./main]{subfiles}

\begin{document}
  \chapter{L'arithmétique de Peano.}
  \begin{itemize}
    \item \textsc{Dedekind} (1988) et \textsc{Peano} (1889) formalisent l'arithmétique.
    \item En 1900, David \textsc{Hilbert}, lors du 2ème ICM à Paris, donne un programme et dont le 2nd problème est la \textit{cohérence de l'arithmétique}.
    \item En 1901, \textsc{Russel} donne son paradoxe concernant l'"ensemble" de tous les ensembles.
    \item En 1930, \textsc(Hilbert) est toujours optimiste : "On doit savoir, on saura !"
  \end{itemize}

  La formalisation de l'arithmétique engendre deux questions :
  \begin{enumerate}
    \item est-ce que tout théorème est prouvable ? ($\triangleright$ complétude)
    \item existe-t-il un algorithme pour décider si un théorème est prouvable ? ($\triangleright$ décidabilité)
  \end{enumerate}

  Le second point est appelé "\textit{Entscheidungsproblem}", le problème de décision, en 1928.

  \begin{itemize}
    \item En 1931, Gödel répond \textsc{non} à ces deux questions.
  \end{itemize}

  On a donné plusieurs formalisations des algorithmes :
  \begin{itemize}
    \item en 1930, le $\lambda$-calcul de Church ;
    \item en 1931--34, les fonctions récursives de Herbrand et Gödel ;
    \item en 1936, les machines de Turing.
  \end{itemize}
  On démontre que les trois modèles sont équivalents.

  La thèse de Church--Turing nous convainc qu'il n'existe pas de modèle plus évolué "dans la vraie vie".

  \section{Les axiomes.}

  On définit le langage $\mathcal{L}_0 = \{\zero, \succ, \oplus, \otimes\}$ où
  \begin{itemize}
    \item $\zero$ est un symbole de constante ;
    \item $\succ$ est un symbole de fonction unaire ;
    \item $\oplus$ et $\otimes$ sont deux symboles de fonctions binaires.
  \end{itemize}

  On verra plus tard que l'on peut ajouter une relation binaire $\le$.

  \begin{rmk}[Convention]
    La structure $\mathds{N}$ représente la $\mathcal{L}_0$-structure dans laquelle on interprète les symboles de manière habituelle :
    \begin{itemize}
      \item pour $\zero$, c'est $0$ ;
      \item pour $\succ$, c'est $\lambda n.  n + 1$ (\textit{i.e.} $x \mapsto x + 1$) ;
      \item pour $\oplus$, c'est $\lambda n \: m.  n + m$ ;
      \item pour $\otimes$, c'est $\lambda n \: m.  n \times m$.
    \end{itemize}
  \end{rmk}

  \subsubsection{Les axiomes de Peano.}

  On se place dans le cas égalitaire.
  L'ensemble $\mathcal{P}$ est composé de $\mathcal{P}_0$ un ensemble fini d'axiomes (A1--A7) et d'un schéma d'induction (SI).

  Trois axiomes pour le successeur :
  \begin{description}
    \item[A1.] $\forall x \: \lnot (\succ x = \zero)$
    \item[A2.] $\forall x \: \exists y \: \big(\lnot (x = \zero) \to x = \succ y\big)$
    \item[A3.] $\forall x \: \forall y \: (\succ x = \succ y \to x = y)$
  \end{description}

  Deux axiomes pour l'addition :
  \begin{description}
    \item[A4.] $\forall x \: (x \oplus \zero = x)$ 
    \item[A5.] $\forall x \: \forall y \: (x \oplus (\succ y) = \succ(x \oplus y))$
  \end{description}

  Deux axiomes pour la multiplication :
  \begin{description}
    \item[A6.] $\forall x \: (x \otimes \zero = \zero)$
    \item[A7.] $\forall x \: \forall y \: \big(x \otimes (\succ y) = (x \otimes y) \oplus x\big)$
  \end{description}

  Et le schéma d'induction :
  \begin{description}
    \item[SI.]
      Pour toute formule $F$ de variables libres $x_0, \ldots, x_n$,
  \end{description}
  \fitbox{$\forall x_1 \: \cdots \: \forall x_n \: \Big(
    \big(F(\zero, \ldots, x_1, \ldots, x_n) \land \forall x \: (F(x, x_1, \ldots, x_n) \to F(\succ x, x_1, \ldots, x_n))\big)
    \to \forall x \: F(x, x_1, \ldots, x_n)
  \Big).$}

  \begin{rmk}
    \begin{itemize}
      \item Le schéma est le SI avec hypothèse faible, qui permet de montrer le SI avec hypothèse forte.
        On adopte la notation $\forall y \le x \: F(y, x_1, \ldots, x_n)$ pour \[
        \forall y \: \big( (\exists z \:  z \oplus y = x) \to  F(y, x_1, \ldots, x_n) \big)
        .\]
        Le SI avec hypothèse forte est :
    \end{itemize}
    \fitbox{
      $\forall x_1 \: \cdots \: \forall x_n \: \Big(
      \big(F(\zero, \ldots, x_1, \ldots, x_n) \land \forall x \: \big((\forall y \le  x \: F(y, x_1, \ldots, x_n)) \to F(\succ x, x_1, \ldots, x_n)\big)\big)
      \to \forall x \: F(x, x_1, \ldots, x_n)
      \Big)$
    }
    \begin{itemize}
      \item L'ensemble $\mathcal{P}$ est non-contradictoire car $\mathds{N}$ est un modèle, appelé \textit{modèle standard}.
      \item On peur remplacer le SI par une nouvelle règle de démonstration :
        \[
        \begin{prooftree}
          \hypo{\Gamma \vdash F(\zero)}
          \hypo{\Gamma \vdash \forall y\: \big(F(y) \to F(\succ y)\big)}
          \infer 2[\mathsf{rec}]{\Gamma \vdash \forall x \: F(x)}
        \end{prooftree}
        .\]
    \end{itemize}
  \end{rmk}

  \begin{exo}
    Montrer l'équivalence entre SI et la nouvelle règle $\mathsf{rec}$, \textit{i.e.} on peut démontrer les mêmes théorèmes.
  \end{exo}

  \begin{nota}
    On note $\repr{n}$ le terme  $\underbrace{\succ \cdots  \succ}_{n \text{ fois}}  \zero$ pour $n \in \mathds{N}$.
  \end{nota}

  \begin{defn}
    Dans une $\mathcal{L}_0$-structure, on dit qu'un élément est \textit{standard} s'il est l'interprétation d'un terme $\repr{n}$ avec  $n \in \mathds{N}$.
  \end{defn}

  \begin{rmk}
    Dans $\mathds{N}$ (le modèle standard), tout élément est standard.
  \end{rmk}

  \begin{thm}
    Il existe des modèles de $\mathcal{P}$ non isomorphes à $\mathds{N}$.
  \end{thm}
  \begin{prv}
    \begin{enumerate}
      \item Avec le théorème de Löwenheim-Skolem, il existe un modèle de $\mathcal{P}$ de cardinal $\kappa$ pour tout $\kappa \ge \aleph_0$, et $\operatorname{card} \mathds{N} = \aleph_0$.
      \item Autre preuve, on considère un symbole de constante $c$ et on pose $\mathcal{L} := \mathcal{L}_0 \cup \{c\}$.
        On considère la théorie \[
          T := \mathcal{P} \cup \mleft\{\,\lnot(c = \repr{n}) \;\middle|\; n \in \mathds{N}\,\mright\}
        .\]
        Montrons que $T$ a un modèle.
        Par le théorème de compacité de la logique du premier ordre, il suffit de montrer que $T$ est finiment satisfiable.
        Soit $T' \subseteq_\mathrm{fini} T$ : par exemple, 
        \[
          T' \subseteq \mathcal{P} \cup \{\lnot(c = \repr{n_1}), \lnot (c = \repr{n_2}), \ldots, (c = \repr{n_k})\} 
        ,\] 
        et $n_k \ge n_1, \ldots, n_{k-1}$.
        On construit un modèle de $T'$ correspondant à $\mathds{N}$ où $c$ est interprété par $n_k + 1$.
        Ainsi, $T'$ est satisfiable et donc $T$ aussi avec un modèle $\mathcal{M}$.

        Montrons que $\mathds{N}$ et $\mathcal{M}$ ne sont pas isomorphes.
        Par l'absurde, supposons que $\varphi : \mathcal{M} \to \mathds{N}$ soit un isomorphisme.
        Alors $\gamma := \varphi(c_{\mathcal{M}})$ satisfait les mêmes formules que $c_{\mathcal{M}}$, par exemple, pour tout $n \in \mathds{N}$, $\mathcal{M} \models \lnot (c = \repr{n})$.
        Or, on ne peut pas avoir $\mathds{N} \models \lnot (\repr{\gamma} = \repr{n})$ pour tout $n \in \mathds{N}$.
        \textbf{\textit{Absurde.}}
    \end{enumerate}
  \end{prv}

  On a montré que tous les modèles isomorphes à $\mathds{N}$ n'ont que des éléments standards.

  \begin{thm}
    Dans tout modèle $\mathcal{M}$ de $\mathcal{P}$, 
    \begin{enumerate}
      \item l'addition est commutative et associative ;
      \item la multiplication aussi ;
      \item la multiplication est distributive par rapport à l'addition ;
      \item tout élément est \textit{régulier} pour l'addition :
        \[
        \mathcal{M} \models \forall x \: \forall y \: \forall z \: (x \oplus y = x \oplus z \to y = z) \;
        ;\]
      \item tout élément non nul est régulier pour la multiplication : 
        \[
        \mathcal{M} \models \forall x\: \forall y \: \forall z \: ((\lnot (x = \zero) \land x \otimes y = x \otimes z) \to y = z)
        \;;\]
      \item la formule suivante définie un ordre total sur $\mathcal{M}$ compatible avec $+$ et $\times$ :
        \[
        x \le y \text{ ssi } \exists z \: (x \oplus z = y)
        .\]
    \end{enumerate}
  \end{thm}
  \begin{prv}
    On prouve la commutativité de $+$ en trois étapes.
    \begin{enumerate}
      \item On montre  $\mathcal{P} \vdash \forall x \: (\zero \oplus x = x)$.
        On utilise le SI avec la formule $F(x) := (\zero \oplus x = x)$.
        \begin{itemize}
          \item On a $\mathcal{P} \vdash \zero \oplus \zero = \zero$ par A4.
          \item On montre $\mathcal{P} \vdash \forall x \: F(x) \to F(\succ x)$, c'est à dire :
            \[
            \forall x \: \big((\zero \oplus x = x) \to (\zero \oplus (\succ x) = \succ x) \big)
            .\]
            On peut le montrer par A5.
        \end{itemize}

        \textbf{Questions/Remarques :}
        \begin{itemize}
          \item Pourquoi pas une récurrence normale ?
            On n'est pas forcément dans $\mathds{N}$ !
          \item Grâce au théorème de complétude, on peut raisonner sur les modèles, donc en maths naïves.
        \end{itemize}
      \item On montre $\mathcal{P} \vdash \forall x \: \forall y \: \succ(x \oplus y) = (\succ x) \oplus y$.
        On veut utiliser le schéma d'induction avec $F(x, y) := \succ(x \oplus y) = (\succ x) \oplus y$.
        Mais ça ne marche pas\ldots (Pourquoi ?)

        La bonne formule est $F(y, x) := \succ(x \oplus y) = (\succ x) \oplus y$.
        \begin{itemize}
          \item On montre $\mathcal{P} \vdash F(\zero, x)$, c'est à dire 
            \[
            \mathcal{P} \vdash \succ(x \oplus \zero) = (\succ x) \oplus \zero.
            \]
            Ceci est vrai car 
            \[
            \succ(x \oplus \zero) \underset {\text{A4}} = \succ x \underset {\text{A4}} = (\succ x) \oplus \zero
            .\]
          \item On a $\mathcal{P} \vdash F(y, x) \to F(\succ y, x)$ car : si $\succ(x \oplus y) = (\succ x) \oplus y$, alors 
        \end{itemize}
        \[
        \succ(x \oplus (\succ y)) \underset {\text{A5}} = \succ \succ (x \oplus y) \underset{\text{hyp}} = \succ((\succ x) \oplus y) \underset {\text{A5}} (\succ x) = \oplus (\succ y)
        .\]
      \item On utilise le SI avec $F(x, y) := (x \oplus y = y \oplus x)$.
        D'une part, on a $F(\zero, y) = (\zero \oplus y = y \oplus \zero)$ par 1 et A4.
        D'autre part, si l'on a $x \oplus y = y \oplus x$ alors  $(\succ x) \oplus y = y \oplus (\succ x)$ par A5 et 2.
        Par le SI, on conclut.
    \end{enumerate}
  \end{prv}

  \begin{exo}
    Finir la preuve du théorème.
  \end{exo}

  \section{Liens entre $\mathds{N}$ et un modèle $\mathcal{M}$ de $\mathcal{P}$.}

  \begin{defn}
     Si $\mathcal{M} \models \mathcal{P}_0$ et $\mathcal{N} \models \mathcal{P}_0$ et $\mathcal{N}$ une sous-interprétation de $\mathcal{M}$, on dit que $\mathcal{N}$ est un segment initial de $\mathcal{M}$, ou que $\mathcal{M}$ est une extension finale de $\mathcal{N}$, si pour tous $a,b,c \in |\mathcal{M}|$ avec $a \in |\mathcal{N}|$ on a :
    \begin{enumerate}
      \item si $\mathcal{M} \models c \le a$ alors $c \in |\mathcal{N}|$ ;
      \item si $b \not\in |\mathcal{N}|$ alors $\mathcal{M} \models a \le b$.
    \end{enumerate}
  \end{defn}

  \begin{figure}[H]
    \centering
    \begin{tikzpicture}
      \filldraw[draw=deepblue,fill=NavyBlue!5] (0, 0) rectangle (5, 1.5);
      \node[deepblue] at (4.6, 1.2) {$\mathcal{M}$};
      \filldraw[draw=deepgreen,fill=ForestGreen!5] (0.1, 0.1) rectangle (2.5, 1.4);
      \node[deepgreen] at (0.5, 1.10) {$\mathcal{N}$};
      \node[fill,deepblue,circle,inner sep=1pt] (C) at (1, 0.35) {};
      \node[deepblue] at (1.5, 0.35) {$\le$};
      \node[deepblue] at (2.5, 0.35) {$\le$};
      \node[fill,deepblue,circle,inner sep=1pt] (A) at (2, 0.35) {};
      \node[fill,deepblue,circle,inner sep=1pt] (B) at (3, 0.35) {};
      \node[deepblue,above=3pt of A] {$a$};
      \node[deepblue,above=3pt of B] {$b$};
      \node[deepblue,above=3pt of C] {$c$};
    \end{tikzpicture}
  \end{figure}

  \begin{rmk}
    \begin{itemize}
      \item Les points peuvent être incomparables et dans $\mathcal{M}$.
      \item L'ensemble $\mathcal{P}_0$ est très faible, on ne montre même pas que $\oplus$ commute ou que $\le$ est une relation d'ordre (\textit{c.f.} TD).
    \end{itemize}
  \end{rmk}

  \begin{thm}
    Soit $\mathcal{M} \models \mathcal{P}_0$.
    Alors, le sous-ensemble de $\mathcal{M}$ suivant est une sous-interprétation de $\mathcal{M}$ qui est un segment initial et qui est isomorphe à $\mathds{N}$ :
    \[
    \mleft\{\,a \in |\mathcal{M}| \;\middle|\;
    \begin{array}{l}
      \text{il existe $n \in \mathds{N}$ et $a$}\\
      \text{est l'interprétation}\\
      \text{de $\repr{n}$ dans $\mathcal{M}$}
    \end{array}\,\mright\} 
    .\]
  \end{thm}

  \begin{prv}[Idée de la preuve]
    \begin{enumerate}
      \item Pour tout $n \in \mathds{N}$, on a $\mathcal{P}_0 \vdash \repr{n+1} = \succ \repr{n}$.
      \item Pour tout $n, m \in \mathds{N}$, on a $\mathcal{P}_0 \vdash \repr{m} \oplus \repr{n} = \repr{m+n}$.
      \item Pour tout $n, m \in \mathds{N}$, on a $\mathcal{P}_0 \vdash \repr{m} \otimes \repr{n} = \repr{m \times n}$.
      \item Pour tout $n \in \mathds{N}_\star$, on a $\mathcal{P}_0 \vdash \lnot (\repr{n} = \zero)$.
      \item Pour tout $n  \neq m$, on a $\mathcal{P}_0 \vdash \lnot (\repr{m} = \repr{n})$.
      \item Pour tout $n \in \mathds{N}$ (admis), on a 
        \[
          \mathcal{P}_0 \vdash \forall x \: \big(x \le \repr{n} \to (x = \zero \lor x = \repr{1} \lor \cdots \lor x = \repr{n}) \big)
        .\] 
      \item Pour tout $x$, on a $\mathcal{P}_0 \vdash \forall x \: (x \le  \repr{n} \lor \repr{n} \le x)$.
    \end{enumerate}
  \end{prv}

  \section{Les fonctions représentables.}

  Cette section détaille un outil technique pour montrer le théorème d'incomplétude de Gödel vu plus tard.
  On code tout avec des entiers !

  \begin{defn}
    Soit $f : \mathds{N}^p \to \mathds{N}$ une fonction totale et $F(x_0, \ldots, x_p)$ une formule de $\mathcal{L}_0$.
    On dit que $F$ \textit{représente} $f$ si, pour tout $p$-uplet d'entiers $(n_1, \ldots, n_p)$ on a : \[
      \mathcal{P}_0 \vdash \forall y \: \big(F(y, \repr{n_1}, \ldots, \repr{n_p}) \leftrightarrow y = \repr{f(n_1, \ldots, n_p)}\big)
    .\]
    On dit que $f$ est \textit{représentable} s'il existe une formule qui la représente.

    Un ensemble de $p$-uplets $A \subseteq \mathds{N}^p$ est \textit{représenté} par $F(x_1, \ldots, x_p)$ si pour tout $p$-uplet d'entiers $(n_1, \ldots, n_p)$, on a 
    \begin{enumerate}
      \item si $(n_1, \ldots, n_p) \in A$ alors $\mathcal{P}_0 \vdash F(n_1, \ldots, n_p)$ ;
      \item si $(n_1, \ldots, n_p) \not\in A$ alors $\mathcal{P}_0 \vdash \lnot F(n_1, \ldots, n_p)$.
    \end{enumerate}
    On dit que $A$ est \textit{représentable} s'il existe une formule qui le représente.
  \end{defn}

  \begin{exo}
    Montrer qu'un ensemble est représentable ssi sa fonction indicatrice l'est.
  \end{exo}

  \begin{exm}[Les briques de base des fonctions récursives]~\\[-\baselineskip]
    \begin{itemize}
      \item La fonction nulle $f : \mathds{N}\to \mathds{N}, x \mapsto 0$ est représentable par $F(x_0, x_1) := x_0 = \zero$.
      \item Les fonctions constantes $f : \mathds{N}\to \mathds{N}, x \mapsto n$ sont représentables par $F(x_0, x_1) := x_0 = \repr{n}$, où $n \in \mathds{N}$.
      \item Les projections $\pi^i_p : \mathds{N}^p \to \mathds{N}, (x_1, \ldots, x_p) \mapsto x_i$ sont représentables par $F(x_0, x_1, \ldots, x_p) := x_0 = x_i$.
      \item La fonction successeur $f : \mathds{N}\to \mathds{N}, x \mapsto x + 1$ est représentable par $F(x_0, x_1) := x_0 = (\succ x_1)$.
      \item L'addition $f : \mathds{N}^2 \to \mathds{N}, (x,y) \mapsto x+y$ est représentable par $F(x_0, x_1, x_2) := x_0 = x_1 \oplus x_2$.
      \item La multiplication $f : \mathds{N}^2 \to \mathds{N}, (x,y) \mapsto x\times y$ est représentable par $F(x_0, x_1, x_2) := x_0 = x_1 \otimes x_2$.
    \end{itemize}
  \end{exm}

  On introduit trois nouvelles opérations.
  \begin{description}
    \item[Récurrence.]
      Soient $g(x_1, \ldots, x_p)$ et $h(x_1, \ldots, x_{p+2})$ des fonctions partielles.
      On définit la fonction partielle $f$ par :
      \begin{itemize}
        \item $f(0, x_1, \ldots, x_p) := g(x_1, \ldots, x_p)$ ;
        \item $f(x_0 + 1, x_1, \ldots, x_p) := h(x_0, f(x_0, \ldots, x_p), x_1, \ldots, x_p)$.
      \end{itemize}
    \item[Composition.]
      Soient $f_1, \ldots, f_n$ des fonctions partielles de $p$ variables et $g$ une fonction partielle de $n$ variables.
      Alors, la fonction composée $g(f_1, \ldots, f_n)$ est définie en $(x_1, \ldots, x_p)$ ssi les fonctions $f_i$ le sont et $g$ est définie en $\big( f_1(x_1, \ldots, x_p), \ldots, f_n(x_1, \ldots, x_p)\big)$.
    \item[Schéma $\mu$.]
      Soit $f(x_1, \ldots x_{p+1})$ une fonction partielle.
      Soit \[
        g(x_1, \ldots, x_p) := \mu y.\: (f(x_1, \ldots, x_p, y) = 0)
      .\]
      Elle est définie en $(x_1, \ldots, x_p)$ si et seulement s'il existe $y$ tel que $f(x_1, \ldots, x_p, y) = 0$ et tous les $f(x_1, \ldots, x_p, x)$ sont définies pour $x \le y$.
      Dans ce cas, $g(x_1, \ldots, x_p)$ est le plus petit $y$ tel que $f(x_1, \ldots, x_p, y) = 0$.
  \end{description}

  \begin{defn}
    L'ensemble des fonctions récursives primitives (\textit{resp.} récursives) est le plus petit ensemble des fonctions contenant les briques de base et stable par composition  et récurrence (\textit{resp.} par composition, récurrence et schéma $\mu$).
  \end{defn}

  \begin{exm}
    Les fonctions \[
    f(x_1, x_2, y) := y^2 - (x_1+x_2) y + x_1 x_2
    \]et\[
    f(x_1, x_2) := \min(x_1, x_2)
    \] sont récursives primitives.
  \end{exm}

  \begin{defn}
    Une fonction récursive \textit{totale} est une fonction récursive définie partout.
  \end{defn}

  \begin{rmk}
    \begin{itemize}
      \item Une fonction récursive primitive est totale.
      \item Une fonction récursive primitive peut se fabriquer avec un seul schéma $\mu$ à la fin (\textit{c.f.} cours de FDI).
      \item \textit{Rappel}.
        Une fonction $f : \mathds{N}^p \to \mathds{N}$ totale est représentée par la formule $F(x_0, \ldots, x_p)$ de $\mathcal{L}_0$ su pour tout $p$-uplet d'entiers $(n_1, \ldots, n_p)$ on a :
        \[
        \mathcal{P}_0 \vdash \forall y \: \big( F(y, \repr {n_1}, \ldots, \repr{n_p}) \leftrightarrow y = \repr{f(n_1, \ldots, n_p)} \big)
        .\] 
      \item \textit{Rappel}.
        Si $\mathcal{M} \models \mathcal{P}_0$ alors l'ensemble de $|\mathcal{M}|$ constitué de l'interprétation des termes standards est une sous-interprétation de $\mathcal{M}$ qui en est un segment initial et qui est isomorphe à $\mathds{N}$.
      \item \textit{Rappel}.
        Une sous-interprétation $\mathcal{N}$ est un segment initial de~$\mathcal{M}$ si 
        \begin{itemize}
          \item $a \in \mathcal{N}$ et $b \in \mathcal{M} \setminus \mathcal{N}$ alors $b \ge a$ ;
          \item $a \in \mathcal{N}$ et $c \le a$ alors $c \in \mathcal{N}$.
        \end{itemize}
    \end{itemize}
  \end{rmk}

  \begin{figure}[H]
    \centering
    \begin{tikzpicture}
      \filldraw[draw=deepblue,fill=NavyBlue!5] (0, 0) rectangle (5, 1.5);
      \node[deepblue] at (4.6, 1.2) {$\mathcal{M}$};
      \filldraw[draw=deepgreen,fill=ForestGreen!5] (0.1, 0.1) rectangle (2.5, 1.4);
      \node[deepgreen] at (0.5, 1.10) {$\mathcal{N}$};
      \node[fill,deepblue,circle,inner sep=1pt] (C) at (1, 0.35) {};
      \node[deepblue] at (1.5, 0.35) {$\le$};
      \node[deepblue] at (2.5, 0.35) {$\le$};
      \node[fill,deepblue,circle,inner sep=1pt] (A) at (2, 0.35) {};
      \node[fill,deepblue,circle,inner sep=1pt] (B) at (3, 0.35) {};
      \node[deepblue,above=3pt of A] {$a$};
      \node[deepblue,above=3pt of B] {$b$};
      \node[deepblue,above=3pt of C] {$c$};
    \end{tikzpicture}
  \end{figure}

  \begin{thm}
    Toute fonction récursive totale est représentable.
    \label{thm:rec-repr}
  \end{thm}
  On a déjà montré que les briques de base sont représentables.
  On montre trois lemmes qui montreront le théorème ci-dessus.

  \begin{lem}
    L'ensemble des fonctions représentables est clos par composition.
  \end{lem}
  \begin{prv}
    Soient $f_1(x_1, \ldots, x_p), \ldots, f_n(x_1, \ldots, x_p)$ et $g(x_1, \ldots, x_n)$ des fonctions représentées par $F_1(x_0, \ldots, x_p), \ldots, F_n(x_0, \ldots, x_p)$ et $G(x_0, \ldots, G_n)$.
    On va montrer que $h = g(f_1, \ldots, f_n)$ est représentée par 
    \[
    H(x_0, \ldots, x_o) := \exists y_0\: \cdots \: \exists y_n \: \big(
      G(x_0, y_1, \ldots, y_n) \land \bigwedge_{1 \le i \le  n} F_i(y_i, x_1, \ldots, x_p)
    \big)
    .\]
    En effet, pour tous entiers $n_1, \ldots, n_{\max(p, n)}$ :
    \begin{itemize}
      \item $\mathcal{P}_0 \vdash \forall y \: F_i(y_1, \repr{n_1}, \ldots, \repr{n_p}) \leftrightarrow y = \repr{f_i(n_1, \ldots, n_p)}$ ;
      \item $\mathcal{P}_0 \vdash \forall y \: G(y_1, \repr{n_1}, \ldots, \repr{n_n}) \leftrightarrow y = \repr{g(n_1, \ldots, n_n)}$.
    \end{itemize}
    Dans tout modèle $\mathcal{M}$ de $\mathcal{P}_0$, pour tout $y \in |\mathcal{M}|$, et tous $n_1, \ldots, n_p \in \mathds{N}$ on a $H(y, n_1, \ldots, n_p)$ est vraie ssi il existe $y_1, \ldots, y_n$ dans $|\mathcal{M}|$ et pour tout $i$, $F_i(y_i, x_1, \ldots, x_p)$ est vrai et $G(y, y_1, \ldots, y_n)$.
    Donc, par les hypothèses précédents, on a $H(y, n_1, \ldots, n_p)$ ssi il existe $y_1, \ldots, y_n$ dans $|\mathcal{M}|$ et pour tout $i$, $y_i = f_i(n_1, \ldots, n_p)$ et $y = g(y_1, \ldots, y_p)$, ssi \[
    y = g(f_1(n_1, \ldots, n_p), \ldots, f_n(n_1, \ldots, n_p))
    \]  ssi $y = h(n_1,\ldots,n_p)$.
    On conclut \[
      \mathcal{P}_0 \vdash \forall y\: \big(H(y, \repr{n_1}, \ldots, \repr{n_p}) \leftrightarrow y = \repr{h(n_1, \ldots, n_p)}\big)
    .\]
  \end{prv}

  \begin{lem}
    Si, à partir d'une fonction représentable totale, on obtient par schéma $\mu$ une fonction totale, alors cette fonction est représentable.
  \end{lem}
  \begin{prv}
    Soit $g : \mathds{N}^{p+1} \to \mathds{N}$ une fonction représentable totale, et soit $f : \mathds{N}^p \to \mathds{N}$ définie par \[
      f(x_1, \ldots, x_p) := \mu x_0.\:\big( g(x_0, \ldots, x_p) = 0 \big)
    .\]
    Montrons que si $f$ est totale alors elle est représentable.
    Soit $G (y, x_0, \ldots, x_p)$ qui représente $g$.
    Alors, pour tous $n_1, \ldots, n_p$ on a 
    \[
      \mathcal{P}_0 \vdash \forall y \: G(y, \repr{n_1}, \ldots, \repr{n_p}) \leftrightarrow y = \repr{g(n_1, \ldots, n_p)}
    .\]
    Considérons la formule \[
    F(y, n_1, \ldots, n_p) := G(0, y, x_1, \ldots, x_p)  \land \forall z < y, \lnot G(0, z, x_1, \ldots, x_p)
    ,\]
    où l'on note $\forall z < y \: H$ pour  $\forall z \: (\exists u \: \lnot (h = \repr{0}) \land z \oplus h = y) \to H$.
    Montrons que $F$ représente $f$.
    Soit $\mathcal{M}$ un modèle de $\mathcal{P}_0$.
    Soient $ n_1, \ldots, n_p$ des entiers et $y \in |\mathcal{M}|$.
    On a $F(y, n_1, \ldots, n_p)$ vrai ssi $G(0, y, n_1, \ldots, n_p)$ vrai et, pour tout $z < y$, $\lnot G(0, z, n_1, \ldots, n_p)$ est vrai.
    Montrons que $b := f(n_1, \ldots, n_p)$ est le seul élément à satisfaire $F(y, n_1, \ldots, n_p)$.
    On a bien $G(0, b, n_1, \ldots, n_p)$ par définition de $f$ et pour tout entier $z < b$, on a  $\lnot G(0, z, n_1, \ldots, n_p)$.
    Mais, si on a $z < b$ et  $z$ n'est pas un entier ?
    Ce cas n'existe pas car la sous-représentation isomorphe à $\mathds{N}$ est un segment initial, il n'y a donc que des entiers qui sont inférieurs à $b$ dans $|\mathcal{M}|$.
    Ainsi, $F(b, n_1, \ldots, n_p)$.
    Montrons que $b$ est le seul.
    Soit $y$ tel que $F(y, n_1, \ldots, n_p)$. Montrons que $y = b$.
    \begin{itemize}
      \item Si $y$ est un entier, c'est vrai par définition de $b$.
      \item Si $y$ n'est pas un entier, alors $y > b$.
        Donc, $g(y, x_1, \ldots, x_p) = 0$ et  $b < y$ avec  $g(b, x_1, \ldots, x_p) = 0$.
        Ainsi, $\forall  z < y \: \lnot G(0, z, x_1, \ldots, x_p)$ est fausse, et donc $F(y, n_1, \ldots, n_p)$ est fausse.
    \end{itemize}
 \end{prv}

 \begin{lem}
   L'ensemble des fonctions totales est stable par définition par récurrence.
   \label{lem:repr-rec}
 \end{lem}
 \begin{prv}
   Soient $f, g, h$ telles que 
    \begin{itemize}
     \item $f(0, x_1, \ldots, x_p) = g(x_1, \ldots, x_p)$
     \item $f(x_0 + 1, x_1, \ldots, x_p) = h(x_0, f(x_0, \ldots, x_p), x_1, \ldots, x_p)$
   \end{itemize}
   Soient $G, H$ représentant $g$ et $h$.
   On a dans $\mathds{N}$ : $y = f(x_0, \ldots, x_p)$ ssi il existe $z_0, \ldots, z_{x_0}$ tel que 
   \begin{itemize}
     \item $z_0 = g(x_1, \ldots, x_p)$
     \item $z_1 = h(0, z_0, x_1, \ldots, x_p)$
     \item $z_2 = h(1, z_1, x_1, \ldots, x_p)$
     \item $\vdots$ 
     \item $z_{x_0} = h(x_0-1, z_{x_0 - 1}, x_1, \ldots, x_p)$
     \item $y = z_{x_0}$
   \end{itemize}
   Zut ! On ne peut pas écrire $\exists z_0 \: \cdots \: \exists z_{x_0}$ !
   On va utiliser une fonction qui permet de coder une suite d'entiers dans un couple d'entier $(a,b)$.
   Interruption de la preuve. 
 \end{prv}

 \begin{lem}[Fonction $\beta$ de Gödel]
   Il existe une fonction $\beta$ à trois variables, récursive primitive et représentable, tel que pour tout  $p \in \mathds{N}$ et toute suite $(n_0, \ldots, n_p) \in \mathds{N}^{p+1}$, il existe des entiers $a$ et $b$ tels que pour tout $0 \le i \le p$, on ait $\beta(i, a, b) = n_i$.
 \end{lem}
 \begin{prv}
   Soient $(a_0, \ldots, a_p)$ une suite d'entiers deux à deux premiers, et $(n_0, \ldots, n_p)$ une suite d'entiers.
   Alors il existe $b \in \mathds{N}$ tel que, pour tout $0 \le i \le p$, $b \equiv n_i \pmod{a_i}$ (par le théorème Chinois).

   Choisissons $a$ et les $a_i$ (qui induisent $b$) ?
   On pose $a = m!$.
   Alors, on pose $a_i := a (i+1) + 1$ pour tout $0 \le i \le p$.
   Les $a_i$ sont bien deux à deux premiers.
   En effet, pour $j > i$, si  $c  \mid a_i$ et $c  \mid a_j$ avec $c$ premier, alors $c  \mid (a_i - a_j)$ donc $c  \mid a (j-i)$ et donc $c \le m$, donc $c  \mid m$.
   Ainsi, il existe bien $b$ tel que $b \equiv n_i \pmod {a_i}$.
   On définit ainsi  $\beta(i, a, b)$ comme le reste de la division de  $b$ par $a(i+1) + 1$.
   La fonction $\beta$ est représentée par  \[
   B(x_0, i, a, b) := \exists x_4 \: b = x_4 \otimes \succ (a \otimes (\succ i)) \land x_4 < \succ(x \otimes \succ i)
   .\]
   On considère $B'(x_0, x_1, x_2, x_3) := B(x_0, x_1, x_2, x_3) \land \forall x_4 < x_0 \:\lnot B(x_4, x_1, x_2, x_3)$.
   Cette dernière formule représente aussi $\beta$ mais aussi que $x_0$ sera un entier standard.
 \end{prv}


 \vspace{1cm}
 \begin{prv}[On reprend la preuve du lemme~\ref{lem:repr-rec}]
   Soient $f, g, h$ telles que 
    \begin{itemize}
     \item $f(0, x_1, \ldots, x_p) = g(x_1, \ldots, x_p)$
     \item $f(x_0 + 1, x_1, \ldots, x_p) = h(x_0, f(x_0, \ldots, x_p), x_1, \ldots, x_p)$
   \end{itemize}
   Soient $G, H$ représentant $g$ et $h$.
   On a dans $\mathds{N}$ : $y = f(x_0, \ldots, x_p)$ ssi il existe $z_0, \ldots, z_{x_0}$ tel que 
   \begin{itemize}
     \item $z_0 = g(x_1, \ldots, x_p)$
     \item $z_1 = h(0, z_0, x_1, \ldots, x_p)$
     \item $z_2 = h(1, z_1, x_1, \ldots, x_p)$
     \item $\vdots$ 
     \item $z_{x_0} = h(x_0-1, z_{x_0 - 1}, x_1, \ldots, x_p)$
     \item $y = z_{x_0}$
   \end{itemize}
   ssi
   \begin{align*}
     \exists a \: \exists b \: \big[ &\\
    & (\exists z_0 \: B'(z_0, \repr{0}, a, b) \land G(z_0, x_1, \ldots, x_p)) \\
       {}\land & \forall i < x_0 \: \exists z \: \exists z' \: \begin{pmatrix}
      &B'(z, i, a, b)\\
      \land &B'(z', \succ i, a, b)\\
      \land &H(z', i, z, x_1, \ldots, x_p)
    \end{pmatrix} \\
      {}\land & B'(y, x_0, a, b)\\
              & \hspace{-1cm}\big]
   \end{align*}
   est vraie.
   Montrons que $F$ représente $f$.

   Soit $\mathcal{M} \models \mathcal{P}_0$, et $n_0, \ldots, n_p$ des entiers et $c \in |\mathcal{M}|$.
   \begin{itemize}
     \item Si $c$ interprète $\repr{f(n_0, \ldots, n_p)}$ alors en choisissant $a$ et $b$ avec le lemme précédent sur la fonction $\beta$, on a bien  $F(c, n_0, \ldots, n_p)$.
     \item Réciproquement, si $\mathcal{M} \models F(d, \repr{n_0}, \ldots, \repr{n_p})$ alors il existe $a, b, z_0$ tels que  $B'(z_0, \repr{0}, a, b)$ et  $G(z_0, n_1, \ldots, n_p)$, et donc $z_0 = g(n_1, \ldots, n_p)$.
       Et, pour tout $i \le  n_0$, il existe $r_i$ et  $s_i$ tels que  \[
       B'(r_i, i, a, b) \land B'(s_i, i + 1, a , b) \land H(s_i, i, r_i, n_1, \ldots, n_p)
       \] 
       donc $r_i = f(i, n_1, \ldots, n_p)$ grâce aux propriétés de $B'$ et car $r_i$ est un entier naturel, et donc par récurrence $d = f(n_0, \ldots, n_p)$.
   \end{itemize}
 \end{prv}

 Ceci conclut la preuve du théorème~\ref{thm:rec-repr}.
\end{document}
