\documentclass{../../notes}

\title{DM n°3 -- FDI}
\author{Hugo \scshape Salou}

\usepackage{multicol}

\def\algorithmicrequire{\textbf{Entrée}}
\def\algorithmicensure{\textbf{Sortie}}
\def\algorithmicend{\textbf{fin}}
\def\algorithmicif{\textbf{si}}
\def\algorithmicthen{\textbf{alors}}
\def\algorithmicelse{\textbf{sinon}}
\def\algorithmicelsif{\algorithmicelse\ \algorithmicif}
\def\algorithmicendif{\algorithmicend\ \algorithmicif}
\def\algorithmicfor{\textbf{pour}}
\def\algorithmicprocedure{\textbf{Procédure}}
\def\algorithmicfunction{\textbf{Fonction}}
\def\algorithmicforall{\textbf{pour tout}}
\def\algorithmicdo{\textbf{faire}}
\def\algorithmicendfor{\algorithmicend\ \algorithmicfor}
\def\algorithmicwhile{\textbf{tant que}}
\def\algorithmicendwhile{\algorithmicend\ \algorithmicwhile}
\def\algorithmicloop{\textbf{répéter indéfiniment}}
\def\algorithmicendloop{\algorithmicend\ \algorithmicloop}
\def\algorithmicrepeat{\textbf{répéter}}
\def\algorithmicuntil{\textbf{jusqu'à que}}
\def\algorithmicprint{\textbf{afficher}}
\def\algorithmicreturn{\textbf{retourner}}
\def\algorithmictrue{\textsc{vrai}}
\def\algorithmicfalse{\textsc{faux}}

\let\Sortie\Ensure
\let\Entree\Require
\floatname{algorithm}{\color{deepblue}Algorithme}

\begin{document}
  \maketitle

  On nomme $L$ le langage \[
    L := \{\langle M \rangle \in \Sigma^\star  \mid M \text{ s'arrête sur l'entrée } \langle M \rangle \} 
  .\]

  Nous allons démontrer que ce langage est
  \begin{itemize}
    \item indécidable (par réduction au langage de l'arrêt $A_\mathsf{TM}$ pour les machines de Turing) ;
    \item récursivement énumérable ;
    \item mais le complémentaire n'est pas récursivement énumérable.
  \end{itemize}

  On rappelle que le langage $A_\mathsf{TM} := \{\langle M, w \rangle  \mid M \text{ accepte } w\}$ est indécidable.

  \begin{multicols}{2}
    On pose la fonction
    \begin{align*}
      f: \quad\quad \Sigma^\star &\longrightarrow \Sigma^\star \\
      \langle N, w \rangle &\longmapsto \langle M_{N,w} \rangle
    \end{align*}
    qui est calculable.
    \columnbreak
    \begin{mdframed}[linecolor=deepblue,innertopmargin=1em,frametitle={\color{white}Machine $M_{N,w}$},frametitlerule=true,frametitlealignment=\centering,frametitlebackgroundcolor=deepblue]
      \begin{algorithmic}
        \Entree Un mot $u \in \Sigma^\star$
        \State On simule $N$ sur l'entrée $w$.
        \If{$N$ accepte $w$}
        \State On accepte.
        \Else
        \State On boucle à l'infini.
        \EndIf
      \end{algorithmic}
    \end{mdframed}
  \end{multicols}

  L'"implémentation haut niveau" de la machine $M_{N,w}$, donnée ci-dessus à droite, permet de construire une telle machine.
  En effet, l'étape "on simule $N$ sur l'entrée $w$" se réalise à l'aide de la machine universelle, et le reste se fait simplement.

  On remarque que $M_{N,w}$ s'arrête sur une entrée quelconque si, et seulement si, $N$ accepte $w$.

  Et, on a l'équivalence 
  \begin{align*}
    \langle M_{N,w} \rangle \in L &\iff M_{N,w} \text{ s'arrête sur } \langle M_{N,w} \rangle\\
    &\iff N \text{ accepte } w\\
    &\iff \langle N, w \rangle \in A_\mathsf{TM}.
  .\end{align*}

  Ainsi, par réduction, on en déduit que $L$ n'est pas décidable.

  Pour démontrer que $L$ est récursivement énumérable, on montre (de manière équivalente) qu'il est Turing-reconnaissable.
  Considérons la machine de Turing $R$ dont l'implémentation est donnée ci-dessous.

  \begin{multicols}{2}
    Sur toutes les entrées $\langle M\rangle \in L$, la machine $R$ accepte cette entrée en un temps fini.
    On en déduit que $R$ reconnait bien $L$, et donc $L$ est Turing-reconnaissable donc récursivement énumérable.

    \begin{mdframed}[linecolor=deepblue,innertopmargin=1em,frametitle={\color{white}Machine $R$},frametitlerule=true,frametitlealignment=\centering,frametitlebackgroundcolor=deepblue]
      \begin{algorithmic}
        \Entree $\langle M \rangle$
        \State Simuler $M$ sur $\langle M \rangle$.
        \If{$M$ accepte $\langle M \rangle$}
        \State On accepte.
        \EndIf
      \end{algorithmic}
    \end{mdframed}
  \end{multicols}
  
  On peut démontrer que $L$ n'est pas de complémentaire récursivement énumérable.
  En effet, par l'absurde, si $L$ l'était, alors $\bar{L}$ serait Turing-reconnaissable.
  Mais, ceci impliquerai (comme $L$ et $\bar{L}$ seraient Turing-reconnaissable) que $L$ serait décidable (il suffit de lancer les deux machines reconnaissant $L$ et $\bar{L}$ en parallèle pour décider $L$).
  Ce qui est \textit{\textbf{absurde}} car on a montré l'indécidabilité de $L$.

  Des résultats précédents, on en conclut que 
  \begin{itemize}
    \item $L$ est indécidable ;
    \item $L$ est récursivement énumérable ;
    \item $L$ n'est pas à complémentaire récursivement énumérable.
  \end{itemize}
\end{document}

