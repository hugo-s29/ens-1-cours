\documentclass{../../notes}

\title{Algèbre 1}
\author{Hugo SALOU}

\DeclareMathOperator{\im}{im}

\begin{document}
  \maketitle

  \begin{prv}
    \begin{itemize}
      \item L'unicité à unique isomorphisme près est formelle (\textit{c.f.} preuve des sommes/produits directs).
      \item On va démontrer l'existence, pour $n = 2$, et on procède par récurrence immédiate pour montrer pour tout $n$.
        On donne deux méthodes.

        \begin{center}
          \textbf{\textit{Méthode rapide}}
        \end{center}

        Soit $(e_1, \ldots, e_n)$ une base de $E_1$.
        Soit $(f_1, \ldots, f_m)$ une base de $E_2$.
        On considère $E_1 \otimes E_2$ le $\mathds{k}$-espace vectoriel de base indexée par $\{1,\ldots,n\} \times \{1,\ldots,m\}$ (c'est à dire $\mathds{k}^n \times \mathds{k}^m$).
        On notera $e_i \otimes f_j$ l'élément de la base correspondante à l'indice $(i,j)$.

        Remarquons que, plus abstraitement, on a considéré \[
          \bigoplus_{(i,j) \in \llbracket 1,n\rrbracket \times \llbracket 1,m\rrbracket   } \mathds{k} (e_i \otimes e_j)
        .\]

        On définit alors l'application \begin{align*}
          \pi: E_1 \times E_2 &\longrightarrow E_1 \otimes E_2 \\
          (e_i, f_j) &\longmapsto e_i \otimes f_j
        ,\end{align*}
        étant étendue par linéarité.

        Justifions que la propriété universelle voulue est satisfaite.

        Soit $\phi : E_1 \times E_2 \to F$.
        On définit l'application linéaire 
        \begin{align*}
          \bar{\phi}: E_1 \otimes E_2 &\longrightarrow F \\
          e_i \otimes e_j &\longmapsto \bar{\phi}(e_i, e_j)
        ,\end{align*}
        étendue par linéarité.

        \begin{obs}
          $\phi  = \bar{\phi} \circ \pi$.
        \end{obs}
        \begin{expl}
          Clair.
        \end{expl}
        L'unicité de $ \bar{\phi}$ est claire, car elle doit prendre les valeurs $\big(\phi(e_i, f_j)\big)_{i,j}$ sur $(e_i \otimes e_j)_{i,j}$, qui est une base de $E_1 \otimes E_2$.


        \begin{center}
          \textbf{\textit{Méthode "M1 Maths"}}
        \end{center}

        On définit le produit tensoriel par un quotient de relation.
        C'est une construction qui sera utilisée notamment en M1 avec les modules.

        On définit\footnote{Cette définition n'a qu'un intérêt théorique : on pose l'espace comme ceci mais, pour pouvoir le calculer, cette définition n'est pas très utile.} : {\footnotesize\[
            \hspace{-4em}
        E_1 \otimes E_2 := \frac{\bigoplus_{e \in E_1, f \in E_2} \mathds{k} (e \otimes f)}{\langle (\lambda e + \lambda' e') \otimes (\mu f + \mu' f') = \lambda \mu e \otimes f + \lambda \mu' e \otimes f' + \lambda' \mu e' \otimes f + \lambda' \mu' e' \otimes f' \rangle}
      .\]}

        On définit alors
        \begin{align*}
          \pi: E_1 \times E_2 &\longrightarrow E_1 \otimes E_2 \\
          (e, f) &\longmapsto \overline{e \otimes f}
        ,\end{align*}
        qui est étendue par bilinéarité puis construction.

        (Exercice) Le couple $(E_1 \otimes E_2, \pi)$ satisfait la propriété universelle recherchée.
    \end{itemize}
  \end{prv}


  \begin{defn}
    Soient $E_1, \ldots, E_n$ des $\mathds{k}$-espaces vectoriels.
    Pour $(e_1, \ldots, e_n) \in E_1 \times \cdots \times E_n$, on note \[
      e_1 \otimes \cdots \otimes e_n := \pi(e_1, \ldots, e_n)
    .\]
  \end{defn}

  \textbf{Terminologie}.
  Les éléments de cette forme sont appelés \textit{tenseurs simples} (ou \textit{pures}).
  Un élément de $E_1 \otimes \cdots \otimes E_n$ est appelé \textit{tenseur}.

  \begin{obs}
    Les tenseurs simples engendrent $E_1 \otimes \cdots \otimes E_n$.
  \end{obs}
  \begin{expl}
    Clair avec les deux constructions données.
  \end{expl}

  \textbf{\textit{Attention !}}
  Un tenseur quelconque n'est en général pas simple.

  Pour bien insister sur le corollaire de la construction :
  \begin{crlr}
    Soient $E_1, \ldots, E_n$ des $\mathds{k}$-espaces vectoriels.
    Soit également $(e_{i_k})_{i_k \in I_k}$ une base de $E_k$ pour $k \in \llbracket 1,n\rrbracket$.

    Alors, $(e_{i_1} \otimes \cdots \otimes e_{i_n})_{(i_1, \ldots, i_n) \in I_1 \times \cdots \times I_n}$ 
    est une base de $E_1 \otimes \cdots \otimes E_n$.

    En particulier, \[
      \mathrm{dim}(E_1 \otimes E_n) = \prod_{i = 1}^n \mathrm{dim}(E_i)
    .\]
  \end{crlr}
  \begin{prv}
    Cela par la construction donnée.
  \end{prv}

  On a les règles de calculs suivantes :
  \begin{enumerate}
    \item {\small $\lambda \cdot (e_1 \otimes \cdots \otimes e_n) = (\lambda \cdot e_1) \otimes e_2 \otimes \cdots \otimes e_n = \cdots = e_1 \otimes \cdots \otimes e_{n-1}  \otimes (\lambda \cdot e_n)$.}
    \item $(e_1 + e_1') \otimes e_2 \otimes \cdots \otimes e_n = e_1 \otimes \cdots \otimes e_n + e_1' \otimes \cdots \otimes e_n$, \[
    \vdots
    \]
      $e_1 \otimes \cdots \otimes e_{n-1} \otimes (e_n + e_n') = e_1 \otimes \cdots \otimes e_n + e_1 \otimes \cdots \otimes e_n'$.
  \end{enumerate}

  Ceci est vrai par bilinéarité de $\pi$.

  \section{Morphismes et produits tensoriels.}

  Dans cette section, on traitera le cas $n = 2$.

  On a le théorème suivant.
  \begin{thm}
    Soient $E, E', F, F'$ quatre  $\mathds{k}$-espaces vectoriels de dimension finie.
    Il y a un isomorphisme canonique
    \begin{align*}
      \mathrm{Hom}(E, E') \otimes \mathrm{Hom}(F, F') &\overset \sim \longrightarrow  \mathrm{Hom}(E \otimes F, E' \otimes F')\\
      u \otimes v &\longmapsto 
      \begin{array}{|rcl}
        E \otimes F &\to& E' \otimes F'\\
        x \otimes y &\mapsto & u(x) \otimes u(y)
      \end{array}
    .\end{align*}

    \textbf{Convention.}
    Lorsqu'on écrit "$u \otimes v$", on l'interprète comme étant un élément de $\mathrm{Hom}(E \otimes F, E' \otimes F')$.
  \end{thm}
  \begin{prv}
    On considère le morphisme $\Phi$ défini par :
    \begin{align*}
      \Phi : \mathrm{Hom}(E, E') \times \mathrm{Hom}(F, F') &\overset \sim \longrightarrow  \mathrm{Hom}(E \otimes F, E' \otimes F')\\
      (u,v) &\longmapsto 
      \begin{array}{|rcl}
        E \otimes F &\to& E' \otimes F'\\
        x \otimes y &\mapsto & u(x) \otimes u(y)
      \end{array}
    .\end{align*}
    \begin{obs}
      $\Phi$ est bien définie.
    \end{obs}
    \begin{expl}
      Ceci découle du fait que \begin{align*}
        E \times F &\longrightarrow E' \otimes F' \\
        (x,y) &\longmapsto E \times u(x)\otimes v(y)
      \end{align*} est bilinéaire, donc unicité de l'application linéaire $E \otimes F \to E' \otimes F'$ (propriété universelle).
    \end{expl}

    \begin{obs}
      L'application $\Phi$ est bilinéaire.
    \end{obs}
    \begin{expl}
      Clair.
    \end{expl}

    Ainsi $\Phi$ induit, par propriété universelle, l'application linéaire  \[
    \bar{\Phi} : \mathrm{Hom}(E, E') \otimes \mathrm{Hom}(F, F') \to \mathrm{Hom}(E \otimes F, E' \otimes F')
    .\]

    \begin{obs}
      $\bar{\Phi}$ est surjective
    \end{obs}
    \begin{expl}
      Il suffit de montrer que l'on peut atteindre les morphismes de la forme
      \begin{align*}
        e_i \otimes f_j &\longmapsto e_k' \otimes f_\ell'\\
        e_r \otimes f_s &\longmapsto 0 \text{ si } (r,s) \neq (i,j)
      \end{align*}où $(e_i)_i$ est une base de  $E$, $(f_j)_j$ est une base de  $F$, $(e'_k)_k$ est une base de  $E'$ et $(f'_\ell)_\ell$ est une base de $F'$.

      Et, pour ce faire, on définit \begin{align*}
        u: e_i &\longmapsto e_k' \\
        e_r &\longmapsto 0 \text{ si } r \neq i
      \end{align*}
      puis \begin{align*}
       v :f_j  &\longmapsto f_j'  \\
        f_s &\longmapsto 0 \text{ si } s \neq j
      .\end{align*}
    \end{expl}

    On conclut puisque \[
    \mathrm{dim}(\mathrm{Hom}(E, E') \otimes \mathrm{Hom}(F, F')) = \mathrm{dim}(\mathrm{Hom}(E \otimes F, E' \otimes F'))
    .\]
  \end{prv}

  On a alors la règle de calculs étendue.
  \begin{prop}
    Si $u : E \to E'$, $u' : E' \to E''$, $v : F \to F'$ et $v' : F' \to F''$ quatre applications linéaires, alors \[
      (u' \otimes v') \circ (u \otimes v) = (u' \circ u) \otimes (v' \circ v)
    .\]
  \end{prop}
  \begin{prv}
    Il suffit d'observer que ces deux applications prennent même valeurs sur les tenseurs simples.
    On conclut car ceux-ci engendrent $E \otimes F$.
  \end{prv}

  \textbf{Interprétation matricielle} (du théorème précédent)

  Soit $\mathcal{B}_E = (e_i)$ une base de $E$, $\mathcal{B}_F = (f_j)$ une base de $F$, $\mathcal{B}_{E'} = (e'_k)$ une base de $E'$ et $\mathcal{B}_{F'} = (f'_\ell)$ une base de $F'$.
  Notons $A := \mathrm{Mat}_{\mathcal{B}_E, \mathcal{B}_F}(u)$ et $B := \mathrm{Mat}_{\mathcal{B}_{E'}, \mathcal{B}_{F'}}(v)$.
  Notons aussi $\mathcal{B}_{E \otimes F} = (e_i \otimes f_j)$ base de $E \otimes F$ et  $\mathcal{B}_{E' \otimes F'} = (e'_k \otimes f_\ell')$ base de $E' \otimes F'$.
  Alors, \[
    \mathrm{Mat}_{\mathcal{B}_{E \otimes F}, \mathcal{B}_{E' \otimes F'}} = \begin{pmatrix} a_{11} B& \ldots& a_{1,m} B \\ 
      \vdots &\ddots &\vdots\\
      a_{n,1} B & \ldots & a_{n,m} B
  \end{pmatrix} \quad\quad (*)
  .\]

  On peut en déduire que (exercice) le corolaire suivant.
  \begin{crlr}
    Si $u \in \mathcal{L}(E)$ et $v \in \mathcal{L}(F)$ alors 
    \begin{enumerate}
      \item $\mathrm{Tr}(u\otimes v) = \mathrm{Tr}(u) \cdot \mathrm{Tr}(v)$ ;
      \item $\det (u \otimes v) = (\det u)^{\operatorname{dim} F} \cdot (\det v)^{\operatorname{dim} E}$ ;
    \end{enumerate}
    et si $u \in \mathcal{L}(E, E')$ et $v \in \mathcal{L}(F, F')$ alors
    \begin{enumerate}[resume*]
      \item $\mathrm{rg}(u\otimes v) = (\operatorname{rg} u) \cdot (\operatorname{rg} v)$.
    \end{enumerate}
  \end{crlr}
  \begin{prv}[Idée]
    On applique l'interprétation matricielle, et on applique les résultats usuels sur les matrices.
  \end{prv}

  \begin{rmk}
    La formule $(*)$ permet de définir le produit tensoriel de matrices.
    On montrerait alors
    \begin{enumerate}
      \item $(A + A') \otimes B = A \otimes B + A' \otimes B$
      \item  $(A A') \otimes (B B') = (A \otimes B)(A' \otimes B')$ (par les règles de calculs)
      \item  $(A \otimes B) \otimes C = A \otimes (B \otimes C)$ (par l'interprétation terme à terme des morphismes).
    \end{enumerate}
  \end{rmk}

  \section{Quelques isomorphismes canoniques et règles de calculs}

  \subsection{Isomorphisme canonique.}

  \begin{prop}
    Soient $E$ et $F$ deux $\mathds{k}$-espaces vectoriels de dimension finie.
    Alors, \textit{canoniquement} on a $\mathrm{Hom}(E, F) \cong E^* \otimes F$.
  \end{prop}
  \begin{prv}
    On considère 
    \begin{align*}
      \phi: E^* \times F &\longrightarrow \mathrm{Hom}(E, F) \\
      (\ell,y) &\longmapsto (x \mapsto \ell(x)\:y)
    .\end{align*}
    \begin{obs}
      $\phi$ est bilinéaire.
    \end{obs}
    Elle induit donc $\bar{\phi} : E^* \otimes F \to \mathrm{Hom}(E, F)$.

    La surjectivité se montre comme à la sections 3, et on conclut par égalité des dimensions.
  \end{prv}

  \subsection{Isomorphismes canoniques et règles de calculs.}
  \begin{prop}
    Si $E,F$ sont deux $\mathds{k}$-espaces vectoriels, alors 
    \begin{align*}
      E \otimes F &\longrightarrow F \otimes E \\
      e \otimes f &\longmapsto f\otimes e
    .\end{align*}
  \end{prop}
  \begin{prv}
    On pose 
    \begin{align*}
      u : E \times F &\longrightarrow F \otimes E \\
      (e,f) &\longmapsto f\otimes e
    ,\end{align*}
    bilinéaire et il induit donc un morphisme \[
    \bar{u} : E \otimes F \to F \otimes E
    .\]
    On construit de même l'inverse.
  \end{prv}

  \begin{prop}
    Si $E_1, E_2, F$ sont trois $\mathds{k}$-espaces vectoriels, alors il y a un isomorphisme canonique
    \begin{align*}
      u: (E_1 \oplus E_2) \otimes F &\longrightarrow (E_1 \otimes F) \oplus (E_2 \otimes F) \\
      (x \oplus y) \otimes z &\longmapsto (x \otimes z) \oplus (y \otimes z)
    .\end{align*}
  \end{prop}
  \begin{prv}
    On considère, pour $i \in \{1,2\}$, l'application $p_i : E_1 \otimes E_2 \to E_i$ la projection canonique.

    \begin{obs}
      $u = (p_1 \otimes \mathrm{id}_F, p_2 \otimes \mathrm{id}_F)$
    \end{obs}
    \begin{expl}
      Clair
    \end{expl}

    On peut construire l'inverse. Soit, pour $i \in \{1,2\}$, l'injection canonique \[
    j_i : E_i \hookrightarrow E_1 \oplus E_2
    .\]
    On considère alors, pour $i \in \{1,2\}$, \[
    j_i \otimes \mathrm{id}_F : E_i \otimes F \to (E_1 \oplus E_2) \otimes F
    .\]
    Ainsi, $v = j_1 \oplus j_2$ est un inverse pour $u$.
  \end{prv}

  \begin{prop}
    Si $E$ et $F$ sont deux $\mathds{k}$-espaces vectoriels de dimension fini, alors il existe un isomorphisme canonique
    \begin{align*}
      \bar{\phi}: E^* \otimes F^* &\longrightarrow (E\otimes F)^* \\
    \mu \otimes \nu &\longmapsto (x \otimes y \mapsto \mu(x) \cdot \nu(y))
    .\end{align*}
  \end{prop}
  \begin{prv}
    \begin{enumerate}
      \item L'application $\bar{\phi}$ est bien définie.
      \item C'est bien un isomorphisme (on montre que c'est surjectif et égalité des dimensions).
    \end{enumerate}
  \end{prv}

  \begin{prop}
    Si $E$, $F$ et $G$ sont trois $\mathds{k}$-espaces vectoriels alors il y a un isomorphisme canonique
    \begin{align*}
      (E \otimes F) \otimes G  &\longrightarrow E \otimes (F \otimes G) \\
      (x \otimes y) \otimes z &\longmapsto x \otimes (y \otimes z)
    .\end{align*}
  \end{prop}
  \begin{prv}
    Exercice.
  \end{prv}

  \chapter{Représentation linéaires des groupes (finis, complexes).}

  \section{Notions de base.}
  Dans cette section, on considère un corps $\mathds{k}$ est un groupe $G$.

  \subsection{Définitions.}

  \begin{defn}
    Une \textit{représentation linéaire de $G$ sur $\mathds{k}$} est la donnée de
    \begin{itemize}
      \item $V$ un $\mathds{k}$-espace vectoriel de dimension finie ;
      \item $\rho : G \to \mathrm{GL}(V)$ morphisme de groupes.
    \end{itemize}
    Autrement dit, c'est une action de $G$ sur $V$ par \textit{automorphismes linéaires}.
  \end{defn}

  \begin{nota}
    On note $(\rho, V)$, souvent abrégé en simplement $V$ ($\rho$ étant sous-entendu).
  \end{nota}

  \textbf{Terminologie.}
  \begin{itemize}
    \item Le \textit{degré} de $(\rho, V)$ est la dimension de $V$.
    \item La représentation est \textit{fidèle} si $\rho$ est injectif.
  \end{itemize}

  \subsection{Sous-représentation (et irréductibilité)}
  \begin{defn}
    Si $(\rho, V)$ une représentation linéaire de $G$ alors un sous-espace vecotirel $W \subseteq V$ est une sous-représentation linéaire de $V$ si $W$ est stable par $G$, \textit{i.e.} \[
    \forall g \in G, g \cdot W \subseteq W
    .\]
  \end{defn}

  \begin{obs}
    La donnée de \[
      \left(
        W,
        \begin{array}{rcl}
          G & \to& \mathrm{GL}(W)\\
          g &\mapsto & \rho(g)_{|W}
        \end{array}
      \right)
    \] est alors évidemment une représentation linéaire de $G$.
  \end{obs}

  \begin{exm}
    \[
    V^G := \{v \in V  \mid \forall g \in G, g \cdot v = v \} 
    \] est une sous-représentation (qui est triviale).
  \end{exm}

  \begin{defn}
    On dit qu'une représentation linéaire $(\rho, V)$ est  \textit{irréductible} si 
    \begin{itemize}
      \item son degré est $\ge 1$ ;
      \item et ses seuls sons-représentations sont $\{0\}$ et $V$, \textit{i.e.} $(\rho,V)$ n'admet pas de sous-représentations non triviales.
    \end{itemize}
  \end{defn}

  \subsection{Morphismes de représentations.}

  \subsubsection{Définition.}

  \begin{defn}
    Soient $(\rho, V)$ et  $(\sigma, W)$ deux représentations linéaires de $G$.
    Un \textit{morphisme de  représentation de $V$ à $W$} est la donnée de \[
    f : F \to W
    ,\] 
    linéaire telle que \[
    \forall g \in G, \forall v \in V, \quad\quad f(g \cdot v) = g \cdot f(v)
    ,\] 
    \textit{i.e.},
    \[
    f(\rho(g)\: v) = \sigma(g)\:  f(v)
    .\]
  \end{defn}

  \begin{nota}
    On note $\mathrm{Hom}_G(V,W)$ l'ensemble des morphismes de représentation linéaire.
    C'est un $\mathds{k}$-espace vectoriel.
  \end{nota}

  \textbf{Terminologie.}
  Les éléments de $\mathrm{Hom}_G(V,W)$ sont appelés les \textit{morphismes $G$-équivariants}.

  \begin{exo}
    L'inclusion d'une sous-représentation est un morphisme de représentations.
  \end{exo}

  \begin{obs}
    Si $f \in \mathrm{Hom}_G(V,W)$ alors
    \begin{itemize}
      \item $\ker f$ est une sous-représentation de $V$ ;
      \item $\im f$ est une sous-représentation de $W$.
    \end{itemize}
  \end{obs}
  \begin{expl}
    Si $v \in \ker f$, alors $f(g \cdot v) = g \cdot f(v) = 0$.
    Si $w \in f(v) \in \im f$ alors $g \cdot w = g \cdot f(v) = f(g \cdot v) \in \im f$.
  \end{expl}

  \subsubsection{Structure de représentation linéaire sur $\mathrm{Hom}(V, W)$}

  On fait agir $G$ par conjugaison : \begin{align*}
    G &\longrightarrow \mathrm{Aut}(\mathrm{Hom}(V, W)) \\
    g &\longmapsto (f \mapsto g \cdot f \cdot g^{-1})
  .\end{align*}

  \begin{obs}
    Cela fait de $\mathrm{Hom}(V,W)$ une représentation linéaire de $G$.
  \end{obs}

  On a alors la proposition suivante.
  \begin{prop}
    $\mathrm{Hom}_G(V,W) = \mathrm{Hom}(V,W)^G$.
  \end{prop}
  \begin{prv}
    C'est un jeu d'écriture :
    \begin{align*}
      f \in \mathrm{Hom}(V,W)^G \iff& g \cdot f \cdot g^{-1} = f\\
      \iff& g \cdot f = f \cdot g\\
      \iff& f \in \mathrm{Hom}_G(V,W)
    .\end{align*}
  \end{prv}


\end{document}
