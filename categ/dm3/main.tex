\documentclass{../../td}

\title{TD n°3\\\itshape Théorie des catégories}
\author{Hugo \textsc{Salou}\\ Dept. Informatique}

\renewcommand\thesection{\arabic{chapter}.\Alph{section}}
\renewcommand\thechapter{\phantom}
\renewcommand\theequation{\arabic{chapter}.\arabic{equation}}
\usetikzlibrary{decorations.pathmorphing}

\let\bm\boldsymbol
\makeatletter
\newcommand\showfootnote{%
  \tfn@tablefootnoteprintout% 
  \gdef\tfn@fnt{0}% 
}
\makeatother

\declaretheorem[style=thmredbox, name=Lemme, numbered=no]{lemm}

\newcommand\pullback{\arrow[very near start,description,phantom]{dr}{\lrcorner}}
\newcommand\pushout{\arrow[very near start,description,phantom]{ul}{\ulcorner}}

\newcommand\id{\ensuremath{\operatorname{id}}}
\newcommand\ord{\ensuremath{\operatorname{ord}}}
\newcommand\colim{\ensuremath{\operatorname{colim}}}

\usepackage{amssymb}

\begin{document}
  \maketitle
  
  {
    \small
    \tableofcontents
  }


  \chapter{Exercice 1.}
  \begin{slshape}
    \color{deepblue}
    Montrer que si un foncteur est un adjoint à droite (\textit{resp}. à gauche) alors il est continue (\textit{resp}. cocontinue).
  \end{slshape}

  Soit $F : \mathbf{D} \to \mathbf{C}$ un foncteur possédant un adjoint à gauche que l'on notera $G : \mathbf{C} \to \mathbf{D}$.
  Ainsi, on sait que $\mathrm{Hom}(-, G-) \cong \mathrm{Hom}(F-,-)$.

  Considérons un petit diagramme $J : \mathbf{J} \to \mathbf{C}$.
  Ainsi, on a la chaîne d'isomorphisme :
  \begin{align*}
    \mathrm{Hom}(A, G(\lim J))
    &\cong \mathrm{Hom}(FA, \lim J) \quad \text{par adjoint}\\
    &\cong \lim \mathrm{Hom}(F A, J) \quad \text{par continuité (TD2)}\\
    &\cong \lim \mathrm{Hom}(A, GJ) \quad \text{par adjoint}\\
    &\cong \mathrm{Hom}(A, \lim GJ) \quad \text{par continuité (TD2)}
  ,\end{align*}
  pour tout $A \in \mathbf{C}_0$.
  Ceci étant vrai quel que soit $A$, on a donc l'isomorphisme $\mathcal{Y}(G(\lim J)) \cong \mathcal{Y}(\lim G J)$.

  Par le lemme de Yoneda, on en déduit que $G(\lim J) \cong \lim G J$.
  On a donc bien la continuité d'un foncteur possédant un adjoint à gauche,~\textit{i.e.} d'un foncteur qui est à un adjoint à droite.

  Par dualité, on a bien le résultat pour les foncteurs possédant un adjoints à droite.

  \chapter{Exercice 2.}
  \begin{slshape}
    \color{deepblue}
    Montrer que si $F : \mathbf{C} \to \mathbf{D}$ est une équivalence de catégories et $G : \mathbf{D} \to \mathbf{C}$ est un quasi-inverse de $F$, alors $F$ est adjoint à gauche à $G$ et $G$ est aussi adjoint à gauche à $F$.
  \end{slshape}

  On veut construire l'isomorphisme \[
    \alpha : \mathrm{Hom}(-, G-) \overset \sim \Rightarrow \mathrm{Hom}(F-,-)
  .\]

  On sait qu'il existe deux isomorphismes naturels \[
  \theta : \mathrm{id}_\mathbf{D} \Rightarrow  F \circ G \quad\quad \text{ et } \quad\quad \eta : \id_\mathbf{C} \Rightarrow G \circ F
  .\]

  Considérons $(f : A \to G B) \in \mathrm{Hom}(-, G -)$, et on veut construire une flèche de la forme $\alpha_{A,B}(f) : FA \to B$.
  On considère le diagramme 
  \[
  \begin{tikzcd}
    A \arrow{dd}{f} &&&& F A \arrow{dd}{Ff} \arrow[dashed]{ddrr}{\alpha_{A,B}(f)} \\
    \quad\quad\quad\arrow[-,very thick]{rr}{} && F \arrow[-stealth,very thick]{rr}{} && \quad\quad\quad \\
    G B &&&& F(GB) \arrow{rr}{\theta^{-1}_B} && B
  \end{tikzcd}
  ,\] 
  et on pose $\alpha_{A,B}(f) := \theta_B^{-1} \circ (F f)$.
  Ceci donne donc un isomorphisme \[
  \alpha_{A,B} : \mathrm{Hom}(A, G B) \overset\sim\longrightarrow \mathrm{Hom}(FA, B)
  .\]
  En effet, l'inverse est $\beta_{A,B} : g \mapsto (G g) \circ \eta_A$ comme le montre le diagramme
  \[
  \begin{tikzcd}
    F A \arrow{dd}{g} &&&& G(F A) \arrow{dd}{Gg} &&  \arrow{ll}{\eta_A} A \arrow[dashed]{ddll}{\beta_{A,B}(f)} \\
    \quad\quad\quad\arrow[-,very thick]{rr}{} && G \arrow[-stealth,very thick]{rr}{} && \quad\quad\quad \\
    B &&&& GB
  \end{tikzcd}
  .\]

  Ceci démontre ainsi que l'on a deux isomorphismes naturels
  \begin{gather*}
    \alpha : \mathrm{Hom}(-, G-) \overset \sim \Rightarrow \mathrm{Hom}(F-,-)\\
    \beta : \mathrm{Hom}(F-, -) \overset \sim \Rightarrow \mathrm{Hom}(-,G-).
  \end{gather*}
  démontrant ainsi que $F$ et $G$ sont adjoints à gauche de $G$ et à droite de $F$ respectivement.
  
  \chapter{Exercice 3.}
  \begin{slshape}
    \color{deepblue}
    On montre que les limites dans $\mathbf{Psh}(\mathbf{C}) := [\mathbf{C}^\mathrm{op}, \mathbf{Ens}]$ existent et se calculent point par point.
    Soit $F : \mathbf{J} \to \mathbf{Psh}(\mathbf{C})$ un diagramme.
    On rappelle que $F$ se voit comme $\hat{F}$ de $[\mathbf{J} \times \mathbf{C}^\mathrm{op}, \mathbf{Ens}]$.
    Vu que $\mathbf{Ens}$ est complet, montrer que $\varprojlim F$ existe et vaut en $X$ : \[
      \big(\!\varprojlim F\big)(X) \cong \varprojlim \hat{F}(-, X)
    ,\] 
    ou écrite de façon plus suggestive, \[
      \big(\!\varprojlim P_\bullet\big)(X) \cong \varprojlim P_\bullet(X)
    ,\] avec $P_\bullet = F$ (modulo curryfication).
    Quel est l'énoncé dual ?
  \end{slshape}

  Comme $\mathbf{Ens}$ est complet, la limite $\varprojlim \hat{F}(-, X)$ existe et on note alors~$\{\phi_{A,X}: \varprojlim \hat{F}(-, X) \to \hat{F}(A, X) \}$ le cône limite.
  De plus, on a que \[
    (\varprojlim F)(Y) = \{\psi_X : \varprojlim (F-)(Y) \to (F X)Y\} 
  .\]
  Et, par curryfication, on a que 
  \[
    \begin{tikzcd}[column sep={160,between origins}]
    \arrow{d}{\phi_X} \varprojlim (F-)(Y) \arrow{r}{\text{curryfication}}[swap]{\sim} & \varprojlim \hat{F}(-, Y) \arrow{d}{\psi_X}\\
    (F X)Y \arrow{r}{\text{curryfication}}[swap]{\sim} & \hat{F}(X,Y).
  \end{tikzcd}
  .\]
  Ceci implique l'isomorphisme \[
    (\varprojlim F)(X) \cong \varprojlim \hat{F}(-, X)
  .\]

  L'énoncé dual est que $\varinjlim F$ existe (car $\mathbf{Ens}$ est co-complet) et peut se calculer point par point avec :
  \[
    (\varinjlim F)(X) \cong \varinjlim \hat{F}(-, X)
  .\] 

  \chapter{Exercice 4.}
  \begin{slshape}
    \color{deepblue}
    \begin{enumerate}
      \item On rappelle que pour $f : A \to B$ une fonction, on peut définir le foncteur $f^{-1} : \wp B \to \wp A$ entre catégories posétales. Montrer qu'il admet un adjoint à gauche (bien connu) et un adjoint à droite (à construire).
      \item En déduire que
        \begin{itemize}
          \item $f\big(\bigcup_{i \in I} A_i\big) = \bigcup_{i \in  I} f(A_i)$ ;
          \item $f^{-1}\big(\bigcup_{i \in I} A_i\big) = \bigcup_{i \in  I} f^{-1}(A_i)$ ;
          \item $f^{-1}\big(\bigcap_{i \in I} A_i\big) = \bigcap_{i \in  I} f^{-1}(A_i)$ ;
        \end{itemize}
      \item Pourquoi $f$ n'a-t-il pas d'adjoint à gauche ?
    \end{enumerate}
  \end{slshape}

  \begin{enumerate}
    \item On pose le foncteur image directe, noté $f : \wp A \to \wp B$.
      Parce que \[
      f(S) \subseteq T \iff S \subseteq f^{-1}(T)
      ,\] quels que soient $S$ et $T$, on sait donc que $f$ est adjoint à gauche de $f^{-1}$.
      Pour l'adjoint à droite, il faut construire un foncteur de la forme~$R : \wp A \to \wp B$ vérifiant l'équivalence 
      \[
      S \subseteq R(T) \iff f^{-1}(S) \subseteq T
      ,\] quels que soient $S$ et $T$.

      On pose \[
      R(T) := f(A) \setminus f(A \setminus T) 
      ,\] 
      et on a bien l'équivalence demandée.

      En effet, si $S \subseteq R(T)$ alors $S \subseteq f(A) = \operatorname{im} f$ et $S \cap f(A \setminus T) = \emptyset$.
      Ceci implique que $f^{-1}(S) \cap f^{-1}(f(A \setminus T)) = \emptyset$ et donc que l'on ait $f^{-1}(S) \cap (A \setminus T) = \emptyset$ (car $f^{-1}(f(A \setminus T)) \supseteq A \setminus T$).
      On en déduit que $f^{-1}(S) \subseteq T$

      Réciproquement, si $f^{-1}(S) \subseteq T$, c'est alors que l'on ait $S \subseteq \operatorname{im} f$ et que $S \cap f(A \setminus T) = \emptyset$.
      On en déduit que $S \subseteq R(T)$.

      Ceci permet de conclure que l'on a bien construit un adjoint à droite du foncteur $f^{-1}$. 
    \item On applique l'exercice 1. En effet, l'union est la limite du diagramme discret (que l'on notera $A_I$ par la suite) dans la catégorie posétale $\wp X$ (pour $X$ un ensemble quelconque) et la colimite est l'intersection.
      \begin{itemize}
        \item On a \[
          f\Big(\bigcup_{i \in  I} A_i\Big) = f(\lim A_I) = \lim f(A_I) = \bigcup_{i \in  I} f(A_i)
          \]  car le foncteur $f$ est continu comme adjoint à droite de~$f^{-1}$.
        \item On a \[
          f^{-1}\Big(\bigcup_{i \in  I} A_i\Big) = f^{-1}(\lim A_I) = \lim f^{-1}(A_I) = \bigcup_{i \in  I} f^{-1}(A_i)
          \]  car le foncteur $f^{-1}$ est continu comme adjoint à droite du foncteur~$R$.
        \item On a \[
          f^{-1}\Big(\bigcap_{i \in  I} A_i\Big) = f^{-1}(\operatorname{colim} A_I) = \operatorname{colim} f^{-1}(A_I) = \bigcap_{i \in  I} f^{-1}(A_i)
          \]  car le foncteur $f^{-1}$ est cocontinu comme adjoint à gauche du foncteur~$f$.
      \end{itemize}
    \item Supposons que $f$ admette un adjoint à gauche, alors $f$ donc un adjoint à droite, et ainsi il est continu.
      En particulier, on a l'égalité $f\big(\bigcap_{i \in  I} A_i \big) = \bigcap_{i \in I} f(A_i)$.
      Sauf que c'est faux !
      On considère par exemple le cas $A = B = \llbracket 1,3\rrbracket$ muni de l'application \begin{align*}
        f: \llbracket 1,3\rrbracket  &\longrightarrow \llbracket 1,3\rrbracket \\
        1 &\longmapsto 1\\
        2 &\longmapsto 2\\
        3 &\longmapsto 1\\
      .\end{align*}
      Ainsi pour $A_1 = \{1,2\}$ et $A_2 = \{2,3\}$, on a \[
      f(A_1 \cap A_2) = f(\{2\}) = \{2\}  \quad\quad \text{mais}\quad\quad f(A_1) \cap f(A_2) = \{1,2\} 
      .\]

      En général, $f$ n'admet pas d'adjoint à gauche.\footnote{À moins que $f$ soit injective, auquel cas $f^{-1}(S) \subseteq T \iff S \subseteq f(T)$ car l'image réciproque $f^{-1}(S)$ ne contient \textit{que} les images de $S$ et rien d'autre.}\showfootnote
  \end{enumerate}

  \chapter{Exercice 5.}
  \begin{slshape}
    \color{deepblue}
    Montrer qu'une transformation naturelle $\alpha : P \Rightarrow Q$ est un monomorphisme dans  $\mathbf{Psh}(\mathbf{C})$ si et seulement si chaque composante $\alpha_X : P(X) \to Q(X)$ l'est. Quel est l'énoncé dual ?

    \textit{Indice.} Utiliser le lemme de Yoneda dans le sens difficile.
  \end{slshape}

  On procède en deux temps.

  Considérons $\eta$ et  $\gamma$ comme indiqué dans le diagramme 
  \[
  \begin{tikzcd}
    O \arrow[Rightarrow,bend left]{r}{\eta}  \arrow[Rightarrow,bend right,swap]{r}{\gamma} & P \arrow[Rightarrow]{r}{\alpha} & Q
  \end{tikzcd}
  .\]
  On sait que $\eta = \gamma$ si et seulement si $\eta_X = \gamma_X$ pour tout $X \in \mathrm{ob}(\mathbf{C})$ (et de même pour $\alpha \circ \eta$,$\alpha \circ \gamma$).
  Supposons que $\alpha \circ \eta = \alpha \circ \gamma$, et que chaque~$\alpha_X$ est un monomorphisme.
  Ainsi, $\alpha_X \circ \eta_X = \alpha_X \circ \gamma_X$ par définition de la composition et donc~$\eta_X = \gamma_X$ quel que soit $X$.
  \[
  \begin{tikzcd}
    O X \arrow[bend left]{r}{\eta_X} \arrow[bend right,swap]{r}{\gamma_X} & P X \arrow{r}{\alpha_X} & Q X
  \end{tikzcd}
  .\] 
  On en déduit $\eta = \gamma$ et ainsi que $\alpha$ est un monomorphisme.

  Pour l'autre sens, le sens difficile, on suppose que $\alpha : P \Rightarrow Q$ est un monomorphisme.
  Fixons un $X$ quelconque.
  On applique le lemme de Yoneda qui donne une transformation naturelle \[\tau : \mathrm{Ev}(-, X) \overset \sim \Rightarrow \mathrm{Nat}(\mathcal{Y}(X), -),\] où l'on a noté $\mathrm{Ev}(F, X)$ le bifoncteur d'évaluation.
  Ainsi, le diagramme \[
  \begin{tikzcd}
    P X \arrow{r}{\alpha_X} \arrow{d}{\tau_P}[swap]{\vertical\sim} & Q X \arrow{d}{\tau_Q}[swap]{\vertical\sim}\\
    \mathrm{Nat}(\mathcal{Y}(X), P) \arrow{r}{- \circ \alpha} & \mathrm{Nat}(\mathcal{Y}(X), Q)
  \end{tikzcd}
  \]
  commute.
  Et, si $\alpha$ est un monomorphisme alors $- \circ \alpha$ l'est aussi et il en va de même pour  \[
    \tau_Q^{-1}\circ (- \circ \alpha) \circ \tau_P
  ,\] par composition de monomorphismes (isomorphisme implique monomorphisme).

  On en déduit que $\alpha_X$ est un monomorphisme, et ce quel que soit $X$. D'où l'équivalence.

  L'énoncé dual est 
  \begin{quote}
    "une transformation naturelle $\alpha : P \Rightarrow Q$ est un épimorphisme dans $\mathbf{Psh}(\mathbf{C})^\mathrm{op}$ si et seulement si chaque composante $\alpha_X : PX \to QX$ l'est".
  \end{quote}

  \chapter{Exercice 6.}
  \begin{slshape}
    \color{deepblue}
    \begin{enumerate}
      \item Montrer que $\wp : A \mapsto \wp(A)$ et $f \mapsto \tilde{f}$ (où $\tilde{f}$ est l'image directe) n'est pas représentable.
      \item Choisir une catégorie d’objet mathématique avec un foncteur d’oubli vers $\mathbf{Ens}$ et montrer qu’il est représentable (ou sinon, pourquoi il ne l’est pas). Les exemples du cours ne sont pas autorisés.
    \end{enumerate}
  \end{slshape}

  \begin{enumerate}
    \item Par l'absurde supposons le représentable.
      Ainsi, il existe $A$ un ensemble tel que $\wp- \overset\sim\Rightarrow \mathrm{Hom}(A, -)$.

      Considérons un ensemble fini $B$ de cardinal $m$.
      Notons $n$ le cardinal de $A$ (potentiellement infini, cela ne posera pas problème s'il l'est).
      On a ainsi \[
        2^m\underset {\substack{\uparrow\\\mathclap{\text{dénombrement}}}} =\operatorname{card} \wp(B) \underset {\substack{\uparrow\\\mathclap{\text{isomorphisme}}}} = \operatorname{card} \mathrm{Hom}(A, B) \underset {\substack{\uparrow\\\mathclap{\text{dénombrement}}}}= (\operatorname{card} B)^{\operatorname{card} A} = m^n
      ,\]
      ce qui est faux pour un ensemble arbitraire $B$ de cardinal $m$.
    \item On considère la catégorie des monoïdes $\mathbf{Mon}$ muni des morphismes de monoïdes.
      Le foncteur d'oubli $U : \mathbf{Mon} \to \mathbf{Ens}$ est défini par :
      \begin{itemize}
        \item à un monoïde $(M, \cdot, \epsilon)$ on associe $M$ l'ensemble sous-jacent ;
        \item à un morphisme $u : (M, \cdot, \epsilon) \to (N, \diamond, \varepsilon)$ on associe l'application $\hat{u} : M \mapsto N, x \mapsto u(x)$ la fonction sous-jacente.
      \end{itemize}

      On représente un tel foncteur d'oubli par le monoïde libre que l'on notera $(\{1\}^\star, \cdot, \varepsilon)$. (Un monoïde libre sur $X$ est l'ensemble des mots sur l'alphabet $X$ donné.)
      L'ensemble $\{1\}^\star$ est ainsi égal à \[
      \{1\}^\star = \{1^n  \mid n \in \mathds{N}\}
      .\] 
      L'opération $\cdot$ correspond à la concaténation usuelle des mots ($1^n \cdot 1^m = 1^{n+m}$ en est une conséquence), et $\varepsilon$ correspond au mot vide ($\varepsilon = 1^0$ est aussi une conséquence).

      Le monoïde libre sur $\{1\}$ est isomorphe à $(\mathds{N}, +, 0)$ mais cette construction n'est plus vraie pour des alphabets plus grands (\textit{c.f.} théorie des langages).

      Construisons l'isomorphisme \[
      \mathrm{Hom}_{\mathbf{Mon}}((\{1\}^\star, \cdot, \varepsilon), (N, \diamond, \epsilon)) \cong N = U(N)
      .\]

      On définit
      \begin{align*}
        \Phi: \mathrm{Hom}_{\mathbf{Mon}}((\{1\}^\star, \cdot, \varepsilon), (N, \diamond, \epsilon)) &\longrightarrow N \\
        (u : \{1\}^\star \to N) &\longmapsto u(1)
      .\end{align*}
      C'est bien un isomorphisme :
      \begin{itemize}
        \item \textit{injectivité}, si $f, g : \{1\}^\star \to N$ vérifient $f(1) = g(1)$ mais par morphisme de monoïde, on a que $f(1^n) = f(1)^n = g(1)^n = g(1^n)$ et les  $(1^n)_{n \in N}$ engendrent le monoïde libre (il n'y a rien de plus en réalité), donc $f = g$ ;
        \item \textit{surjectivité}, pour un élément $x \in N$ on pose le morphisme~$u$ défini par $u(1^n) := x^n$, il vérifie bien que $u(1) = x$, d'où la surjectivité.
      \end{itemize}

      On en conclut quant à la représentabilité du foncteur d'oubli sur les monoïdes $U : \mathbf{Mon} \to \mathbf{Ens}$.
  \end{enumerate}

  \chapter{Exercice 7.}

  \begin{slshape}
    \color{deepblue}

    Dans une catégorie posétale admettant tout produit fini (dite "\textit{cartésienne}"), on appelle (s'il existe) l'exponentiation par $X$ le foncteur~$(-)^{X}$ (s'il existe) adjoint à gauche de
    \begin{align*}
      - \times X: \mathbf{C} &\longrightarrow \mathbf{C} \\
      A &\longmapsto A \times X\\
      (f : A \to B) &\longmapsto (f \times \id_X : A \times X \to B \times X)
    .\end{align*}

    \begin{enumerate}
      \item Décrire l'exponentiation dans $\mathbf{Ens}$.
      \item Montrer que dans une catégorie admettant tout objet exponentiel $(A^B)^C \cong A^{B \times C}$.
      \item Dans une catégorie admettant tout produit fini et tout objet exponentiel (c'est à dire "\textit{clos cartésien}") montrer que si de plus $\mathbf{C}$ est localement petite et contient les coproduits alors \[
      A \times (B + C) \cong A \times B + A \times C
      .\]
    \end{enumerate}
  \end{slshape}

  \begin{enumerate}
    \item Montrons que dans $\mathbf{Ens}$, l'exponentiation $(-)^X$ correspond au foncteur $\mathrm{Hom}(X, -)$.
      Soient $A,B,X$ trois ensembles.
      On a donc l'isomorphisme
      \begin{align*}
        \Phi: \mathrm{Hom}(A, \mathrm{Hom}(X, B)) &\longrightarrow \mathrm{Hom}(A \times X, B) \\
        (f : A \to \mathrm{Hom}(X, B)) &\longmapsto \left(
        \begin{array}{rcl}
          g : (A \times X) & \to & B\\
          (a,x) &\mapsto & f(a)(x)
        \end{array}
        \right)
      ,\end{align*}
      qui est juste une curryfication.
      D'où $\mathrm{Hom}(X, -)$ est adjoint à gauche de $- \times X$.
      On en déduit que dans $\mathbf{Ens}$ l'exponentiation existe toujours et qu'il vaut $\mathrm{Hom}(X, -) \cong (-)^X$.
    \item Soient $X,A,B,C$ des objets.
      On a la chaine d'isomorphisme suivante :
      \begin{align*}
        \mathrm{Hom}(X, (A^B)^C) &\cong \mathrm{Hom}(X \times C, A^B) & \text{par adjoint}\\
        &\cong \mathrm{Hom}((X \times C) \times B, A) & \text{par adjoint}\\
        &\cong \mathrm{Hom}(X \times (C \times B), A) & \text{par TD2 (ex1)}\\
        &\cong \mathrm{Hom}(X \times (B \times C), A) & \text{par TD2 (ex1)}\\
        &\cong \mathrm{Hom}(X, A^{B \times C}) & \text{par adjoint}
      .\end{align*}

      On en déduit que $\mathcal{Y}((A^{B})^{C}) \cong \mathcal{Y}(A^{B \times C})$.
      D'où, $(A^B)^C \cong A^{B \times C}$ par le lemme de Yoneda.
    \item Le foncteur $A \times -$ admet un adjoint à gauche.
      Il est donc co-continu (exercice 1).

      Considérons $F : \mathbf{J} \to \mathbf{C}$ le diagramme discret  donné ci-dessous \[
      \begin{tikzcd}
        B\arrow[loop above]{}{\mathrm{id}_B} & 
        C\arrow[loop above]{}{\mathrm{id}_C}
      \end{tikzcd}
      .\]
      On a la chaîne d'isomorphismes suivante :
      \begin{align*}
        A \times (B + C) &\cong A \times (\operatorname{colim} F)\\
        &\cong \operatorname{colim} (A \times F)\\
        &\cong A \times B + A \times C
      ,\end{align*}
      car le diagramme $A \times F$ est :\[
      \begin{tikzcd}
        A \times B\arrow[loop above]{}{\mathrm{id}_{A \times B}} & 
        A \times C\arrow[loop above]{}{\mathrm{id}_{A \times C}}
      \end{tikzcd}
      .\]
  \end{enumerate}

  \chapter{Exercice 8.}

  \begin{slshape}
    \color{deepblue}
    On dit qu'une catégorie est \textit{filtrante} si tout diagramme fini admet un cocône.

    \begin{enumerate}
      \item Pourquoi cela généralise la notion d'ordre filtrant (\textit{c.f.} TD 2) ?
    \end{enumerate}
    On appelle \textit{(co)limite filtrante} une (co)limite indexée par une catégorie filtrante.
    Dans une catégorie localement petite, on appelle un objet $X$ \textit{compact} dès lors que $h^X := \mathrm{Hom}(X, -)$ commute avec les limites filtrées.

    \begin{enumerate}[resume*]
      \item Montrer qu'un objet $X$ est compact si et seulement si toute flèche $X \overset u \to \operatorname{colim} U_i$, pour un diagramme $U_\bullet$ filtrant, se factorise par  $X \overset f \to U_j \overset {\iota_j} \to \operatorname{colim} U_i$. \label{ex8-q2}
      \item Quels sont les objets compacts de $\mathcal{O}_X$ la catégorie des ouverts munie des inclusions de l'espace topologique $X$ ? (\textit{Indice} : le nom est bien choisi).
      \item Quels sont les objets compacts de $\mathbf{Ens}$ ?
      \item Quels sont les objets compacts de $\mathbf{Vect}_\mathds{k}$ ?
      \item Qu'en est-il de $\mathbf{Grp}$ ?
    \end{enumerate}
  \end{slshape}

  \begin{enumerate}
    \item Considérons une catégorie posétale $\mathbf{C}$ munie d'un ordre filtrant noté $\preceq$.
      Montrons que c'est une catégorie filtrante.

      Montrons que tout diagramme fini de cardinal $n$ admet un cocône par récurrence sur $n \in \mathds{N}^\star$.

      \begin{itemize}
        \item Pour $n = 1$, on considère un diagramme $F$ de cardinal 1.
          Un cocône de ce diagramme est celui ci-dessous :
          \[
          \begin{tikzcd}
            F X \arrow{d}{\id_{F X}}\\
            F X
          \end{tikzcd}
          .\] 
        \item Pour un diagramme $F$ de cardinal $n+1$, on isole un élément de ce diagramme $X_0$ et on considère le diagramme~$F'$ contenant les $n$ autres éléments.
          Par hypothèse de récurrence, ce diagramme possède un cocône $C'$ muni des morphismes $\phi'_X : F' X \to C'$.
          Parce que $\preceq$ est filtrant, il existe~$C \in \mathbf{C}$ tel que $X_0\preceq C$ et $C' \preceq C$.
          Le diagramme suivant est donc un cocône sur $F$ :
          \[
          \begin{tikzcd}[column sep={40,between origins}, row sep={40, between origins}]
            F X_0 \arrow{ddrr}{} & & F X_1 \arrow[swap]{drr}{} & F X_2 \arrow[swap]{dr}{} & \cdots  & F X_{n-1} \arrow[swap]{dl}{} & F X_n \arrow{dll}{}\\
                  &&&& C' \arrow{dll}{}\\
                  &&C
          \end{tikzcd}
          .\]
      \end{itemize}
      On en conclut que toute catégorie posétale sur un ordre filtrant est une catégorie filtrante.
    \item On procède par équivalence.
      \begin{align*}
        & \text{$X$ compact} \\
        \iff & \forall U_\bullet \text{ diag. filtrant}, \colim h^X(U_i) \cong h^X(\colim U_i)\\
        \iff & \forall U_\bullet \text{ diag. filtrant}, \underbrace{\colim \mathrm{Hom}(X, U_i)}_{\{\phi_i : \mathrm{Hom}(X, U_i) \to N\} } \cong \mathrm{Hom}(X, \colim U_i) \\
        \iff & \forall U_\bullet \text{ diag. filtrant}, \forall u : X \to \colim U_\bullet, \exists ! i \in I,\\
             & \quad\quad\quad\quad\quad \text{le diagramme suivant commute}
      \end{align*}
      \[
        \begin{tikzcd}
          X \arrow{dd}{u} \arrow{dr}{\phi_i} \\
          & U_i \arrow{dl}{\iota_i}\\
          \colim U_\bullet
        \end{tikzcd}
      ,\]
      où l'on a noté $(\iota_i : U_i \to \colim U_\bullet)$ le cocône limite sur $U_\bullet$ et $N$ la pointe du cocône limite $\colim \mathrm{Hom}(X, U_i)$.

    \item Montrons que $Y$ est compact (au sens que chaque couverture par ouvert de $Y$ admet une sous-couverture finie de $Y$) si et seulement si $Y$ est un objet compact dans $\mathcal{O}_X$.
      Je noterai "objet compact" pour la définition catégorique et "compact topologique" pour la définition topologique.

      Dans cette catégorie, la colimite correspond à l'union ensembliste.
      Considérons un recouvrement $\bigcup_{i \in  I} U_i = Y$, pour $Y$ quelconque.
      Le diagramme \begin{align*}
        K: \wp_\text{finies}(I) &\longrightarrow \mathcal{O}_X \\
        I\supseteq J &\longmapsto \bigcup_{j \in J} U_j
      \end{align*} est filtrant (toute famille finie de partie finie admet une union finie) et sa colimite vaut $\bigcup_{i \in  I} U_i = Y$.

      Ainsi, si $Y$ est un objet compact, alors il existe $J \subseteq I$ fini et $f$ tel que le diagramme suivant commute :
      \[
      \begin{tikzcd}
        Y \arrow{dr}{f} \arrow{dd}{\id_Y}\\
        & \bigcup_{j \in J} U_j \arrow{dl}{\iota_J}\\
        \bigcup_{i \in  I} U_i = Y
      \end{tikzcd}
      .\]
      Ceci implique nécessairement que $\bigcup_{j \in J} U_j = Y$ avec $J$ fini.
      On en conclut que tout recouvrement de $Y$ admet un sous-recouvrement fini, donc que $Y$ est un compact topologique.

      Réciproquement, si $Y$ est un compact topologique, et soit $U_\bullet : I \to \mathcal{O}_X$ un diagramme filtrant quelconque.
      Montrons l'isomorphisme \[
        \mathrm{Hom}_{\mathcal{O}_X}(Y, \colim U_\bullet) \cong \colim \mathrm{Hom}_{\mathcal{O}_X}(Y, U_\bullet)
      .\] 
      L'ensemble à droite est vide si et seulement si $\colim U_\bullet = \bigcup_{i \in  I} U_i \subsetneq Y$, et c'est un singleton sinon (argument de continuité).
      Dans le premier cas, ceci implique que tous les ensembles $\mathrm{Hom}_{\mathcal{O}_X}(Y, U_i)$ par le même argument et donc $\colim \mathrm{Hom}_{\mathcal{O}_X}(Y, U_\bullet)$ est vide.
      Dans l'autre cas, par compacité topologique de $Y$, on sait que l'on peut extraire un sous-ensemble fini $J \subseteq I$ tel que $Y = \bigcup_{j \in  J} U_j$.
      Et, vu que le diagramme $U_\bullet$ est filtré, alors il existe un cocône sur $J$, de pointe $p$.
      On en déduit que~$U_c \supseteq \bigcup_{j \in J}U_j = X$, d'où $U_c = X$.
      On a donc bien l'isomorphisme demandé.


    \item Montrons qu'un ensemble est compact (au sens d'être un objet compact dans $\mathbf{Ens}$) si et seulement s'il est fini.

      Remarquons que l'union ensembliste est une colimite filtrée, et que pour une famille finie d'ensembles finis, leur union est fini.

      Considérons un ensemble compact $X = \bigcup_{i \in  I} U_i$ où les éléments~$U_i \subseteq X$ sont des sous-ensembles finis, et où $I$ est une catégorie filtrante.
      Alors, par la question~\ref{ex8-q2}, il existe~$i \in I$ et $f$ tel que le diagramme suivant commute :
      \[
      \begin{tikzcd}
        X \arrow{dd}{\id_X} \arrow{dr}{f} \\
        & U_i \arrow{dl}{\iota_i} \\
        \colim U_\bullet = X
      \end{tikzcd}
      .\]
      On en déduit que nécessairement $U_i \cong X$, ce qui implique $X$ fini.

      Réciproquement, soit $U_\bullet : I \to \mathbf{Ens}$  un diagramme filtrant quelconque.
      On a l'isomorphisme : \[
        \mathrm{Hom}(X, \underbrace{\colim U_\bullet}_{\bigcup_{i \in  I} U_i}) \cong \underbrace{\colim \mathrm{Hom}(X, U_\bullet)}_{\bigcup_{i \in  I} \mathrm{Hom}(X, U_\bullet)}
      ,\]
      par les propriétés des applications usuelles sur les ensembles.



    \item Considérons un espace vectoriel $E$.
      On pose $E = \bigoplus_{i \in I} \operatorname{vect}(e_i)$.
      On "décompose" ainsi $E$ en somme directe de sous-espaces vectoriels de dimension $1$.
      On note $U_i = \operatorname{vect} e_i$.

      Le diagramme \begin{align*}
        K: \wp_\text{finies}(I) &\longrightarrow \mathbf{Vect}_\mathds{k} \\
        J &\longmapsto \bigoplus_{j \in J} U_j
      ,\end{align*}
      c'est un diagramme filtrant (la somme d'espace vectoriels existe toujours, et c'est un espace vectoriel) et sa colimite vaut $\bigoplus_{i \in I} U_i = E$.

      Ce diagramme filtrant permet la factorisation
      \[
      \begin{tikzcd}
        E \arrow{dd}{\id_E} \arrow{dr}{f}\\
        & \bigoplus_{j \in J} U_j \arrow{dl}{\iota_j}\\
        E = \bigoplus_{i \in I} U_i
      \end{tikzcd}
      .\]
      On en déduit que $E = \bigoplus_{j \in J} U_j$ avec $J$ fini, donc que $E$ est de dimension finie.

      Réciproquement, si $U_\bullet : I \to \mathbf{Vect}_\mathds{k}$ est un diagramme filtrant quelconque, où $X$ est fini. Montrons l'isomorphisme \[
        \mathrm{Hom}(X, \underbrace{\colim U_\bullet}_{\bigoplus_{i \in  I} U_i}) \cong \underbrace{\colim \mathrm{Hom}(X, U_\bullet)}_{\bigoplus_{i \in  I} \mathrm{Hom}(X, U_\bullet)}
      ,\] en décomposant les applications sous chacune de leurs coordonnées (possible car $X$ est fini).
      D'où la réciproque.
    \item Pour $\mathbf{Grp}$, on procède quasi-exactement que pour $\mathbf{Vect}_\mathds{k}$.
      On décompose un groupe $G$ en $\langle g_i \mid i \in I \rangle$, et on pose le diagramme 
      \begin{align*}
        K: \wp_\text{finies}(I) &\longrightarrow \mathbf{Grp} \\
        J &\longmapsto \langle g_i  \mid i \in J \rangle
      .\end{align*}
      On montre, très similairement au raisonnement réalisé dans les précédente question, que $G$ est compact si et seulement si $G = \langle g_i  \mid i \in J \rangle$ avec $J$ fini.
      Autrement dit, $G$ est compact si et seulement s'il est engendré par un nombre fini d'éléments.
  \end{enumerate}

  \chapter{Exercice 10.}

  \begin{slshape}
    \color{deepblue}
    \begin{enumerate}
      \item Montrer que $\mathbf{Co} \backslash \mathds{K} \cong \mathbf{Ext}_\mathds{K}$ où $\mathbf{Co}$ est la catégorie des corps et~$\mathbf{Ext}_\mathds{K}$ est la catégorie des extensions $\mathds{L}$ de $\mathds{K}$ ayant pour flèches les $\mathds{K}$-morphismes de corps.
      \item Donner de même une description équivalente de $\mathbf{Ens} \backslash \{*\}$ avec~$\{*\}$ un singleton.
    \end{enumerate}
  \end{slshape}

  \begin{enumerate}
    \item Une extension d'un corps $\mathds{K}$ est un couple $(\mathds{L}, \iota)$ où $\iota : \mathds{K} \to \mathds{L}$ est un morphisme de corps.
      Et, un objet de $\mathbf{Co} \backslash \mathds{K}$ est un couple~$(\mathds{L}, \iota)$ où $\mathds{L}$ est un corps et  $\iota : \mathds{K} \to \mathds{L}$ est un morphisme de corps (car morphisme de $\mathbf{Co}$.)

      On a donc correspondance exacte des objets.

      Si on a un morphisme d'extension $u : \mathds{L} \to \mathds{L}'$, alors $u$ vaut l'identité sur $\mathds{K}$ (par définition de morphisme d'extensions) et c'est un morphisme de corps.
      Par ces deux contraintes, le diagramme ci-dessous commute :
      \[
      \begin{tikzcd}
                                                   & \mathds{K}\\
        \mathds{L} \arrow{rr}{u} \arrow[swap]{ur}{\iota} && \mathds{L}'\arrow{ul}{\iota'}
      \end{tikzcd}
      .\]
      
      Ceci conclut quant à l'isomorphisme des deux catégories.

    \item Construisons l'isomorphisme entre les catégories $\mathbf{Ens} \backslash \{*\}$ et~$\mathbf{Ens_p}$ où $\mathbf{Ens_p}$ est la catégorie des espaces pointés :
      \begin{align*}
        \mathbf{Ens} \backslash \{*\} &\overset \sim \longleftrightarrow \mathbf{Ens_p} \\
         (A, \pi : \{*\} \to A) &\longmapsto (A,  \pi(*))\\
         (A, \pi : {*} \mapsto p) &\longmapsfrom (A,  p)\\
         ((A, \pi) \overset u \to (A', \pi')) &\longmapsto ((A, \pi(*)) \overset u \to (A', \pi'(*)))\\
         u \circ \pi = \pi' & \phantom{\longmapsto} \quad u(\pi(*)) = \pi'(*)\\
         ((A, \pi : * \mapsto p) \overset u \to (A', \pi' : * \mapsto p')) &\longmapsfrom ((A, p) \overset u \to (A', p))\\
         u \circ \pi = \pi' & \phantom{\longmapsto} \quad u(p) = p'
      .\end{align*}
  \end{enumerate}

  \chapter{Exercice 11.}
  
  \begin{slshape}
    \color{deepblue}
    Soient $\mathbf{C}$ une petite catégorie, $\mathbf{E}$ une catégorie, et $F : \mathbf{C} \to \mathbf{E}$ un diagramme tel que $\lim_\mathbf{C} F$ existe.
    Montrer que la catégorie des cônes sur $F$ est isomorphe à la catéogorie relative à $\lim_\mathbf{C} F$.
    Si $\colim_\mathbf{C} F$ existe, énoncer et démontrer la proposition duale.
  \end{slshape}

  Notons $\{\psi_X : \lim_\mathbf{C} \to F X\}_{X \in \mathbf{C}}$ le cône limite.

  On construit l'isomorphisme suivant :
  \begin{align*}
    \textbf{Cônes}_\mathbf{C}(F) &\overset \sim \longleftrightarrow \mathbf{C} / {\lim}_\mathbf{C} F \\
    \{\phi_X : C \to F X\}_{X \in \mathbf{C}} &\longmapsto (C, \Phi : C \to {\lim}_\mathbf{C} F)\\
    \{\phi_X := \psi_X \circ \Phi \}_{X \in \mathbf{C}} & \longmapsfrom (C, \Phi : C \to {\lim}_\mathbf{C} F)\\
    (u : C \to C') & \longleftrightarrow (u : (C, \Phi) \to (C', \Phi'))\\
    \psi_X \circ u = \psi'_X & \phantom{\longmapsto} \quad \Phi \circ u = \Phi'
  .\end{align*}
  où, dans le premier cas, on utilise la propriété universelle de la limite sur $F$.
  Pour l'équivalence entre $\psi_X \circ u = \psi'_X$ et  $\Phi \circ u = \Phi'$ on procède par :
  \begin{itemize}
    \item "$\implies$". propriété universelle ;
    \item "$\impliedby$". par morphisme de cocône.
  \end{itemize}

  La propriété duale est : la catégorie des cocônes sur $F$ est isomorphe à la catégorie corelative à $\colim_\mathbf{C} F$.
  On procède par dualité.
  \begin{itemize}
    \item la colimite $\colim_\mathbf{C} F$ est une limite dans la catégorie opposée ;
    \item un cocône sur $F$ est un cône dans la catégorie opposée ;
    \item on a $(\mathbf{C} \backslash X)^\mathrm{op} = \mathbf{C}^\mathrm{op}/ X$.
  \end{itemize}

  \chapter{Exercice 12.}

  \begin{slshape}
    \color{deepblue}
    On donne (pour la première fois!) la définition de propriété universelle la plus commune, celle de morphisme universelle.
    Soit $G : \mathbf{D} \to \mathbf{C}$ un foncteur, et soit $X \in \mathbf{C}$.
    On définit le \textit{morphisme universel} de $X$ vers~$G$ par un couple $(A, u : X \to GA)$ tel que tout morphisme $f : X \to G B$ on dispose d'un unique $g : A \to B$ tel que le triangle suivant  (à droite) commute :
  \[
  \begin{tikzcd}
    A \arrow[dashed]{dd}{g} &&&& G A \arrow{dd}{G g} & X \arrow{l}{u} \arrow{ddl}{f} \\
    \quad\quad\quad\arrow[-,very thick]{rr}{} && G \arrow[-stealth,very thick]{rr}{} && \quad\quad\quad \\
    B &&&& GB
  \end{tikzcd}
  .\]
  \begin{enumerate}
    \item Montrer qu'un morphisme universel de $X$ vers $G : \mathbf{D} \to \mathbf{C}$ est tout simplement un objet initial de la catégorie fléchée $\mathbf{D} \backslash X$.
    \item Montrer que le foncteur $\mathrm{Hom}(X, G -) : \mathbf{D} \to \mathbf{Ens}$ est représenté par $A$ où $\eta : h^A \overset \sim \Rightarrow \mathrm{Hom}(X, G-)$ revient à dire que $(A, \eta_A(\id_A) : X \to G A)$ est le morphisme universel de $X$ dans $G$.
    \item Montrer aussi qu'une propriété universelle est une version locale d'un adjoint, c'est-à-dire un adjoint $F$ à gauche de $G$ répond pour tout $X$ au problème universel de $X$ vers $G$ avec $(F X, \eta_X : X \to GFX)$ où $\eta$ est l'unité de  $F \dashv G$.
  \end{enumerate}
  \end{slshape}

  \begin{enumerate}
    \item Un objet initial  de la catégorie relative $\mathbf{D} \backslash X$ est un objet de la forme $(A, u: X \to GA)$ tel que pour tout autre objet $(B, v : X \to G B)$ on a un unique morphisme $g : A \to B$ qui fait commuter le diagramme suivant :
      \[
  \begin{tikzcd}
    A \arrow[dashed]{dd}{g} &&&& G A \arrow{dd}{G g} & X \arrow{l}{u} \arrow{ddl}{v} \\
    \quad\quad\quad\arrow[-,very thick]{rr}{} && G \arrow[-stealth,very thick]{rr}{} && \quad\quad\quad \\
    B &&&& GB
  \end{tikzcd}
      .\]
      Ainsi, un morphisme universel de $X$ vers $G : \mathbf{D} \to \mathbf{C}$ est tout simplement un objet initial de la catégorie $\mathbf{D} \backslash X$.
    \item On pose l'isomorphisme 
      \begin{align*}
        \eta_B: \mathrm{Hom}(A,B) &\longrightarrow \mathrm{Hom}(X, GB) \\
        g &\longmapsto G g \circ u
      ,\end{align*}
      car $(A, u)$ est est un morphisme universel.
      Le carré suivant commute
      \[
      \begin{tikzcd}
        \mathrm{Hom}(A, B)  \arrow{d}{\eta_B}[swap]{\sim} \arrow{r}{f \circ -} & \mathrm{Hom}(A, C) \arrow{d}{\eta_C}[swap]{\sim} \\
        \mathrm{Hom}(X, GB) \arrow{r}{G f \circ -} & \mathrm{Hom}(X, GC)
      \end{tikzcd}
      ,\] 
      d'où $\eta$ est une transformation naturelle $\mathrm{Hom}(A, -) \Rightarrow \mathrm{Hom}(X, G -)$.
      On conclut que $A$ représente $\mathrm{Hom}(X, G -)$.
    \item 
      On a l'isomorphisme naturel $\alpha : \mathrm{Hom}(X, G -) \Rightarrow \mathrm{Hom}(F X, -)$ où 
      \begin{align*}
        \alpha_B: \mathrm{Hom}(X, G B) &\longrightarrow \mathrm{Hom}(F X, B) \\
        f &\longmapsto \varepsilon_B \circ F f
      ,\end{align*}
      où $\varepsilon : FG \Rightarrow \id_\mathbf{C}$.
      Ceci implique que, pour $f : X \to GB$, il existe un unique morphisme $g \in \mathrm{Hom}(FX, B)$.
      Ainsi, par adjoint \[
      G (\alpha_B(f)) \circ \eta_X = G (\varepsilon_B \circ F f) \circ \eta_X = f
      .\]
      Ceci implique que le diagramme suivant commute : \[
      \begin{tikzcd}
        FX \arrow[dashed,swap]{dd}{\alpha_B(f)} &&&& G FX \arrow[swap]{dd}{G (\alpha_B f)} & X \arrow{l}{\eta_X} \arrow{ddl}{f} \\
        \quad\quad\quad\arrow[-,very thick]{rr}{} && G \arrow[-stealth,very thick]{rr}{} && \quad\quad\quad\quad\quad \\
        B &&&& GB
      \end{tikzcd}
      .\]
      Ceci permet d'en conclure que $(FX, \eta_X)$ est un morphisme universel.
  \end{enumerate}

\end{document}
