\documentclass{../../notes}

\title{DM n°1\\[-0.3em] \itshape Algorithmique 2}
\author{Hugo \scshape Salou}

\tikzset{
  initial text=,
  node distance=2cm,
  auto,
  every state/.style={draw=deepblue,thick},
  every path/.style={draw=deepblue},
}

\begin{document}
  \maketitle

  \chapter{Condition nécessaire et suffisante.}

  Une condition nécessaire pour avoir une attribution "Noël canadien" est la suivante :
  en notant $j = \arg\max_{i \in \llbracket 1,n\rrbracket  } d_i$, \[
  d_j \le  \sum_{i \in \llbracket 1,n\rrbracket \setminus \{j\}} d_i
  .\]

  Cette condition est nécessaire car, si l'on a $d_j > \sum_{i \neq j} d_i$, alors on ne peut pas construire d'attribution pour la famille (il y a au plus $\sum_{i \neq j} d_i$ échanges de cadeaux, et il en faudrait plus pour en attribuer à toute la famille $j$).

  Pour montrer que c'est une condition suffisante, on propose un algorithme (en l'occurrence l'algorithme~\ref{algo:2-3}) qui maintient cette condition comme invariant et qui donne une attribution.

  De manière équivalente, la condition s'écrit \[
    (\star) :  \quad 2 \max_{i \in \llbracket 1,n\rrbracket} d_i \le \sum_{i=1}^n d_i
  .\] 


  \chapter{Algorithme pour les cycles courts.}

  Pour cet algorithme, on utilise une structure de données un peu particulière : un tableau de piles (où les piles sont implémentées avec des listes chaînées).
  La structure de donnée ressemble donc à un rideau de perles.

  \begin{figure}[H]
    \centering
    \begin{tikzpicture}[node distance=8]
      \node[draw, rectangle, scale=1.5] (x1) {};
      \node[draw, rectangle, scale=1.5, left of=x1] (x2) {};
      \node[draw, rectangle, scale=1.5, left of=x2] (x3) {};
      \node[draw, rectangle, scale=1.5, left of=x3] (x4) {};
      \node[draw, rectangle, scale=1.5, left of=x4] (x5) {};
      \node[draw, rectangle, scale=1.5, left of=x5] (x6) {};
      \node[draw, rectangle, scale=1.5, left of=x6] (x7) {};
      \node[draw, rectangle, scale=1.5, left of=x7] (x8) {};
      \node[draw, rectangle, scale=1.5, left of=x8] (x9) {};
      \node[draw, rectangle, scale=1.5, left of=x9] (x10) {};
      \node[draw, rectangle, scale=1.5, left of=x10] (x11) {};
      \node[draw, rectangle, scale=1.5, left of=x11] (x12) {};
      \node[draw, rectangle, scale=1.5, left of=x12] (x13) {};
      \node[draw, rectangle, scale=1.5, left of=x13] (x14) {};
      \node[draw, rectangle, scale=1.5, left of=x14] (x15) {};
      \node[draw, rectangle, scale=1.5, left of=x15] (x16) {};
      \node[draw, rectangle, scale=1.5, left of=x16] (x17) {};
      \node[draw, rectangle, scale=1.5, left of=x17] (x18) {};
      \node[draw, rectangle, scale=1.5, left of=x18] (x19) {};
      \node[draw, rectangle, scale=1.5, left of=x19] (x20) {};
      \node[draw, rectangle, scale=1.5, left of=x20] (x21) {};
      \node[draw, rectangle, scale=1.5, left of=x21] (x22) {};
      \node[draw, rectangle, scale=1.5, left of=x22] (x23) {};
      \node[draw, rectangle, scale=1.5, left of=x23] (x24) {};
      \node[draw, rectangle, scale=1.5, left of=x24] (x25) {};
      \node[draw, rectangle, scale=1.5, left of=x25] (x26) {};
      \node[draw, rectangle, scale=1.5, left of=x26] (x27) {};
      \node[draw, rectangle, scale=1.5, left of=x27] (x28) {};
      \node[draw, rectangle, scale=1.5, left of=x28] (x29) {};
      \node[draw, rectangle, scale=1.5, left of=x29] (x30) {};
      \node[draw, circle, scale=0.6, below=0.2em of x2] (x2a1) {};
      \node[draw, circle, scale=0.6, below=0.2em of x2a1] (x2a2) {};
      \node[draw, circle, scale=0.6, below=0.2em of x3] (x3a1) {};
      \node[draw, circle, scale=0.6, below=0.2em of x5] (x5a1) {};
      \node[draw, circle, scale=0.6, below=0.2em of x5a1] (x5a2) {};
      \node[draw, circle, scale=0.6, below=0.2em of x6] (x6a1) {};
      \node[draw, circle, scale=0.6, below=0.2em of x8] (x8a1) {};
      \node[draw, circle, scale=0.6, below=0.2em of x9] (x9a1) {};
      \node[draw, circle, scale=0.6, below=0.2em of x9a1] (x9a2) {};
      \node[draw, circle, scale=0.6, below=0.2em of x9a2] (x9a3) {};
      \node[draw, circle, scale=0.6, below=0.2em of x15] (x15a1) {};
      \node[draw, circle, scale=0.6, below=0.2em of x16] (x16a1) {};
      \node[draw, circle, scale=0.6, below=0.2em of x16a1] (x16a2) {};
      \node[draw, circle, scale=0.6, below=0.2em of x17] (x17a1) {};
      \node[draw, circle, scale=0.6, below=0.2em of x17a1] (x17a2) {};
      \node[draw, circle, scale=0.6, below=0.2em of x17a2] (x17a3) {};
      \node[draw, circle, scale=0.6, below=0.2em of x17a3] (x17a4) {};
      \node[draw, circle, scale=0.6, below=0.2em of x17a4] (x17a5) {};
      \node[draw, circle, scale=0.6, below=0.2em of x19] (x19a1) {};
      \node[draw, circle, scale=0.6, below=0.2em of x19a1] (x19a2) {};
      \node[draw, circle, scale=0.6, below=0.2em of x22] (x22a1) {};
      \node[draw, circle, scale=0.6, below=0.2em of x22a1] (x22a2) {};
      \node[draw, circle, scale=0.6, below=0.2em of x24] (x24a1) {};
      \node[draw, circle, scale=0.6, below=0.2em of x26] (x26a1) {};
      \node[draw, circle, scale=0.6, below=0.2em of x26a1] (x26a2) {};
      \node[draw, circle, scale=0.6, below=0.2em of x27] (x27a1) {};
      \node[draw, circle, scale=0.6, below=0.2em of x28] (x28a1) {};
      \node[draw, circle, scale=0.6, below=0.2em of x28a1] (x28a2) {};
      \node[draw, circle, scale=0.6, below=0.2em of x28a2] (x28a3) {};
      \node[draw, circle, scale=0.6, below=0.2em of x29] (x29a1) {};
      \draw (x2) -- (x2a1);
      \draw (x2a1) -- (x2a2);
      \draw (x3) -- (x3a1);
      \draw (x5) -- (x5a1);
      \draw (x5a1) -- (x5a2);
      \draw (x6) -- (x6a1);
      \draw (x8) -- (x8a1);
      \draw (x9) -- (x9a1);
      \draw (x9a1) -- (x9a2);
      \draw (x9a2) -- (x9a3);
      \draw (x15) -- (x15a1);
      \draw (x16) -- (x16a1);
      \draw (x16a1) -- (x16a2);
      \draw (x17) -- (x17a1);
      \draw (x17a1) -- (x17a2);
      \draw (x17a2) -- (x17a3);
      \draw (x17a3) -- (x17a4);
      \draw (x17a4) -- (x17a5);
      \draw (x19) -- (x19a1);
      \draw (x19a1) -- (x19a2);
      \draw (x22) -- (x22a1);
      \draw (x22a1) -- (x22a2);
      \draw (x24) -- (x24a1);
      \draw (x26) -- (x26a1);
      \draw (x26a1) -- (x26a2);
      \draw (x27) -- (x27a1);
      \draw (x28) -- (x28a1);
      \draw (x28a1) -- (x28a2);
      \draw (x28a2) -- (x28a3);
      \draw (x29) -- (x29a1);
    \end{tikzpicture}
    \caption{Structure "rideau de perles"\footnote{Après le cours sur les tables de hachages, la structure "rideau de perles" est en réalité une simple table de hachage.}}
  \end{figure}
  \showfootnote

  L'initialisation d'une telle structure (chaque pile étant initialement vide) se fait en $\mathrm{O}(N)$ où $N$ est la taille du tableau que l'on veut allouer.
  Pour simplifier, on supposera que le tableau est indicé à partir de $0$.

  En l'occurrence, on utilise une telle structure pour catégoriser les différentes familles en fonction du nombre de personnes respectives. J'appellerai cardinal d'une famille le nombre de personnes dans cette famille.
  Sur cette structure, on garde deux indices : les indices (donc ici les cardinaux) des deux plus grandes familles considérés.

  \begin{algorithm}[H]
    \centering
    \begin{algorithmic}[1]
      \Statex \Comment {Initialisation et précalculs}
      \State $N \gets \sum_{i = 1}^n d_i$
      \State On initialise $R$ un rideau de perles de taille $N + 1$.
      \State $m, m' \gets 0, 0$
      \For{$i = 1$ à $n$}
      \State $R[d_i]$.push($i$)
      \State $m, m' \gets$ les deux plus grands éléments de  $[m, m', d_i]$\footnote{Ce n'est pas un ensemble ici mais bien une liste de $3$ éléments. De plus, on choisit bien $m \ge m'$}
      \EndFor

      \vspace{0.5em}
      \Statex \Comment {Vérification de faisabilité}

      \If{$N < 2m$}
      \State \Return Impossible de trouver une attribution \label{algo:2-3-line-check}
      \EndIf

      \vspace{0.5em}
      \Statex \Comment {Cas de $N$ impair}

      \If{$N$ est impair}
      \State $i \gets R[m]$.pop()
      \State $j \gets R[m']$.pop()
      \State On choisit $k$ différent de $i$ et $j$. 
      \State On choisit $p_i$ une personne non-attribuée de la famille $i$.
      \State On choisit $p_j$ une personne non-attribuée de la famille $j$.
      \State On choisit $p_k$ une personne non-attribuée de la famille $k$.
      \State On attribue $p_i \to p_j \to p_k \to p_i$.
      \State $R[m - 1]$.push($i$)
      \State $R[m' - 1]$.push($j$) \label{algo:2-3-odd}
      \If{$R[m]$ est vide} $m \gets m - 1$ \EndIf
      \If{$R[m']$ est vide} $m' \gets m' - 1$ \EndIf
      \EndIf

      \vspace{0.5em}
      \Statex \Comment {Partie gloutonne avec structure de donnée}

      \While{$m > 0$} \label{algo:2-3-while-loop}
      \State $i \gets R[m]$.pop()
      \State $j \gets R[m']$.pop()
      \State On choisit $p_i$ une personne non-attribuée de la famille $i$.
      \State On choisit $p_j$ une personne non-attribuée de la famille $j$.
      \State On attribue $p_i \rightleftarrows p_j$.

      \State $R[m - 1]$.push($i$)
      \State $R[m' - 1]$.push($j$)

      \If{$R[m]$ est vide} $m \gets m - 1$ \EndIf
      \If{$R[m']$ est vide} $m' \gets m' - 1$ \EndIf
      \EndWhile
    \end{algorithmic}
    \caption{Attribution avec des $2$- ou des $3$-cycles}
    \label{algo:2-3}
  \end{algorithm}
  \showfootnote

  Afin de simplifier les explications, on suppose que, lorsque quelqu'un a eu une attribution, il disparait de la famille.
  Ainsi, le cardinal d'une famille $i$, que l'on notera $d'_i$, correspond maintenant au nombre de personnes n'ayant pas été attribuées.

  \textbf{Théorème.} \textsl{L'algorithme~\ref{algo:2-3} calcule en $\mathrm{O}(N)$, où $N = \sum_{i=1}^n d_i$, une attribution "Noël canadien" maximisant le nombre de cycles, si cela est possible, ou dit "impossible" si cela n'est pas possible.}

  \textit{Preuve.} L'algorithme~\ref{algo:2-3} maintient quelques invariants :
  \begin{enumerate}
    \item pour le rideau de perles, on a que $R[i]$ contient des familles de cardinal $i$ ;
    \item une fois la ligne~\ref{algo:2-3-line-check} passée, on a toujours $\sum_{i=1}^n d'_i \ge 2 m$ ;
    \item une fois la ligne~\ref{algo:2-3-odd} passée, on a toujours $\sum_{i=1}^n d'_i$ pair ;
    \item $m, m'$ correspondent aux deux plus grands cardinaux de familles.
  \end{enumerate}

  On montre les invariants ci-dessus.
  \begin{enumerate}
    \item Initialement, c'est vrai, puis après les ajouts de la partie "initialisation", on a toujours que la famille $i$ est de cardinal $d_i$, et enfin, on attribue a exactement un élément, donc le cardinal décroit de $1$.
    \item Après~\ref{algo:2-3-line-check}, on a bien l'inégalité, et elle est conservée car, on décroit la somme de $2$ (car deux attributions), et la valeur de $m$ de un, sauf dans le cas où $m$ reste fixée, ce qui correspond à $m = m'$, et dans ce cas, on a bien $\sum_{i=1}^n d_i \ge 2m + 2$ (on a au moins $2$ de leste dans l'inégalité), d'où la conservation.
    \item On décroît à chaque fois de $2$, et c'est vrai après la ligne~\ref{algo:2-3-odd}.
    \item On remplit bien $R$ en maintenant cette propriété, et on la conserve bien à chaque tour de boucle.
  \end{enumerate}

  On n'attribue jamais un cadeau à deux personnes de la même famille car on retire la famille du rideau de perles avant de prendre la famille suivante.
  Les attributions en $2$-cycles et un éventuel $3$-cycle sont optimales (dans le but de minimiser la taille des cycles), en effet on devrait avoir des $1$-cycles si l'on voulait faire plus court, mais cela n'est pas autorisé.

  Et, il n'y a pas de personne n'ayant pas été attribuée, car, à la fin de l'algorithme, $m = \max_{i \in \llbracket 1,n\rrbracket  } d'_i = 0$, donc toutes les familles ont été complètement attribuée.

  On en conclut que l'algorithme fournit une attribution lorsque la condition $(\star)$ est vérifiée.
  Ceci justifie également que la condition $(\star)$ est suffisante.

  Avant la boucle de la ligne~\ref{algo:2-3-while-loop}, la complexité est en $\mathrm{O}(N)$.
  Et, la boucle réalise au plus~$N / 2$ tours de boucles, car on réalise au plus~$(N / 2) \times 2$ attributions (on néglige le $3$-cycle, car il n'a lieu qu'une fois).
  D'où une complexité en $\mathrm{O}(N)$.
  \hfill $\square$

  \chapter{Algorithme pour l'unique cycle long.}

  On part de la solution en $2$- et $3$-cycles donné par l'algorithme précédent, et on se donne un algorithme permettant d'agrandir un cycle en ajoutant un $2$-cycle dedans.
  Ainsi, on commence, soit par un $2$-cycle choisi au hasard, soit par l'éventuel $3$-cycle créé (si $N$ est impair).

  \begin{figure}[H]
    \centering
    \begin{tikzpicture}
      \begin{scope}
        \node[draw, circle] (a) {A};
        \node[draw, circle, right of=a] (b) {B};
        \node[draw, circle, below of=a] (c) {C};
        \node[draw, circle, right of=c] (d) {D};

        \node[above of=a] (a0) {};
        \node[above of=b] (b0) {};

        \draw[decorate, decoration=snake, -latex] (b) to[bend right] (b0) to[bend right] (a0) to[bend right] (a);

        \draw[-latex, thick] (a) to (b);
        \draw[<->, thick] (c) to (d);
      \end{scope}

      \begin{scope}[shift={(5,0)}]
        \node[draw, circle] (a) {A};
        \node[draw, circle, right of=a] (b) {B};
        \node[draw, circle, below of=a] (c) {C};
        \node[draw, circle, right of=c] (d) {D};

        \node[above of=a] (a0) {};
        \node[above of=b] (b0) {};

        \draw[decorate, decoration=snake, -latex] (b) to[bend right] (b0) to[bend right] (a0) to[bend right] (a);

        \draw[-latex, thick] (a) to (d);
        \draw[-latex, thick] (d) to (c);
        \draw[-latex, thick] (c) to (b);
      \end{scope}
    \end{tikzpicture}
    \caption{Agrandir un cycle}
  \end{figure}

  Dans la figure ci-dessus, on distingue les attributions \tikz \draw[->, thick] (0, 0) to (0.5, 0);, qui sont celles donné par l'algorithme précédent, et les attributions \tikz[baseline=-2] \draw[-latex, thick] (0, 0) to (0.5, 0);, qui sont celles générées par l'algorithme de construction de cycles.

  Il est important de justifier la construction.
  Comme on considère des $2$-cycles, on peut inverser les deux personnes comme on le veut sans casser l'attribution.
  Ainsi, quitte à permuter C et D, on peut s'assurer que A n'est pas dans la même famille que D, et que C et B ne sont pas dans la même famille non plus.
  En effet, si avec les deux permutations on ne peut pas garantir cette condition, c'est que A, B, C et D sont dans la même famille.
  Mais, c'est impossible car notre algorithme maintient l'invariant que l'on conserve une attribution valide.

  \begin{algorithm}
    \centering
    \begin{algorithmic}[1]
      \State On exécute l'algorithme~\ref{algo:2-3}.

      \State On considère $C$ le premier cycle produit par l'algorithme~\ref{algo:2-3}. C'est un 2- ou un 3-cycle.

      \Statex

      \While{il existe un $2$-cycle non considéré}
      \State Soit $p_i \rightleftarrows p_j$ un tel $2$-cycle.
      \State On l'ajoute dans $C$ comme montré sur la figure précédente (en mettant à jour $C$ par la suite).
      \Statex \Comment{On doit donc faire la disjonction de cas avec les familles de $\mathrm{A}$, $\mathrm{B}$, $\mathrm{C}$ et $\mathrm{D}$. Mais, ça peut se faire en $\mathrm{O}(1)$ si l'on représente une personne comme un couple numéro de famille, numéro de personne dans la famille.}
      \EndWhile
    \end{algorithmic}
    \caption{Calcul d'un cycle de taille maximale}
    \label{algo:max}
  \end{algorithm}

  \textbf{Théorème.} \textsl{L'algorithme~\ref{algo:max} renvoie un cycle de longueur maximale (en supposant la condition nécessaire et suffisante précédente) avec une complexité en $\mathrm{O}(N)$ où $N$ est le nombre total de personnes.}

  \textit{Preuve}.
  On maintient comme invariant que l'on a une attribution valide.
  Le nombre de cycles dans le graphe diminue strictement, jusqu'à atteindre $1$ : on a donc un unique cycle à la fin de l'algorithme.
  En effet, ceci est assuré car on produit toujours $N / 2$ $2$-cycles, ou un $3$-cycle et $(N-3) / 2$ $2$-cycles, avec l'algorithme~\ref{algo:2-3}.
  On a donc moins de $N / 2$ tours de boucle, et on manipule des cycles comme des listes chaînées, donc le traitement dans la boucle se fait en $\mathrm{O}(1)$. (On suppose avoir une file de cycles pour récupérer les cycles produits par l'algorithme~\ref{algo:2-3}.)
  D'où la complexité en $\mathrm{O}(N)$.
  \hfill $\square$
\end{document}
