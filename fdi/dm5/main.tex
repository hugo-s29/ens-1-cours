\documentclass{../../notes}

\title{DM n°5 -- FDI}
\author{Hugo \scshape Salou}

\usepackage{multicol}

\def\algorithmicrequire{\textbf{Entrée}}
\def\algorithmicensure{\textbf{Sortie}}
\def\algorithmicend{\textbf{fin}}
\def\algorithmicif{\textbf{si}}
\def\algorithmicthen{\textbf{alors}}
\def\algorithmicelse{\textbf{sinon}}
\def\algorithmicelsif{\algorithmicelse\ \algorithmicif}
\def\algorithmicendif{\algorithmicend\ \algorithmicif}
\def\algorithmicfor{\textbf{pour}}
\def\algorithmicprocedure{\textbf{Procédure}}
\def\algorithmicfunction{\textbf{Fonction}}
\def\algorithmicforall{\textbf{pour tout}}
\def\algorithmicdo{\textbf{faire}}
\def\algorithmicendfor{\algorithmicend\ \algorithmicfor}
\def\algorithmicwhile{\textbf{tant que}}
\def\algorithmicendwhile{\algorithmicend\ \algorithmicwhile}
\def\algorithmicloop{\textbf{répéter indéfiniment}}
\def\algorithmicendloop{\algorithmicend\ \algorithmicloop}
\def\algorithmicrepeat{\textbf{répéter}}
\def\algorithmicuntil{\textbf{jusqu'à que}}
\def\algorithmicprint{\textbf{afficher}}
\def\algorithmicreturn{\textbf{retourner}}
\def\algorithmictrue{\textsc{vrai}}
\def\algorithmicfalse{\textsc{faux}}

\let\Sortie\Ensure
\let\Entree\Require
\floatname{algorithm}{\color{deepblue}Algorithme}

\begin{document}
  \maketitle

  \begin{enumerate}
    \item Considérons un ensemble fini $E$.
      On montre, par récurrence sur~$|E|$, que $E$ est récursif primitif.\label{q1}
      \begin{itemize}
        \item Dans le cas où $|E| = 0$, alors $E = \emptyset$ alors $\chi_E = \mathbf{0}$, où $\mathbf{0}$ est la fonction constante nulle.
        \item Considérons $|E| = n + 1$.
          Soit $x \in E$ et $E' := E \setminus \{x\}$.
          Par hypothèse de récurrence $\chi_{E'}$ est primitive récursive.
          En TD, on a vu que la construction "$\mathtt{if}\ldots \mathtt{then}\ldots \mathtt{else}$"\footnote{Pour la lisibilité, j'écrirais cette construction comme une commande, et non comme un appel à une fonction (c'est un petit abus de notations).}\showfootnote\ (avec $\mathtt{true}$ représenté par $1$ et $\mathtt{false}$ par $0$) pouvait être réalisée par une fonction primitive récursive, on s'appuie donc sur cette construction.
          De même, le test d'égalité peut être réalisé (il suffit de calculer les deux quasi-différences dans $\mathds{N}$ et de vérifier que les deux sont nulles).
          On a donc \[
            \chi_{E}(y) = \mathtt{if}\ \mathtt{eq}(x,y)\ \mathtt{then}\ \mathbf{1}(y)\ \mathtt{else}\ \chi_{E'}(y)
          .\]
      \end{itemize}
      Ainsi, tout ensemble fini est récursif primitif.

      En notant $E = \{x_1, \ldots, x_n\}$, le raisonnement par récurrence donne la fonction primitive récursive :
      \begin{align*}
        \chi_E(y) ={}& \mathtt{if}\ \mathtt{eq}(x_1, y)\ \mathtt{then}\ \mathbf{1}(x)\ \mathtt{else}\\
                     & \mathtt{if}\ \mathtt{eq}(x_2, y)\ \mathtt{then}\ \mathbf{1}(x)\ \mathtt{else}\\
                     & \mathtt{if}\ \mathtt{eq}(x_3, y)\ \mathtt{then}\ \mathbf{1}(x)\ \mathtt{else}\\
                     & \qquad \qquad \qquad \vdots \\
                     & \mathtt{if}\ \mathtt{eq}(x_n, y)\ \mathtt{then}\ \mathbf{1}(x)\ \mathtt{else}\\
                     & \mathbf{0}(x)
      .\end{align*}

      Montrons que $P_k$ est récursif primitif en montrant que $k\mathds{N}$ l'est.
      En TD, on a implémenté la fonction $\mathtt{div}$  calculant le quotient modulo un nombre précisé en argument, et $\mathtt{mult}$ calculant le produit.
      On pose donc \[
        \chi_{k\mathds{N}}(x) = \mathtt{eq}(\mathtt{mult}(\mathtt{div}(x, k), k), x)
      .\]
      En effet, on a \[
      x \in k \mathds{N} \iff k  \mid x \iff \left\lfloor \frac{x}{k} \right\rfloor \times k = x
      .\]

      (On aurait aussi pu passer par la fonction $\mathtt{mod}$ définie en TD.)
    \item Supposons $A$ et $B$ primitifs récursifs.
      Alors, les ensembles $A \cap B$, $A \cup B$ et $A^\mathrm{C}$ le sont.
      En effet, on a :
      \begin{itemize}
        \item $\chi_{A \cap B}(x) = \mathtt{mult}(\chi_A(x), \chi_B(x))$ ;
        \item $\chi_{A \cup B}(x) = \mathtt{sub}(\mathtt{add}(\chi_A(x), \chi_B(x)), \chi_{A \cap B}(x))$ ;
        \item $\chi_{A^\mathrm{C}}(x) = \mathtt{sub}(\mathbf{1}(x), \chi_A(x))$.
      \end{itemize}
      Ces relations proviennent des relations entre fonctions indicatrices :
      \begin{gather*}
        \chi_{A \cap B} = \chi_A \times \chi_B\\
        \chi_{A \cup B} = \chi_A + \chi_B - \chi_{A \cap B}\\
        \chi_{A^\mathrm{C}} = 1 - \chi_A
      \end{gather*}

      On en conclut que les ensembles primitifs récursifs sont clos par les opérations ensemblistes $\cap$, $\cup$ et $-^\mathrm{C}$.
    \item On a 
      \begin{align*}
        g(x) = {}&\mathtt{if}\ \chi_{A_1}(x)\ \mathtt{then}\ f_1(x)\ \mathtt{else}\\
               &\mathtt{if}\ \chi_{A_2}(x)\ \mathtt{then}\ f_2(x)\ \mathtt{else}\\
               &\mathtt{if}\ \chi_{A_3}(x)\ \mathtt{then}\ f_3(x)\ \mathtt{else}\\
               & \qquad \qquad \qquad \vdots \\
               &\mathtt{if}\ \chi_{A_k}(x)\ \mathtt{then}\ f_k(x)\ \mathtt{else}\\
               & f(x)
      .\end{align*}
      (La fonction définie en question~\ref{q1} est un cas particulier où l'on a posé~$A_i = \{x_i\}$ (et donc $\chi_{A_i}(x) = \mathtt{eq}(x,x_i)$), $f_i = \mathbf{1}$ et $f = \mathbf{0}$, on retrouve ainsi la même fonction primitive récursive.)
      Formellement, on procède par récurrence sur $k$.
      \begin{itemize}
        \item Pour $k = 0$, on pose $g = f$, qui est primitive récursive.
        \item Pour $k > 0$, on considère $g'$ définie "cas par cas" (par hypothèse de récurrence) sur les $k$ cas~$((A_1, f_1), \ldots, (A_k, f_k))$, primitive récursive, et on a \[
          g(x) = \mathtt{if}\ \chi_{A_{k+1}}(x)\ \mathtt{then}\ f_{k+1}(x)\ \mathtt{else}\ g'(x)
          ,\]
          qui est aussi primitive récursive, d'où l'hérédité.
      \end{itemize}
      Ceci permet de conclure que les fonctions primitives récursives sont closes par schéma de définition par cas.
  \end{enumerate}
\end{document}

