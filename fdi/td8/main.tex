\documentclass{../../notes}

\title{TD n°8 -- FDI}
\author{Hugo \scshape Salou}

\tikzset{
  initial text=,
  node distance=2cm,
  auto,
  every state/.style={draw=deepblue,thick},
  every path/.style={draw=deepblue},
}

\begin{document}
  Considérons le langage \[
    L := \{\langle M_1, M_2 \rangle \in \Sigma^\star \mid \mathcal{L}(M_1) = \mathcal{L}(M_2) \}
  .\]

  Avant de commencer, on considère deux machines :
  \begin{itemize}
    \item $M_\emptyset$ une machine qui rejette toutes les entrées ;
    \item $M_{\Sigma^\star}$ une machine qui accepte toutes les entrées ;
    \item $M'_{M,w}$ est la machine  donc le code est :
  \end{itemize}
  On a donc \[
    \mathcal{L}(M_\emptyset) = \emptyset \quad\quad \text{et}\quad\quad \mathcal{L}(M_{\Sigma^\star}) = \Sigma^\star
  .\]

  \begin{itemize}
    \item Le langage $L$ est indécidable.
      En effet, on procède par réduction au langage $\textsc{Vide}_\mathsf{TM}$.
      La fonction \begin{align*}
        f:\Sigma^\star &\longrightarrow \Sigma^\star \\
        \langle M \rangle &\longmapsto \langle M, M_{\emptyset} \rangle
      \end{align*}
      est calculable et on a l'équivalence
      \begin{align*}
        \langle M, M_\emptyset\rangle \in L \iff& \mathcal{L}(M) = \mathcal{L}(M_\emptyset)\\
        \iff& \mathcal{L}(M) = \emptyset\\
        \iff& \langle M \rangle \in \textsc{Vide}_\mathsf{TM}
      .\end{align*}
      D'où l'indécidabilité.
    \item Le langage $L$ n'est pas récursivement énumérable.
      En effet, par l'absurde, supposons $L$ récursivement énumérable.
      Ainsi, $L$ est Turing-reconnaissable par une machine $R$.
      Le langage $\textsc{Vide}_\mathsf{TM}$ n'est pas Turing-reconnaissable.
      Cependant, la machine dont le code est ci-dessous reconnait $\textsc{Vide}_\mathsf{TM}$.
      \textit{\textbf{Absurde !}}
      \begin{itemize}
        \item On simule $R$ sur l'entrée $\langle M, M_\emptyset \rangle$.
        \item Si $R$ accepte, alors on accepte.
      \end{itemize}
  \end{itemize}
\end{document}
