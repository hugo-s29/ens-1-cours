\documentclass[../main]{subfiles}

\begin{document}
  \chapter{Sémantique opérationnelle des expressions arithmétiques avec déclarations locales ($\mathsf{LEA}$).} \label{thprog-chap05}
  \minitoc

  On suppose donnés $\mathds{Z}$ les entiers relatifs et $\mathcal{V}$ un ensemble infini de variables (d'identifiants/d'identificateurs/de noms).

  On définit $\mathsf{LEA}$ par la grammaire suivante :
  \[
  a ::= \ubar k  \mid a_1 \oplus a_2  \mid\letin x {a_1} {a_2} \mid  x
  ,\]
  où $x \in \mathcal{V}$ et $k \in \mathds{Z}$.

  En Rocq, on peut définir :
  \begin{lstlisting}[language=Coq,caption=Définition inductive de $\mathsf{LEA}$]
Inductive $\mathsf{LEA}$ : Set :=
| $\mathtt{Cst}$ : $\mathds{Z}$ -> $\mathsf{LEA}$
| $\mathtt{Add}$ : $\mathsf{LEA}$ -> $\mathsf{LEA}$ -> $\mathsf{LEA}$
| $\mathtt{Let}$ : $\mathcal{V}$ -> $\mathsf{LEA}$ -> $\mathsf{LEA}$ -> $\mathsf{LEA}$
| $\mathtt{Var}$ : $\mathcal{V}$ -> $\mathsf{LEA}$.
  \end{lstlisting}

  \begin{exm}
    Voici quelques exemples d'expressions avec déclarations locales :
    \begin{enumerate}
      \item $\letin x 3 x \oplus x$ ;
      \item $\letin x {\ubar 2} \letin y {x \oplus \ubar 2} x \oplus y$ ;
      \item $\letin x {(\letin y {\ubar 5} y \oplus y)} (\letin z {\ubar 6} z \oplus \ubar 2) \oplus x $ ;
      \item $\letin {\tikzmarknode{xodef}x} {\ubar 7 \oplus \ubar 2} (\letin {\tikzmarknode{xidef}x} 5 {\tikzmarknode{xiused1}x \oplus \tikzmarknode{xiused2}x}) \oplus \tikzmarknode{xoused}x$.
    \begin{tikzpicture}[overlay, remember picture,shorten >= 5pt,shorten <= 5pt,]
      \draw[->] (xidef.south) to[bend right=15] (xiused1.south);
      \draw[->] (xidef.south) to[bend right=15] (xiused2.south);
      \draw[->] (xodef.south) to[bend right=15] (xoused.south);
    \end{tikzpicture}
    \end{enumerate}
    \vspace{0.5cm}
  \end{exm}

  \section{Sémantique à grands pas sur $\mathsf{LEA}$.}
  On définit une relation d'\textit{évaluation} ${\Downarrow} : \mathsf{LEA} * \mathds{Z}$\footnote{On surcharge encore les notations.}
  définie par :
  \[
  \begin{prooftree}
    \infer 0{\ubar k \Downarrow k}
  \end{prooftree}
  \quad\quad
  \begin{prooftree}
    \hypo{a_1\Downarrow k_1}
    \hypo{a_2\Downarrow k_2}
    \infer[left label={k=k_1+k_2}] 2{a_1 \oplus a_2 \Downarrow k}
  \end{prooftree}
  ,\] et on ajoute une règle pour le $\letin x \ldots \ldots$ :
  \[
  \begin{prooftree}
    \hypo{a_1 \Downarrow k_1}
    \hypo{a_2\left[\sfrac{\ubar{k}_1}{x}\right] \Downarrow k_1}
    \infer 2{(\letin x {a_1} {a_2}) \Downarrow k_2}
  \end{prooftree}
  .\]
  On note ici $a[\sfrac{\ubar{k}}{x}]$ la substitution de $\ubar k$ à la place de $x$ dans l'expression $a$.
  Ceci sera défini après.

  Attention : on n'a pas de règles de la forme \[
    \cancel{
      \begin{prooftree}
        \infer 0{x \Downarrow \redQuestionBox}
      \end{prooftree}
    }
  ,\]
  les variables sont censées disparaitre avant qu'on arrivent à elles.

  \begin{defn}
    Soit $a \in \mathsf{LEA}$.
    L'ensemble des \textit{variables libres} d'une expression $a$ noté $\Vl(a)$, et est défini par induction sur $a$ de la manière suivante :
    \begin{itemize}
      \item $\Vl(\ubar k) = \emptyset$ ;
      \item $\Vl(x) = \{x\}$ ;
      \item $\Vl(a_1 \oplus a_2) = \Vl(a_1) \cup \Vl(a_2)$ ;
      \item $\Vl(\letin x {a_1} {a_2}) = \Vl(a_1) \cup (\Vl(a_2) \setminus \{x\})$.
    \end{itemize}
  \end{defn}

  \begin{exm}
    \[
      \Vl(\letin {\tikzmarknode{xdef}x} {\ubar3} \letin {\tikzmarknode{ydef}y} {\tikzmarknode{xused}x \oplus \ubar2} \tikzmarknode{yused}y \oplus (z \oplus \underline{15})) = \{z\} 
    .\] 
    \begin{tikzpicture}[overlay, remember picture,shorten >= 5pt,shorten <= 5pt,]
      \draw[<-] (xdef.south) to[bend right] (xused.south);
      \draw[<-] (ydef.north) to[bend left] (yused.north);
    \end{tikzpicture}
  \end{exm}

  \begin{defn}
    Une expression $a \in \mathsf{LEA}$ est \textit{close} si $\Vl(a) = \emptyset$.
    On note $\mathsf{LEA}_0 \subseteq \mathsf{LEA}$ l'ensemble des expressions arithmétiques de closes.
  \end{defn}

  \begin{defn}
    Soient $a \in \mathsf{LEA}$, $x \in \mathcal{V}$ et $k \in \mathds{Z}$.
    On définit par induction sur $a$ (\textit{quatre cas}) le résultat de la \textit{substitution} de $x$ par~$\ubar k$ dans $a$, noté $a[\sfrac{\ubar k}{x}]$ de la manière suivante :
    \begin{itemize}
      \item $\ubar{k'}[\sfrac{\ubar k}{x}] = \ubar{k'}$ ;
      \item $(a_1 \oplus a_2)[\sfrac{\ubar k}{x}] = (a_1[\sfrac{\ubar k}{x}]) \oplus (a_2[\sfrac{\ubar k}{x}])$ ;
      \item $y[\sfrac{\ubar k}{x}] = \begin{cases}
          \ubar k \quad& \text{si}\ x = y\\
          y \quad& \text{si}\ x \neq y\: ;
      \end{cases}$
    \item $(\letin y {a_1} {a_2})[\sfrac{\ubar k}{x}] = \begin{cases}
        \letin y {a_1[\sfrac{\ubar k}{x}]} a_2 & \text{si}\ x = y\\
        \letin y {a_1[\sfrac{\ubar k}{x}]} a_2[\sfrac{\ubar k}{x}] & \text{si}\ x \neq y.
    \end{cases}$
    \end{itemize}
  \end{defn}

  \section{Sémantique à petits pas sur $\mathsf{LEA}$.}

  On définit la relation ${\to} \subseteq \mathsf{LEA} * \mathsf{LEA}$ inductivement par :
  \[
  \begin{prooftree}
    \infer[left label={k = k_1 + k_2}] 0[\mathcal{A}]{\ubar{k}_1 \oplus \ubar{k}_2 \to \ubar{k}}
  \end{prooftree},
  \]\[
  \begin{prooftree}
    \hypo{a_2 \to a_2'}
    \infer 1[\mathcal{C}_\mathrm{d}]{a_1 \oplus a_2 \to a_1 \oplus a_2'}
  \end{prooftree}
  \quad\text{et}\quad
  \begin{prooftree}
    \hypo{a_1 \to a_1'}
    \infer 1[\mathcal{C}_\mathrm{g}]{a_1 \oplus \ubar k \to a_1' \oplus \ubar k}
  \end{prooftree}
  ,\] 
  puis les nouvelles règles pour le $\letin x \ldots \ldots$ :
  \[
  \begin{prooftree}
    \hypo{a_1 \to a_1'}
    \infer 1[\mathcal{C}_\mathrm{l}]{\letin x {a_1} {a_2} \to \letin x {a_1'} {a_2}}
  \end{prooftree}
  \] 
  \[
  \begin{prooftree}
    \infer 0{\letin x {\ubar k} a \to a[\sfrac{\ubar k}{x}]}
  \end{prooftree}
  .\]

  On peut démontrer l'équivalence des sémantiques à grands pas et à petits pas.

  \section{Sémantique contextuelle pour $\mathsf{LEA}$.}

  On définit les contextes d'évaluations par la grammaire suivante :
  \begin{align*}
    E ::= \ &\ [\;]\\
          \mid&\ a \oplus E\\
          \mid&\ E \oplus \ubar k\\
          \mid&\ \letin x E a
  .\end{align*}
  % À FAIRE : Refaire la présentation de cette grammaire

  On définit \textit{deux} relations $\mapsto_\mathrm{a}$ et $\mathrel{\color{deepblue}\mapsto}$ par les règles :
  \[
  \begin{prooftree}
    \infer[left label={k = k_1 + k_2}] 0{\ubar k_1 \oplus \ubar k_2 \mapsto_\mathrm{a} k_2}
  \end{prooftree}
  \quad\quad
  \begin{prooftree}
    \infer 0 {\letin x {\ubar k} a \mapsto_\mathrm{a} a[\sfrac{\ubar k}{x}]}
  \end{prooftree}
  ,\]
  et \[
  \begin{prooftree}
    \hypo{a \mapsto_\mathrm{a} a'}
    \infer 1{E[a] \mathrel{\color{deepblue}\mapsto} E[a']}
  \end{prooftree}
  .\] 


  \section{Sémantique sur $\mathsf{LEA}$ avec environnement.}

  \begin{defn}
    Soient $A$ et $B$ deux ensembles.
    Un \textit{dictionnaire} sur $(A,B)$ est une fonction partielle à domaine fini de $A$ dans $B$.

    Si $D$ est un dictionnaire sur $(A,B)$, on note $D(x) = y$ lorsque $D$ associe $y \in B$ à $x \in A$.

    Le domaine d'un dictionnaire $D$ est \[
    \mathrm{dom}(D) = \{x \in A \mid \exists y \in B, D(x) = y \} 
    .\]

    On note $\emptyset$ le dictionnaire vide.

    Pour un dictionnaire $D$ sur $(A,B)$, deux éléments $x \in A$ et $y \in B$, on note $D[x \mapsto y]$ est le dictionnaire $D'$ défini par 
    \begin{itemize}
      \item $D'(x) = y$ ;
      \item $D'(z) = D(z)$ pour $z \in \mathrm{dom}(D)$ tel que $z \neq x$.
    \end{itemize}
  \end{defn}

  On ne s'intéresse pas à la construction d'un tel type de donné, mais juste son utilisation.

  On se donne un ensemble $\mathrm{Env}$ d'\textit{environnements} notés $\mathcal{E}, \mathcal{E}', \ldots$ qui sont des dictionnaires sur $(\mathcal{V}, \mathds{Z})$.

  \subsection{Sémantique à grands pas sur $\mathsf{LEA}$ avec environnements.}

  On définit la relation ${\Downarrow} \subseteq \mathsf{LEA} * \mathrm{Env} * \mathds{Z}$, noté $a, \mathcal{E} \Downarrow k$ ("$a$ s'évalue en $k$ dans $\mathcal{E}$") défini par \[
  \begin{prooftree}
    \infer 0{\ubar k, \mathcal{E} \Downarrow k}
  \end{prooftree}
  \quad\quad
  \begin{prooftree}
    \hypo{a_1, \mathcal{E} \Downarrow k_1}
    \hypo{a_2, \mathcal{E} \Downarrow k_2}
    \infer[left label={k = k_1 + k_2}] 2{a_1 \oplus a_2, \mathcal{E} \Downarrow k}
  \end{prooftree}
  \quad\quad
  \begin{prooftree}
    \infer[left label={\mathcal{E}(x) = k}] 0{x, \mathcal{E} \Downarrow k}
  \end{prooftree}
  ,\] 
  \[
  \begin{prooftree}
    \hypo{a_1, \mathcal{E} \Downarrow k_1}
    \hypo{a_2, \mathcal{E}[x \mapsto k_1] \Downarrow k_2}
    \infer 2{\letin x {a_1} {a_2} \Downarrow k_2}
  \end{prooftree}
  .\]

  \begin{rmk}
    \begin{itemize}
      \item Dans cette définition, on n'a pas de substitutions (c'est donc plus facile à calculer).
      \item Si $\Vl(a) \subseteq \mathrm{dom}(\mathcal{E})$, alors il existe $k \in \mathds{Z}$ tel que $a, \mathcal{E} \Downarrow k$.
      \item On a $a\Downarrow k$ (sans environnement) si, et seulement si $a, \emptyset \Downarrow k$ (avec environnement).
    \end{itemize}
  \end{rmk}

  Pour les petits pas avec environnements, c'est un peu plus compliqué\ldots On verra ça en TD. (Écraser les valeurs dans un dictionnaire, ça peut être problématique avec les petits pas.)
\end{document}
