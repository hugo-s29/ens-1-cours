\documentclass[../main]{subfiles}

\begin{document}
  \chapter{Les bases de Rocq.} {
  \lstset{language=Coq}
  \section{Les définitions par induction : \lstinline|Inductive|.}

  En Rocq (anciennement Coq), on peut définir des ensembles par induction.
  Pour cela, on utilise le mot \lstinline|Inductive|.

  Par exemple, pour définir un type de liste d'entiers, on utilise le code ci-dessous.

  \begin{lstlisting}[caption=Définition du type $\mathsf{nlist}$ en Rocq]
    Inductive $\mathsf{nlist}$ : Set:=
    | Nil : $\mathsf{nlist}$
    | Cons : $\mathsf{nat} \to \mathsf{nlist} \to \mathsf{nlist}$.
  \end{lstlisting}

  En Rocq, au lieu de définir la fonction \texttt{Cons} comme une "fonction" de la forme $\mathtt{Cons} : \mathsf{nat} * \mathsf{nlist} \to \mathsf{nlist}$, on la \textit{curryfie} en une "fonction" de la forme $\mathtt{Cons} : \mathsf{nat} \to \mathsf{nlist} \to \mathsf{nlist}$. Les types définis par les deux versions sont isomorphes.


  Pour définir une relation, on utilise aussi le mot clé \lstinline|Inductive| :

  \begin{lstlisting}[caption=Définition de la relation $\mathsf{le}$]
    Inductive $\mathsf{le}$ : $\mathsf{nat}$ -> $\mathsf{nat}$ -> Prop :=
    | $\mathsf{le}$_refl : forall n, $\mathsf{le}$ n n
    | $\mathsf{le}$_S : forall n k, $\mathsf{le}$ n k -> $\mathsf{le}$ (S n) (S k).
  \end{lstlisting}

  Aux types définis par induction, on associe un principe d'induction (qu'on voir avec \lstinline|Print $\mathsf{le}$_ind.| ou   \lstinline|Print $\mathsf{nlist}$_ind.|).
  Ce principe d'induction permet de démontrer une propriété $\mathcal{P}$ sur un ensemble/une relation définie par induction.

  \section{Quelques preuves avec Rocq.}

  On décide de prouver le lemme suivant avec Rocq.

  \begin{not-lem}
    Soit $\ell$ une liste triée, et soient $a$ et $b$ deux entiers tels que~$a \le b$.
    Alors la liste $a :: b :: \ell$ est triée.
  \end{not-lem}

  Pour cela, on écrit en Rocq :
  \begin{lstlisting}
    Lemma exemple_triee :
      forall l, $\mathsf{tri\acute{e}e}$ l ->
        forall a b, $\mathsf{le}$ a b ->
          $\mathsf{tri\acute{e}e}$ (Cons a (Cons b l)).
  \end{lstlisting}

  Il ne reste plus qu'à prouver ce lemme.
  On commence la démonstration par introduire les variables et hypothèses : les variables \texttt{l}, \texttt{a}, \texttt{b}, et les hypothèses $(\mathtt{H1}) : \mathsf{tri\acute{e}e}\ \mathtt{l}$, et $(\mathtt{H2}) : \mathsf{le}\ \mathtt{a}\ \mathtt{b}$.
  On commence par introduire la liste \texttt{l} et l'hypothèse \texttt{H1} et on s'occupera des autres un peu après.
  \begin{lstlisting}
    Proof.
      intros l H1.
  \end{lstlisting}
  On décide de réaliser une preuve par induction sur la relation $\mathsf{tri\acute{e}e}$, qui est en hypothèse $(\mathtt{H1})$.
  \begin{lstlisting}
      induction H1.
  \end{lstlisting}
  Dans le cas d'une preuve par induction sur $\mathsf{tri\acute{e}e}$, on a \textit{trois} cas.
  \begin{itemize}
    \item \textsl{Cas 1}. On se trouve dans le cas $\mathtt{l} = \mathtt{Nil}$.
      Pas trop de problèmes pour prouver que  \texttt{[a;b]} est triée avec l'hypothèse $\mathtt{a} \le  \mathtt{b}$.
      On introduit les variables et hypothèses \texttt{a}, \texttt{b} et \texttt{H2}.
      \begin{lstlisting}
  - intros a b H2.
      \end{lstlisting}
      À ce moment de la preuve, l'objectif est de montrer :
      \[
      \mathsf{tri\acute{e}e}\ \mathtt{Cons}(\mathtt{a}, \mathtt{Cons}(\mathtt{b}, \mathtt{Nil}))
      .\]
      Pour cela, on utilise deux fois les propriétés de la relation $\mathsf{tri\acute{e}e}$ :
      \begin{lstlisting}
    apply t_cons.
    apply t_singl.
      \end{lstlisting}
      Notre objectif a changé, on doit maintenant démontrer $\mathsf{le}\ \mathtt{a}\ \mathtt{b}$.
      C'est une de nos hypothèses, on peut donc utiliser :
      \begin{lstlisting}
    assumption.
      \end{lstlisting}
      Ceci termine le cas 1.
    \item \textsl{Cas 2}. On se trouve dans le cas $\mathtt{l} = \texttt{[k]}$.
      On doit de démontrer que la liste \texttt{[a;b;k]} est triée.
      On a l'hypothèse $\mathtt{a} \le \mathtt{b}$, mais aucune hypothèse de la forme $\mathtt{b} \le \mathtt{k}$.
      On est un peu coincé pour ce cas\ldots
  \end{itemize}
  }


  (Un jour je finirai d'écrire cette partie\ldots\
  Malheureusement, ce n'est pas aujourd'hui\ldots)
\end{document}
