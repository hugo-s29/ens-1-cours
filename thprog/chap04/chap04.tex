% FELA DE LA YASIIN MARVIN PTCQ THE PARACYDE
\documentclass[../main]{subfiles}

\begin{document}
  \chapter{Sémantique opérationnelle pour les expressions arithmétiques simples ($\mathsf{EA}$).} \label{thprog-chap04}
  \minitoc

  Depuis le début du cours, on s'est intéressé à la \textit{méthode inductive}.
  On essaie d'appliquer cette méthode à "l'exécution" des "programmes".

  On définira un programme comme un ensemble inductif : un programme est donc une structure de donnée.
  L'exécution d'un programme sera décrit comme des relations inductives (essentiellement binaires) sur les programmes.
  Définir ces relations, cela s'appelle la \textit{sémantique opérationnelle}.

  On considèrera deux sémantiques opérationnelles
  \begin{itemize}
    \item la sémantique à grands pas, où l'on associe un résultat à un programme ;
    \item la sémantique à petits pas, où l'on associe un programme "un peu plus tard" à un programme.
  \end{itemize}

  Notre objectif, dans un premier temps, est de définir OCaml, ou plutôt un plus petit langage fonctionnel inclus dans OCaml.

  On se donne l'ensemble $\mathds{Z}$ (on le prend comme un postulat).
  On définit l'ensemble $\mathsf{EA}$ en Rocq par :
  \begin{lstlisting}[language=coq,caption=Définition des expressions arithmétiques simples]
    Inductive $\mathsf{EA}$ : Set :=
    | $\mathtt{Cst}$ : $\mathds{Z}$ -> $\mathsf{EA}$
    | $\mathtt{Add}$ : $\mathsf{EA}$ -> $\mathsf{EA}$ -> $\mathsf{EA}$.
  \end{lstlisting}

  \begin{note}
    On se donne $\mathds{Z}$ et on note $k \in \mathds{Z}$ (vu comme une métavariable).
    On définit (inductivement) l'ensemble $\mathsf{EA}$ des expressions arithmétiques, notées $a, a', a_1, \ldots$ par la grammaire
    \[
      a ::= \ubar{k}  \mid a_1 \oplus a_2
    .\] 
  \end{note}

  \begin{exm}
    L'expression $\ubar 1 \oplus (\ubar 3 \oplus \ubar 7)$ représente l'expression Rocq 
    \[
    \mathtt{Add}(\mathtt{Cst}\ 1, \mathtt{Add}\ (\mathtt{Cst}\ 3)\ (\mathtt{Cst}\ 7))
    ,\]
    que l'on peut représenter comme l'arbre de syntaxe\ldots
  \end{exm}

  \begin{rmk}
    Dans le but de définir un langage minimal, il n'y a donc pas d'intérêt à ajoute $\ominus$ et $\otimes$, représentant la soustraction et la multiplication.
  \end{rmk}

  \section{Sémantique à grands pas sur $\mathsf{EA}$.}

  On définit la sémantique opérationnelle à grands pas pour $\mathsf{EA}$.
  L'intuition est d'associer l'exécution d'un programme avec le résultat.
  On définit la relation d'évaluation ${\Downarrow} \subseteq \mathsf{EA} * \mathds{Z}$, avec une notation infixe, définie par les règles d'inférences suivantes :
  \[
  \begin{prooftree}
    \infer 0{\ubar k \Downarrow k}
  \end{prooftree}
  \quad \text{et} \quad
  \begin{prooftree}
    \hypo{a_1 \Downarrow k_1}
    \hypo{a_2 \Downarrow k_2}
    \infer 2{a_1 \oplus a_2 \Downarrow k}
  \end{prooftree}
  ,\]
  où, dans la seconde règle d'inférence, $k = k_1 + k_2$.
  Attention, le $+$ est la somme dans $\mathds{Z}$, c'est une opération \textit{externalisée}.
  Vu qu'on ne sait pas comment la somme a été définie dans $\mathds{Z}$ (on ne sait pas si elle est définie par induction/point fixe, ou pas du tout), on ne l'écrit pas dans la règle d'inférence.

  La forme générale des règles d'inférences est la suivante :
  \[
  \begin{prooftree}
    \hypo{P_1}
    \hypo{\ldots}
    \hypo{P_m}
    \infer[left label=\text{Cond. App.}] 3[\mathcal{R}_i]{C}
  \end{prooftree}
  \] 
  où l'on donne les conditions d'application (ou \textit{side condition} en anglais).
  Les $P_1,\ldots,P_m,C$ sont des relations inductives, mais les conditions d'applications \textbf{ne sont pas} forcément inductives.

  \begin{exm}
    \[
    \begin{prooftree}
      \hypo{\ubar 3 \Downarrow 3}
      \hypo{\ubar 2 \Downarrow 2}
      \hypo{\ubar 5 \Downarrow 5}
      \infer[left label={2 + 5 = 7}] 2{(\ubar 2 \oplus \ubar 5) \Downarrow 7}
      \infer[left label={3 + 7 = 10}] 2{\ubar 3 \oplus (\ubar 2 \oplus \ubar 5) \Downarrow 10}
    \end{prooftree}
    .\]
  \end{exm}

  \section{Sémantique à petits pas sur $\mathsf{EA}$.}

  On définit ensuite la sémantique opérationnelle à \textit{petits pas} pour $\mathsf{EA}$.
  L'intuition est de faire un pas exactement (la relation n'est donc pas réflexive) dans l'exécution d'un programme et, si possible, qu'elle soit déterministe.

  Une relation \textit{déterministe} (ou \textit{fonctionnelle}) est une relation $\mathcal{R}$ telle que, si $a \mathrel{\mathcal{R}} b$ et $a \mathrel{\mathcal{R}} c$ alors $b = c$.

  La relation de réduction ${\to} \subseteq \mathsf{EA} * \mathsf{EA}$, notée infixe, par les règles d'inférences
  suivantes 

  \[
  \begin{prooftree}
    \infer[left label={k = k_1 + k_2}] 0[\mathcal{A}]{\ubar{k}_1 \oplus \ubar{k}_2 \to \ubar{k}}
  \end{prooftree},
  \]\[
  \begin{prooftree}
    \hypo{a_2 \to a_2'}
    \infer 1[\mathcal{C}_\mathrm{d}]{a_1 \oplus a_2 \to a_1 \oplus a_2'}
  \end{prooftree}
  \quad\text{et}\quad
  \begin{prooftree}
    \hypo{a_1 \to a_1'}
    \infer 1[\mathcal{C}_\mathrm{g}]{a_1 \oplus \ubar k \to a_1' \oplus \ubar k}
  \end{prooftree}
  .\] 

  Il faut le comprendre par "quand c'est fini à droite, on passe à gauche".

  Les règles $\mathcal{C}_\mathrm{g}$ et $\mathcal{C}_\mathrm{d}$ sont nommées respectivement \textit{règle contextuelle droite} et \textit{règle contextuelle gauche}.
  Quand $a \to a'$, on dit que $a$ se \textit{réduit} à $a'$.

  \begin{rmk}
    La notation $\ubar k \not\to$ indique que, quelle que soit l'expression $a \in \mathsf{EA}$ ,on n'a pas $\ubar k \to a$.
    Les constantes ne peuvent pas être exécutées.
  \end{rmk}

  \begin{exo}
    Et si on ajoute la règle \[
    \begin{prooftree}
      \hypo{a_1 \to a_1'} \hypo{a_2 \to a_2'}
      \infer 2{a_1 \oplus a_2 \to a_1' \oplus a_2'}
    \end{prooftree}
    ,\]
    appelée \textit{réduction parallèle}, que se passe-t-il ?
  \end{exo}

  \begin{rmk}
    Il n'est pas possible de démontrer $\ubar 2 \oplus (\ubar 3 \oplus \ubar 4) \to \ubar 9$.
    En effet, on réalise \textit{deux} pas.
  \end{rmk}

  \section{Coïncidence entre grands pas et petits pas.}

  On définit la clôture réflexive et transitive d'une relation binaire $\mathcal{R}$ sur un ensemble $E$, notée $\mathcal{R}^\star$.
  On la définit par les règles d'inférences suivantes :
  \[
  \begin{prooftree}
    \infer 0{x \mathrel{\mathcal{R}^\star} x}
  \end{prooftree}
  \quad\text{et}\quad
  \begin{prooftree}
    \hypo{x \mathrel{\mathcal{R}} y}
    \hypo{y \mathrel{\mathcal{R}^\star} z}
    \infer 2{x \mathrel{\mathcal{R}^\star} z}
  \end{prooftree}
  .\]

  \begin{lem}\label{lem:petit-pas-trans-transitif}
    La relation $\mathcal{R}^\star$ est transitive.
  \end{lem}
  \begin{prv}
    On démontre \[
      \forall x,y \in E, \quad \text{si } x \mathrel{\mathcal{R}^\star} y \text{ alors } \underbrace{\forall z, \ y \mathrel{\mathcal{R}^\star} z \implies x \mathrel{\mathcal{R}^\star} z}_{\mathcal{P}(x,y)}
    \] par induction sur $x \mathrel{\mathcal{R}^\star} y$.
    Il y a \textit{deux} cas.
    \begin{itemize}
      \item \textsl{Réflexivité}. On a donc $x = y$ et, par hypothèse, $y \mathrel{\mathcal{R}^\star} z$.
      \item \textsl{Transitivité}.
        On sait que $x \mathrel{\mathcal{R}} a$ et $a \mathrel{\mathcal{R}^\star} y$.
        De plus, on a l'hypothèse d'induction \[\mathcal{P}(a,y) : \forall z, y \mathrel{\mathcal{R}^\star} z \implies a \mathrel{\mathcal{R}^\star} z.\]
        Montrons $\mathcal{P}(x,y)$.
        Soit $z$ tel que $y \mathrel{\mathcal{R}^\star} z$. Il faut donc montrer $x \mathrel{\mathcal{R}^\star} z$.
        On sait que $x \mathrel{\mathcal{R}} a$ et, par hypothèse d'induction, $a \mathrel{\mathcal{R}^\star} z$.
        Ceci nous donne $x \mathrel{\mathcal{R}^\star} z$ en appliquant la seconde règle d'inférence.
    \end{itemize}
  \end{prv}

  \begin{lem} \label{lem:petit-pas-trans-compose}
    Quelles que soient $a_2$ et $a_2'$, si $a_2 \to^\star a_2'$, alors pour tout $a_1$, on a $a_1 \oplus a_2 \to^\star a_1 \oplus a_2'$.
  \end{lem}
  \begin{prv}
    On procède par induction sur $a_2 \to^\star a_2'$.
    Il y a \textit{deux} cas.
    \begin{enumerate}
      \item On a $a_2' = a_2$.
        Il suffit donc de montrer que l'on a \[a_1 \oplus a_2 \to^\star a_1 \oplus a_2,\] ce qui est vrai par réflexivité.
      \item On sait que $a_2 \to a$ et $a \to^\star a_2'$.
        On sait de plus que \[
        \forall a_1, \quad a_1 \oplus a \to^\star a_1 \oplus a_2'
        \] par hypothèse d'induction.
        On veut montrer que \[
        \forall a_1, \quad a_1 \oplus a_2 \to^\star a_1 \oplus a_2'
        .\]
        On se donne $a_1$.
        On déduit de $a_2 \to a$ que $a_1 \oplus a_2 \to a_1 \oplus a$ par $\mathcal{C}_\mathrm{d}$.
        Par hypothèse d'induction, on a $a_1 \oplus a \to^\star a_1 \oplus a_2'$.
        Par la seconde règle d'inférence, on conclut.
    \end{enumerate}
  \end{prv}

  \begin{lem}\label{lem:petit-pas-trans-plus}
    Quelles que soient les expressions $a_1$ et $a_1'$, si $a_1 \to^\star a_1'$ alors, pour tout $k$, $a_1 \oplus \ubar k \to^\star a_1' \oplus \ubar k$.
    \qed
  \end{lem}

  Attention, le lemme précédent est faux si l'on remplace $\ubar k$ par une expression $a_2$.
  En effet, $a_2$ ne peut pas être "spectateur" du calcul de $a_1$.

  \begin{prop}
    Soient $a$ une expression et $k$ un entier.
    On a l'implication \[
    a \Downarrow k \implies a \to^\star \ubar k
    .\]
  \end{prop}

  \begin{prv}
    On le démontre par induction sur la relation $a\Downarrow k$.
    Il y a \textit{deux} cas.
    \begin{enumerate}
      \item Dans le cas $a = \ubar k$, alors on a bien $\ubar k \to^\star \ubar k$.
      \item On sait que $a_1 \Downarrow k_1$ et $a_2 \Downarrow k_2$, avec $k = k_1 + k_2$.
        On a également deux hypothèses d'induction : 
        \begin{itemize}
          \item $(H_1)$ : $a_1 \to^\star \ubar{k}_1$ ;
          \item $(H_2)$ : $a_2 \to^\star \ubar{k}_2$.
        \end{itemize}
        On veut montrer $a_1 \oplus a_2 \to^\star \ubar{k}$, ce que l'on peut faire par :
        \[
          a_1 \oplus a_2
          \xrightarrow{\smash{(H_2) + \text{lemme~\ref{lem:petit-pas-trans-compose}}}}^\star
          a_1 \oplus \ubar{k}_2
          \xrightarrow{\smash{(H_1) + \text{lemme~\ref{lem:petit-pas-trans-plus}}}}^\star
          \ubar{k}_1 \oplus \ubar{k}_2
          \xrightarrow{\mathcal{A}} \ubar k
        .\] 
    \end{enumerate}
  \end{prv}

  \begin{prop}
    Soient $a$ une expression et $k$ un entier.
    On a l'implication \[
    a \to^\star \ubar k \implies a \Downarrow k
    .\]
    \qed
  \end{prop}

  \section{L'ensemble $\mathsf{EA}$ avec des erreurs à l'exécution.}

  On exécute des programmes de $\mathsf{EA}$.
  On considère que $\ubar{k}_1 \oplus \ubar{k}_2$ s'évalue comme \[
    \frac{(k_1+k_2) \times k_2}{k_2}
  .\]
  Le cas $k_2 = 0$ est une situation d'erreur, une "\textit{\textbf{situation catastrophique}}".
  (C'est une convention : quand un ordinateur divise par zéro, il explose !)

  \subsection{Relation à grands pas.}

  On note encore $\Downarrow$ la relation d'évaluation sur $\mathsf{EA} * \mathds{Z}_\bot$, où l'on définit l'ensemble $\mathds{Z}_\bot  = \mathds{Z} \cup \{\bot\}$.
  Le symbole $\bot$ est utilisé pour représenter un cas d'erreur.

  Les règles d'inférences définissant $\Downarrow$ sont :
  \[
  \begin{prooftree}
    \infer 0{\ubar k \Downarrow k}
  \end{prooftree}
  \quad\quad
  \begin{prooftree}
    \hypo{a_1 \Downarrow k_1}
    \hypo{a_2 \Downarrow k_2}
    \infer[left label={\begin{array}{r}k = k_1 + k_2\\ k \neq 0\end{array}}] 2{a_1 \oplus a_2 \Downarrow k}
  \end{prooftree}
  \quad\quad
  \begin{prooftree}
    \hypo{a_1 \Downarrow k_1}
    \hypo{a_2 \Downarrow 0}
    \infer 2{a_1 \oplus a_2 \Downarrow \bot}
  \end{prooftree}
  ,\] 
  et les règles de propagation du $\bot$ :
  \[
  \begin{prooftree}
    \hypo{a_1 \Downarrow \bot}
    \hypo{(a_2 \Downarrow r)}
    \infer 2{a_1 \oplus a_2 \Downarrow \bot}
  \end{prooftree}
  \quad\quad
  \begin{prooftree}
    \hypo{(a_1 \Downarrow r)}
    \hypo{a_2 \Downarrow \bot}
    \infer 2{a_1 \oplus a_2 \Downarrow \bot}
  \end{prooftree}
  .\]

  \subsection{Relation à petits pas.}

  On (re)-définit la relation ${\to} \subseteq \mathsf{EA} * \mathsf{EA}_\bot$, où $\mathsf{EA}_\bot = \mathsf{EA} \cup \{\bot\}$, par les règles d'inférences 
  \[
  \begin{prooftree}
    \infer[left label={\begin{array}{r}k = k_1 + k_2\\ k_2 \neq 0\end{array}}] 0{\ubar{k}_1 \oplus \ubar{k}_2 \to \ubar k}
  \end{prooftree}
  \quad
  \begin{prooftree}
    \hypo{a_2 \to a_2'}
    \infer[left label=a_2 \neq \bot] 1{a_1 \oplus a_2 \to a_1 \oplus a_2'}
  \end{prooftree}
  \]
  \[
  \begin{prooftree}
    \hypo{a_1 \to a_1'}
    \infer[left label=a_1 \neq \bot] 1{a_1 \oplus \ubar k \to a_1' \oplus \ubar k}
  \end{prooftree}
  \quad\quad
  \begin{prooftree}
    \infer 0{\ubar{k}_1 \oplus \ubar{0} \to \bot}
  \end{prooftree}
  ,\] et les règles de propagation du $\bot$ :
  \[
  \begin{prooftree}
    \hypo{a_1 \to \bot}
    \infer 1{a_1 \oplus \ubar{k} \to \bot}
  \end{prooftree}
  \quad\text{et}\quad
  \begin{prooftree}
    \hypo{a_2 \to \bot}
    \infer 1{a_1 \oplus a_2 \to \bot}
  \end{prooftree}
  .\] 

  Pour démontrer l'équivalence des relations grand pas et petits pas, ça semble un peu plus compliqué\ldots

  \section{Sémantique contextuelle pour $\mathsf{EA}$.}
  
  On définit la relation ${\mapsto} : \mathsf{EA} \times \mathsf{EA}$ par la règle :
  \[
  \begin{prooftree}
    \infer[left label ={k=k_1 + k_2}] 0{E[\ubar{k}_1 \oplus \ubar{k}_2] \mapsto E[\ubar{k}]}
  \end{prooftree}
  ,\]
  où $E$ est un \textit{contexte d'évaluation} que l'on peut définir par la grammaire \[
    E ::= [\;]  \mid \redQuestionBox
  .\]
  Le \textit{trou} est une constante, notée $[\;]$ qui n'apparaît qu'une fois par contexte d'évaluation.
  Pour $E$ un contexte d'évaluation et $a \in \mathsf{EA}$, alors $E[a]$ désigne l'expression arithmétique obtenue en remplaçant le trou par $a$ dans $E$.

  \begin{exm}
    On note $E_0 = \ubar 3 \oplus ([\;] \oplus \ubar 5)$ et $a_0 = \ubar 1 \oplus \ubar 2$.
    Alors \[
      \ubar 3 \oplus ((\ubar 1 \oplus \ubar 2) \oplus \ubar 5)
    .\] 
  \end{exm}

  \textbf{Que faut-il mettre à la place de $\redQuestionBox$ ?}

  \begin{exm}[Première tentative]
    On pose \[
      E ::= [\;]  \mid \ubar{k}  \mid E_1 \oplus E_2
    .\]
    Mais, ceci peut introduire \textit{plusieurs} trous (voire aucun) dans un même contexte.
    C'est raté.
  \end{exm}

  \begin{exm}[Seconde tentative]
    On pose \[
      E ::= [\;]  \mid a \oplus E \mid E \oplus a
    .\]
    Mais, on pourra réduire une expression à droite avant de réduire à gauche.
    C'est encore raté.
  \end{exm}

  \begin{exm}[Troisième (et dernière) tentative]
    On pose \[
      E ::= [\;]  \mid a \oplus E \mid E \oplus \ubar k
    .\]
    Là, c'est réussi !
  \end{exm}

  \begin{lem}
    Pour toute expression arithmétique $a \in \mathsf{EA}$ qui n'est pas une constante, il existe un unique triplet $(E, k_1, k_2)$ tel que \[
      a = E[\ubar{k}_1 \oplus \ubar{k}_2]
    .\]
  \end{lem}

  Ceci permet de justifier la proposition suivante, notamment au niveau des notations.

  \begin{prop}
    Pour tout $a, a'$, on a \[
    a \to a' \quad \text{si, et seulement si,} \quad a \mapsto a'
    .\]
  \end{prop}
  
  \begin{prv}
    Pour démontrer cela, on procède par double implication :
    \begin{itemize}
      \item "$\implies$" par induction sur $a \to a'$ ;
      \item "$\impliedby$" par induction sur $E$.
    \end{itemize}
  \end{prv}
\end{document}
