\documentclass{../notes}

\title{Projet fonctionnel}
\author{Basé sur le cours de Daniel \textsc{Hirschkoff}\\ Notes prises par Hugo \textsc{Salou}}

\usepackage{pdflscape}

\begin{document}
  \maketitle

  \dominitoc
  \tableofcontents

  \pagebreak

  \chapter*{Introduction.}

  Le cours se décompose en deux parties :
  \begin{itemize}
    \item une partie \textit{pratique} ($\sim 2/3$ du cours) avec de la programmation OCaml en binôme ;
    \item une partie \textit{théorique} : avec du $\lambda$-calcul, typage et lien avec la logique.
  \end{itemize}

  Pour la partie pratique, on développe \textit{fouine}, un sur-ensemble du langage $\mathsf{FUN}$.
  Le produit attendu est un interprète OCaml, c'est-à-dire :
  \[
    \text{fichier texte} \xrightarrow{\text{frontend}} \mathtt{expr} \xrightarrow{\text{fonction \texttt{eval}}} \text{valeur / affichage console}
  .\]

\end{document}

