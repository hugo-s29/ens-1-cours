\documentclass[./main]{subfiles}

\begin{document}
  \chapter{Idéaux et anneaux noethériens}

  \section{Exercice 1.}

  \begin{enonce}
    \begin{enumerate}
      \item Se répéter dix fois : "Un idéal n'est pas un sous-anneau et l'image d'un idéal par un morphisme d'anneau n'est pas nécessairement un idéal" (et trouver des exemples).
      \item Montrer que l'image d'un idéal par un morphisme d'anneaux surjectif est un idéal.
    \end{enumerate}
  \end{enonce}

  \begin{enumerate}
    \item L'idéal $I = 2\mathds{Z}$ n'est pas un sous-anneau car $1 \not\in I$.
      On considère le morphisme d'anneau \begin{align*}
        f: \mathds{Z} &\longrightarrow \mathds{Q} \\
        x &\longmapsto x
      ,\end{align*}
      et on a $f(n\mathds{Z}) = n \mathds{Z}$ qui n'est pas un idéal de $\mathds{Q}$ (car $\mathds{Q}$ est un corps, donc n'a pour idéaux que $\mathds{Q}$ et $\{0\}$).
    \item Soit $f : A \to B$ un morphisme d'anneaux surjectif, et $I$ un idéal de $A$.
      \begin{itemize}
        \item Soient $x_1, x_2 \in f(I)$. Soient $y_1, y_2 \in I$ tels que $x_1 = f(y_1)$ et $x_2 = f(y_2)$.
          D'où, $x_1 + x_2 = f(y_1 + y_2) \in f(I)$ car $y_1 + y_2 \in I$.
        \item Soit $x \in f(I)$ et $b \in B$. Il existe $y \in I$ tel que $f(y) = x$.
          Par surjectivité, il existe $a \in A$ tel que $b = f(a)$.
          Ainsi, $bx = f(ay) \in f(I)$, car $ay \in I$.
      \end{itemize}
  \end{enumerate}

  \section{Exercice 2. \textit{Terminologie}}

  \begin{enonce}
    Soit $A$ un anneau et soient $a, b$ deux éléments de $A$.
    \begin{enumerate}
      \item Montrer que $a$ est inversible si et seulement si $(a) = A$.
      \item Montrer que $a$ divise $b$ si et seulement si $(b) \subseteq (a)$.
      \item On suppose que $A$ est intègre.
        \begin{enumerate}
          \item Montrer que $a \in A$ est premier si et seulement si $(a)$ est un idéal premier.
          \item Montrer que $a$ est irréductible dans $A$ si et seulement si $(a)$ est maximal parmi les idéaux principaux de $A$. En déduire que, si tout idéal de $A$ est principal, alors $a$ est irréductible si et seulement si $(a)$ est maximal.
          \item Donner un exemple d'anneau $A$ et d'élément irréductible $a \in A$ tel que $(a)$ n'est pas maximal parmi les idéaux de~$A$.
        \end{enumerate}
    \end{enumerate}
  \end{enonce}

  \begin{enumerate}
    \item Montrons que $a \in A^\times \iff (a) = A$.
      On a la chaîne d'équivalences :
      \[
        (a) = A \iff 1 \in (a) \iff \exists b \in A, ab = 1 \iff a \in A^\times
      .\] 
    \item Montrons que $a  \mid b \iff (b) \subseteq (a)$.
      D'une part, on a \[
        a  \mid b \iff \exists c \in A, b  = a c \underset {(\star)}\implies (b) \subseteq (a)
      .\]
      Et d'autre part, on a bien l'implication réciproque de $(\star)$, car si~$(b) \subseteq (a)$ alors $b \in (a)$ et donc il existe $c \in A$ tel que $b = a c$.
    \item 
      \begin{enumerate}
        \item On a :
          {
          \footnotesize
          \begin{align*}
            a \text{ premier}
            \iff& [\forall b, c \in A, a  \mid b c \implies a \mid b \text{ ou } a \mid c]\\
            \iff& [\forall b, c \in A, (bc) \subseteq (a) \implies (b) \subseteq (a) \text{ ou } (c) \subseteq (a)]\\
            \iff& (a) \text{ est un idéal premier}
          .\end{align*}
          }
        \item 
          \begin{itemize}
            \item Si $(a)$ est maximal parmi les idéaux principaux, et si $a = b c$ alors $b  \mid a$, et donc $(a) \subseteq (b)$.
              \begin{itemize}
                \item Soit $(b) = A$ et donc  $b \in A^\times$.
                \item Soit $(b) = (a)$ alors, il existe $d \in A$ tel que $b = a d$.
                  D'où,  $b = bcd$ et, car  $A$ intègre, $cd = 1$ d'où on a que $c \in A^\times$.
              \end{itemize}
            \item Réciproquement, on a 
              \begin{align*}
                (a) \subseteq (b) \implies& b  \mid a\\
                \implies& \exists c \in A, a = b c\\
                \implies& \exists (b,c) \in A^\times \times A \cup A \times A^\times, a = b c\\
                \implies& (b) = A \text{ ou } (b) = (a)
              .\end{align*}
          \end{itemize}
        \item On considère $A = \mathds{Z}[X]$ et $a = 2$ irréductible.
          Alors, $(2) \subseteq (2, X) \neq \mathds{Z}[X]$ et $(X) \subseteq (2, X) \neq \mathds{Z}[X]$.
      \end{enumerate}
  \end{enumerate}

  \section{Exercice 3. \textit{Quotienteries}}

  \begin{enumerate}
    \item On a, pour $a, b \in A$ :
      \[
      \bar{a} \bar{b} = \bar{0} \in A / I \iff a b \in I
      .\]
      D'où, 
      si $I$ premier, alors $\bar{a} = \bar{0}$ ou $\bar{b} = \bar{0}$.
      Réciproquement, si $ab \in I$ alors $\bar{a} = \bar{0}$ ou $\bar{b} = \bar{0}$, et donc $a \in I$ ou $b \in I$.

      Ensuite, on a pour $a \in A$, \[
      \bar{a} \in (A / I)^\times \iff (a) + I = A
      .\] 
      Ainsi, si $I$ est maximal, et si  $\bar{a} \in (A / I) \setminus \{0\}$, alors $a\not\in I$ et $(a) + I$ contient strictement $I$, d'où $(a) + I = A$ et  $\bar{a} \in (A / I)^\times$.
      Réciproquement, soit $I \subseteq J \subseteq A$, et s'il existe $x \in J \setminus I$, alors $xnar \in (A / I) \setminus \{0\}$ et donc il existe $y \in A$ tel que $\bar{x} \bar{y} = \bar{1}$ et donc $A = (x) + I \subseteq J$ et donc $J = A$.
    \item Soit $B$ un anneau et $f : A \to B$ un morphisme d'anneau tel que $I \subseteq \ker f$.
      Montrons qu'il existe un unique morphisme $\bar{f} : A / I \to B$ tel que $f = \bar{f} \circ \pi$.
      \begin{itemize}
        \item \textit{Unicité}.
          Si $\bar{f} : A / I \to B$ est tel que $\bar{f} \circ \pi = f$.
          Alors, parce que $\pi$ est surjectif, on a que $\forall x \in A/I, \exists a \in A, x = \pi(a)$, et donc $\bar{f}(x) = f(a)$.
        \item \textit{Existence}.
          On pose \begin{align*}
            \bar{f}: A / I &\longrightarrow B \\
            \bar{x} &\longmapsto f(x)
          .\end{align*}
          \begin{itemize}
            \item C'est bien défini car si $\pi(a) = \pi(b)$ alors $a-b \in I \subseteq \ker f$ et donc $f(a) = f(b)$.
            \item C'est bien un morphisme :
               \begin{itemize}
                \item $\bar{f}(\bar{1}) = f(1) = 1$ ;
                \item $\bar{f}(\bar{a} + \bar{b}) = \bar{f}(\overline{a + b}) = f(a+b) = f(a) + f(b) = \bar{f}(\bar{a}) + \bar{f}(\bar{b})$ ;
                \item $\bar{f}(\bar{a} \times  \bar{b}) = \bar{f}(\overline{a \times  b}) = f(a \times b) = f(a) \times  f(b) = \bar{f}(\bar{a}) \times  \bar{f}(\bar{b})$.
              \end{itemize}
          \end{itemize}
      \end{itemize}
    \item Soit $f : A \to B$, alors il existe un unique morphisme $\bar{f} : A / \ker f \to B$ tel que $f = \bar{f} \circ \pi$.
      Par construction $\im \bar{f} = \im f$ et donc $\bar{f} : A / \ker f \to \im f$ est surjectif.
      Montrons que $\bar{f}$ est injective.
      Si $\bar{f}(\bar{x}) = 0$ alors $f(x) = 0$ et donc  $x \in \ker f$ d'où on a $\bar{x} = \bar{0} \in A / \ker f$.
      On en conclut : \[
      A / \ker f \cong \im f
      .\] 
    \item Soient $I \subseteq J \subseteq A$ et on note $J / I = \pi_I(J)$.
      Alors, $(A / I) / (J / I) \cong A / J$.
      En effet, on pose 
      \begin{align*}
        f: A/I &\longrightarrow A/J \\
        a + I &\longmapsto a + J
      ,\end{align*}
      qui est un morphisme d'anneaux bien défini.

      Pour $a \in A$, $f(a + I) = a + J$ donc  $f$ est surjective.

      Et, 
      \begin{align*}
        \ker f &= \mleft\{\,a + I \in  A / I \;\middle|\; a + J = \bar{0} \in A / J\,\mright\} \\ 
        &= \mleft\{\,a + I \in A / I \;\middle|\; a \in J\,\mright\} \\
        &= \pi_I(J) = J / I \\
      .\end{align*}
      On en conclut, par le premier théorème d'isomorphisme que \[
        (A / I) / (J / I) \cong A / J
      .\]
    \item On considère la bijection croissante 
      \begin{align*}
        \{I \subseteq J \triangleleft A \}   &\longleftrightarrow \{\bar{K} \triangleleft A / I\}  \\ 
        J &\longmapsto \pi_I(J) \\
        \pi^{-1}(\bar{K}) & \longmapsfrom \bar{K}
      .\end{align*}
      On vérifie aisément que :
      \begin{itemize}
        \item pour $J \supseteq I$, $\pi ^{-1}(\pi(J)) = J$ ;
        \item pour $\bar{K} \triangleleft A / I$, $\pi(\pi^{-1}(\bar{K})) = \bar{K}$ ;
        \item l'application est croissante.
      \end{itemize}

      Si $P \triangleleft A$ est premier et tel que  $I \subseteq P$ alors, par le 3ème théorème d'isomorphisme, on a \[
        (A / I) / \pi(P) = (A / I) / (P / I) \cong A / P
      ,\] et ce dernier est intègre et on conclut.

      Réciproquement, si $\bar{P} \triangleleft A / I$ est premier, alors $\bar{P} = \pi(\pi^{-1}(\bar{P}))$ et donc $(A / I) / \bar{P} \cong A / \pi^{-1}(\bar{P})$ et le premier est intègre, donc on conclut.
    \item On pose \begin{align*}
          f: A[X] &\longrightarrow (A / I)[X] \\
        \sum a_i X^i &\longmapsto \sum \bar{a}_i X^i
      .\end{align*}
      On montre que $\im f = (A / I)[X]$ et que  $\ker f = I \: A[X]$.
  \end{enumerate}

  \section{Exercice 4. \textit{Division euclidienne par un polynôme unitaire}}

  \section{Exercice 5. \textit{Application : étude d'une courbe algébrique}}

  \section{Exercice 6. \textit{Lemme d'évitements des premiers}}

  \section{Exercice 7. \textit{Anneaux noethériens}}

  \section{Exercice 8. \textit{Existence et finitude des idéaux premiers minimaux}}
\end{document}
