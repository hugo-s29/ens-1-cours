\chapter{Diagramme dans une catégorie.}

\begin{defn}
  Soit $\mathbf{J}$ une petite catégorie, on appelle $\mathbf{J}$-diagramme dans une catégorie $\mathbf{C}$ tout foncteur $F : \mathbf{J} \to  \mathbf{C}$.
\end{defn}

\begin{exm}
  \begin{itemize}
    \item 
      Le diagramme \[
      \begin{tikzcd}
        A_1\arrow[loop above]{}{\mathrm{id}_{A_1}}
        &
        A_2\arrow[loop above]{}{\mathrm{id}_{A_2}}
      \end{tikzcd}
      ,\] 
      est défini par : $J_0 = \{1,2\}$ et $J_1 = \{\mathrm{id}_1, \mathrm{id}_2\}$ avec 
      \begin{align*}
        F: \mathbf{J} &\longrightarrow \mathbf{C} \\
        i &\longmapsto A_i\\
        \mathrm{id}_i &\longmapsto \mathrm{id}_{A_i}
      .\end{align*}
    \item 
      \[
      \begin{tikzcd}
        A_1 \arrow{r}{} & A_2 \arrow{r}{} & A_3
      \end{tikzcd}
      .\] 
  \end{itemize}
\end{exm}

\begin{defn}
  On dit qu'un diagramme $F : \mathbf{J} \to \mathbf{C}$ est \textit{commutatif} si pour tous $F(L)$ et $F(K)$ avec $L$ et $K$ deux objets de $\mathbf{J}$,
  tous les morphismes de source $F(L)$ et de but $F(K)$ sont égaux.
\end{defn}
