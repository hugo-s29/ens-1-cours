\documentclass{../../notes}

\title{DM n°1 -- FDI}
\author{Hugo \scshape Salou}

\tikzset{
  initial text=,
  node distance=2cm,
  auto,
  every state/.style={draw=deepblue,thick},
  every path/.style={draw=deepblue},
}

\begin{document}
  \maketitle
  \chapter*{Exercice 1.}

  On applique l'algorithme d'éliminations d'états sur l'automate.
  Les différentes étapes sont représentées.

  On commence par "détourer" l'automate (on ajoute $q_\mathrm{i}$ et $q_\mathrm{f}$).

  \begin{figure}[H]
    \centering
    \begin{tikzpicture}
      \node[state,initial] (init) {$q_\mathrm{i}$};
      \node[state,right of=init] (q1) {$1$};
      \node[state,below right=1.5cm and 1.2cm of q1] (q3) {$3$};
      \node[state,above right=1.5cm and 1.2cm of q3] (q2) {$2$};
      \node[state,accepting,left of=q3] (final) {$q_\mathrm{f}$};

      \draw[->] (init) edge node{$\varepsilon$} (q1)
      (q1) edge node{$a$} (q2)
      (q2) edge[bend left] node{$b$} (q3)
      (q3) edge[bend left] node{$a$} (q2)
      (q1) edge node[swap]{$b$} (q3)
      (q3) edge node{$\varepsilon$} (final);
    \end{tikzpicture}
  \end{figure}

  On supprime l'état $2$.
  \vspace{-1.2cm}

  \begin{figure}[H]
    \centering
    \begin{tikzpicture}
      \node[state,initial] (init) {$q_\mathrm{i}$};
      \node[state,right of=init] (q1) {$1$};
      \node[state,right=2cm of q1] (q3) {$3$};
      \node[state,accepting,right of=q3] (final) {$q_\mathrm{f}$};

      \draw[->] (init) edge node{$\varepsilon$} (q1)
      (q1) edge node[swap]{$b\mid ab$} (q3)
      (q3) edge[loop above] node{$ab$} (q3)
      (q3) edge node{$\varepsilon$} (final);
    \end{tikzpicture}
  \end{figure}

  On supprime les états $2$ et $3$.

  \begin{figure}[H]
    \centering
    \begin{tikzpicture}
      \node[state,initial] (init) {$q_\mathrm{i}$};
      \node[state,accepting,right=3cm of init] (final) {$q_\mathrm{f}$};

      \draw[->] (init) edge node{$(b \mid ab) (ab)^\star$} (final);
    \end{tikzpicture}
  \end{figure}

  On en déduit une expression régulière équivalente à l'automate initial : \[
    (b  \mid ab) (ab)^\star
  .\]

  \chapter*{Exercice 2.}
  \vspace{-0.5cm}
  On procède en deux temps :
  \begin{enumerate}
    \item on réalise un DFA $\mathcal{A}_\text{\ref{step1}}$ reconnaissant le langage $\mathcal{L}((aba)^\star) = L$ ; \label{step1}
    \item et on inverse les états finaux et non finaux (donnant $\mathcal{A}_\text{\ref{step2}}$).\label{step2}
  \end{enumerate}
  L'automate déterministe obtenu reconnaitra le langage $\Sigma^\star \setminus L$.


  \begin{figure}[H]
    \centering
    \begin{tikzpicture}
      \node[state,initial,accepting] (q1) {1};
      \node[state,right of=q1] (q2) {2};
      \node[state,right of=q2] (q3) {3};
      \node[state,below of=q2,gray] (trash) {4};

      \draw[->]
        (q1) edge node{$a$} (q2)
        (q2) edge node{$b$} (q3)
        (q3) edge[bend right=45] node[swap]{$a$} (q1)
        ;

      \draw[->,gray]
        (q3) edge[gray,bend left=10] node{$b$} (trash)
        (q1) edge[gray,bend right=10] node[swap]{$b$} (trash)
        (q2) edge[gray] node {$a$} (trash)
        (trash) edge[gray,loop left] node{$a,b$} (trash)
        ;
    \end{tikzpicture}
    \caption{Automate fini déterministe pour l'étape \ref{step1}}
  \end{figure}

  \begin{figure}[H]
    \centering
    \begin{tikzpicture}
      \node[state,initial] (q1) {1};
      \node[state,right of=q1,accepting] (q2) {2};
      \node[state,right of=q2,accepting] (q3) {3};
      \node[state,below of=q2,accepting] (trash) {4};

      \draw[->]
        (q1) edge node{$a$} (q2)
        (q2) edge node{$b$} (q3)
        (q3) edge[bend right=45] node[swap]{$a$} (q1)
        (q3) edge[bend left=10] node{$b$} (trash)
        (q1) edge[bend right=10] node[swap]{$b$} (trash)
        (q2) edge node {$a$} (trash)
        (trash) edge[loop left] node{$a,b$} (trash)
        ;
    \end{tikzpicture}
    \caption{Automate fini déterministe pour l'étape \ref{step2}}
  \end{figure}


\end{document}
