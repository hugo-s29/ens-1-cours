\documentclass[./main]{subfiles}

\begin{document}
  \chapter{Table de caractères.}
  \minitoc

  \section{Exercice 1. \textit{Caractères linéaires}}
  \label{td12-ex1}

  \begin{enonce}
    Soit $G$ un groupe fini.
    \begin{enumerate}
      \item Si $G$ est abélien, montrer qu'il admet $\#G$ représentations de degré $1$ à isomorphisme près.
      \item En déduire que, dans le cas général, il en admet $[G : \mathrm{D}(G)]$.
    \end{enumerate}
  \end{enonce}

  \begin{enumerate}
    \item On sait que $G$ est abélien. \label{td12-ex1-q1}
      Alors, toutes les représentations irréductibles de $G$ sont de degré $1$.
      Ainsi, \[
      \# G = \sum_{V \text{ irréductible }} (\dim V)^2 = \# \{\text{représentations irréductibles}\} 
      .\]

      Justifions le "toutes les représentations irréductibles de $G$ sont de degré 1".
      Soit $(V, \rho)$ une représentation irréductible de  $G$.
      Alors, pour tout $g, h \in G$ alors $\rho(g)\rho(h) = \rho(h)\rho(g)$ et ainsi  $\rho(g)$ et  $\rho(h)$ sont diagonalisables.
      Donc elles sont co-diagonalisable.
      Alors il existe une base $\mathcal{B}$ de $V$ qui co-diagonalise $\rho(g)$ et donc le premier vecteur de $\mathcal{B}$ engendre une droite propre $D$ pour chaque $\rho(g)$.
      Et,  $D$ est donc stable par tous les $\rho(g)$, c'est donc une sous-représentation de $V$.
      Par irréductibilité de $V$, on a $D = V$ et donc $\dim V = 1$.
    \item Le dual de $G$, noté $G^\star$, est l'ensemble des caractères linéaires.
      On a vu dans le DM n°1 que $G^\star \cong (G^\mathrm{ab})^\star$, où $G^\mathrm{ab} := G / \mathrm{D}(G)$.
      Ainsi, d'après la question~\ref{td12-ex1-q1}, on sait que~$G^\mathrm{ab}$ admet exactement~$|G^\mathrm{ab}|$ caractères linéaires.
      D'où, $|(G^\mathrm{ab})^\star| = |G^\mathrm{ab}|$. On en conclut que \[
        |G^\star| = [G : \mathrm{D}(G)]
      .\] 
  \end{enumerate}

  \section{Exercice 2. \textit{Certaines propriétés des représentations de~$\mathfrak{S}_n$}.}

  \begin{enonce}
    Soit $n \ge 2$ un entier.
    \begin{enumerate}
      \item Soit $\sigma \in \mathfrak{S}_n$.
        Justifier que $\sigma$ et $\sigma^{-1}$ sont conjuguées dans $\mathfrak{S}_n$.
      \item En déduire que la table de caractère de $\mathfrak{S}_n$ est à valeurs réelles.
    \end{enumerate}

    \textit{Remarque} : On peut même montrer que la table de caractère de $\mathfrak{S}_n$ est toujours à valeurs entières, mais cela nécessite des arguments de théorie des corps du cours d'Algèbre 2.
  \end{enonce}

  \begin{enumerate}
    \item La classe de conjugaison de $\sigma$ est déterminée par les longueurs des cycles apparaissant dans la décomposition en cycles à supports disjoints (\textit{i.e.} le \textit{\textbf{type}}).
      L'inverse d'un $p$-cycle est un $p$-cycle par tout $p \in \llbracket 2,n\rrbracket$ donc $\sigma$ et $\sigma^{-1}$ ont même type.
      On en conclut que $\sigma$ et $\sigma^{-1}$ sont conjugués.
    \item Pour tout caractère $\chi$, pour toute permutation  $\sigma \in \mathfrak{S}_n$, on a \[
          \chi(\sigma) = \overline{\chi(\sigma^{-1})} = \overline{\chi(\sigma)}
      ,\]
      car $\chi$ est constant sur les classes de conjugaisons. Ainsi,  $\chi(\sigma) \in \mathds{R}$ et la table de caractères de $\mathfrak{S}_n$ est réelle.
  \end{enumerate}

  \section{Exercice 3. \textit{Table de caractères de $\mathfrak{A}_4$.}}
  \begin{enonce}
    \begin{enumerate}
      \item Montrer que $\mathfrak{A}_4$ a  $4$ classes de conjugaison : l'identité, la classe de $(1\:2\:3)$, la classe de  $(1\:3\:2)$, et les doubles transpositions.
      \item Montrer que le groupe dérivé de  $\mathfrak{A}_4$ est le sous-groupe des doubles transpositions, et en déduire  $3$ caractères linéaires de~$\mathfrak{A}_4$. 
      \item Déterminer la dimension de la dernière représentation irréductible de $\mathfrak{A}_4$ grâce aux propriétés de la représentation régulière.
      \item En utilisant l'orthogonalité des colonnes, déterminer alors la table de caractère de $\mathfrak{A}_4$.
    \end{enumerate}
  \end{enonce}

  \begin{enumerate}
    \item On connait les classes de conjugaisons dans $\mathfrak{S}_4$, et on regardent celles qui sont dans $\mathfrak{A}_4$.
      Il faudra après re-vérifier que ces classes de conjugaisons ne se re-découpent pas dans $\mathfrak{A}_4$.

      Dans $\mathfrak{S}_4$, on a 
       \begin{itemize}
         \item $\{\id\} \subseteq \mathfrak{A}_4$ ;
         \item $\{\text{transpositions}\} \not \subseteq \mathfrak{A}_4$ ;
         \item $\{\text{3-cycles}\} \subseteq \mathfrak{A}_4$ ;
         \item $\{\text{bi-transpositions}\} \subseteq \mathfrak{A}_4$ ;
         \item $\{\text{4-cycles}\} \not \subseteq \mathfrak{A}_4$.
      \end{itemize}

      Les classes $\{\id\}$ et $\{\text{bi-transpositions}\}$ ne se re-découpent pas.
      Cependant, pour les $3$-cycles, on les décompose en deux classes :
      celle de $(1\:2\:3)$ et  $(1\:3\:2)$.
       \begin{itemize}
         \item Les deux permutions ne sont pas conjuguées car, si elles l'étaient, alors il existerait $\sigma \in \mathfrak{A}_4$ telle que 
           \[
             (\sigma(1)\:\sigma(2)\:\sigma(3)) = \sigma\: (1\:2\:3)\: \sigma^{-1} = (1\:3\:2)
           .\]
           Et, $\sigma(4) = 4$ donc  $\sigma$ permute $1,2,3$.
           Par $\mathfrak{A}_3$, on en déduit que l'on a~$\sigma \in \{\id, (1\:2\:3), (1\:3\:2)\}$.
           On en conclut que $\sigma$ et~$(1\:2\:3)$ commutent : \textit{\textbf{absurde}} car \[
           \sigma\:(1\:2\:3)\: \sigma^{-1} = (1\:2\:3) \neq (1\:3\:2)
           .\]
         \item On sait que $\# \mathrm{Cl}_{\mathfrak{A}_4}((1\:2\:3)) = \# \mathfrak{A}_4 / \# \mathrm{C}_{\mathfrak{A}_4}((1\:2\:3))$ (par relation orbite-stabilisateur pour la conjugaison).
           De plus, on sait que $\# \mathrm{Cl}_{\mathfrak{S}_4}((1\:2\:3)) = \#\mathfrak{S}_4 / \# \mathrm{C}_{\mathfrak{S}_4}((1\:2\:3))$.
           Ainsi, on a que $\# \mathrm{C}_{\mathfrak{S}_4}((1\:2\:3)) = 3$.
           On a $\mathrm{C}_{\mathfrak{S}_4}((1\:2\:3)) = \langle (1\:2\:3)\rangle$.
           Or,~$\mathrm{C}_{\mathfrak{A}_4}((1\:2\:3)) = \mathfrak{A}_4 \cap \mathrm{C}_{\mathfrak{S}_4}((1\:2\:3))$.
           Ainsi, $\# \mathrm{Cl}_{\mathfrak{S}_4}((1\:3\:2)) = 4$.

           Tous les $3$-cycles de $\mathfrak{A}_4$ sont répartis dans deux classes de conjugaisons : celle de  $(1\:2\:3)$ et celle de  $(1\:3\:2)$.

         \item Et $\mathfrak{A}_4$ est  $2$-transitif donc $(1\:2)(3\:4)$ est conjugué à  $(a\:b)(c\:d)$ pour tout  $a,b,c,d$ distincts
           avec  $\sigma : 1 \mapsto a, 2 \mapsto b$ car  \[
           \sigma (1\:2)(3\:4) \sigma^{-1} = \cdots = (a\:b)(c\:d)
           .\]
      \end{itemize}
      Donc, les classes de conjugaisons de $\mathfrak{A}_4$ sont :
      \[
      \{\id\} \quad \{\text{classe de }(1\:2\:3)\}  \quad \{\text{classe de }(1\:3\:2)\} \quad \text{et} \quad \{\text{bi-transpositions}\} 
      .\]
    \item Si $H \triangleleft G$ et $G / H$ est abélien alors $\mathrm{D}(G) \subseteq H$.
      Le sous-groupe distingué $V_4 \triangleleft \mathfrak{A}_4$ est le sous-groupe contenant l'identité et les bi-transpositions.
      On a $|\mathfrak{A}_4 / V_4| = 3$ donc $\mathfrak{A}_4 / V_4$ est abélien, d'où on a $ \mathrm{D}(\mathfrak{A}_4) \subseteq V_4$.
      Or, $\mathrm{D}(\mathfrak{A}_4) \triangleleft \mathfrak{A}_4$ donc c'est une union de classe de conjugaisons.
      Ainsi $\mathrm{D}(\mathfrak{A}_4) = \{\id\}$ et $\mathrm{D}(\mathfrak{A}_4) = V_4$.
      Et, puisque $\mathfrak{A}_4$ est non-abélien, alors  $\mathrm{D}(\mathfrak{A}_4) \neq \{\id\}$.
      On en déduit que $\mathrm{D}(\mathfrak{A}_4) = V_4$.
      On a que $\mathfrak{A}_4$ a  $3 = [\mathfrak{A}_4 : V_4]$ caractères linéaires (\textit{c.f.} exercice~\ref{td12-ex1}).
      Un caractère linéaire $\chi$ de $\mathfrak{A}_4$ vérifie donc  $\chi(V_4) = 1$ et est uniquement déterminé par  $\chi(1\:2\:3) \in \{1,\mathrm{j}, \mathrm{j}^2\}$ où $\mathrm{j} = \mathrm{e}^{2 \mathrm{i} \pi / 3}$.
    \item On a que $\# \mathfrak{A}_4 = 12 = 1^2 + 1^2 + 1^2 + 3^2$.
    \item On en déduit la table suivante.
      % \[
      %\begin{array}{c|c|c|c|c}
      %  \,&\id&(1\:2\:3)&(1\:3\:2)&(1\:2)(3\:4)\\ \hline
      %  \mathds{1}&1&1&1&1\\ \hline
      %  V_{\mathrm{j}} & 1 & \mathrm{j} & \mathrm{j}^2 & 1\\ \hline
      %  V_{\mathrm{j}^2} & 1 & \mathrm{j}^2 & \mathrm{j} & 1\\ \hline
      %  W & 1 & 0 & 0 & -1\\
      %\end{array}
      %.\] 
  \end{enumerate}

  \section{Exercice 4. \textit{Tables de caractères de $D_8$ et~$H_8$.}}
  \begin{enonce}
    On va calculer les tables de caractères des groupes $D_8$ et $H_8$.
    \begin{enumerate}
      \item Soit $D_8$ le groupe diédral d'ordre $8$. Il est engendré par deux éléments $r$ et $s$ tels que l'élément $r$ est d'ordre $4$, l'élément $s$ est d'ordre $2$ et l'égalité $s r s^{-1} = r^{-1}$ est vérifiée.
        \begin{enumerate}
          \item Montrer que les classes de conjugaisons de $D_8$ sont $\{1\}$, $\{r, r^3\}$, $\{r^2\}$ $\{s, sr^2\}$ et $\{s r, s r^3\}$.
          \item Montrer que le groupe dérivé de $D_8$ est $\{1,r^2\}$.
          \item En déduire que $D_8$ a $4$ représentations de degré 1, et une irréductible de degré $2$, ainsi que la table de caractère de~$D_8$. À quelle action géométrique correspond la représentation irréductible de degré 2 ? 
        \end{enumerate}
    \end{enumerate}
  \end{enonce}

  \begin{enumerate}
    \item
  \end{enumerate}

\end{document}
