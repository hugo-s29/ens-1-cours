\documentclass[../main]{subfiles}

\begin{document}
  \chapter{Mémoire structurée, logique de séparation} \label{thprog-chap09}
  \minitoc

  La syntaxe des commandes de $\mathsf{IMP}$ avec tas est définie par la grammaire 
  \begin{align*}
    c ::= \ &x \ceq a  \mid \mathtt{skip}  \mid c_1 \col c_2  \mid \ifte b {c_1} {c_2}  \mid \while b c\\
            &x \ceq \ptr{a}  \mid \ptr{x_1} \ceq a  \mid x \ceq \mathtt{alloc}(k)  \mid \mathtt{free}(a).
  \end{align*}
  Les expressions arithmétiques et booléennes restent inchangées.

  Ces quatre nouvelles constructions correspondent respectivement à 
  \begin{itemize}
    \item l'accès à une case mémoire ;
    \item la modification d'une valeur d'une case mémoire ;
    \item l'allocation de $k$ cases mémoires ;
    \item la libération de cases mémoires.
  \end{itemize}

  \begin{rmk}
    C'est un langage impératif de bas niveau : on manipule de la mémoire directement.
    On ne s'autorise pas tout, cependant. On ne s'autorise pas, par exemple, \[
      \ptr{x + i + 1} \ceq \ptr{t + i} + \ptr{t + i - 1}
    ,\] mais on demande d'écrire \[
      x \ceq \ptr{t+i}\col y \ceq \ptr{t + i - 1} \col \ptr{x + i + 1} \ceq x + y
    .\]
  \end{rmk}

  On raffine $\mathsf{IMP}$ : avant, les cases vivaient quelque part, mais on ne sait pas où, ce sont des registres ; maintenant, peut aussi allouer des "blocs" de mémoire, et on peut donc parler de cases mémoires adjacentes.

  L'allocation $\mathtt{alloc}$ est \textit{dynamique}, similaire à \texttt{malloc} en C, où l'on alloue de la mémoire dans l'espace mémoire appelé \textit{tas}.

  \section{Sémantique opérationnelle de $\mathsf{IMP}$ avec tas.}

  \begin{defn}[États mémoire]
    Un état mémoire est la donnée de 
    \begin{itemize}
      \item $\sigma$ un \textit{registre}, c-à-d un dictionnaire sur $(\mathrm{V}, \mathds{Z})$ ;
      \item $h$ un \textit{tas}, c-à-d un dictionnaire sur $(\mathds{N}, \mathds{Z})$, c'est un gros tableau.\footnote{On appelle parfois $\mathsf{IMP}$ avec tas, $\mathsf{IMP}$ avec tableau.}
    \end{itemize}

    On définit $h[k_1 \mapsto k_2]$ que si $k_1 \in \mathrm{dom}(h)$ et alors il vaut le dictionnaire où l'on assigne $k_2$ à $k_1$.
    
    On définit $h_1 \uplus h_2$ que si $\mathrm{dom}(h_1) \cap \mathrm{dom}(h_2) = \emptyset$ et vaut l'union de dictionnaires $h_1$ et $h_2$.
  \end{defn}

  On définit ainsi la sémantique dénotationnelle comme la relation ${\Downarrow}$ est une relation quinaire (\textit{i.e.} avec $5$ éléments) notée $c, \sigma, h \Downarrow \sigma', h'$ où  $c$ est une commande, $h,h'$ sont deux tas, et $\sigma, \sigma'$ sont deux registres.

  \begin{gather*}
    \begin{prooftree}
      \hypo{a, \sigma \Downarrow k}
      \infer[left label={\sigma' = \sigma[x \mapsto k]}] 1{x \ceq a, \sigma, h \Downarrow \sigma', h}
    \end{prooftree}
    \quad\quad
    \begin{prooftree}
      \hypo{c_1, \sigma, h \Downarrow \sigma', h'}
      \hypo{c_2, \sigma', h' \Downarrow \sigma'', h''}
      \infer 2{c_1 \col c_2, \sigma, h \Downarrow \sigma'', h''}
    \end{prooftree}
    \\[1em]
    \begin{prooftree}
      \hypo{a, \sigma \Downarrow k}
      \infer[left label={\substack{k' = h(k)\\ \sigma' = \sigma[x \mapsto k']}}] 1{x \ceq \ptr{a}, \sigma, h \Downarrow \sigma' h}
    \end{prooftree}
    \quad\quad
    \begin{prooftree}
      \hypo{a_1, \sigma \Downarrow k_1}
      \hypo{a_2, \sigma \Downarrow k_2}
      \infer[left label={h' = h[k_1 \mapsto k_2]\text{\footnote{On sous-entend ici que $k_1 \in \mathrm{dom}(h)$.}\showfootnote}}] 2{\ptr{a_1} \ceq a_2, \sigma, h \Downarrow \sigma, h'}
    \end{prooftree}
    \\[1em]
    \begin{prooftree}
      \infer[left label={\substack{\{k', \ldots, k' + k - 1\} \cap \mathrm{dom}(h) = \emptyset \\ k > 0\\ h' = h \uplus \{k' \mapsto 0, \ldots, k' + k - 1 \mapsto 0\}\\ \sigma' = \sigma[x \mapsto k']}}] 0{x \ceq \mathtt{alloc}(k), \sigma, h \Downarrow \sigma', h'}
    \end{prooftree}
  \end{gather*}
  \[
    \begin{prooftree}
      \hypo{a, \sigma \Downarrow k}
      \infer[left label={h = h' \uplus \{k \mapsto k'\}}] 1{\mathtt{free}(a), \sigma, h \Downarrow \sigma, h'}
    \end{prooftree}
  .\]

  \section{Logique de séparation.}

  \begin{defn}
    On définit les assertions dans $\mathsf{IMP}$ avec tas comme l'enrichissement des assertions $\mathsf{IMP}$ avec les constructions $\mathtt{emp}$, $\mapsto$ et $*$ :
    \[
      A ::= {\cdots}  \mid \mathtt{emp}  \mid a_1 \mapsto a_2  \mid A_1 * A_2
    .\]

    On enrichit ainsi la relation de satisfaction :
    \begin{itemize}
      \item $\sigma, h \models \mathtt{emp}$ si et seulement si $h = \emptyset$ ;
      \item $\sigma, h \models a_1 \to a_2$ si et seulement si $a_1, \sigma \Downarrow k_1$, et $a_2, \sigma \Downarrow k_2$ et~$h = [k_1 = k_2]$ (le tas est un singleton) ;
      \item $\sigma, h \models A_1 * A_2$ si et seulement si $h = h_1 \uplus h_2$ avec $\sigma, h_1 \models A_1$ et $\sigma, h_2 \models A_2$.
    \end{itemize}
  \end{defn}

  L’opérateur $*$ est appelé \textit{conjonction séparante} : on découpe le tas en deux, chaque partie étant "observée" par une sous-assertion.

  \begin{note}[Notations]
    \begin{itemize}
      \item On note $a \mapsto \texttt{\_}$ pour $\exists i, a \mapsto i$.
      \item On note $a_1 \hookrightarrow a_2$ pour $(a_1 \mapsto a_2) * \mathtt{true}$.
      \item On note $a \mapsto (a_1, \ldots, a_n)$ pour \[
          (a \mapsto a_1) * (a + 1 \mapsto a_2) * \cdots * (a + n - 1 \mapsto a_n)
        .\]
      \item On note $[A]$ pour $A \land \mathtt{emp}$.
    \end{itemize}
  \end{note}
  Ces notations signifient respectivement :
  \begin{itemize}
    \item La première formule dit qu'une case mémoire est allouée en $a$ (ou plutôt que si $a$ s'évalue en $k$, alors il y a une case mémoire allouée en $k$).
    \item La seconde formule dit que la case mémoire $a_1$ est allouée, et contient $a_2$ quelque part dans le tas.
    \item La troisième formule dit que les $n$ cases mémoires suivant $a$ contiennent respectivement $a_1$, puis $a_2$, \textit{etc}.
    \item La quatrième formule permet de ne pas parler du tas. Intuitivement $A$ "observe" uniquement la composante $\sigma$ et alors $A$ est une formule de la logique de Hoare.
  \end{itemize}

  \section{Triplets de Hoare pour la logique de séparation.}

  On rappelle  que $ \{A\} c \{A'\} $ est définir avec $A,A'$ closes.
  La validité d'un triplet de Hoare en logique de séparation est défini par~$\models \{A\} c \{A'\}$ si et seulement si dès que $\sigma, h \models A$ et $c, \sigma, h \Downarrow \sigma', h'$ alors  $\sigma', h' \models A'$.

  Parmi les règles d'inférences pour la logique de séparation, il y a la "\textit{frame rule}" ou "\textit{règle d'encadrement}" :
  \[
  \begin{prooftree}
    \hypo{\vdash \{A\} c \{A'\}}
    \infer[left label=\substack{\text{aucune variable $x \in \mathsf{vars}(B)$}\\\text{n'est modifiée par $c$}}] 1{\vdash \{A*B\} c \{A'*B\}}
  \end{prooftree}
  .\]
  Elle permet de \textit{zoomer}, et de ce concentrer sur un comportent local.

  Les règles suivantes définissent une logique sur les commandes de $\mathsf{IMP}$ avec tas.
  \begin{gather*}
    \begin{prooftree}
      \infer 0{\vdash \{\mathtt{emp}\}x \ceq \mathtt{alloc}(k) \{x \mapsto (0, \ldots, 0)\}}
    \end{prooftree}\\[1em]
    \begin{prooftree}
      \infer 0{\vdash \{a \mapsto \texttt{\_}\} \mathtt{free}(a) \{\mathtt{emp}\}}
    \end{prooftree}\quad\quad
    \begin{prooftree}
      \infer 0{\vdash \{a_1 \mapsto \texttt{\_}\} \ptr{a_1} \ceq a_2 \{a_1 \mapsto a_2\}}
    \end{prooftree}\\[1em]
    \begin{prooftree}
      \infer 0{\vdash \{[x = x_0] * (a \mapsto x_1)\} x\ceq \ptr{a} \{[x = x_1] * (a[\sfrac{x_0}{x}] \mapsto x_1)\}}
    \end{prooftree}
  \end{gather*}
\end{document}
