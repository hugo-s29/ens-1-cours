\documentclass{../notes}

\title{Logique}

\usepackage{multicol}
\usepackage{lscape}
%\usepackage{pdflscape}
%\usepackage{cellspace}

\newlength\stextwidth
\newcommand\makesamewidth[3][c]{%
  \settowidth{\stextwidth}{#2}%
  \makebox[\stextwidth][#1]{#3}%
}

\newcounter{questioncounter}
\setcounter{questioncounter}{0}

\newcommand\question[1]{
  \vspace{1em}
  \refstepcounter{questioncounter}
  \makesamewidth[r]{\textbf{Q99.}}{\textbf{Q\thequestioncounter.}}~
  \parbox[t]{11cm}{#1}
  \vspace{1em}
}

\newcommand\monoit[1]{ \texttt{\textit{#1}} }

\newcommand\indication[2][\relax]{
  \textit{Indication.}

  \raisebox{\depth}{\rotatebox{180}{\parbox{12.3cm}{#2}}}
}

\newcommand\ptree[1]{
  \begin{prooftree}
    #1
  \end{prooftree}
}

\begin{document}
  \section{Introduction.}

  Étant donnée une formule logique $\varphi$, comment montrer qu'elle est tautologique (\textit{i.e.} toujours vraie) ? Comment montrer $\models \varphi$ ?
  \begin{enumerate}
    \item On \textit{brute force} toutes les combinaisons possibles des variables, c'est-à-dire on teste les $2^{|{\operatorname{vars} \varphi}|}$ valuations et on vérifie que l'on ait $v(\varphi) = \textsc{Vrai}$ pour toute valuation $v$.
      C'est l'équivalent de construire une table de vérité pour $\varphi$.
    \item On calcule $v(\varphi)$ de manière symbolique et on essaie de simplifier les termes.
      Ceci implique de savoir factoriser/développer de manière astucieuse. 
    \item On réalise un raisonnement plus mathématique avec des "montrons", des "supposons", \textit{etc}.
  \end{enumerate}

  La déduction naturelle, et les preuves formelles, c'est ce dernier point.

  Voici un exemple de raisonnement formel pour montrer \[
    \varphi := \big((p \to q) \land (q \to r)\big) \to (p \to r)
  .\]
  \begin{exm}
    \label{ex1}
    Montrons $\big((p \to q) \land (q \to r)\big) \to (p \to r)$.
    \begin{itemize}
      \item[$\hookrightarrow$] Supposons $(p\to q) \land (q \to r)$, et montrons $p \to r$.
        \begin{itemize}
          \item[$\hookrightarrow$] Supposons $p$, et montrons $r$.
            \begin{itemize}
              \item[$\hookrightarrow$] Montrons $q$.
                \begin{itemize}
                  \item[$\hookrightarrow$] Montrons $p$. C'est vrai par hypothèse.
                  \item[$\hookrightarrow$] Montrons $p \to q$. C'est vrai par hypothèse.
                \end{itemize}
              \item[$\hookrightarrow$] Montrons $q \to r$. C'est vrai par hypothèse.
            \end{itemize}
        \end{itemize}
    \end{itemize}
  \end{exm}

  On voit une structure arborescente de la preuve : on a des cas, des sous-cas, des sous-sous-cas, \textit{etc}.
  Seules deux types d'informations nous intéressent : les \textit{hypothèses courantes} et l'\textit{objectif de preuve}.
  Ces deux informations changent au cours de la preuve.
  Localement, on a supposé $p$, mais ce n'est pas le cas dans le cas général. Les hypothèses sont donc locales.

  \begin{defn}
    Un \textit{séquent}, noté $\Gamma \vdash \varphi$, est la donnée d'un ensemble d'hypothèses $\Gamma$ et d'un objectif de preuve $\varphi$.

    L'ensemble $\Gamma$ est généralement écrit sous forme de liste mais c'est bien un ensemble.
  \end{defn}

  Avec une écriture en séquent, on obtient une preuve comme celle ci-dessous.

  \begin{exm}
    $\vdash \big((p \to q) \land (q \to r)\big) \to (p \to r)$
    \begin{itemize}
      \item[$\hookrightarrow$] $(p\to q) \land (q \to r) \vdash p \to r$
        \begin{itemize}
          \item[$\hookrightarrow$] $(p\to q) \land (q \to r), p \vdash r$
            \begin{itemize}
              \item[$\hookrightarrow$] $(p\to q) \land (q \to r), p \vdash q$
                \begin{itemize}
                  \item[$\hookrightarrow$] $(p\to q) \land (q \to r), p \vdash p$, hyp.
                  \item[$\hookrightarrow$] $(p\to q) \land (q \to r), p \vdash p \to q$, hyp.
                \end{itemize}
              \item[$\hookrightarrow$] $(p\to q) \land (q \to r), p \vdash q\to r$, hyp.
            \end{itemize}
        \end{itemize}
    \end{itemize}
  \end{exm}

  Pour faire mieux "ressortir" la structure d'arbre, on peut l'écrire comme ci-dessous.
  L'arbre est à l'envers, avec la racine en bas (en dépit de ce que disent les biologistes).

  On va d'abord définir formellement le "format" de l'arbre avant de montrer quelques exemples.

  \section{Définitions formelles.}

  \begin{defn}
    Une \textit{règle (de construction) de preuves} est de la forme 
    \[
    \begin{prooftree}
      \hypo{\Gamma_1 \vdash \varphi_1} 
      \hypo{\Gamma_2 \vdash \varphi_2}
      \hypo{\cdots} 
      \hypo{\Gamma_n \vdash \varphi_n} 
      \infer 4[\mathsf{nom}]{\Gamma \vdash \varphi}
    \end{prooftree}
    .\]
    Ce qu'il y a au dessus de la barre est appelé \textit{prémisses}.
    Ce qu'il y a en dessous, c'est la \textit{conclusion}.

    On peut avoir $n = 0$, on l'écrira comme \[
    \begin{prooftree}
      \infer 0[\mathsf{nom}]{\Gamma \vdash \varphi}
    \end{prooftree}
    .\]
    On appelle cela une \textit{règle de base}.

    Un ensemble de règles de preuves est appelé \textit{système de preuves}.
    La déduction naturelle est un exemple de système de preuves, mais ce n'est pas le seul, il en existe d'autre.
  \end{defn}

  Les règles de constructions, ce sont les nœuds (internes et feuilles) de notre arbre de preuve.
  Les règles de base, ce sont les feuilles.

  L'idée c'est que les $\Gamma_i \vdash \varphi_i$ sont les prémisses d'une règle, mais chaque prémisse est conclusion d'une autre règle.
  En gros, on peut coller les règles entre-elles pour former un arbre de preuve.

  \begin{defn}
    Un \textit{arbre de preuve} est un arbre étiqueté par des séquents, et dont le lien parent-enfants est fait par une règle du système de preuve.

    Si $\Gamma \vdash \varphi$ est la racine d'un arbre de preuve, alors on dit qu'il est~\textit{prouvable}.
    (On pourra parfois le noter $\Gamma \vdash \varphi$ comme si c'était une propriété mathématique et non un coupe).
  \end{defn}

  \section{Déduction naturelle intuitionniste.}

  Le système de preuves classique est appelé la \textit{déduction naturelle}. Il existe en réalité deux versions de ce système, la version \textit{intuitionniste} et la version \textit{classique} :
  \[
    \text{déduction naturelle intuitionniste} \subsetneq \text{déduction naturelle classique}
  .\] 
  L'inclusion stricte ci-dessous a deux sens :
  \begin{itemize}
    \item les règles intuitionnistes sont strictement inclues dans les règles classiques ;
    \item les résultats prouvables en intuitionniste sont strictement inclus dans les résultats prouvables en classique.
  \end{itemize}

  \begin{landscape}
    \begin{table}[hptb]
      \centering
      \begin{tabular}{c|c|c}
        \hline
        \textsc{Symbole} & \textsc{Règle(s) d'introduction} & \textsc{Règle(s) d'élimination}\\ \hline \hline 
        ~&~&~\\[-5pt]
        $\top$\/&$\ptree{\infer0[\top_\mathsf{i}]{\Gamma \vdash \top}}$&\\[-5pt]
        ~&~&~\\[-5pt] \hline ~ & ~ & ~\\[-5pt]
        $\bot$\/ &&$\ptree{\hypo{\Gamma\vdash\bot}\infer1[\bot_\mathsf{e}]{\Gamma \vdash G}}$\\[-5pt]
        ~&~&~\\[-5pt] \hline ~ & ~ & ~\\[-5pt]
        $\lnot$\/ &$\ptree{\hypo{\Gamma,G\vdash \bot}\infer1[\lnot_\mathsf{i}]{\Gamma\vdash\lnot G}}$&$\ptree{\hypo{\Gamma\vdash G}\hypo{\Gamma\vdash \lnot G}\infer2[\lnot_\mathsf{e}]{\Gamma\vdash \bot}}$ \\
        ~&~&~\\[-5pt] \hline ~ & ~ & ~\\[-5pt]
        $\to$ & $\ptree{\hypo{\Gamma,G \vdash H}\infer1[\to_\mathsf{i}]{\Gamma \vdash G \to H}}$ & $\ptree{\hypo{\Gamma \vdash H \to G}\hypo{\Gamma \vdash H}\infer2[\to_\mathsf{e}]{\Gamma \vdash G}}$\\
        ~&~&~\\[-5pt] \hline ~ & ~ & ~\\[-5pt]
        $\land$\/ & $\ptree{\hypo{\Gamma \vdash G}\hypo{\Gamma \vdash H}\infer2[\land_\mathsf{i}]{\Gamma \vdash G \land H}}$\/ & $\ptree{\hypo{\Gamma \vdash G \land H}\infer1[\land_\mathsf{e}^\mathsf{g}]{\Gamma \vdash G}}$ \quad\quad $\ptree{\hypo{\Gamma \vdash G \land H}\infer1[\land_\mathsf{e}^\mathsf{d}]{\Gamma \vdash H}}$ \\
        ~&~&~\\[-5pt] \hline ~ & ~ & ~\\[-5pt]
        $\lor$&$\ptree{\hypo{\Gamma\vdash G}\infer1[\lor_\mathsf{i}^\mathsf{g}]{\Gamma\vdash G \lor H}}$\quad\quad$\ptree{\hypo{\Gamma\vdash H}\infer1[\lor_\mathsf{i}^\mathsf{d}]{\Gamma\vdash G \lor H}}$&$\ptree{\hypo{\Gamma \vdash A \lor B}\hypo{\Gamma,A\vdash G}\hypo{\Gamma,B\vdash G}\infer3[\lor_\mathsf{e}]{\Gamma \vdash G}}$\\
        ~&~&~\\[-5pt] \hline
        \multicolumn{3}{c}{}\\
        \multicolumn{3}{c}{$\ptree{\infer 0[\mathsf{ax}]{\Gamma, \varphi\vdash \varphi}} \quad \quad \ptree{\hypo{\Gamma \vdash \varphi} \infer 1[\mathsf{aff}]{\Gamma, \Gamma' \vdash \varphi}}$}\\
        \multicolumn{3}{c}{}\\ \hline
      \end{tabular}
    \end{table}
  \end{landscape}

  Sur la page précédente se trouvent les règles de la déduction naturelle intuitionniste et classique.
  Les règles sont regroupées en trois types :
  \begin{multicols}{3}
    \begin{enumerate}
      \item Introduction 
      \item Élimination
      \item Autres
    \end{enumerate}
  \end{multicols}

  Voici, en français, ce qu'elles veulent dire.
  \begin{itemize}
    \item $\top_\mathsf{i}$ On peut toujours prouver "vrai".
    \item $\bot_\mathsf{e}$ Si l'on sait que l'on peut montrer une absurdité, alors on peut oublier l'objectif de preuve et montrer cette absurdité.
    \item $\land_\mathsf{i}$ Pour montrer $G$ \textit{et} $H$, il suffit montrer $G$ puis montrer $H$.
    \item  $\land_\mathsf{e}^\mathsf{g}$ Si on peut montrer $G$ \textit{et} $H$, alors on peut montrer $G$ 
    \item  $\land_\mathsf{e}^\mathsf{d}$ Si on peut montrer $G$ \textit{et} $H$, alors on peut montrer $H$.
    \item $\lor_\mathsf{i}^\mathsf{g}$ Si on veut montrer $G$ \textit{ou} $H$, alors il suffit de montrer $G$.
    \item $\lor_\mathsf{i}^\mathsf{d}$ Si on veut montrer $G$ \textit{ou} $H$, alors il suffit de montrer $H$.
    \item $\lor_\mathsf{e}$ C'est la disjonction de cas : si on veut montrer $G$, et qu'on sait qu'on a $A$ \textit{ou} $B$, alors on il suffit de montrer $G$ dans le cas~$A$ puis de montrer $G$ dans le cas $B$.
    \item $\to_\mathsf{i}$ Si on veut montrer $G$ \textit{implique} $H$, il suffit de supposer $G$, et de montrer $H$.
    \item $\to_\mathsf{e}$ Si on peut montrer $H$ \textit{implique} $G$ et qu'on peut montrer $H$, alors on peut montrer $G$.
    \item $\lnot_\mathsf{i}$ Pour montrer \textit{non} $G$, il suffit de montrer qu'en supposant $G$, on a une absurdité.
    \item $\lnot_\mathsf{e}$ Si on a montré $G$ et  \textit{non} $G$, alors on a une absurdité.
    \item $\mathsf{ax}$ (axiome) Pour montrer $\varphi$, il suffit de l'avoir en hypothèse.
    \item $\mathsf{aff}$ (affaiblissement) Si on peut montrer $\varphi$ avec certaines hypothèses, alors on peut le faire avec encore plus d'hypothèses, quitte à ne pas les utiliser.
  \end{itemize}

  Il est enfin temps de faire un arbre de preuve.
  En voici un :
  \[
  \begin{prooftree}
    \infer 0[\mathsf{ax}]{ p, \lnot p \vdash p }
    \infer 0[\mathsf{ax}]{ p, \lnot p \vdash \lnot p }
    \infer 2[\lnot_\mathsf{e}]{ p, \lnot p \vdash \bot }
    \infer 1[\lnot_\mathsf{i}]{ p \vdash \lnot \lnot p }
    \infer 1[\to_\mathsf{i}]{  \vdash p \to (\lnot \lnot p) }
  \end{prooftree}
  .\]

  \question{ Re-dérouler la démonstration ci-dessus pour vérifier qu'on a bien compris. }

  \question{ Construire l'arbre de preuve pour le séquent de l'exemple~\ref{ex1}. }

  \question{
    Au fur et à mesure, prouvez les séquents ci-dessous.
    C'est en faisant que l'on apprend.
    Si vous bloquez sur un, prévenez-moi et je vous donnerai des indications.
  }

  Quelques preuves simples pour commencer :
  \begin{enumerate}
    \item $\vdash p \to p$
    \item $p, \lnot p \vdash \bot$
    \item $p, q \vdash p \land q$
    \item $p \land q \vdash q \land p$
    \item $p\lor q \vdash q \lor p$
    \item $\vdash \lnot (p \land \lnot p)$
  \end{enumerate}

  Preuves un peu plus poussées :
  \begin{enumerate}[resume*]
    \item $p \lor (p \land q) \vdash p$
    \item $p \land , r \land s \vdash p \land s$
    \item $p, q \land r \vdash p \land q$
    \item $p\lor q \vdash q \lor p$
    \item $\lnot \lnot \lnot p \vdash \lnot p$
  \end{enumerate}

  Les lois de De Morgan (sauf une) :
  \begin{enumerate}[resume*]
    \item $\lnot (p\lor q) \vdash \lnot p \land \lnot q$
    \item $\lnot p \land \lnot q \vdash  \lnot (p \lor q)$
    \item $\lnot p \lor \lnot q \vdash \lnot (p \land q)$
  \end{enumerate}

  Distributivité $\lor$ et $\land$ :
  \begin{enumerate}[resume*]
    \item $p \land (q \lor r) \vdash (p \land q) \lor (p \land r)$
    \item $(p \land q) \lor (p \land r) \vdash  p \land (q \lor r)$
    \item $p \lor (q \land r) \vdash (p \lor q) \land (p \lor r)$
    \item $(p \lor q) \land (p \lor r) \vdash p \lor (q \land r)$
  \end{enumerate}

  Bonus :
  \begin{enumerate}[resume*]
    \item $\lnot p \vdash p \to q$
    \item $\lnot (p \to q) \vdash q \to p$
    \item $(p \land q) \to r \vdash p \to (q \to r)$
    \item $p \to (q \to r) \vdash (p \land q) \to r$
  \end{enumerate}

  \section{Logique classique.}

  Une fois que vous savez faire la plupart, on peut passer à la logique \textit{classique}.
  En logique classique, on ajoute une des trois règles suivantes.

  \begin{description}
    \item[Tiers Exclu.]
      On a toujours, soit $F$, soit \textit{non} $F$.

      \[
      \begin{prooftree}
        \infer 0[\mathsf{te}]{\Gamma \vdash F \lor \lnot F}
      \end{prooftree}
      \] 

    \item[Absurde.]
      Au lieu de montrer $F$, supposons \textit{non} $F$ et trouvons une contradiction.
      \[
      \begin{prooftree}
        \hypo{\Gamma, \lnot F \vdash \bot}
        \infer 1[\mathsf{abs}]{\Gamma \vdash F}
      \end{prooftree}
      \]
      On la note parfois $\mathsf{raa}$.
    \item[Double \textit{non}.]
      Deux \textit{non} s'annulent.
      \[
      \begin{prooftree}
        \hypo{\Gamma \vdash \lnot \lnot F}
        \infer 1[\lnot\lnot_\mathsf{e}]{\Gamma \vdash F}
      \end{prooftree}
      \] 
  \end{description}

  \question{
    En utilisant ces règles, démontrer les séquents suivants :
  }
  \begin{enumerate}[resume*]
    \item $\lnot p \to p \vdash p$
    \item $p\to q \vdash \lnot p \lor q$
    \item $\lnot p \lor q \vdash p \to q$
    \item $p \to q \vdash \lnot q \to \lnot p$
    \item $p \lor q, \lnot q \lor r \vdash p \lor r$
  \end{enumerate}

  \begin{defn}
    Une règle 
    \[
    \begin{prooftree}
      \hypo{\Gamma_1 \vdash \varphi_1}
      \hypo{\Gamma_2 \vdash \varphi_2}
      \hypo{\cdots}
      \hypo{\Gamma_n \vdash \varphi_n}
      \infer 4[\mathsf{r\grave{e}gle}]{\Gamma \vdash \varphi}
    \end{prooftree}
    \]
    est \textit{dérivable} dans un système de preuve (la règle ci-dessus n'appartient pas au système de preuve, sinon cela ne sert à rien) si l'on peut dériver sa conclusion en supposant pouvoir dériver ses prémisses.

    Deux règles $\mathsf{r}_1$ et $\mathsf{r}_2$ sont équivalentes dans un système $S$ si on peut prouver $\mathsf{r}_2$ dans $S \cup \{\mathsf{r}_1\}$ et si on peut prouver $\mathsf{r}_1$ dans $S \cup \{\mathsf{r}_2\}$.
    On peut généraliser cette définition à l'équivalence entre $n$ règles.
  \end{defn}

  \begin{exm}
    La règle $\mathsf{cut}$ définie par 
    \[
    \begin{prooftree}
      \hypo{\Gamma, A \vdash B}
      \hypo{\Gamma \vdash A}
      \infer 2[\mathsf{cut}]{\Gamma \vdash B}
    \end{prooftree}
    \] est dérivable en logique intuitionniste :
    \[
    \begin{prooftree}
      \hypo{\Gamma, A \vdash B}
      \infer 1[\to_\mathsf{i}]{\Gamma \vdash A \to B}
      \hypo{\Gamma \vdash A}
      \infer 2[\to_\mathsf{e}]{\Gamma \vdash B}
    \end{prooftree}
    .\] 
  \end{exm}

  \question{
    Quelle différence y a-t-il entre $\mathsf{abs}$ et $\lnot_\mathsf{i}$ ?
  }

  \question{
    Peut-on dériver $\mathsf{aff}$ depuis la déduction naturelle intuitionniste (sans $\mathsf{aff}$ bien sûr) ?
    Si oui, en intuitionniste aussi ?
  }

  \question{
    Montrer que les règles $\mathsf{te}$, $\mathsf{abs}$ et $\lnot\lnot_\mathsf{e}$ sont équivalentes.
    Par ordre de difficulté :
    \begin{itemize}
      \item $\lnot\lnot_\mathsf{e}$ à l'aide de $\mathsf{abs}$ 
      \item $\mathsf{abs}$ à l'aide de $\lnot\lnot_\mathsf{e}$ 
      \item $\mathsf{abs}$ à l'aide de $\mathsf{te}$ 
      \item $\lnot\lnot_\mathsf{e}$ à l'aide de $\mathsf{te}$
      \item $\mathsf{te}$ à l'aide de $\lnot\lnot_\mathsf{e}$ 
      \item $\mathsf{te}$ à l'aide de $\mathsf{abs}$
    \end{itemize}
  }

  Lorsqu'on a une règle dérivable dans un certain système de preuve, on peut donc l'utiliser comme si de rien n'était : en effet, il suffira de remplacer son utilisation par le morceau d'arbre donné par la dérivation de la règle.

  \question{
    Réaliser une preuve avec $\mathsf{cut}$ et remplacer l'utilisation par le morceau d'arbre donné ci-avant.
  }

  \section{Correction, complétude.}

  \begin{defn}
    On dit qu'un système de preuves est \textit{correct} si, pour toute formule $F$, si $\vdash F$ alors $\models F$.
    On dit qu'un système de preuves est \textit{complet} si, pour toute formule $F$, si $\models F$ alors $\vdash F$.
  \end{defn}

  Pour faire simple,
  \begin{itemize}
    \item dans un système correct, ce qui est prouvable est vrai ;
    \item dans un système complet, ce qui est vrai est prouvable.
  \end{itemize}

  \begin{thm}
    \begin{itemize}
      \item La déduction naturelle intuitionniste est incomplète mais correcte.
      \item La déduction naturelle classique est complète et correcte. \qed
    \end{itemize}
  \end{thm}

  On se propose de prouver le théorème ci-dessus en exercice.

  \subsection{Preuve de la correction (exercice).}

  On montre par induction sur l'arbre de preuve de $\Gamma\vdash F$, pour montrer que $\Gamma\models F$.

  Pour le cas de $\mathsf{ax}$, si $F \in \Gamma$ alors toute valuation satisfaisant $\Gamma$ satisfait les éléments de  $\Gamma$, en particulier $F$.

  Pour le cas de $\top_\mathsf{i}$, on a toujours $\Gamma\models\top$.

  Pour le cas de  $\land_\mathsf{i}$, si on a une valuation satisfaisant $\Gamma$, alors d'une part, on sait que cette valuation satisfait  $H$ par hypothèse d'induction (car $\Gamma \vdash H$), et aussi, cette valuation satisfait $G$ par hypothèse d'induction.
  Or, cette valuation satisfait $G \land H$ ssi elle satisfait $G$ et $H$ (par définition).
  D'où $\Gamma \models G \land H$.

  \question{
    Faire de même pour quelques uns des autres cas d'introduction ($\to_\mathsf{i}$, $\lor_\mathsf{i}^\mathsf{g}$ par exemple).
  } 

  Pour l'élimination, on peut regarder la règle $\to_\mathsf{e}$.
  Si on suppose avoir montré $\Gamma \vdash H \to G$ et $\Gamma \vdash H$, alors toute valuation satisfaisant $\Gamma$ satisfait $H \to G$ et $H$, donc elle satisfait $G$.

  \question{
    Faire de même pour quelques uns des autres cas d'introduction ($\lor_\mathsf{e}$, $\land_\mathsf{e}^\mathsf{d}$ par exemple).
  }

  En admettant que tous autres les cas se passent bien, on a montré la correction de la logique intuitionniste.

  Pour ce qui est de la logique classique, il reste à traiter les cas du tiers exclu, ou de la double négation, ou du raisonnement par l'absurde.
  Par exemple, on a toujours $\models F \lor \lnot F$ avec  $\mathsf{te}$ ou alors $\lnot \lnot F \models F$ avec  $\lnot \lnot_\mathsf{e}$.

  \question{Conclure quant à la partie "correction" (intuitionniste/classique) du théorème.}

  \subsection{Preuve de complétude (exercice).}

  On fixe $\mathcal{V} := \{x_1, \ldots, x_n\}$  l'ensemble des variables propositionnelles.
  Pour une valuation $v$ donnée, on construit la formule $\varphi(v)$ comme  \[
  \ell_1 \land (\ell_2 \land (\cdots (\ell_{n-1} \land \ell_n)\cdots ))
  ,\] 
  où $\ell_i := x_i$ si $v(x_i) = \textsc{Vrai}$ et  $\ell_i := \lnot x_i$ sinon.

  La formule $\varphi(v)$ est satisfaite par une et une seule valuation, $v$.

  On définit un ordre sur l'ensemble des valuations sur $\mathcal{V}$ induit en identifiant $\mathds{B}^\mathcal{V}$ et $\{0,1\}^n$, où l'on utilise l'ordre lexicographique.
  On note ainsi $\mathds{B}^\mathcal{V} := \{v_1, \ldots, v_{2^n}\}$ dans l'ordre induit.

  On note \[
  \Phi(\mathcal{V}) := \varphi(v_1) \lor (\varphi(v_2) \lor (\cdots ( \varphi(v_{2^n - 1}) \lor \varphi(v_{2^n}) )\cdots ))
  .\]

  On se place en logique classique.

  \question{
    Montrer que $\vdash \Phi(\mathcal{V})$.
    On pourra procéder par récurrence sur~$|\mathcal{V}|$.
  }

  \question{
    Démontrer que, pour tout valuation $v$ et toute formule $F$ à variables dans $\mathcal{V}$, si $v(F) = \textsc{Vrai}$, alors
    $\varphi(v) \vdash F$, et si $v(F) = \textsc{Faux}$, alors $\varphi(v) \models \lnot F$.
  }

  \question{
    Soit $F$ une tautologie à valeurs sur $\mathcal{V}$.
    Montrer que $\Phi(\mathcal{V})\models F$.
  }

  \question{
    Conclure.
  }

  \section{Équivalences de logiques (exercice).}

  On note $\mathbf{FLP}$ l'ensemble des formules de la logique propositionnelle (utilisant $\top$,  $\bot$, $\lnot$,  $\to $, $\land$ et  $\lor$).

  On dit que deux ensembles $A, B \subseteq \mathbf{FLP}$ sont équivalents si
  \begin{itemize}
    \item pour toute formule $\varphi \in A$, il existe $\psi \in B$ telle que pour tout $\Gamma$,
      \[
      \Gamma \models \varphi \text{ ssi } \Gamma \models \psi
      .\] 
    \item pour toute formule $\psi \in B$, il existe $\varphi \in A$ telle que pour tout $\Gamma$,
      \[
      \Gamma \models \psi \text{ ssi } \Gamma \models \varphi
      .\] 
  \end{itemize}

  On notera $\mathbf{FLP}_{o_1, o_2, \ldots, o_n}$ où les $o_i$ sont des opérateurs de $\mathbf{FLP}$ l'ensemble des formules de la logique propositionnelle n'utilisant que ces opérateurs.

  \question{
    Montrer que $\mathbf{FLP}$ est équivalente à $\mathbf{FLP}_{\lnot,\land}$.
  }

  \question{
    Justifier que $\mathbf{FLP}$ n'est pas équivalente à $\mathbf{FLP}_{\land,\lor,\to}$.
  }

  \indication{ Chercher un équivalent de la formule $\lnot p$. }

  L'équivalence ci-avant est une équivalence \textit{sémantique} (car $\models$).
  L'\textit{équivalence syntaxique} est comme l'équivalence sémantique mais où l'on remplace $\models$ par $\vdash$.

  Ceci dépend évidemment de la "version" de la déduction naturelle choisie : intuitionniste ou classique.
  Mais, vu ce le théorème de complétude/correction précédent pour la logique classique, il n'y a pas d'intérêt de différencier l'équivalence sémantique de l'équivalence syntaxique en logique classique.


  \question{ La logique $\mathbf{FLP}$ est-elle syntaxiquement équivalente (en logique intuitionniste) à $\mathbf{FLP}_{\lnot,\lor}$ ? }
  
  \indication{
    On pourra essayer de dériver le tiers-exclu depuis la formule équivalente à l'implication logique.
  }
\end{document}
