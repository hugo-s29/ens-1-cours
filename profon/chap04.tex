\documentclass[./main]{subfiles}

\begin{document}
  \chapter{Le $\lambda$-calcul polymorphe.}

  On étend la grammaire des types :
  \[
  A, B ::= X  \mid A \to B  \mid \forall X \: A
  .\] 
  Ici, les $X$ ne sont plus les types de base (noté $\boldsymbol{X}$ précédemment), mais ce sont des \textit{variables de types}.

  Pour mieux refléter les notations de la littérature, on aura plutôt :
  \[
  A, B ::= \alpha  \mid A \to B  \mid \forall \alpha \: A
  .\]

  Le "$\forall \alpha \: A$" est une structure qui lie, $\forall$ est un lieur.
  Il y a donc une notion de variables libres d'un type  $\Vl(A)$, d'$\alpha$-conversion, et de substitution.
  Par exemple, \[
  \Vl(\forall \alpha \: \forall \beta \: (\alpha \to  \beta \to \gamma \to \alpha)) = \{\gamma\} 
  \] et \[
  \forall \alpha \: \alpha \to \alpha =_\alpha \forall \beta \: \beta \to \beta
  .\]

  \section{Typage, version implicite.}
  Aux trois règles des types simples, on ajoute les deux règles ci-dessous :
  \[
  \begin{prooftree}
    \hypo{\Gamma \vdash M : A}
    \infer[left label={\alpha \not\in \Vl(\Gamma)}] 1[\mathcal{T}_\mathsf{g}]{\Gamma \vdash M : \forall \alpha \: A}
  \end{prooftree}
  \quad\quad
  \begin{prooftree}
    \hypo{\Gamma \vdash M : \forall \alpha \: A}
    \infer 1[\mathcal{T}_\mathsf{i}]{\Gamma \vdash M : A[\sfrac{B}{\alpha}]}
  \end{prooftree}
  .\]
  Les règles correspondent à la \textit{généralisation} et à l'\textit{instantiation}.

  Et, \textit{\textbf{on ne change pas les $\lambda$-termes}}.

  \section{Typage, version explicite.}

  On change les $\lambda$-termes :
  \[
  M, N ::= x  \mid M \: N  \mid \lambda x: A. \: M  \mid \Lambda \alpha. \: M \mid M \: A
  .\]
  La construction $\Lambda \alpha. \: M$ est une  \textit{abstraction de types}, et la construction $M\: A$ est l'\textit{application d'un terme à un type}.
  On note \textit{\textbf{explicitement}} le type "d'entrée" de l'abstraction.

  On change les règles de $\beta$-réduction :
  \[
  \begin{prooftree}
    \infer 0{(\Lambda \alpha. \: M) \: A \to_\beta M[\sfrac{A}{\alpha}]}
  \end{prooftree}
  \quad\quad
  \begin{prooftree}
    \hypo{M \to_\beta M'}
    \infer 1{\Lambda \alpha. \: M \to_\beta \Lambda \alpha. \: M'}
  \end{prooftree}
  \]\[
  \begin{prooftree}
    \hypo{M \to_\beta M'}
    \infer 1{M \: A \to_\beta M' \: A}
  \end{prooftree}
  .\]

  On change également les règles de typage :
  \begin{gather*}
    \begin{prooftree}
      \infer[left label={\Gamma(x) = A}] 0{\Gamma \vdash x : A}
    \end{prooftree}
    \quad\quad\quad
    \begin{prooftree}
      \hypo{\Gamma \vdash M : A \to B}
      \hypo{\Gamma \vdash N : A}
      \infer 2{\Gamma \vdash M \: N : B}
    \end{prooftree}\\
    \begin{prooftree}
      \hypo{\Gamma, x : A \vdash M : B}
      \infer[left label={x \not\in \operatorname{dom}(\Gamma)}] 1{\Gamma \vdash \lambda x : A.\: M : A \to B}
    \end{prooftree}\\
    \begin{prooftree}
      \hypo{\Gamma \vdash M : A}
      \infer[left label={\alpha \not\in \Vl(\Gamma)}] 1{\Gamma \vdash \Lambda \alpha.\: M : \forall \alpha \: A}
    \end{prooftree}
    \begin{prooftree}
      \hypo{\Gamma \vdash M : \forall \alpha\: A}
      \infer 1{\Gamma \vdash M \: B : A[\sfrac{B}{\alpha}]}
    \end{prooftree}
  \end{gather*}


  \begin{lem}
    On a $\Gamma \vdash M : A$ dans le système explicite ssi il existe $M'$ dans le système explicite avec $\Gamma \vdash M' : A$ (explicite) et $M$ est "l'effacement" de $M'$.
  \end{lem}

  \begin{itemize}
    \item La représentation implicite est plus utilisée dans les langages comme OCaml, où l'on doit inférer les types.
    \item La représentation explicite correspond plus aux langages comme Rocq, avec un lien plus naturel avec la logique.
  \end{itemize}

  \begin{exm}
    Soit le $\lambda$-terme non typé $\underline{2} := \lambda f. \: \lambda z.\: f \: (f \: z)$.
    Comment le typer en version explicite ?

    \begin{enumerate}
      \item \[
          \begin{prooftree}
            \hypo{\vdots}
            \infer 1{f : \alpha \to \alpha, z : \alpha \vdash f \: (f \: z) : \alpha}
            \infer 1{f : \alpha \to \alpha \vdash \lambda z : \alpha.\: f \: (f \: z) : \alpha \to \alpha}
            \infer 1{\emptyset \vdash \lambda f : \alpha \to \alpha. \: \lambda z : \alpha.\: f \: (f \: z) : (\alpha \to \alpha) \to \alpha \to \alpha}
            \infer 1{\emptyset \vdash \Lambda \alpha.\: \lambda f : \alpha \to \alpha. \: \lambda z : \alpha.\: f \: (f \: z) : \forall \alpha \: (\alpha \to \alpha) \to \alpha \to \alpha}
          \end{prooftree}
      .\] 
      \item \[
          \begin{prooftree}
            \hypo{\vdots}
            \infer 1{f : \forall  \alpha \: \alpha \to \alpha, z : \beta \vdash f \: \beta \: (f \: \beta\: z) : \beta}
            \infer 1{f : \forall  \alpha \: \alpha \to \alpha \vdash \lambda z : \beta.\: f \: \beta \: (f \: \beta\: z) : \beta \to \beta}
            \infer 1{\emptyset \vdash \lambda f : \forall \alpha \: \alpha \to \alpha. \: \lambda z : \beta. \: f \: \beta (f\: \beta \: z) : (\forall \alpha \: \alpha \to \alpha) \to \beta \to \beta}
          \end{prooftree}
      .\]
    \end{enumerate}
  \end{exm}

  \begin{exm}
    On suppose que l'on a les couples.
    Comment compléter le séquent \[
    y : \beta, z : \gamma \vdash \lambda f : \redQuestionBox. \: (f \: y, f \: z) : \redQuestionBox
    .\]
    \begin{itemize}
      \item Pour les types simples, on ne peut pas si $\beta \neq \gamma$.
      \item Mais, avec polymorphisme, on a : \[
        y : \beta, z : \gamma \vdash \lambda f : \forall  \alpha \: \alpha \to \delta. \: (f \: y, f \: z) : (\forall  \alpha \: \alpha \to \delta) \to \delta \times \delta
      .\]
    \end{itemize}
  \end{exm}

  \begin{exm}
    \[
    \begin{prooftree}
      \infer 0{x : \alpha \vdash x : \alpha}
      \infer 1{\emptyset \vdash \lambda x. \:x  : \alpha \to \alpha}
      \infer 1{\emptyset \vdash \lambda x. \:x : \forall  \alpha \: \alpha \to \alpha}
      \infer 1{\emptyset \vdash \lambda x. \: x : (\forall \beta \: \beta) \to (\forall \delta \: \delta)}
    \end{prooftree}
    .\]
    En effet,  $(\forall \beta \: \beta) \to (\forall \beta \: \beta) =_\alpha (\forall \beta \: \beta) \to (\forall \delta \: \delta)$.
  \end{exm}

  \begin{exm}[Ouch !]
    \[
    \begin{prooftree}
      \infer 0{x : \alpha \vdash x : \alpha}
      \infer[left label={\text{NON ! On a $\alpha \in \Vl(x : \alpha)$ !}}] 1{ x : \alpha \vdash x : \forall \alpha \: \alpha}
      \rewrite{\color{red}\box\treebox}
      \infer 1{x : \alpha \vdash x : \beta}
      \infer 1{\emptyset \vdash \lambda x.\: x : \alpha \to \beta}
      \infer 1{\emptyset \vdash \lambda x. \: x : \forall \alpha \: \forall \beta \: \alpha \to \beta}
    \end{prooftree}
    .\] 
  \end{exm}

  \section{Point de vue logique sur le polymorphisme, système F.}

  On se place du point de vue Curry-Howard. On ajoute deux règles de déduction supplémentaire :
  \[
  \begin{prooftree}
    \hypo{\Gamma \vdash A}
    \infer[left label={\alpha \not\in \Vl(\Gamma)}] 1[\forall_\mathsf{I}]{\Gamma \vdash \forall \alpha \: A}
  \end{prooftree}
  \quad\quad
  \begin{prooftree}
    \hypo{\Gamma \vdash \forall \alpha \: A}
    \infer 1[\forall_\mathsf{E}]{\Gamma \vdash A[\sfrac{B}{\alpha}]}
  \end{prooftree}
  .\]

  On a, encore, un possibilité d'éliminer les coupures : si $\alpha \not\in \Vl(\Gamma)$,

  \[
  \begin{prooftree}
    \hypo{\color{deepblue} \delta}
    \infer 1{\Gamma \vdash A}
    \infer 1[\forall_\mathsf{I}]{\Gamma \vdash \forall \alpha \: A}
    \infer 1[\forall_\mathsf{E}]{\Gamma \vdash A[\sfrac{{\color{nicered} B}}{\alpha}]}
  \end{prooftree}
  \quad\quad
  \leadsto
  \quad\quad
  \begin{prooftree}
    \hypo{{\color{deepgreen} \delta}[\sfrac{{\color{nicered} B}}{\alpha}]}
    \infer 1{\Gamma \vdash A[\sfrac{{\color{nicered} B}}{\alpha}]}
  \end{prooftree}
  \] 

  Ceci correspond, par correspondance de Curry-Howard, a une $\beta$-réduction :
  \[
    (\Lambda \alpha \: M) \: {\color{nicered} B} \to_\beta M[\sfrac{{\color{nicered} B}}{\alpha}]
  .\]

  On a aussi un théorème de normalisation forte.
  \begin{thm}
    Si $\Gamma \vdash M : A$ en $\lambda$-calcul polymorphe, alors $M$ est fortement normalisant.
  \end{thm}

  Pour démontrer ce théorème, on utilise encore les candidats de réductibilité (il ne faut définir que $\mathcal{R}_{\forall \alpha \: A}$), mais la preuve est bien plus complexe. En effet, les types polymorphes ont un grand pouvoir expressif (\textit{c.f.} la section suivante).

  Le système que l'on a s'appelle \textit{système F}.\footnote{À ne pas confondre avec \href{https://fr.wikipedia.org/wiki/Station_F}{\textit{Station F}}.}\showfootnote\ C'est de la logique du second ordre : on peut quantifier sur les variables propositionnelles.
  On quantifie donc sur des ensembles de valeurs au lieu de se limiter uniquement à quantifier sur les valeurs.

  \section{Représentation des connecteurs logiques.}

  On peut représenter les connecteurs logiques différemment, sans les ajouter explicitement, mais uniquement avec le fragment contenant $\to$ et $\forall$.

  \begin{itemize}
    \item On peut représenter $\bot  := \forall \alpha \: \alpha$. En effet, on a la correspondance suivante :
      \[
        \begin{prooftree}
          \hypo{\Gamma \vdash \forall \alpha \: \alpha}
          \infer 1[\forall_\mathsf{I}]{\Gamma \vdash A}
        \end{prooftree}
        \quad\leadsto\quad
        \begin{prooftree}
          \hypo{\Gamma \vdash \bot}
          \infer 1[\bot_\mathsf{E}]{\Gamma \vdash A}
        \end{prooftree}
      .\]
    \item On peut représenter $\top := \forall \alpha \: \alpha \to \alpha$.
    \item On peut représenter $A \land B := \forall \alpha \: (A \to B \to \alpha) \to \alpha$ :

      \fitbox{$ \hspace{-2cm}\begin{prooftree}
        \infer 0{\Gamma, c : A \to B \to \alpha \vdash c :A \to B \to \alpha}
        \hypo{\Gamma \vdash M  : A}
        \infer 1{\Gamma, c : A \to B \to \alpha \vdash M : A}
        \infer 2{\Gamma, c : A \to B \to \alpha \vdash c\: M : B \to \alpha}
        \hypo{\Gamma\vdash N  : B}
        \infer 1{\Gamma, c : A \to B \to \alpha \vdash M : A}
        \infer 2{\Gamma, c : A \to B \to \alpha \vdash c\: M \: N \: \alpha}
        \infer 1{\Gamma \vdash \lambda c. \: c \: M \: N : (A \to B \to \alpha) \to \alpha}
        \infer 1{\Gamma \vdash \lambda c. \: c \: M \: N : \forall  \alpha \: (A \to B \to \alpha) \to \alpha}
      \end{prooftree}$}

      d'où l'introduction du $\land$.
      Pour l'élimination, si \[\Gamma \vdash M : \forall \alpha \: (A \to B \to \alpha) \to \alpha,\] alors $\Gamma \vdash M \: (\lambda x. \: \lambda y.\: x) : A$ et $\Gamma \vdash M \: (\lambda x. \: \lambda y.\: y) : B$.
  \end{itemize}


  \section{Le $\lambda$-calcul polymorphe, côté programmation.}

  \begin{description}
    \item["Généricité".]
      On peut avoir plusieurs types en même temps.
      Par exemple, une fonction de tri a un type \[
      \forall \alpha \: (\alpha \to \alpha \to \mathtt{bool}) \to \mathtt{list}\ \alpha \to \mathtt{list} \ \alpha
      .\] 
    \item["Paramétricité".]
      Lorsque l'on a une fonction $f : (\forall \alpha : \alpha) \to A$, la fonction n'inspecte pas son argument.
      Elle ne fait que le dupliquer, le passer à d'autres fonctions, le rejeter, mais elle ne peut pas voir les données sous-jacentes.
      (On exclue ici les fonctions types \lstinline|Obj.magic|).
  \end{description}

  Quelques conséquences sur les formes normales :
  \begin{itemize}
    \item Il n'y a qu'une seule forme normale ayant un type $\forall \alpha \: \alpha \to \alpha$.
    \item Il n'y a que deux formes normales ayant un type $\forall \alpha \: \alpha \to \alpha \to \alpha$ : $\lambda u.\: \lambda v.\: u$ et $\lambda u.\: \lambda v.\: v$.
  \end{itemize}
  On a aussi quelques "théorèmes gratuits", comme par exemple : si~$\mathtt{f} : \alpha \to \beta$ est croissante vis à vis de $\mathtt{ordre}_\alpha : \alpha \to \alpha \to \mathtt{bool}$ et~$\mathtt{ordre}_\beta : \beta \to \beta \to \mathtt{bool}$, alors 
  \[
  \mathtt{map} \ \mathtt{f} \ (\mathtt{sort}\ \mathtt{ordre}_\alpha\ \mathtt{l}) = \mathtt{sort} \ \mathtt{ordre}_\beta \ (\mathtt{map}\ \mathtt{f}\ \mathtt{l})
  .\] 


  \section{Polymorphisme et inférence de types.}

  Pour l'inférence de types, on se place dans la représentation implicite du typage polymorphique.
  On s'intéresse à la question suivante : pour $M$ un $\lambda$-terme donné, existe-t-il $\Gamma$ et $A$ tels que $\Gamma \vdash M : A$ ?

  Cependant, il y a un problème :
  \begin{thm}[\~1992]
    L'inférence de types pour le typage polymorphique implicite est \textit{\textbf{indécidable}}.
  \end{thm}

  Ceci conclut l'étude du Système F. On va maintenant s'intéresser à une variant de système F développée en 1978, le \textit{$\lambda$-calcul avec schéma de types}.
  On note $\rho$ le schéma de types définis par :
  \[
  A ::= \alpha  \mid A \to B \quad\quad\quad \rho ::= A  \mid \forall \alpha \: \rho
  .\]

  \begin{rmk}
    Il n'y a pas de types de la forme $(\forall \alpha \: \alpha \to \alpha) \to (\forall \beta \: A)$.
    La quantification est \textit{\textbf{prénexe}}.
  \end{rmk}

  Les termes sont donnés par la grammaire :
  \[
  M ::= x  \mid \lambda x. \: M  \mid M \: N  \mid \letin x M N
  .\]

  Pour le système de types, on considère $\Gamma$ un dictionnaire sur (variables du $\lambda$-calcul $x$, schémas de types $\rho$).
  La relation de typage est notée~${\Gamma \vdash M : A}$ où $A$ est un type \textit{\textbf{simple}}.
  On la définit par induction à l'aide des règles :
  \begin{gather*}
    \begin{prooftree}
      \hypo{\Gamma \vdash M : A \to B}
      \hypo{\Gamma \vdash N : A}
      \infer 2{\Gamma \vdash M \: N : B}
    \end{prooftree}
    \quad\quad
    \begin{prooftree}
      \hypo{\Gamma, x : A \vdash M : B}
      \infer 1{\Gamma \vdash \lambda x.\: M : A \to B}
    \end{prooftree}\\
    \begin{prooftree}
      \hypo{\Gamma \vdash M : A}
      \hypo{\Gamma, x : \forall \vec{\alpha}\: A \vdash N : B}
      \infer[left label={\vec{\alpha}\cap \Vl(\Gamma) = \emptyset}] 2{\Gamma \vdash \letin x M N : B}
    \end{prooftree}
    \quad\quad
    \begin{prooftree}
      \infer[left label={\Gamma(x) = \forall \vec{\alpha} \: A}] 0{\Gamma \vdash x : A[\sfrac{\vec{B}}{\vec{\alpha}}]}
    \end{prooftree}
  \end{gather*}

  Ici, on note $\forall \vec{\alpha} \: A$ pour  $\forall \alpha_1\: \cdots\: \forall \alpha_n \: A$ avec possiblement aucun $\forall$. De même, la substitution $A[\sfrac{\vec{B}}{\vec{\alpha}}]$ se fait simultanément.

  La règle d'\textit{\textbf{instantiation}} de système F se retrouve partiellement dans la règle de l'axiome.
  La règle d'\textit{\textbf{généralisation}} de système F se retrouve uniquement dans la règle du $\letin \ldots \ldots \ldots$, et uniquement après le $\mathtt{in}$.


  \begin{exm}
    Le terme \[
      \letin f {\lambda x. \: x} f
    \] a pour type $\beta \to \beta$, sans généralisation !

    \[
    \begin{prooftree}
      \infer 0{x : \alpha \vdash x : \alpha}
      \infer 1{\emptyset \vdash \lambda x. \: x : \alpha \to \alpha}
      \infer 0{f : \forall \alpha \: \alpha \to \alpha \vdash f : \beta \to \beta}
      \infer 2{\emptyset\vdash \letin f {\lambda x. \: x} f : \beta \to \beta}
    \end{prooftree}
    .\] 
  \end{exm}


  En OCaml, lorsqu'on essaie de typer
  \begin{center}
    \lstinline|(fun z -> z) (fun x -> x)|
  \end{center}
  on se rend compte que l'on n'a pas un type \lstinline|'a -> 'a|, mais
  \begin{center}
    \lstinline|'_weak1 -> '_weak1|.
  \end{center}

  \begin{thm}
    Il existe un algorithme correct et complet pour l'inférence de types pour les schémas de types.
  \end{thm}
  
  Pour ce faire, on fait comme en \thprog 7, où l'on part de $M$, on génère des contraintes, et à l'aide de l'algorithme d'unification, on trouve le type "le plus général".
\end{document}
