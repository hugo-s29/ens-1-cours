\documentclass[./main]{subfiles}

\begin{document}
  \chapter{Introduction à la théorie de la démonstration.}

  \section{Formules et preuves.}

  \begin{defn}
    On se donne un ensemble de \textit{variables propositionnelles}, qui seront notées $X, Y, Z$, \textit{etc}.
    L'ensemble des \textit{formules} est défini par la grammaire :
    \[
      A, B ::= X  \mid A \Rightarrow B
    .\]
    Cet ensemble de formules s'appelle le "\textit{fragment implicatif de la logique propositionnelle intuitionniste}".
  \end{defn}

  Cela peut sembler inhabituel car, généralement, on commence par introduire $\lnot$, $\lor$ et $\land$, car on a en tête les booléens.

  \begin{defn}
    Les \textit{séquents}, notés $\Gamma \vdash A$, un couple formé de $\Gamma$ une  \textit{\textbf{liste}} de formules, et $A$ une formule.
    La liste $\Gamma$ est une \textit{liste d'hypothèses}.
    On notera $\Gamma, A$ la notation pour l'extension de la liste.
  \end{defn}

  \begin{defn}
    On  \textit{prouve} (\textit{dérive}) les séquents à l'aides des \textit{règles de déduction} (\textit{d'inférence}) :
    \[
    \begin{prooftree}
      \infer[left label={A \in \Gamma}] 0[\mathsf{Ax}]{\Gamma \vdash A}
    \end{prooftree}
    \quad
    \begin{prooftree}
      \hypo{\Gamma, A \vdash B}
      \infer 1[\Rightarrow_\mathsf{I}]{\Gamma \vdash A \Rightarrow B}
    \end{prooftree}
    \quad
    \begin{prooftree}
      \hypo{\Gamma \vdash A \Rightarrow B}
      \hypo{\Gamma \vdash A}
      \infer 2[\Rightarrow_\mathsf{E}]{\Gamma \vdash B}
    \end{prooftree}
    .\] 
    On les appelle, dans l'ordre, règle de l'\textit{axiome}, règle de l'\textit{introduction de $\Rightarrow$} et règle de l'\textit{élimination de $\Rightarrow$}.
    Il s'agit des \textit{règles de déduction naturelle pour le fragment implicatif de la logique propositionnelle intuitionniste}.
  \end{defn}

  \begin{defn}
    Le séquent $\Gamma \vdash A$ est \textit{prouvable} s'il existe une \textit{preuve} (\textit{dérivation}) ayant $\Gamma \vdash A$ à la racine et des axiomes aux feuilles. La formule $A$ est \textit{prouvable} si $\vdash A$ l'est.
  \end{defn}

  \begin{exm}
    \[
    \begin{prooftree}
      \infer 0[\mathsf{Ax}]{X \Rightarrow Y, X \vdash X \Rightarrow Y}
      \infer 0[\mathsf{Ax}]{X \Rightarrow Y, X \vdash X}
      \infer 2[\Rightarrow_\mathsf{E}]{X \Rightarrow Y, X \vdash Y}
      \infer 1[\Rightarrow_\mathsf{I}]{X \Rightarrow Y \vdash X \Rightarrow Y}
      \infer 1[\Rightarrow_\mathsf{I}]{\vdash (X \Rightarrow Y) \Rightarrow (X \Rightarrow Y)}
    \end{prooftree}
    .\] 
  \end{exm}

  On écrit généralement des "preuves génériques", en utilisant $A, B$ plutôt que $X, Y$.

  \begin{exm}
    \[
    \begin{prooftree}
      \infer 0[\mathsf{Ax}]{(A \Rightarrow A) \Rightarrow B \vdash (A \Rightarrow A) \Rightarrow B}
      \infer 0[\mathsf{Ax}]{(A \Rightarrow A) \Rightarrow B, A \vdash A}
      \infer 1[\Rightarrow_\mathsf{I}]{(A \Rightarrow A) \Rightarrow B \vdash A \Rightarrow A}
      \infer 2[\Rightarrow_\mathsf{E}]{(A \Rightarrow A) \Rightarrow B \vdash B}
      \infer 1[\Rightarrow_\mathsf{I}]{\vdash ((A \Rightarrow A) \Rightarrow B) \Rightarrow B}
    \end{prooftree}
    .\]
  \end{exm}

  \section{Et en Rocq ?}

  En Rocq, un objectif de preuve 
  \[
    \begin{array}{c}
      {\left.\begin{array}{rl}
        H_1 &: A_1\\
        H_2 &: A_2\\
        H_3 &: A_3\\
        &\vdots\\ \hline \hline
      \end{array}
    \right\} \Gamma}\\
          A
    \end{array}
  \] correspond au séquent
  \[
  \Gamma \vdash A
  .\] 

  Chaque tactique correspond à des opérations sur l'arbre de preuve.
  On construit "au fur et à mesure" l'arbre de preuve montrant $\Gamma \vdash A$.
  Voici ce que quelques tactiques Rocq font.
  \newcommand\arrowspacing{\phantom{aaaaaaaa}}
  \begin{align*}
    \begin{prooftree}
      \hypo{??}
      \infer 1{\Gamma, A, B, A \vdash A}
    \end{prooftree}&
    \quad \xrightarrow[\mathtt{assumption}]{\arrowspacing} \quad
    \begin{prooftree}
      \infer 0[\mathsf{Ax}]{\Gamma, A, B, A \vdash A}
    \end{prooftree}\\
    \begin{prooftree}
      \hypo{??}
      \infer 1{\Gamma \vdash C}
    \end{prooftree}&
    \quad \xrightarrow[\mathtt{assert}\ A]{\arrowspacing} \quad
    \begin{prooftree}
      \hypo{??}
      \infer 1{\Gamma, A \vdash B}
      \infer 1{\Gamma \vdash A \Rightarrow B}
      \hypo{??}
      \infer 1{\Gamma \vdash A}
      \infer 2[\Rightarrow_\mathsf{E}]{\Gamma \vdash B}
    \end{prooftree}\\
    \begin{prooftree}
      \hypo{??}
      \infer 1{\Gamma \vdash C}
    \end{prooftree}&
    \quad \xrightarrow[\mathtt{cut}\ A]{\arrowspacing} \quad
    \begin{prooftree}
      \hypo{??}
      \infer 1{\Gamma \vdash A \Rightarrow B}
      \hypo{??}
      \infer 1{\Gamma \vdash A}
      \infer 2[\Rightarrow_\mathsf{E}]{\Gamma \vdash B}
    \end{prooftree}\\
    \begin{prooftree}
      \hypo{??}
      \infer 1{\Gamma \vdash C}
    \end{prooftree}&
    \quad \xrightarrow[\mathtt{apply}\ H]{\arrowspacing} \quad
    \begin{prooftree}
      \infer 0[\mathsf{Ax}]{\Gamma, A \Rightarrow B \vdash A \Rightarrow B}
      \hypo{??}
      \infer 1{\Gamma, A \Rightarrow B \vdash A}
      \infer 2[\Rightarrow_\mathsf{E}]{\Gamma, \underbrace{A \Rightarrow B}_H \vdash B}
    \end{prooftree}\\
    \begin{prooftree}
      \hypo{??}
      \infer 1{\Gamma \vdash B \Rightarrow C}
    \end{prooftree}&
    \quad \xrightarrow[\mathtt{intro}]{\arrowspacing} \quad
    \begin{prooftree}
      \hypo{??}
      \infer 1{\Gamma, B \vdash C}
      \infer 1[\Rightarrow_\mathsf{I}]{\Gamma \vdash B \Rightarrow C}
    \end{prooftree}
  \end{align*}

  \section{Liens avec le $\lambda$-calcul simplement typé : \textit{correspondance de Curry-Howard}.}

  Les règles de typage du $\lambda$-calcul correspondent au règles d'inférences du fragment implicatif :
  \begin{align*}
    \begin{prooftree}
      \infer[left label={{\color{deepblue}x : {}} A \in \Gamma}] 0{\Gamma \vdash {\color{deepblue}x : {}} A}
    \end{prooftree} \quad \quad
    &\xleftrightarrow{\arrowspacing\:} \quad \quad
    \begin{prooftree}
      \infer[left label={A \in \Gamma}] 0[\mathsf{Ax}]{\Gamma \vdash A}
    \end{prooftree}\\
    \begin{prooftree}
      \hypo{\Gamma , {\color{deepblue} x : {}} A \vdash {\color{deepblue} M : {}} B}
      \infer 1{\Gamma \vdash {\color{deepblue} \lambda x.\: M : {}} A \to B}
    \end{prooftree} \quad \quad
    &\xleftrightarrow{\arrowspacing\:} \quad \quad
    \begin{prooftree}
      \hypo{\Gamma, A \vdash B}
      \infer 1[\Rightarrow_\mathsf{I}]{\Gamma \vdash A \Rightarrow B}
    \end{prooftree}\\
    \begin{prooftree}
      \hypo{\Gamma \vdash {\color{deepblue} M : {}} A \to B}
      \hypo{\Gamma \vdash {\color{deepblue} N : {}} A}
      \infer 2{\Gamma \vdash {\color{deepblue} M\: N : {}} B}
    \end{prooftree} \quad \quad
    &\xleftrightarrow{\arrowspacing\:} \quad \quad
    \begin{prooftree}
      \hypo{\Gamma, A \vdash B}
      \infer 1[\Rightarrow_\mathsf{I}]{\Gamma \vdash A \Rightarrow B}
    \end{prooftree}\\
  .\end{align*}
  En retirant les $\lambda$-termes en bleu (incluant les "$:$"), et en changeant~$\to$ en $\Rightarrow$, on obtient exactement les mêmes règles.

  Si on sait que $\Gamma \vdash {\color{deepblue} x : {}} A$ alors, en effaçant les parties en bleu, on obtient une preuve de $\tilde{\Gamma} \vdash A$.

  \textit{\textbf{Inversement}}, on se donne une preuve de $\Gamma \vdash A$. On se donne des variables $x_i$ pour transformer $\Gamma = A_1, \ldots, A_k$ en $\hat{\Gamma} = {x_1 : A_1}, \ldots, {x_k : A_k}$.
  Par induction sur $\Gamma \vdash A$, on montre qu'il existe un $\lambda$-terme tel que $\hat{\Gamma} \vdash M : A$. On a trois cas.
  \begin{itemize}
    \item Pour $\Rightarrow_\mathsf{I}$, par induction, si $\hat{\Gamma}, x = A \vdash M : B$, on déduit $\hat{\Gamma} \vdash \lambda x.\: M : A \to B$.
    \item Pour $\Rightarrow_\mathsf{I}$, par induction, si $\hat{\Gamma} \vdash M : A \to B$ et $\hat{\Gamma} \vdash N : A$, on déduit $\hat{\Gamma} \vdash  M \: N : B$.
    \item Pour $\mathsf{Ax}$, on sait $A \in \Gamma$ donc il existe $x$ tel que $x : A \in \hat{\Gamma}$, et on conclut $\hat{\Gamma} \vdash x_i : A$.
  \end{itemize}

  On a les propriétés suivantes pour la relation de déduction :
  \begin{itemize}
    \item \textit{affaiblissement} : si $\Gamma \vdash B$ (implicitement "est prouvable") alors $\Gamma, A \vdash B$ ;
    \item \textit{contraction} : si $\Gamma, A, A \vdash B$ alors $\Gamma, A \vdash B$ ;
    \item \textit{renforcement} si $\Gamma, {\color{nicered} A} \vdash B$ alors $\Gamma \vdash B$ \textit{à condition qu'on n'utilise pas l'axiome avec l'hypothèse {\color{nicered} A} (celle là uniquement, les $A$ intermédiaires ne posent pas de problèmes) pour déduire $B$}.
    \item \textit{échange} ; su $\Gamma, A, B, \Gamma' \vdash C$ alors $\Gamma, B, A, \Gamma' \vdash C$.
  \end{itemize}
  C'est analogue aux propriétés du typage en $\lambda$-calcul.

  En effet, la propriété de renforcement, très imprécise dans sa formulation logique, est simplement : si $\hat{\Gamma}, {\color{nicered} x : A} \vdash M : B$ alors $\hat{\Gamma} \vdash M : B$ \textit{à condition que $x\not\in \Vl(M)$}.

  Si on veut prouver ces propriétés (au lieu d'utiliser la correspondance de Curry-Howard), on ferait une induction sur la preuve du séquent qui est donné.

  La règle \[
  \begin{prooftree}
    \hypo{\Gamma \vdash B}
    \infer 1[\mathsf{aff}]{\Gamma, A \vdash B}
  \end{prooftree}
  \] 
  est \textit{admissible}.
  En effet, si on sait prouver les prémisses (ici, $\Gamma \vdash B$) alors on sait prouver la conclusion (ici, $\Gamma, A \vdash B$).
  Ceci dépend fortement de la logique que l'on utilise.

  \section{Curry-Howard du côté calcul : les coupures.}

  Typons un redex :
  \[
  \begin{prooftree}
    \hypo{\Gamma, x : A \vdash M : B}
    \infer 1[\Rightarrow_\mathsf{I}]{\lambda x.\: M : A \to B}
    \hypo{\Gamma \vdash N : A}
    \infer 2[\Rightarrow_\mathsf{E}]{\Gamma \vdash (\lambda x.\: M) \: N : M}
  \end{prooftree}
  .\]
  Oui, c'est exactement la même chose que la tactique \texttt{assert} en Rocq.

  \begin{defn}
    Une \textit{coupure} est un endroit dans la preuve où il y a un usage d'une règle d'élimination ($\Rightarrow_\mathsf{E}$) dont la prémisse principale est déduite à l'aide d'une règle d'introduction ($\Rightarrow_\mathsf{I}$) pour le même connecteur logique.
  \end{defn}
  \begin{rmk}
    Ici, on n'a qu'un seul connecteur logique, $\Rightarrow$, mais cela s'étend aux autres connecteurs que l'on pourrait ajouter.
    La \textit{prémisse principale} est, par convention, la première.
  \end{rmk}

  On peut \textit{éliminer une coupure} pour $\Rightarrow$, c'est-à-dire transformer une preuve (\textit{c.f.} contracter un $\beta$-redex) en passant de 
  \[
  \begin{prooftree}
    \hypo{\color{deepgreen} \delta}
    \infer 1{\Gamma, {\color{nicered} A} \vdash B}
    \infer 1[\Rightarrow_\mathsf{I}]{\Gamma \vdash {\color{nicered} A} \Rightarrow B}
    \hypo{\color{deepblue} \delta'}
    \infer 1{\Gamma \vdash {\color{nicered} A}}
    \infer 2[\Rightarrow_\mathsf{E}]{\Gamma \vdash B}
  \end{prooftree}
  \] à
  \[
    \begin{prooftree}
      \hypo{{\color{deepgreen} \delta}[\sfrac{{\color{deepblue} \delta'}}{{\color{nicered} A}}]}
      \infer 1{\Gamma \vdash B}
    \end{prooftree}
  \] 
  où l'on note ${\color{deepgreen} \delta}[\sfrac{{\color{deepblue} \delta'}}{{\color{nicered} A}}]$ la preuve obtenue en remplaçant dans ${\color{deepgreen} \delta}$ chaque usage de l'axiome avec ${\color{nicered} A}$ par ${\color{deepblue} \delta'}$.

  On a le même séquent en conclusion (\textit{c.f.} préservation du typage en $\lambda$-calcul simplement typé).

  La correspondance de Curry-Howard c'est donc :
  \[
  \begin{array}{rcl}
    \text{Types} & \longleftrightarrow & \text{Formules}\\
    \text{Programmes} & \longleftrightarrow & \text{Preuves}\\
    \phantom{aaaaaaaaaaaaaa}\text{$\beta$-réduction} & \longleftrightarrow & \text{Élimination d'une coupure}\\
    \text{\textbf{\textit{Programmation}}} & \longleftrightarrow & \text{\textbf{\textit{Logique}}}\.\
  \end{array}
  \]

  \section{Faux, négation, consistance.}
  On modifie nos formules :
  \[
  A, B ::= X  \mid A \Rightarrow B  \mid \bot
  \]
  et on ajoute la règle d'élimination du $\bot$ (il n'y a pas de règle d'introduction) :
  \[
  \begin{prooftree}
    \hypo{\Gamma \vdash \bot}
    \infer 1[\bot_\mathsf{E}]{\Gamma \vdash A}
  \end{prooftree}
  .\]

  La négation $\lnot A$ est une notation pour  $A \Rightarrow \bot$.
  On peut donc prouver le séquent $\vdash A \Rightarrow \lnot \lnot A$ :
  \[
  \begin{prooftree}
    \infer 0[\mathsf{Ax}]{A, \lnot A \vdash \lnot A}
    \infer 0[\mathsf{Ax}]{A, \lnot A \vdash A}
    \infer 2[\Rightarrow_\mathsf{E}]{A, \lnot A \vdash \bot}
    \infer 1[\Rightarrow_\mathsf{I}]{A \vdash \lnot \lnot A}
    \infer 1[\Rightarrow_\mathsf{I}]{\vdash A \Rightarrow \lnot \lnot A}
  \end{prooftree}
  .\]

  \begin{thm}[Élimination des coupures]
    Si $\Gamma \vdash A$ (est prouvable) alors il existe une preuve \textit{\textbf{sans coupure}} de $\Gamma \vdash A$.
  \end{thm}
  \begin{prv}
    \textit{c.f.} TD.
  \end{prv}

  \begin{rmk}[Lien avec normalisation forte en $\lambda$-calcul simplement typé]
    Ici, on veut la normalisation faible ("il existe une  forme normale \ldots"). On ne peut pas appliquer \textit{stricto sensu} la normalisation forte pour le $\lambda$-calcul simplement typé car le système de type contient $\bot$.
  \end{rmk}

  \begin{lem}
    Une preuve sans coupure de $\vdash A$ en logique intuitionniste se termine (à la racine) nécessairement par une règle d'introduction.
  \end{lem}
  \begin{prv}
    Par induction sur $\vdash A$. Il y a 4 cas.
    \begin{itemize}
      \item $\mathsf{Ax}$ : absurde car $\Gamma = \emptyset$.
      \item $\Rightarrow_\mathsf{I}$ : OK
      \item $\Rightarrow_\mathsf{E}$ : on récupère une preuve de $\vdash B \Rightarrow A$ qui termine (par induction) par une introduction $\Rightarrow_\mathsf{I}$. Absurde car c'est une coupure.
      \item $\bot_\mathsf{E}$ : On récupère une preuve de $\bot$ qui termine par une règle d'induction : impossible.
    \end{itemize}
  \end{prv}

  \begin{crlr}[Consistance de la logique]
    Il n'y a pas de preuve de $\vdash$ en logique propositionnelle intuitionniste dans le fragment avec $\Rightarrow$ et  $\bot$.
  \end{crlr}
  \begin{prv}
    S'il y en avait une, il y en aurait une sans coupure, qui se termine par une règle d'introduction, impossible.
  \end{prv}

  \section{Et en Rocq ? (partie 2)}

  On étend les formules avec $\forall$, $\exists$, $\lnot$, $\lor$, $\land$, \textit{etc}.
  Les preuves sont des $\lambda$-termes.
  En effet, dans une preuve de $\vdash X \to X \to X$ on peut écrire \[
    \mathtt{exact}\ (\fun{x\: y}{x})
  ,\]
  pour démontrer le séquent.

  Le mot clé \texttt{Qed} prend le $\lambda$-terme construit par la preuve et calcule $M'$ sous forme normale tel que $M \to_\beta^\star M'$.
  La logique de Rocq est \textit{constructive}. C'est-à-dire qu'une preuve de $A \Rightarrow B$ c'est une fonction qui transforme une preuve de $A$ en une preuve de $B$.
  Après avoir appelé  \texttt{Qed}, il est possible d'extraire le $\lambda$-terme construit en un programme OCaml, Haskell, \textit{etc}.
\end{document}
