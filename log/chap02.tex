\documentclass[./main]{subfiles}

\begin{document}
  \chapter{La logique du premier ordre.}

  \section{Les termes.}

  On commence par définir les \textit{termes}, qui correspondent à des objets mathématiques.
  Tandis que les formules relient des termes et correspondent plus à des énoncés mathématiques.

  \begin{defn}
    Le langage $\mathcal{L}$ (du premier ordre)
    est la donnée d'une famille (pas nécessairement finie) de symboles de trois sortes :
    \begin{itemize}
      \item les symboles de \textit{constantes}, notées $c$ ;
      \item les symboles de \textit{fonctions}, avec un entier associé, leur \textit{arité}, notées $f(x_1, \ldots, x_n)$ où $n$ est l'arité ;
      \item les symboles de \textit{relations}, avec leur arité, notées $R$, appelés \textit{prédicats}.
    \end{itemize}
    Les trois ensembles sont disjoints.
  \end{defn}

  \begin{rmk}
    \begin{itemize}
      \item Les constantes peuvent être vues comme des fonctions d'arité $0$.
      \item On aura toujours dans les relations : "$=$" d'arité $2$, et "$\bot$" d'arité $0$.
      \item On a toujours un ensemble de variables $\mathcal{V}$.
    \end{itemize}
  \end{rmk}

  \begin{exm}
    Le langage $\mathcal{L}_\mathrm{g}$ de la théorie des groupes est défini par :
    \begin{itemize}
      \item une constante : $c$,
      \item deux fonctions : $f_1$ d'arité $2$ et $f_2$ d'arité $1$ ;
      \item la relation $=$.
    \end{itemize}
    Ces symboles sont notés usuellement $e$, $*$, $\square^{-1}$ ou bien $0$, $+$, $-$.
  \end{exm}

  \begin{exm}
    Le langage $\mathcal{L}_\mathrm{co}$ des corps ordonnés est défini par :
    \begin{itemize}
      \item deux constantes $0$ et $1$,
      \item quatre fonctions $+$, $\times$, $-$ et $\square^{-1}$,
      \item deux relations $=$ et $\le$.
    \end{itemize}
  \end{exm}

  \begin{exm}
    Le langage $\mathcal{L}_\mathrm{ens}$ de la théorie des ensembles est défini par :
    \begin{itemize}
      \item une constante $\emptyset$,
      \item trois fonctions $\cap$, $\cup$ et $\square^\mathsf{c}$,
      \item trois relations $=$, $\in$ et $\subseteq$.
    \end{itemize}
  \end{exm}

  \begin{defn}
    \begin{description}
      \item[Par le haut.]
        L'ensemble $\mathcal{T}$ des termes sur le langage $\mathcal{L}$ est le plus petit ensemble de mots sur $\mathcal{L} \cup \mathcal{V} \cup \{\verb|(|, \verb|)|, \verb|,|\}$ tel
        \begin{itemize}
          \item qu'il contienne $\mathcal{V}$ et les constantes ;
          \item qui est stable par application des fonctions, c'est-à-dire que pour des termes $t_1, \ldots, t_n$ et un symbole de fonction $f$ d'arité $n$, alors $f(t_1, \ldots, t_n)$ est un terme.\footnote{Attention : le "$\ldots$" n'est pas un terme mais juste une manière d'écrire qu'on place les termes à côté des autres.}
        \end{itemize}
      \item[Par le bas.]
        On pose \[
          \mathcal{T}_0 = \mathcal{V} \cup \{c  \mid c \text{ est un symbole de constante de } \mathcal{L} \} 
        ,\] 
        puis \[
        \mathcal{T}_{k+1} = \mathcal{T}_k \cup \mleft\{\,f(t_1, \ldots t_n) \;\middle|\;
        \begin{array}{c}
          f \text{ fonction d'arité } n\\
          t_1, \ldots, t_n \in \mathcal{T}_k
        \end{array}\,\mright\}  
        ,\] et enfin \[
        \mathcal{T} = \bigcup_{n \in \mathds{N}} \mathcal{T}_n
        .\]
    \end{description}
  \end{defn}

  \begin{rmk}
    Dans la définition des termes, un n'utilise les relations.
  \end{rmk}

  \begin{exm}
    \begin{itemize}
      \item Dans $\mathcal{L}_\mathrm{g}$, $*(*(x,\square^{-1}(y)), e)$ est un terme, qu'on écrira plus simplement en $(x * y^{-1}) * e$.
      \item Dans $\mathcal{L}_\mathrm{co}$, $(x + x) + (-0)^{-1}$ est un terme.
      \item Dans $\mathcal{L}_\mathrm{ens}$, $(\emptyset^\mathsf{c} \cup \emptyset) \cap (x \cup y)^\mathsf{c}$ est un terme.
    \end{itemize}
  \end{exm}

  \begin{defn}
    Si $t$ et $u$ sont des termes et $x$ est une variable, alors~$t[x:u]$ est le mot dans lequel les lettres de  $x$ ont été remplacées par le mot $u$.
    Le mot $t[x:u]$ est un terme (preuve en exercice).
  \end{defn}

  \begin{exm}
    Avec $t = (x * y^{-1}) * e$ et $u = x * e$, alors on a \[
      t[x:u] = ((x*e) * y^{-1}) * e
    .\]
  \end{exm}

  \begin{defn}
    \begin{itemize}
      \item Un terme \textit{clos} est un terme sans variable (par exemple $(0 + 0)^{-1}$).
      \item La \textit{hauteur} d'un terme est le plis petit $k$ tel que $t \in \mathcal{T}_k$.
    \end{itemize}
  \end{defn}

  \begin{exo}
    \begin{itemize}
      \item Énoncer et prouver le lemme de lecture unique pour les termes.
      \item Énoncer et prouver un lemme de bijection entre les termes et un ensemble d'arbres étiquetés.
    \end{itemize}
  \end{exo}

  \section{Les formules.}

  \begin{defn}
    \begin{itemize}
      \item Les formules sont des mots sur l'alphabet \[ \mathcal{L} \cup \mathcal{V} \cup \{\verb|(|, \verb|)|, \verb|,|, \exists, \forall, \land, \lor, \lnot, \to \} .\] 
      \item Une \textit{formule atomique} est une formule de la forme $R(t_1, \ldots, t_n)$ où $R$ est un symbole de relation d'arité $n$ et $t_1, \ldots, t_n$ des termes.
      \item L'ensemble des \textit{formules} $\mathcal{F}$ du langage $\mathcal{L}$ est défini par 
        \begin{itemize}
          \item on pose $\mathcal{F}_0$ l'ensemble des formules atomiques ;
          \item on pose $\mathcal{F}_{k+1} = \mathcal{F}_k \cup \mleft\{\,
              \begin{array}{c}
                (\lnot F)\\
                (F \to G)\\
                (F \lor G)\\
                (F \land G)\\
                \exists x\: F\\
                \exists x\: G\\
              \end{array}
            \;\middle|\; 
            \begin{array}{c}
              F,G \in \mathcal{F}_k\\
              x \in \mathcal{V}
            \end{array}
          \,\mright\}$ ;
        \item et on pose enfin $\mathcal{F} = \bigcup_{n \in \mathds{N}} \mathcal{F}_n$.
        \end{itemize}
    \end{itemize}
  \end{defn}

  \begin{exo}
    La définition ci-dessus est "par le bas". Donner une définition par le haut de l'ensemble $\mathcal{F}$.
  \end{exo}

  \begin{exm}
    \begin{itemize}
      \item Dans $\mathcal{L}_\mathrm{g}$, un des axiomes de la théorie des groupes s'écrit \[
        \forall x\: \exists x\: (x * y = e \land y * x = e)
        .\]
      \item Dans $\mathcal{L}_\mathrm{co}$, l'énoncé "le corps est de caractéristique 3" s'écrit \[
        \forall x\: (x + (x + x) = 0)
        .\] 
      \item Dans $\mathcal{L}_\mathrm{ens}$, la loi de De Morgan s'écrit \[
          \forall x\: \forall y\: (x^\mathsf{c} \cup y^\mathsf{c} = (x \cap y)^\mathsf{c})
        .\]
    \end{itemize}
  \end{exm}

  \begin{exo}
    \begin{itemize}
      \item Donner et montrer le lemme de lecture unique.
      \item Énoncer et donner un lemme d'écriture en arbre.
    \end{itemize}
  \end{exo}

  \begin{rmk}[Conventions d'écriture.]
    On note :
    \begin{itemize}
      \item $x \le  y$ au lieu de ${\le}(x,y)$ ;
      \item $\exists x \ge 0\: (F)$ au lieu de $\exists x\: (x \ge 0 \land F)$ ;
      \item $\forall x \ge 0\: (F)$ au lieu de $\forall x\: (x \ge 0 \to F)$ ;
      \item $A \leftrightarrow B$ au lieu de $(A \to B) \land (B \to A)$ ;
      \item $t \neq u$ au lieu de $\lnot (t = u)$.
    \end{itemize}

    On enlèves les parenthèses avec les conventions de priorité 
    \begin{enumerate}
      \item[0.] les symboles de relations (le plus prioritaire) ;
      \item les symboles $\lnot, \exists, \forall$ ;
      \item les symboles $\land$ et $\lor$ ;
      \item le symbole $\to$ (le moins prioritaire).
    \end{enumerate}
  \end{rmk}

  \begin{exm}
    Ainsi, $\forall x \: A \land B \to \lnot C \lor D$ s'écrit \[
      (((\forall x\:A) \land B) \to ((\lnot C) \lor D))
    .\]
  \end{exm}

  \begin{rmk}
    Le calcul propositionnel est un cas particulier de la logique du premier ordre
    où l'on ne manipule que des relations d'arité $0$ (pas besoin des fonctions et des variables) : les "variables" du calcul propositionnel sont des formules atomiques ; et on n'a pas de relation "$=$".
  \end{rmk}

  \begin{rmk}
    On ne peut pas exprimer \textit{a priori} :
    \begin{itemize}
      \item  des quantifications sur en ensemble\footnote{En dehors de $\mathcal{L}_\mathrm{ens}$, en tout cas.} ;
      \item "$\exists n\ \exists x_1\: \ldots \: \exists x_n $" une formule qui dépend d'un paramètre ;
      \item le principe de récurrence : si on a $\mathcal{P}(0)$ pour une propriété $\mathcal{P}$ et que si $\mathcal{P}(n) \to \mathcal{P}(n+1)$ alors on a $\mathcal{P}(n)$ pour tout $n$.
    \end{itemize}
  \end{rmk}

  Quelques définitions techniques qui permettent de manipuler les formules.

  \begin{defn}
    L'ensemble des sous-formules de $F$, noté $\mathrm{S}(F)$ est défini par induction :
    \begin{itemize}
      \item si $F$ est atomique, alors on définit $\mathrm{S}(F) = \{F\}$ ;
      \item si $F = F_1 \oplus F_2$ (avec $\oplus$ qui est $\lor$, $\to$ ou $\land$)
         alors on définit~ $\mathrm{S}(F) = \mathrm{S}(F_1) \cup \mathrm{S}(F_2) \cup \{F\}$ ;
      \item si $F = \lnot F_1$, ou $F = \mathsf{Q}x\: F_1$ avec~$\mathsf{Q} \in \{\forall, \exists\}$, alors on définit $\mathrm{S}(F) = \mathrm{S}(F_1) \cup \{F\}$.
    \end{itemize}
    C'est l'ensemble des formules que l'on voit comme des sous-arbres de l'arbre équivalent à la formule $F$.
  \end{defn}

  \begin{defn}
    \begin{itemize}
      \item La \textit{taille} d'une formule, est le nombre de connecteurs ($\lnot$,  $\lor$,  $\land$,  $\to$), et de quantificateurs ($\forall $, $\exists $).
      \item La racine de l'arbre est 
        \begin{itemize}
          \item rien su la formule est atomique ;
          \item "$\oplus$" si $F = F_1 \oplus F_2$ avec $\oplus$ un connecteur (binaire ou unaire) ;
          \item "$\mathsf{Q}$" si $F = \mathsf{Q}x\: F_1$ avec $\mathsf{Q}$ un quantificateur.
        \end{itemize}
    \end{itemize}
  \end{defn}

  \begin{defn}
    \begin{itemize}
      \item Une \textit{occurrence} d'une variable est un endroit où la variable apparait dans la formule (\textit{i.e.} une feuille étiquetée par cette variable).
      \item Une occurrence d'une variable est \textit{liée} si elle se trouve dans une sous-formule dont l'opérateur principal est un quantificateur appelé à cette variable (\textit{i.e.} un $\forall x\: F'$ ou un $\exists x\: F'$).
      \item Une occurrence d'une variable est \textit{libre} quand elle n'est pas liée.
      \item Une variable est libre si elle a au moins une occurrence libre, sinon elle est liée.
    \end{itemize}
  \end{defn}

  \begin{rmk}
    On note $F(x_1, \ldots, x_n)$ pour dire que les variables libres sont $F$ sont parmi $\{x_1, \ldots, x_n\}$.
  \end{rmk}

  \begin{defn}
    Une formule est \textit{close} si elle n'a pas de variables libres.
  \end{defn}

  \begin{defn}[Substitution]
    On note $F[x := t]$ la formule obtenue en remplaçant toutes les occurrences libres de  $x$ par $t$, après renommage éventuel des occurrences des variables liées de $F$ qui apparaissent dans $t$.
  \end{defn}

  \begin{defn}[Renommage]
    On donne une définition informelle et incomplète ici.
    On dit que les formules $F$ et $G$ sont $\alpha$-équivalentes si elle sont syntaxiquement identiques à un renommage près des occurrences liées des variables.
  \end{defn}

  \begin{exm}
    On pose \[
    F(x,z) := \forall y\:(x * y = y * z) \land \forall x\: (x * x = 1)
    ,\]
    et alors 
    \begin{itemize}
      \item $F(z,z) = F[x := z] = \forall y\:(z * y = y * z) \land \forall x\: (x * x = 1)$ ;
      \item $F(y^{-1}, x) = F[x := y^{-1}] = \forall {\color{nicered} u}\:(y^{-1} * {\color{nicered} u} = {\color{nicered} u} * z) \land \forall x\: (x * x = 1)$.
    \end{itemize}
    On a procédé à un renommage de $y$ à ${\color{red} u}$.
  \end{exm}

  \section{Les démonstrations en déduction naturelle.}

  \begin{defn}
    Un \textit{séquent} est un coupe noté $\Gamma \vdash F$ (où $\vdash $ se lit "montre" ou "thèse") tel que $\Gamma$ est un ensemble de formules appelé \textit{contexte} (\textit{i.e.} l'ensemble des hypothèses), la formule $F$ est la \textit{conséquence} du séquent.
  \end{defn}

  \begin{rmk}
    Les formules ne sont pas nécessairement closes. Et on note souvent $\Gamma$ comme une liste.
  \end{rmk}

  \begin{defn}
    On dit que $\Gamma \vdash F$ est \textit{prouvable}, \textit{démontrable} ou \textit{dérivable}, s'il peut être obtenu par une suite finie de règles (\textit{c.f.} ci-après).
    On dit qu'une formule $F$ est \textit{prouvable} si $\emptyset\vdash F$ l'est.
  \end{defn}

  \begin{defn}[Règles de la démonstration]
    Une règle s'écrit \[
    \begin{prooftree}
      \hypo{\text{\textit{prémisses} : des séquents}}
      \infer 1[\text{nom de la règle}] {\text{\textit{conclusion} : un séquent}}
    \end{prooftree}
    .\]

    \begin{description}
      \item[Axiome.]
        \[
        \begin{prooftree}
          \infer 0[\mathsf{ax}] {\Gamma, A \vdash A}
        \end{prooftree}
        \]
      \item[Affaiblissement.]
        \[
          \begin{prooftree}
            \hypo{\Gamma \vdash A}
            \infer 1[\mathsf{aff}] {\Gamma, B \vdash A}
          \end{prooftree}
        \]
      \item[Implication.]
        \[
        \begin{prooftree}
          \hypo{\Gamma, A \vdash B}
          \infer 1[\to_\mathsf{i}] {\Gamma \vdash A \to B}
        \end{prooftree}
        \quad\quad
        \begin{prooftree}
          \hypo{\Gamma \vdash A \to B}
          \hypo{\Gamma \vdash A}
          \infer 2[\to_\mathsf{e}\footnote{Aussi appelée \textit{modus ponens}}] {\Gamma \vdash B}
        \end{prooftree}
        \]
      \item[Conjonction.]
        \[
        \begin{prooftree}
          \hypo{\Gamma \vdash A}
          \hypo{\Gamma \vdash B}
          \infer 2[\land_\mathsf{i}] {\Gamma \vdash A \land B}
        \end{prooftree}
        \quad
        \begin{prooftree}
          \hypo{\Gamma \vdash A \land B}
          \infer 1[\lor_\mathsf{e}^\mathsf{g}]{\Gamma \vdash A}
        \end{prooftree}
        \quad
        \begin{prooftree}
          \hypo{\Gamma \vdash A \land B}
          \infer 1[\lor_\mathsf{e}^\mathsf{d}]{\Gamma \vdash B}
        \end{prooftree}
        \] 
      \item[Disjonction.]
        \[
        \begin{prooftree}
          \hypo{\Gamma \vdash A}
          \infer 1[\lor_\mathsf{i}^\mathsf{g}]{\Gamma \vdash A \lor B}
        \end{prooftree}
        \quad
        \begin{prooftree}
          \hypo{\Gamma \vdash B}
          \infer 1[\lor_\mathsf{i}^\mathsf{d}]{\Gamma \vdash A \lor B}
        \end{prooftree}
        \]~ \[
        \begin{prooftree}
          \hypo{\Gamma \vdash A \lor B}
          \hypo{\Gamma, A \vdash C}
          \hypo{\Gamma, B \vdash C}
          \infer 3[\lor_\mathsf{e}\footnote{C'est un raisonnement par cas}]{\Gamma \vdash C}
        \end{prooftree}
        .\] 
      \item[Négation.]
        \[
        \begin{prooftree}
          \hypo{\Gamma, A \vdash \bot}
          \infer 1[\lnot_\mathsf{i}]{\Gamma \vdash \lnot A}
        \end{prooftree}
        \quad\quad
        \begin{prooftree}
          \hypo{\Gamma \vdash A}
          \hypo{\Gamma \vdash \lnot A}
          \infer 2[\lnot_\mathsf{e}]{\Gamma \vdash \bot}
        \end{prooftree}
        \]
      \item[Absurdité classique.]
        \[
        \begin{prooftree}
          \hypo{\Gamma, \lnot A \vdash \bot}
          \infer 1[\bot_\mathsf{e}]{\Gamma \vdash A}
        \end{prooftree}
        \]
        (En logique intuitionniste, on retire l'hypothèse $\lnot A$ dans la prémisse.)
      \item[Quantificateur universel.]
        \[
        \begin{prooftree}
          \hypo{\Gamma \vdash A}
          \infer[left label=
          \begin{array}{r}
            \text{si $x$ n'est pas libre}\\
            \text{dans les formules de $\Gamma$}
          \end{array}
          ] 1[\forall_\mathsf{i}]{\Gamma \vdash \forall x \: A}
        \end{prooftree}
        \]~\[
        \begin{prooftree}
          \hypo{\Gamma \vdash \forall x\:A}
          \infer[left label=
          \begin{array}{r}
            \text{quitte à renommer les}\\
            \text{variables liées de $A$ qui}\\
            \text{apparaissent dans $t$}
          \end{array}
          ] 1[\forall_\mathsf{e}]{\Gamma \vdash A[x := t]}
        \end{prooftree}
        \]
      \item[Quantificateur existentiel.]
        \[
          \begin{prooftree}
            \hypo{\Gamma \vdash A[x := t]}
            \infer 1[\exists_\mathsf{i}]{\Gamma \vdash \exists x \: A}
          \end{prooftree}
         \]~
        \[
          \begin{prooftree}
            \hypo{\Gamma \vdash \exists x\: A}
            \hypo{\Gamma, A \vdash C}
            \infer[left label={
              \begin{array}{r}
                \text{avec $x$ ni libre dans $C$ ou}\\
                \text{dans les formules de $\Gamma$}
              \end{array}
            }] 2[\exists_\mathsf{e}]{\Gamma \vdash C}
          \end{prooftree}
         \]
    \end{description}
  \end{defn}
  % TODO


  \section{La sémantique.}

  \begin{defn}
    Soit $\mathcal{L}$ un langage de la sémantique du premier ordre.
    On appelle \textit{interprétation} (ou \textit{modèle}, ou \textit{structure}) du langage $\mathcal{L}$ l'ensemble $\mathcal{M}$ des données suivantes :
    \begin{itemize}
      \item un ensemble non vide, noté $|\mathcal{M}|$, appelé \textit{domaine} ou \textit{ensemble de base} de $\mathcal{M}$ ;
      \item pour chaque symbole $c$ de constante, un élément $c_\mathcal{M}$ de $|\mathcal{M}|$ ;
      \item pour chaque symbole $f$ de fonction $n$-aire, une fonction $f_\mathcal{M} : |\mathcal{M}|^n \to |\mathcal{M}|$ ;
      \item pour chaque symbole $R$ de relation $n$-aire (sauf pour l'égalité "$=$"), un sous-ensemble $R_\mathcal{M}$ de $|\mathcal{M}|^n$.
    \end{itemize}
  \end{defn}

  \begin{rmk}
    \begin{itemize}
      \item La relation "$=$" est toujours interprétée par la vraie égalité :
        \[
        \{(a,a)  \mid a \in |\mathcal{M}|\} 
        .\]
      \item On note, par abus de notation, $\mathcal{M}$ pour $|\mathcal{M}|$.
      \item Par convention, $|\mathcal{M}|^0 = \{\emptyset\}$.
    \end{itemize}
  \end{rmk}

  \begin{exm}
      Avec $\mathcal{L}_\mathrm{corps} = \{0, 1, +, \times , -, \square^{-1}\} $, on peut choisir 
      \begin{itemize}
        \item $|\mathcal{M}| = \mathds{R}$ avec $0_\mathds{R}$, $1_\mathds{R}$, $+_\mathds{R}$, $\times_\mathds{R}$, $-_\mathds{R}$ et $\square^{-1}_\mathds{R}$ ;
        \item ou $|\mathcal{M}| = \mathds{R}$ avec $2_\mathds{R}$, $2_\mathds{R}$, $-_\mathds{R}$, $+_\mathds{R}$, \textit{etc}.
      \end{itemize}
  \end{exm}

  Définissions la \textit{vérité}.

  \begin{defn}
    Soit $\mathcal{M}$ une interprétation de $\mathcal{L}$.
    \begin{itemize}
      \item Un \textit{environnement} est une fonction de l'ensemble des variables dans $|\mathcal{M}|$.
      \item Si $e$ est un environnement et $a \in |\mathcal{M}|$, on note $e[x:=a]$ l'environnement $e'$ tel que $e'(x) = a$ et pour $y \neq x$, $e(y) = e'(y)$.
      \item La \textit{valeur} d'un terme $t$ dans l'environnement $e$, noté $\Val_\mathcal{M}(t,e)$, est définie par induction sur l'ensemble des termes de la façon suivante :
        \begin{itemize}
          \item $\Val_\mathcal{M}(c, e) = c_\mathcal{M}$ si $c$ est une constante ;
          \item $\Val_\mathcal{M}(c,e) = e(x)$ si $x$ est une variable ;
          \item $\Val_\mathcal{M}(f(t_1, \ldots, t_n), e) = f_\mathcal{M}(\Val_\mathcal{M}(t_1, e), \ldots, \Val_\mathcal{M}(t_n, e))$.
        \end{itemize}
    \end{itemize}
  \end{defn}

  \begin{rmk}
    La valeur est $\Val_\mathcal{M}(t, e)$ est un élément de $|\mathcal{M}|$.
  \end{rmk}

  \begin{exm}
    Dans $\mathcal{L}_\mathrm{arith} = \{0, 1, +, \times\}$, avec le modèle \[
    \mathcal{M} : \mathds{N}, 0_\mathds{N}, 1_\mathds{N}, +_\mathds{N}, \times_\mathds{N}
    ,\] et l'environnement \[
    e : \quad x_1 \mapsto 2_\mathds{N} \quad x_2 \mapsto 0_\mathds{N} \quad x_3 \mapsto 3_\mathds{N}
    ,\] 
    alors la valeur du terme $t := (1 \times x_1) + (x_2 \times x_3) + x_2$ est $2_\mathds{N} = (1 \times 2) + (0 \times 3) + 0 $.
  \end{exm}

  \begin{lem}
    La valeur $\Val_\mathcal{M}(t, e)$ ne dépend que de la valeur de $e$ sur les variables de $t$.
    \qed
  \end{lem}

  \begin{nota}
    \begin{itemize}
      \item Lorsque cela est possible, on oublie $\mathcal{M}$ et $e$ dans la notation, et on note $\Val(t)$.
      \item À la place de  $\Val_\mathcal{M}(t,e)$  quand $x_1, \ldots, x_n$ sont les variables de $t$ et $e(x_1) = a_1, \ldots, e(x_n) = a_n$, on note $t[a_1, \ldots, a_n]$ ou aussi~$t[x_1 := a_1, \ldots, x_n := a_n]$.
        C'est un \textit{terme à paramètre}, mais attention ce n'est \textit{\textbf{ni un terme, ni une substitution}}.
    \end{itemize}
  \end{nota}

  \begin{defn}
    Soit $\mathcal{M}$ une interprétation d'un langage $\mathcal{L}$.
    La \textit{valeur} d'une formule $F$ de $\mathcal{L}$ dans l'environnement $e$ est un élément de $\{0,1\}$ noté $\Val_\mathcal{M}(F, e)$ et définie par induction sur l'ensemble des formules par 

    {
      \footnotesize
    \begin{itemize}
      \item $\Val_\mathcal{M}(R(t_1, \ldots, t_n), e) = 1$  ssi $(\Val_\mathcal{M}(t_1, e), \ldots, \Val_\mathcal{M}(t_n ,e)) \in R_\mathcal{M}$ ;
      \item $\Val_\mathcal{M}(\bot, e) = 0$ ;
      \item $\Val_\mathcal{M}(\lnot F, e) = 1 - \Val_\mathcal{M}(F, e)$ ;
      \item $\Val_\mathcal{M}(F \land G, e) = 1$ ssi $\Val_\mathcal{M}(F, e) = 1$ et $\Val_\mathcal{M}(G, e) = 1$ ;
      \item $\Val_\mathcal{M}(F \lor G, e) = 1$ ssi $\Val_\mathcal{M}(F, e) = 1$ ou $\Val_\mathcal{M}(G, e) = 1$ ;
      \item $\Val_\mathcal{M}(F \to G, e) = 1$ ssi $\Val_\mathcal{M}(F, e) = 0$ ou $\Val_\mathcal{M}(G, e) = 1$ ;
      \item $\Val_\mathcal{M}(\forall x\:F, e) = 1$ ssi pour tout $a \in |\mathcal{M}|$, $\Val_\mathcal{M}(F, e[x := a]) = 1$ ;
      \item $\Val_\mathcal{M}(\exists x\:F, e) = 1$ ssi il existe $a \in |\mathcal{M}|$, $\Val_\mathcal{M}(F, e[x := a]) = 1$.
    \end{itemize}
    }
  \end{defn}


  \begin{rmk}
    \begin{itemize}
      \item On se débrouille pour que les connecteurs aient leur sens courant, les "mathématiques naïves".
      \item Dans le cas du calcul propositionnel, si $R$ est d'arité $0$, \textit{i.e.} une variable propositionnelle, comme $|\mathcal{M}|^0 = \{\emptyset\}$ alors on a deux possibilité :
        \begin{itemize}
          \item ou bien $R = \emptyset$, et alors on convient que $\Val_\mathcal{M}(R, e) = 0$ ;
          \item ou bien $R = \{\emptyset\} $, et alors on convient que $\Val_\mathcal{M}(R, e) = 1$.
        \end{itemize}
    \end{itemize}
  \end{rmk}

  \begin{rmk}
    On verra plus tard qu'on peut construire les entiers avec 
    \begin{itemize}
      \item $0 : \emptyset$,
      \item $1 : \{\emptyset\}$,
      \item $2 : \{\emptyset, \{\emptyset\}\}$,
      \item $\;\vdots \quad \vdots$
      \item  $n + 1 : n \cup \{n\}$,
      \item $\;\vdots \qquad \vdots$
    \end{itemize}
  \end{rmk}

  \begin{nota}
    À la place de $\Val_\mathcal{M}(F, e) = 1$, on notera $\mathcal{M}, e \models F$ ou bien~$\mathcal{M} \models F$.
    On dit que $\mathcal{M}$ \textit{satisfait} $F$, que $\mathcal{M}$ est un \textit{modèle} de $F$ (dans l'environnement $e$), que $F$ est est vraie dans $\mathcal{M}$.
  \end{nota}

  \begin{lem}
    La valeur $\Val_\mathcal{M}(F, e)$ ne dépend que de la valeur de~$e$ sur les variables libres de $F$.
  \end{lem}
  \begin{prv}
    En exercice.
  \end{prv}

  \begin{crlr}
    Si $F$ est close, alors $\Val_\mathcal{M}(F, e)$ ne dépend pas de $e$ et on note $\mathcal{M} \models F$ ou $\mathcal{M} \not\models F$.
  \end{crlr}

  \begin{rmk}
    Dans le cas des formules closes, on doit passer un environnement à cause de $\forall $ et $\exists $.
  \end{rmk}

  \begin{nota}
    On note $F[a_1, \ldots, a_n]$ pour $\Val_\mathcal{M}(F, e)$ avec $e(x_1) = a_1, \ldots, e(x_n) = a_n$.
    C'est une \textit{formule à paramètres}, mais ce n'est \textit{\textbf{pas une formule}}.
  \end{nota}


  \begin{exm}
    Dans $\mathcal{L} = \{S\}$ où $S$ est une relation binaire, on considère deux modèles :
    \begin{itemize}
      \item $\mathcal{N} : |\mathcal{N}| = \mathds{N}$ avec $S_\mathcal{N} = \{(x,y)  \mid x < y\}$,
      \item $\mathcal{R} : |\mathcal{R}| = \mathds{R}$ avec $S_\mathcal{R} = \{(x,y)  \mid x < y\}$ ;
    \end{itemize}
    et deux formules
    \begin{itemize}
      \item $F = \forall x \: \forall y \: (S\: x \: y \to \exists z\:(S \: x \: z \land S \: z \: y))$,
      \item $G = \exists x \: \forall y \: (x = y \lor S \: x \: y)$ ;
    \end{itemize}
    alors on a 
    \[
    \mathcal{N} \not\models F \quad \mathcal{R} \models F \quad \mathcal{N} \models G \quad \mathcal{R} \not\models G
    .\]
    En effet, la formule $F$ représente le fait d'être un ordre dense, et $G$ d'avoir un plus petit élément.
  \end{exm}

  \begin{defn}
    Dans un langage $\mathcal{L}$, une formule $F$ est un \textit{théorème} (\textit{logique}) si pour toute structure $\mathcal{M}$ et tout environnement $e$, on a $\mathcal{M}, e \models F$.
  \end{defn}

  \begin{exm}
    Quelques théorèmes simples :
    $\forall x \: \lnot \bot$, et $\forall x \: x = x$ et même $x = x$ car on ne demande pas que la formule soit clause.

    Dans $\mathcal{L}_\mathrm{g} = \{e, *, \square^{-1}\}$, on considère deux formules 
    \begin{itemize}
      \item $F = \forall x \: \forall y \: \forall z \: ((x * (y * z) = (x * y) * z) \land x * e = e * x = x \land \exists t \: (x * t = e \land t * x = e))$ ;
      \item et $G = \forall e' = \forall e'\: (\forall x\: (x * e' = e' * x = x) \to  e = e')$.
    \end{itemize}
    Aucun des deux n'est un théorème (il n'est vrai que dans les groupes pour $F$ (c'est même la définition de groupe) et dans les monoïdes pour $G$ (unicité du neutre)), mais $F \to G$ est un théorème logique.
  \end{exm}

  \begin{defn}
    Soient $\mathcal{L}$ et $\mathcal{L}'$ deux langages. On dit que $\mathcal{L}'$ \textit{enrichit} $\mathcal{L}$ ou que $\mathcal{L}$ est une \textit{restriction}  de $\mathcal{L}'$  si $\mathcal{L} \subseteq \mathcal{L}'$.

    Dans ce cas, si $\mathcal{M}$ est une interprétation de $\mathcal{L}$, et si $\mathcal{M}'$ est une interprétation de $\mathcal{L}'$ alors on dit que $\mathcal{M}'$ est un \textit{enrichissement} de $\mathcal{M}$ ou que $\mathcal{M}$ est une \textit{restriction} de $\mathcal{M}'$ ssi $|\mathcal{M}| = |\mathcal{M}'|$ et chaque symbole de $\mathcal{L}$ a la même interprétation dans $\mathcal{M}$ et $\mathcal{M}'$, \textit{i.e.} du point de vue de $\mathcal{L}$, $\mathcal{M}$ et $\mathcal{M}'$ sont les mêmes.
  \end{defn}

  \begin{exm}
    Avec $\mathcal{L} = \{e, *\}$ et $\mathcal{L}' = \{e, *, \square^{-1}\}$ alors $\mathcal{L}'$ est une extension de $\mathcal{L}$.
    On considère 
    \begin{itemize}
      \item $\mathcal{M} : \quad |\mathcal{M}| = \mathds{Z} \quad e_\mathcal{M} = 0_\mathds{Z} \quad *_\mathcal{M} = +_\mathds{Z}$ ;
      \item $\mathcal{M}' : \quad |\mathcal{M}'| = \mathds{Z} \quad e_\mathcal{M'} = 0_\mathds{Z} \quad *_\mathcal{M'} = +_\mathds{Z} \quad \square^{-1}_\mathcal{M'} = \mathrm{id}_\mathds{Z}$,
    \end{itemize}
    et alors $\mathcal{M}'$ est une extension de $\mathcal{M}$.
  \end{exm}

  \begin{prop}
    Si $\mathcal{M}$ une interprétation de $\mathcal{L}$ est un enrichissement de $\mathcal{M}'$, une interprétation de $\mathcal{L}'$, alors pour tout environnement $e$, 
    \begin{enumerate}
      \item si $t$ est un terme de $\mathcal{L}$, alors $\Val_\mathcal{M}(t, e) = \Val_{\mathcal{M}'}(t, e)$ ;
      \item si $F$ est une formule de $\mathcal{L}$ alors $\Val_\mathcal{M}(F, e) = \Val_\mathcal{M'}(F, e)$.
    \end{enumerate}
  \end{prop}
  \begin{prv}
    En exercice.
  \end{prv}

  \begin{crlr}
    La vérité d'une formule dans une interprétation ne dépend que de la restriction de cette interprétation au langage de la formule.
    \qed
  \end{crlr}

  \begin{defn}
    Deux formules $F$ et $G$ sont \textit{équivalentes} si $F \leftrightarrow G$ est un théorème logique.
  \end{defn}

  \begin{prop}
    Toute formule est équivalente à une formule n'utilisant que les connecteurs logiques $\lnot$, $\lor$ et $\exists$.
    \qed
  \end{prop}


  \begin{defn}
    Soient $\mathcal{M}$ et $\mathcal{N}$ deux interprétations de $\mathcal{L}$.
    \begin{enumerate}
      \item Un \textit{$\mathcal{L}$-morphisme} de $\mathcal{M}$ est une fonction $\varphi : |\mathcal{M}| \to |\mathcal{N}|$ telle que 
        \begin{itemize}
          \item pour chaque symbole de constante $c$, on a $\varphi(c_\mathcal{M}) = c_\mathcal{N}$ ;
          \item pour chaque symbole $f$ de fonction $n$-aire, on a 
            \[
            \varphi(f_\mathcal{M}(a_1, \ldots, a_n)) = f_\mathcal{N}(\varphi(a_1), \ldots, \varphi(a_n))\;
            ;\] 
          \item pour chaque symbole $R$ de relation $n$-aire (autre que "$=$"), on a 
            \[
              (a_1, \ldots, a_n) \in R_\mathcal{M} \text{ ssi } (\varphi(a_1), \ldots, \varphi(a_n)) \in R_\mathcal{N}
            .\]
          \item Un \textit{$\mathcal{L}$-isomorphisme} est un $\mathcal{L}$-morphisme bijectif.
          \item Si $\mathcal{M}$ et $\mathcal{N}$ sont \textit{isomorphes} s'il existe un $\mathcal{L}$-isomorphisme de $\mathcal{M}$ à $\mathcal{N}$.
        \end{itemize}
    \end{enumerate}
  \end{defn}

  \begin{rmk}
    \begin{enumerate}
      \item On ne dit rien sur "$=$" car si on impose la même condition que pour les autres relations alors nécessairement $\varphi$ est injectif.
      \item La notion dépend du langage $\mathcal{L}$.
      \item Lorsqu'on a deux structures isomorphes, on les confonds, ce sont les mêmes, c'est un renommage.
    \end{enumerate}
  \end{rmk}

  \begin{exm}
    Avec $\mathcal{L}_\mathrm{ann} = \{0, +, \times ,-\}$ et $\mathcal{L}' = \mathcal{L}_\mathrm{ann}\cup  \{1\}$, et les deux modèles $\mathcal{M} : \mathds{Z} / 3\mathds{Z}$ et $\mathcal{N} = \mathds{Z} / 12 \mathds{Z}$, on considère la fonction définie (on néglige les cas inintéressants) par $\varphi(\bar{n}) = \overline{4n}$.

    Est-ce que $\varphi$ est un morphisme de $\mathcal{M}$ dans $\mathcal{N}$ ? Oui\ldots\ et non\ldots\ Dans $\mathcal{L}$ c'est le cas, mais pas dans $\mathcal{L}'$  car $\varphi(1) = 4$.
  \end{exm}

  \begin{exm}
    Dans $\mathcal{L} = \{c, f, R\}$ avec $f$ une fonction binaire, et $R$ une relation binaire, on considère 
    \begin{itemize}
      \item $\mathcal{M} : \mathds{R}, 0, +, \le$ ;
      \item $\mathcal{N} : {]{0,+\infty}[}, 1, \times , \le$.
    \end{itemize}
    Existe-t-il un morphisme de $\mathcal{M}$ dans $\mathcal{N}$ ?
    Oui, il suffit de poser le morphisme $\varphi : x \mapsto \mathrm{e}^x$.
  \end{exm}

  \begin{prop}
    La composée de deux morphismes (\textit{resp}. isomorphisme) est un morphisme (\textit{resp}. un isomorphisme).
    \qed
  \end{prop}

  \begin{nota}
    Si $\varphi$ est un morphisme de $\mathcal{M}$ dans $\mathcal{N}$ et $e$ un environnement de $\mathcal{M}$,
    alors on note $\varphi(e)$ pour  $\varphi \circ e$. C'est un environnement de $\mathcal{N}$.
  \end{nota}

  \begin{lem}
    Soient $\mathcal{M}$ et $\mathcal{N}$ deux interprétations de $\mathcal{L}$, et $\varphi$ un morphisme de $\mathcal{M}$ dans $\mathcal{N}$. Alors pour tout terme $t$ et environnement $e$, on a \[
    \varphi(\Val_\mathcal{M}(t, e)) = \Val_\mathcal{N}(t, \varphi(e))
    .\]
    \qed
  \end{lem}

  \begin{lem}
    Soient $\mathcal{M}$ et $\mathcal{N}$ deux interprétations de $\mathcal{L}$, et $\varphi$ un morphisme \textit{\textbf{injectif}} de $\mathcal{M}$ dans $\mathcal{N}$. Alors pour toute formule atomique $F$ et environnement $e$, on a \[
      \mathcal{M}, e \models F \text{ ssi } \mathcal{N}, \varphi(e) \models F
    \]
  \end{lem}

  \begin{lem}
    Soient $\mathcal{M}$ et $\mathcal{N}$ deux interprétations de $\mathcal{L}$, et $\varphi$ un \textit{\textbf{isomorphisme}}\footnote{On utilise ici la \textit{surjectivité} pour le "$\exists $".} de $\mathcal{M}$ dans $\mathcal{N}$. Alors pour toute formule $F$ et environnement $e$, on a \[
      \mathcal{M}, e \models F \text{ ssi } \mathcal{N}, \varphi(e) \models F
    \]
  \end{lem}

  \begin{crlr}
    Deux interprétations isomorphismes satisfont les mêmes formules closes.
  \end{crlr}

  \begin{exo}
    Les groupes $\mathds{Z} / 4 \mathds{Z}$ et $\mathds{Z} / 2 \mathds{Z} \times \mathds{Z} / 2 \mathds{Z}$ sont-ils isomorphes ?
    Non. En effet, les deux formules 
    \begin{itemize}
      \item $\exists x\: (x \neq e \land x * x \neq e \land x * (x * x) \neq e \land x * (x * (x * x)) = e)$,
      \item $\forall x \: (x*x) = e$
    \end{itemize}
    ne sont pas vraies dans les deux (pour la première, elle est vraie dans $\mathds{Z} / 4 \mathds{Z}$ mais pas dans $(\mathds{Z} / 2 \mathds{Z})^2$ et pour la seconde, c'est l'inverse).
  \end{exo}

  \begin{rmk}
    La réciproque du corollaire est \textit{\textbf{fausse}} : deux interprétations qui satisfont les mêmes formules closes ne sont pas nécessairement isomorphes.
    Par exemple, avec $\mathcal{L} = \{{\le}\}$, les interprétations $\mathds{R}$ et $\mathds{Q}$ satisfont les mêmes formules closes, mais ne sont pas isomorphes.
  \end{rmk}

  \begin{defn}
    Soit $\mathcal{L}$ un langage, $\mathcal{M}$ et $\mathcal{N}$ deux interprétations de $\mathcal{L}$.
    On dit que $\mathcal{N}$ est une \textit{extension} de $\mathcal{M}$ (ou $\mathcal{M}$ est une \textit{sous-interprétation} de $\mathcal{N}$) si les conditions suivants sont satisfaites :
    \begin{itemize}
      \item $|\mathcal{M}| \subseteq |\mathcal{N}|$ ;
      \item pour tout symbole de constante $c$, on a $c_\mathcal{M} = c_\mathcal{N}$ ;
      \item pour tout symbole de fonction $n$-aire $f$, on a $f_\mathcal{M} = f_\mathcal{N}\big|_{|\mathcal{M}|^n}$ (donc en particulier $f_\mathcal{N}(|\mathcal{M}|^n) \subseteq |\mathcal{M}|$) ;
      \item pour tout symbole de relation $n$-aire $R$, on a $R_\mathcal{M} = R_\mathcal{N} \cap |\mathcal{M}|^n$.
    \end{itemize}
  \end{defn}

  \begin{prop}
    Soient $\mathcal{M}$ et $\mathcal{N}$ deux interprétations de $\mathcal{L}$. Alors $\mathcal{M}$ est isomorphe à une sous-interprétation $\mathcal{M}'$ de $\mathcal{N}$ si et seulement si, il existe un morphisme injectif de $\mathcal{M}$ dans $\mathcal{N}$.
  \end{prop}

  \begin{exm}
    Construction de $\mathds{Z}$ à partir de $\mathds{N}$.
    On pose la relation $(p, q) \sim (p', q')$ si  $p + q' = p' + q$.
    C'est une relation d'équivalence sur  $\mathds{N}^2$.
    On pose $\mathds{Z} := \mathds{N}^2 / \sim$ (il y a un isomorphisme $\mathds{N}^2 / {\sim} \to \mathds{Z}$ par $(p,q) \mapsto p - q$).
    Est-ce qu'on a $\mathds{N} \subseteq \mathds{N}^2 / {\sim}$ ?
    D'un point de vue ensembliste, non.
    Mais, généralement, l'inclusion signifie avoir un morphisme injectif de $\mathds{N}$ dans $\mathds{N}^2 / {\sim}$.
  \end{exm}
\end{document}
