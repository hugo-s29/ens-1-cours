\documentclass[../main]{subfiles}

\begin{document}
  \chapter*{Introduction.} \addstarredchapter{Introduction.} \label{thprog-chap00}

  Dans ce cours, on étudie la \textit{sémantique des langages de programmation}. On présente des approches pour
  \begin{itemize}
    \item définir rigoureusement ce qu'est/ce que fait un programme ;
    \item établir mathématiquement des propriétés sur des programmes. 
  \end{itemize}

  Par exemple,
  \begin{itemize}
    \item démontrer l'absence de \textit{bug} dans un programme ;
    \item démontrer des propriétés sur des programmes de transformation de programme ;
    \item l'étude des nouveaux langages de programmation.
  \end{itemize}

  Dans ce cours, les langages fonctionnels (OCaml, Haskell, Scheme,~\ldots) auront un rôle central.
\end{document}
