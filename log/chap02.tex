\documentclass[./main]{subfiles}

\begin{document}
  \chapter{La logique du premier ordre.}

  \section{Les termes.}

  On commence par définir les \textit{termes}, qui correspondent à des objets mathématiques.
  Tandis que les formules relient des termes et correspondent plus à des énoncés mathématiques.

  \begin{defn}
    Le langage $\mathcal{L}$ (du premier ordre)
    est la donnée d'une famille (pas nécessairement finie) de symboles de trois sortes :
    \begin{itemize}
      \item les symboles de \textit{constantes}, notées $c$ ;
      \item les symboles de \textit{fonctions}, avec un entier associé, leur \textit{arité}, notées $f(x_1, \ldots, x_n)$ où $n$ est l'arité ;
      \item les symboles de \textit{relations}, avec leur arité, notées $R$, appelés \textit{prédicats}.
    \end{itemize}
    Les trois ensembles sont disjoints.
  \end{defn}

  \begin{rmk}
    \begin{itemize}
      \item Les constantes peuvent être vues comme des fonctions d'arité $0$.
      \item On aura toujours dans les relations : "$=$" d'arité $2$, et "$\bot$" d'arité $0$.
      \item On a toujours un ensemble de variables $\mathcal{V}$.
    \end{itemize}
  \end{rmk}

  \begin{exm}
    Le langage $\mathcal{L}_\mathrm{g}$ de la théorie des groupes est défini par :
    \begin{itemize}
      \item une constante : $c$,
      \item deux fonctions : $f_1$ d'arité $2$ et $f_2$ d'arité $1$ ;
      \item la relation $=$.
    \end{itemize}
    Ces symboles sont notés usuellement $e$, $*$, $\square^{-1}$ ou bien $0$, $+$, $-$.
  \end{exm}

  \begin{exm}
    Le langage $\mathcal{L}_\mathrm{co}$ des corps ordonnés est défini par :
    \begin{itemize}
      \item deux constantes $0$ et $1$,
      \item quatre fonctions $+$, $\times$, $-$ et $\square^{-1}$,
      \item deux relations $=$ et $\le$.
    \end{itemize}
  \end{exm}

  \begin{exm}
    Le langage $\mathcal{L}_\mathrm{ens}$ de la théorie des ensembles est défini par :
    \begin{itemize}
      \item une constante $\emptyset$,
      \item trois fonctions $\cap$, $\cup$ et $\square^\mathsf{c}$,
      \item trois relations $=$, $\in$ et $\subseteq$.
    \end{itemize}
  \end{exm}

  \begin{defn}
    \begin{description}
      \item[Par le haut.]
        L'ensemble $\mathcal{T}$ des termes sur le langage $\mathcal{L}$ est le plus petit ensemble de mots sur $\mathcal{L} \cup \mathcal{V} \cup \{\verb|(|, \verb|)|, \verb|,|\}$ tel
        \begin{itemize}
          \item qu'il contienne $\mathcal{V}$ et les constantes ;
          \item qui est stable par application des fonctions, c'est-à-dire que pour des termes $t_1, \ldots, t_n$ et un symbole de fonction $f$ d'arité $n$, alors $f(t_1, \ldots, t_n)$ est un terme.\footnote{Attention : le "$\ldots$" n'est pas un terme mais juste une manière d'écrire qu'on place les termes à côté des autres.}
        \end{itemize}
      \item[Par le bas.]
        On pose \[
          \mathcal{T}_0 = \mathcal{V} \cup \{c  \mid c \text{ est un symbole de constante de } \mathcal{L} \} 
        ,\] 
        puis \[
        \mathcal{T}_{k+1} = \mathcal{T}_k \cup \mleft\{\,f(t_1, \ldots t_n) \;\middle|\;
        \begin{array}{c}
          f \text{ fonction d'arité } n\\
          t_1, \ldots, t_n \in \mathcal{T}_k
        \end{array}\,\mright\}  
        ,\] et enfin \[
        \mathcal{T} = \bigcup_{n \in \mathds{N}} \mathcal{T}_n
        .\]
    \end{description}
  \end{defn}

  \begin{rmk}
    Dans la définition des termes, un n'utilise les relations.
  \end{rmk}

  \begin{exm}
    \begin{itemize}
      \item Dans $\mathcal{L}_\mathrm{g}$, $*(*(x,\square^{-1}(y)), e)$ est un terme, qu'on écrira plus simplement en $(x * y^{-1}) * e$.
      \item Dans $\mathcal{L}_\mathrm{co}$, $(x + x) + (-0)^{-1}$ est un terme.
      \item Dans $\mathcal{L}_\mathrm{ens}$, $(\emptyset^\mathsf{c} \cup \emptyset) \cap (x \cup y)^\mathsf{c}$ est un terme.
    \end{itemize}
  \end{exm}

  \begin{defn}
    Si $t$ et $u$ sont des termes et $x$ est une variable, alors~$t[x:u]$ est le mot dans lequel les lettres de  $x$ ont été remplacées par le mot $u$.
    Le mot $t[x:u]$ est un terme (preuve en exercice).
  \end{defn}

  \begin{exm}
    Avec $t = (x * y^{-1}) * e$ et $u = x * e$, alors on a \[
      t[x:u] = ((x*e) * y^{-1}) * e
    .\]
  \end{exm}

  \begin{defn}
    \begin{itemize}
      \item Un terme \textit{clos} est un terme sans variable (par exemple $(0 + 0)^{-1}$).
      \item La \textit{hauteur} d'un terme est le plis petit $k$ tel que $t \in \mathcal{T}_k$.
    \end{itemize}
  \end{defn}

  \begin{exo}
    \begin{itemize}
      \item Énoncer et prouver le lemme de lecture unique pour les termes.
      \item Énoncer et prouver un lemme de bijection entre les termes et un ensemble d'arbres étiquetés.
    \end{itemize}
  \end{exo}

  \section{Les formules.}

  \begin{defn}
    \begin{itemize}
      \item Les formules sont des mots sur l'alphabet \[ \mathcal{L} \cup \mathcal{V} \cup \{\verb|(|, \verb|)|, \verb|,|, \exists, \forall, \land, \lor, \lnot, \to \} .\] 
      \item Une \textit{formule atomique} est une formule de la forme $R(t_1, \ldots, t_n)$ où $R$ est un symbole de relation d'arité $n$ et $t_1, \ldots, t_n$ des termes.
      \item L'ensemble des \textit{formules} $\mathcal{F}$ du langage $\mathcal{L}$ est défini par 
        \begin{itemize}
          \item on pose $\mathcal{F}_0$ l'ensemble des formules atomiques ;
          \item on pose $\mathcal{F}_{k+1} = \mathcal{F}_k \cup \mleft\{\,
              \begin{array}{c}
                (\lnot F)\\
                (F \to G)\\
                (F \lor G)\\
                (F \land G)\\
                \exists x\: F\\
                \exists x\: G\\
              \end{array}
            \;\middle|\; 
            \begin{array}{c}
              F,G \in \mathcal{F}_k\\
              x \in \mathcal{V}
            \end{array}
          \,\mright\}$ ;
        \item et on pose enfin $\mathcal{F} = \bigcup_{n \in \mathds{N}} \mathcal{F}_n$.
        \end{itemize}
    \end{itemize}
  \end{defn}

  \begin{exo}
    La définition ci-dessus est "par le bas". Donner une définition par le haut de l'ensemble $\mathcal{F}$.
  \end{exo}

  \begin{exm}
    \begin{itemize}
      \item Dans $\mathcal{L}_\mathrm{g}$, un des axiomes de la théorie des groupes s'écrit \[
        \forall x\: \exists x\: (x * y = e \land y * x = e)
        .\]
      \item Dans $\mathcal{L}_\mathrm{co}$, l'énoncé "le corps est de caractéristique 3" s'écrit \[
        \forall x\: (x + (x + x) = 0)
        .\] 
      \item Dans $\mathcal{L}_\mathrm{ens}$, la loi de De Morgan s'écrit \[
          \forall x\: \forall y\: (x^\mathsf{c} \cup y^\mathsf{c} = (x \cap y)^\mathsf{c})
        .\]
    \end{itemize}
  \end{exm}

  \begin{exo}
    \begin{itemize}
      \item Donner et montrer le lemme de lecture unique.
      \item Énoncer et donner un lemme d'écriture en arbre.
    \end{itemize}
  \end{exo}

  \begin{rmk}[Conventions d'écriture.]
    On note :
    \begin{itemize}
      \item $x \le  y$ au lieu de ${\le}(x,y)$ ;
      \item $\exists x \ge 0\: (F)$ au lieu de $\exists x\: (x \ge 0 \land F)$ ;
      \item $\forall x \ge 0\: (F)$ au lieu de $\forall x\: (x \ge 0 \to F)$ ;
      \item $A \leftrightarrow B$ au lieu de $(A \to B) \land (B \to A)$ ;
      \item $t \neq u$ au lieu de $\lnot (t = u)$.
    \end{itemize}

    On enlèves les parenthèses avec les conventions de priorité 
    \begin{enumerate}
      \item[0.] les symboles de relations (le plus prioritaire) ;
      \item les symboles $\lnot, \exists, \forall$ ;
      \item les symboles $\land$ et $\lor$ ;
      \item le symbole $\to$ (le moins prioritaire).
    \end{enumerate}
  \end{rmk}

  \begin{exm}
    Ainsi, $\forall x \: A \land B \to \lnot C \lor D$ s'écrit \[
      (((\forall x\:A) \land B) \to ((\lnot C) \lor D))
    .\]
  \end{exm}

  \begin{rmk}
    Le calcul propositionnel est un cas particulier de la logique du premier ordre
    où l'on ne manipule que des relations d'arité $0$ (pas besoin des fonctions et des variables) : les "variables" du calcul propositionnel sont des formules atomiques ; et on n'a pas de relation "$=$".
  \end{rmk}

  \begin{rmk}
    On ne peut pas exprimer \textit{a priori} :
    \begin{itemize}
      \item  des quantifications sur en ensemble\footnote{En dehors de $\mathcal{L}_\mathrm{ens}$, en tout cas.} ;
      \item "$\exists n\ \exists x_1\: \ldots \: \exists x_n $" une formule qui dépend d'un paramètre ;
      \item le principe de récurrence : si on a $\mathcal{P}(0)$ pour une propriété $\mathcal{P}$ et que si $\mathcal{P}(n) \to \mathcal{P}(n+1)$ alors on a $\mathcal{P}(n)$ pour tout $n$.
    \end{itemize}
  \end{rmk}

  Quelques définitions techniques qui permettent de manipuler les formules.

  \begin{defn}
    L'ensemble des sous-formules de $F$, noté $\mathrm{S}(F)$ est défini par induction :
    \begin{itemize}
      \item si $F$ est atomique, alors on définit $\mathrm{S}(F) = \{F\}$ ;
      \item si $F = F_1 \oplus F_2$ (avec $\oplus$ qui est $\lor$, $\to$ ou $\land$)
         alors on définit~ $\mathrm{S}(F) = \mathrm{S}(F_1) \cup \mathrm{S}(F_2) \cup \{F\}$ ;
      \item si $F = \lnot F_1$, ou $F = \mathsf{Q}x\: F_1$ avec~$\mathsf{Q} \in \{\forall, \exists\}$, alors on définit $\mathrm{S}(F) = \mathrm{S}(F_1) \cup \{F\}$.
    \end{itemize}
    C'est l'ensemble des formules que l'on voit comme des sous-arbres de l'arbre équivalent à la formule $F$.
  \end{defn}

  \begin{defn}
    \begin{itemize}
      \item La \textit{taille} d'une formule, est le nombre de connecteurs ($\lnot$,  $\lor$,  $\land$,  $\to$), et de quantificateurs ($\forall $, $\exists $).
      \item La racine de l'arbre est 
        \begin{itemize}
          \item rien su la formule est atomique ;
          \item "$\oplus$" si $F = F_1 \oplus F_2$ avec $\oplus$ un connecteur (binaire ou unaire) ;
          \item "$\mathsf{Q}$" si $F = \mathsf{Q}x\: F_1$ avec $\mathsf{Q}$ un quantificateur.
        \end{itemize}
    \end{itemize}
  \end{defn}

  \begin{defn}
    \begin{itemize}
      \item Une \textit{occurrence} d'une variable est un endroit où la variable apparait dans la formule (\textit{i.e.} une feuille étiquetée par cette variable).
      \item Une occurrence d'une variable est \textit{liée} si elle se trouve dans une sous-formule dont l'opérateur principal est un quantificateur appelé à cette variable (\textit{i.e.} un $\forall x\: F'$ ou un $\exists x\: F'$).
      \item Une occurrence d'une variable est \textit{libre} quand elle n'est pas liée.
      \item Une variable est libre si elle a au moins une occurrence libre, sinon elle est liée.
    \end{itemize}
  \end{defn}

  \begin{rmk}
    On note $F(x_1, \ldots, x_n)$ pour dire que les variables libres sont $F$ sont parmi $\{x_1, \ldots, x_n\}$.
  \end{rmk}

  \begin{defn}
    Une formule est \textit{close} si elle n'a pas de variables libres.
  \end{defn}

  \begin{defn}[Substitution]
    On note $F[x := t]$ la formule obtenue en remplaçant toutes les occurrences libres de  $x$ par $t$, après renommage éventuel des occurrences des variables liées de $F$ qui apparaissent dans $t$.
  \end{defn}

  \begin{defn}[Renommage]
    On donne une définition informelle et incomplète ici.
    On dit que les formules $F$ et $G$ sont $\alpha$-équivalentes si elle sont syntaxiquement identiques à un renommage près des occurrences liées des variables.
  \end{defn}

  \begin{exm}
    On pose \[
    F(x,z) := \forall y\:(x * y = y * z) \land \forall x\: (x * x = 1)
    ,\]
    et alors 
    \begin{itemize}
      \item $F(z,z) = F[x := z] = \forall y\:(z * y = y * z) \land \forall x\: (x * x = 1)$ ;
      \item $F(y^{-1}, x) = F[x := y^{-1}] = \forall {\color{nicered} u}\:(y^{-1} * {\color{nicered} u} = {\color{nicered} u} * z) \land \forall x\: (x * x = 1)$.
    \end{itemize}
    On a procédé à un renommage de $y$ à ${\color{red} u}$.
  \end{exm}

  \section{Les démonstrations en déduction naturelle.}

  \begin{defn}
    Un \textit{séquent} est un coupe noté $\Gamma \vdash F$ (où $\vdash $ se lit "montre" ou "thèse") tel que $\Gamma$ est un ensemble fini de formules appelé \textit{contexte} (\textit{i.e.} l'ensemble des hypothèses), la formule $F$ est la \textit{conséquence} du séquent.
  \end{defn}

  \begin{rmk}
    Les formules ne sont pas nécessairement closes. Et on note souvent $\Gamma$ comme une liste.
  \end{rmk}

  \begin{defn}
    On dit que $\Gamma \vdash F$ est \textit{prouvable}, \textit{démontrable} ou \textit{dérivable}, s'il peut être obtenu par une suite finie de règles (\textit{c.f.} ci-après).
    On dit qu'une formule $F$ est \textit{prouvable} si $\emptyset\vdash F$ l'est.
  \end{defn}

  \begin{defn}[Règles de la démonstration]
    Une règle s'écrit \[
    \begin{prooftree}
      \hypo{\text{\textit{prémisses} : des séquents}}
      \infer 1[\text{nom de la règle}] {\text{\textit{conclusion} : un séquent}}
    \end{prooftree}
    .\]

    \begin{description}
      \item[Axiome.]
        \[
        \begin{prooftree}
          \infer 0[\mathsf{ax}] {\Gamma, A \vdash A}
        \end{prooftree}
        \]
      \item[Affaiblissement.]
        \[
          \begin{prooftree}
            \hypo{\Gamma \vdash A}
            \infer 1[\mathsf{aff}] {\Gamma, B \vdash A}
          \end{prooftree}
        \]
      \item[Implication.]
        \[
        \begin{prooftree}
          \hypo{\Gamma, A \vdash B}
          \infer 1[\to_\mathsf{i}] {\Gamma \vdash A \to B}
        \end{prooftree}
        \quad\quad
        \begin{prooftree}
          \hypo{\Gamma \vdash A \to B}
          \hypo{\Gamma \vdash A}
          \infer 2[\to_\mathsf{e}\footnote{Aussi appelée \textit{modus ponens}}] {\Gamma \vdash B}
        \end{prooftree}
        \]
      \item[Conjonction.]
        \[
        \begin{prooftree}
          \hypo{\Gamma \vdash A}
          \hypo{\Gamma \vdash B}
          \infer 2[\land_\mathsf{i}] {\Gamma \vdash A \land B}
        \end{prooftree}
        \quad
        \begin{prooftree}
          \hypo{\Gamma \vdash A \land B}
          \infer 1[\lor_\mathsf{e}^\mathsf{g}]{\Gamma \vdash A}
        \end{prooftree}
        \quad
        \begin{prooftree}
          \hypo{\Gamma \vdash A \land B}
          \infer 1[\lor_\mathsf{e}^\mathsf{d}]{\Gamma \vdash B}
        \end{prooftree}
        \] 
      \item[Disjonction.]
        \[
        \begin{prooftree}
          \hypo{\Gamma \vdash A}
          \infer 1[\lor_\mathsf{i}^\mathsf{g}]{\Gamma \vdash A \lor B}
        \end{prooftree}
        \quad
        \begin{prooftree}
          \hypo{\Gamma \vdash B}
          \infer 1[\lor_\mathsf{i}^\mathsf{d}]{\Gamma \vdash A \lor B}
        \end{prooftree}
        \]~ \[
        \begin{prooftree}
          \hypo{\Gamma \vdash A \lor B}
          \hypo{\Gamma, A \vdash C}
          \hypo{\Gamma, B \vdash C}
          \infer 3[\lor_\mathsf{e}\footnote{C'est un raisonnement par cas}]{\Gamma \vdash C}
        \end{prooftree}
        .\] 
      \item[Négation.]
        \[
        \begin{prooftree}
          \hypo{\Gamma, A \vdash \bot}
          \infer 1[\lnot_\mathsf{i}]{\Gamma \vdash \lnot A}
        \end{prooftree}
        \quad\quad
        \begin{prooftree}
          \hypo{\Gamma \vdash A}
          \hypo{\Gamma \vdash \lnot A}
          \infer 2[\lnot_\mathsf{e}]{\Gamma \vdash \bot}
        \end{prooftree}
        \]
      \item[Absurdité classique.]
        \[
        \begin{prooftree}
          \hypo{\Gamma, \lnot A \vdash \bot}
          \infer 1[\bot_\mathsf{e}]{\Gamma \vdash A}
        \end{prooftree}
        \]
        (En logique intuitionniste, on retire l'hypothèse $\lnot A$ dans la prémisse.)
      \item[Quantificateur universel.]
        \[
        \begin{prooftree}
          \hypo{\Gamma \vdash A}
          \infer[left label=
          \begin{array}{r}
            \text{si $x$ n'est pas libre}\\
            \text{dans les formules de $\Gamma$}
          \end{array}
          ] 1[\forall_\mathsf{i}]{\Gamma \vdash \forall x \: A}
        \end{prooftree}
        \]~\[
        \begin{prooftree}
          \hypo{\Gamma \vdash \forall x\:A}
          \infer[left label=
          \begin{array}{r}
            \text{quitte à renommer les}\\
            \text{variables liées de $A$ qui}\\
            \text{apparaissent dans $t$}
          \end{array}
          ] 1[\forall_\mathsf{e}]{\Gamma \vdash A[x := t]}
        \end{prooftree}
        \]
      \item[Quantificateur existentiel.]
        \[
          \begin{prooftree}
            \hypo{\Gamma \vdash A[x := t]}
            \infer 1[\exists_\mathsf{i}]{\Gamma \vdash \exists x \: A}
          \end{prooftree}
         \]~
        \[
          \begin{prooftree}
            \hypo{\Gamma \vdash \exists x\: A}
            \hypo{\Gamma, A \vdash C}
            \infer[left label={
              \begin{array}{r}
                \text{avec $x$ ni libre dans $C$ ou}\\
                \text{dans les formules de $\Gamma$}
              \end{array}
            }] 2[\exists_\mathsf{e}]{\Gamma \vdash C}
          \end{prooftree}
         \]
    \end{description}
  \end{defn}
  % TODO


  \section{La sémantique.}

  \begin{defn}
    Soit $\mathcal{L}$ un langage de la sémantique du premier ordre.
    On appelle \textit{interprétation} (ou \textit{modèle}, ou \textit{structure}) du langage $\mathcal{L}$ l'ensemble $\mathcal{M}$ des données suivantes :
    \begin{itemize}
      \item un ensemble non vide, noté $|\mathcal{M}|$, appelé \textit{domaine} ou \textit{ensemble de base} de $\mathcal{M}$ ;
      \item pour chaque symbole $c$ de constante, un élément $c_{\mathcal{M}}$ de $|\mathcal{M}|$ ;
      \item pour chaque symbole $f$ de fonction $n$-aire, une fonction $f_{\mathcal{M}} : |\mathcal{M}|^n \to |\mathcal{M}|$ ;
      \item pour chaque symbole $R$ de relation $n$-aire (sauf pour l'égalité "$=$"), un sous-ensemble $R_{\mathcal{M}}$ de $|\mathcal{M}|^n$.
    \end{itemize}
  \end{defn}

  \begin{rmk}
    \begin{itemize}
      \item La relation "$=$" est toujours interprétée par la vraie égalité :
        \[
        \{(a,a)  \mid a \in |\mathcal{M}|\} 
        .\]
      \item On note, par abus de notation, $\mathcal{M}$ pour $|\mathcal{M}|$.
      \item Par convention, $|\mathcal{M}|^0 = \{\emptyset\}$.
    \end{itemize}
  \end{rmk}

  \begin{exm}
      Avec $\mathcal{L}_\mathrm{corps} = \{0, 1, +, \times , -, \square^{-1}\} $, on peut choisir 
      \begin{itemize}
        \item $|\mathcal{M}| = \mathds{R}$ avec $0_\mathds{R}$, $1_\mathds{R}$, $+_\mathds{R}$, $\times_\mathds{R}$, $-_\mathds{R}$ et $\square^{-1}_\mathds{R}$ ;
        \item ou $|\mathcal{M}| = \mathds{R}$ avec $2_\mathds{R}$, $2_\mathds{R}$, $-_\mathds{R}$, $+_\mathds{R}$, \textit{etc}.
      \end{itemize}
  \end{exm}

  Définissions la \textit{vérité}.

  \begin{defn}
    Soit $\mathcal{M}$ une interprétation de $\mathcal{L}$.
    \begin{itemize}
      \item Un \textit{environnement} est une fonction de l'ensemble des variables dans $|\mathcal{M}|$.
      \item Si $e$ est un environnement et $a \in |\mathcal{M}|$, on note $e[x:=a]$ l'environnement $e'$ tel que $e'(x) = a$ et pour $y \neq x$, $e(y) = e'(y)$.
      \item La \textit{valeur} d'un terme $t$ dans l'environnement $e$, noté $\Val_{\mathcal{M}}(t,e)$, est définie par induction sur l'ensemble des termes de la façon suivante :
        \begin{itemize}
          \item $\Val_{\mathcal{M}}(c, e) = c_{\mathcal{M}}$ si $c$ est une constante ;
          \item $\Val_{\mathcal{M}}(c,e) = e(x)$ si $x$ est une variable ;
          \item $\Val_{\mathcal{M}}(f(t_1, \ldots, t_n), e) = f_{\mathcal{M}}(\Val_{\mathcal{M}}(t_1, e), \ldots, \Val_{\mathcal{M}}(t_n, e))$.
        \end{itemize}
    \end{itemize}
  \end{defn}

  \begin{rmk}
    La valeur est $\Val_{\mathcal{M}}(t, e)$ est un élément de $|\mathcal{M}|$.
  \end{rmk}

  \begin{exm}
    Dans $\mathcal{L}_\mathrm{arith} = \{0, 1, +, \times\}$, avec le modèle \[
    \mathcal{M} : \mathds{N}, 0_\mathds{N}, 1_\mathds{N}, +_\mathds{N}, \times_\mathds{N}
    ,\] et l'environnement \[
    e : \quad x_1 \mapsto 2_\mathds{N} \quad x_2 \mapsto 0_\mathds{N} \quad x_3 \mapsto 3_\mathds{N}
    ,\] 
    alors la valeur du terme $t := (1 \times x_1) + (x_2 \times x_3) + x_2$ est $2_\mathds{N} = (1 \times 2) + (0 \times 3) + 0 $.
  \end{exm}

  \begin{lem}
    La valeur $\Val_{\mathcal{M}}(t, e)$ ne dépend que de la valeur de $e$ sur les variables de $t$.
    \qed
  \end{lem}

  \begin{nota}
    \begin{itemize}
      \item Lorsque cela est possible, on oublie $\mathcal{M}$ et $e$ dans la notation, et on note $\Val(t)$.
      \item À la place de  $\Val_{\mathcal{M}}(t,e)$  quand $x_1, \ldots, x_n$ sont les variables de $t$ et $e(x_1) = a_1, \ldots, e(x_n) = a_n$, on note $t[a_1, \ldots, a_n]$ ou aussi~$t[x_1 := a_1, \ldots, x_n := a_n]$.
        C'est un \textit{terme à paramètre}, mais attention ce n'est \textit{\textbf{ni un terme, ni une substitution}}.
    \end{itemize}
  \end{nota}

  \begin{defn}
    Soit $\mathcal{M}$ une interprétation d'un langage $\mathcal{L}$.
    La \textit{valeur} d'une formule $F$ de $\mathcal{L}$ dans l'environnement $e$ est un élément de $\{0,1\}$ noté $\Val_{\mathcal{M}}(F, e)$ et définie par induction sur l'ensemble des formules par 

    {
      \footnotesize
    \begin{itemize}
      \item $\Val_{\mathcal{M}}(R(t_1, \ldots, t_n), e) = 1$  ssi $(\Val_{\mathcal{M}}(t_1, e), \ldots, \Val_{\mathcal{M}}(t_n ,e)) \in R_{\mathcal{M}}$ ;
      \item $\Val_{\mathcal{M}}(\bot, e) = 0$ ;
      \item $\Val_{\mathcal{M}}(\lnot F, e) = 1 - \Val_{\mathcal{M}}(F, e)$ ;
      \item $\Val_{\mathcal{M}}(F \land G, e) = 1$ ssi $\Val_{\mathcal{M}}(F, e) = 1$ et $\Val_{\mathcal{M}}(G, e) = 1$ ;
      \item $\Val_{\mathcal{M}}(F \lor G, e) = 1$ ssi $\Val_{\mathcal{M}}(F, e) = 1$ ou $\Val_{\mathcal{M}}(G, e) = 1$ ;
      \item $\Val_{\mathcal{M}}(F \to G, e) = 1$ ssi $\Val_{\mathcal{M}}(F, e) = 0$ ou $\Val_{\mathcal{M}}(G, e) = 1$ ;
      \item $\Val_{\mathcal{M}}(\forall x\:F, e) = 1$ ssi pour tout $a \in |\mathcal{M}|$, $\Val_{\mathcal{M}}(F, e[x := a]) = 1$ ;
      \item $\Val_{\mathcal{M}}(\exists x\:F, e) = 1$ ssi il existe $a \in |\mathcal{M}|$, $\Val_{\mathcal{M}}(F, e[x := a]) = 1$.
    \end{itemize}
    }
  \end{defn}


  \begin{rmk}
    \begin{itemize}
      \item On se débrouille pour que les connecteurs aient leur sens courant, les "mathématiques naïves".
      \item Dans le cas du calcul propositionnel, si $R$ est d'arité $0$, \textit{i.e.} une variable propositionnelle, comme $|\mathcal{M}|^0 = \{\emptyset\}$ alors on a deux possibilité :
        \begin{itemize}
          \item ou bien $R = \emptyset$, et alors on convient que $\Val_{\mathcal{M}}(R, e) = 0$ ;
          \item ou bien $R = \{\emptyset\} $, et alors on convient que $\Val_{\mathcal{M}}(R, e) = 1$.
        \end{itemize}
    \end{itemize}
  \end{rmk}

  \begin{rmk}
    On verra plus tard qu'on peut construire les entiers avec 
    \begin{itemize}
      \item $0 : \emptyset$,
      \item $1 : \{\emptyset\}$,
      \item $2 : \{\emptyset, \{\emptyset\}\}$,
      \item $\;\vdots \quad \vdots$
      \item  $n + 1 : n \cup \{n\}$,
      \item $\;\vdots \qquad \vdots$
    \end{itemize}
  \end{rmk}

  \begin{nota}
    À la place de $\Val_{\mathcal{M}}(F, e) = 1$, on notera $\mathcal{M}, e \models F$ ou bien~$\mathcal{M} \models F$.
    On dit que $\mathcal{M}$ \textit{satisfait} $F$, que $\mathcal{M}$ est un \textit{modèle} de $F$ (dans l'environnement $e$), que $F$ est est vraie dans $\mathcal{M}$.
  \end{nota}

  \begin{lem}
    La valeur $\Val_{\mathcal{M}}(F, e)$ ne dépend que de la valeur de~$e$ sur les variables libres de $F$.
  \end{lem}
  \begin{prv}
    En exercice.
  \end{prv}

  \begin{crlr}
    Si $F$ est close, alors $\Val_{\mathcal{M}}(F, e)$ ne dépend pas de $e$ et on note $\mathcal{M} \models F$ ou $\mathcal{M} \not\models F$.
  \end{crlr}

  \begin{rmk}
    Dans le cas des formules closes, on doit passer un environnement à cause de $\forall $ et $\exists $.
  \end{rmk}

  \begin{nota}
    On note $F[a_1, \ldots, a_n]$ pour $\Val_{\mathcal{M}}(F, e)$ avec $e(x_1) = a_1, \ldots, e(x_n) = a_n$.
    C'est une \textit{formule à paramètres}, mais ce n'est \textit{\textbf{pas une formule}}.
  \end{nota}


  \begin{exm}
    Dans $\mathcal{L} = \{S\}$ où $S$ est une relation binaire, on considère deux modèles :
    \begin{itemize}
      \item $\mathcal{N} : |\mathcal{N}| = \mathds{N}$ avec $S_{\mathcal{N}} = \{(x,y)  \mid x < y\}$,
      \item $\mathcal{R} : |\mathcal{R}| = \mathds{R}$ avec $S_{\mathcal{R}} = \{(x,y)  \mid x < y\}$ ;
    \end{itemize}
    et deux formules
    \begin{itemize}
      \item $F = \forall x \: \forall y \: (S\: x \: y \to \exists z\:(S \: x \: z \land S \: z \: y))$,
      \item $G = \exists x \: \forall y \: (x = y \lor S \: x \: y)$ ;
    \end{itemize}
    alors on a 
    \[
    \mathcal{N} \not\models F \quad \mathcal{R} \models F \quad \mathcal{N} \models G \quad \mathcal{R} \not\models G
    .\]
    En effet, la formule $F$ représente le fait d'être un ordre dense, et $G$ d'avoir un plus petit élément.
  \end{exm}

  \begin{defn}
    Dans un langage $\mathcal{L}$, une formule $F$ est un \textit{théorème} (\textit{logique}) si pour toute structure $\mathcal{M}$ et tout environnement $e$, on a $\mathcal{M}, e \models F$.
  \end{defn}

  \begin{exm}
    Quelques théorèmes simples :
    $\forall x \: \lnot \bot$, et $\forall x \: x = x$ et même $x = x$ car on ne demande pas que la formule soit clause.

    Dans $\mathcal{L}_\mathrm{g} = \{e, *, \square^{-1}\}$, on considère deux formules 
    \begin{itemize}
      \item $F = \forall x \: \forall y \: \forall z \: ((x * (y * z) = (x * y) * z) \land x * e = e * x = x \land \exists t \: (x * t = e \land t * x = e))$ ;
      \item et $G = \forall e' = \forall e'\: (\forall x\: (x * e' = e' * x = x) \to  e = e')$.
    \end{itemize}
    Aucun des deux n'est un théorème (il n'est vrai que dans les groupes pour $F$ (c'est même la définition de groupe) et dans les monoïdes pour $G$ (unicité du neutre)), mais $F \to G$ est un théorème logique.
  \end{exm}

  \begin{defn}
    Soient $\mathcal{L}$ et $\mathcal{L}'$ deux langages. On dit que $\mathcal{L}'$ \textit{enrichit} $\mathcal{L}$ ou que $\mathcal{L}$ est une \textit{restriction}  de $\mathcal{L}'$  si $\mathcal{L} \subseteq \mathcal{L}'$.

    Dans ce cas, si $\mathcal{M}$ est une interprétation de $\mathcal{L}$, et si $\mathcal{M}'$ est une interprétation de $\mathcal{L}'$ alors on dit que $\mathcal{M}'$ est un \textit{enrichissement} de $\mathcal{M}$ ou que $\mathcal{M}$ est une \textit{restriction} de $\mathcal{M}'$ ssi $|\mathcal{M}| = |\mathcal{M}'|$ et chaque symbole de $\mathcal{L}$ a la même interprétation dans $\mathcal{M}$ et $\mathcal{M}'$, \textit{i.e.} du point de vue de $\mathcal{L}$, $\mathcal{M}$ et $\mathcal{M}'$ sont les mêmes.
  \end{defn}

  \begin{exm}
    Avec $\mathcal{L} = \{e, *\}$ et $\mathcal{L}' = \{e, *, \square^{-1}\}$ alors $\mathcal{L}'$ est une extension de $\mathcal{L}$.
    On considère 
    \begin{itemize}
      \item $\mathcal{M} : \quad |\mathcal{M}| = \mathds{Z} \quad e_{\mathcal{M}} = 0_\mathds{Z} \quad *_{\mathcal{M}} = +_\mathds{Z}$ ;
      \item $\mathcal{M}' : \quad |\mathcal{M}'| = \mathds{Z} \quad e_{\mathcal{M'}} = 0_\mathds{Z} \quad *_{\mathcal{M'}} = +_\mathds{Z} \quad \square^{-1}_{\mathcal{M'}} = \mathrm{id}_\mathds{Z}$,
    \end{itemize}
    et alors $\mathcal{M}'$ est une extension de $\mathcal{M}$.
  \end{exm}

  \begin{prop}
    Si $\mathcal{M}$ une interprétation de $\mathcal{L}$ est un enrichissement de $\mathcal{M}'$, une interprétation de $\mathcal{L}'$, alors pour tout environnement $e$, 
    \begin{enumerate}
      \item si $t$ est un terme de $\mathcal{L}$, alors $\Val_{\mathcal{M}}(t, e) = \Val_{\mathcal{M}'}(t, e)$ ;
      \item si $F$ est une formule de $\mathcal{L}$ alors $\Val_{\mathcal{M}}(F, e) = \Val_{\mathcal{M'}}(F, e)$.
    \end{enumerate}
  \end{prop}
  \begin{prv}
    En exercice.
  \end{prv}

  \begin{crlr}
    La vérité d'une formule dans une interprétation ne dépend que de la restriction de cette interprétation au langage de la formule.
    \qed
  \end{crlr}

  \begin{defn}
    Deux formules $F$ et $G$ sont \textit{équivalentes} si $F \leftrightarrow G$ est un théorème logique.
  \end{defn}

  \begin{prop}
    Toute formule est équivalente à une formule n'utilisant que les connecteurs logiques $\lnot$, $\lor$ et $\exists$.
    \qed
  \end{prop}


  \begin{defn}
    Soient $\mathcal{M}$ et $\mathcal{N}$ deux interprétations de $\mathcal{L}$.
    \begin{enumerate}
      \item Un \textit{$\mathcal{L}$-morphisme} de $\mathcal{M}$ est une fonction $\varphi : |\mathcal{M}| \to |\mathcal{N}|$ telle que 
        \begin{itemize}
          \item pour chaque symbole de constante $c$, on a $\varphi(c_{\mathcal{M}}) = c_{\mathcal{N}}$ ;
          \item pour chaque symbole $f$ de fonction $n$-aire, on a 
            \[
            \varphi(f_{\mathcal{M}}(a_1, \ldots, a_n)) = f_{\mathcal{N}}(\varphi(a_1), \ldots, \varphi(a_n))\;
            ;\] 
          \item pour chaque symbole $R$ de relation $n$-aire (autre que "$=$"), on a 
            \[
              (a_1, \ldots, a_n) \in R_{\mathcal{M}} \text{ ssi } (\varphi(a_1), \ldots, \varphi(a_n)) \in R_{\mathcal{N}}
            .\]
          \item Un \textit{$\mathcal{L}$-isomorphisme} est un $\mathcal{L}$-morphisme bijectif.
          \item Si $\mathcal{M}$ et $\mathcal{N}$ sont \textit{isomorphes} s'il existe un $\mathcal{L}$-isomorphisme de $\mathcal{M}$ à $\mathcal{N}$.
        \end{itemize}
    \end{enumerate}
  \end{defn}

  \begin{rmk}
    \begin{enumerate}
      \item On ne dit rien sur "$=$" car si on impose la même condition que pour les autres relations alors nécessairement $\varphi$ est injectif.
      \item La notion dépend du langage $\mathcal{L}$.
      \item Lorsqu'on a deux structures isomorphes, on les confonds, ce sont les mêmes, c'est un renommage.
    \end{enumerate}
  \end{rmk}

  \begin{exm}
    Avec $\mathcal{L}_\mathrm{ann} = \{0, +, \times ,-\}$ et $\mathcal{L}' = \mathcal{L}_\mathrm{ann}\cup  \{1\}$, et les deux modèles $\mathcal{M} : \mathds{Z} / 3\mathds{Z}$ et $\mathcal{N} = \mathds{Z} / 12 \mathds{Z}$, on considère la fonction définie (on néglige les cas inintéressants) par $\varphi(\bar{n}) = \overline{4n}$.

    Est-ce que $\varphi$ est un morphisme de $\mathcal{M}$ dans $\mathcal{N}$ ? Oui\ldots\ et non\ldots\ Dans $\mathcal{L}$ c'est le cas, mais pas dans $\mathcal{L}'$  car $\varphi(1) = 4$.
  \end{exm}

  \begin{exm}
    Dans $\mathcal{L} = \{c, f, R\}$ avec $f$ une fonction binaire, et $R$ une relation binaire, on considère 
    \begin{itemize}
      \item $\mathcal{M} : \mathds{R}, 0, +, \le$ ;
      \item $\mathcal{N} : {]{0,+\infty}[}, 1, \times , \le$.
    \end{itemize}
    Existe-t-il un morphisme de $\mathcal{M}$ dans $\mathcal{N}$ ?
    Oui, il suffit de poser le morphisme $\varphi : x \mapsto \mathrm{e}^x$.
  \end{exm}

  \begin{prop}
    La composée de deux morphismes (\textit{resp}. isomorphisme) est un morphisme (\textit{resp}. un isomorphisme).
    \qed
  \end{prop}

  \begin{nota}
    Si $\varphi$ est un morphisme de $\mathcal{M}$ dans $\mathcal{N}$ et $e$ un environnement de $\mathcal{M}$,
    alors on note $\varphi(e)$ pour  $\varphi \circ e$. C'est un environnement de $\mathcal{N}$.
  \end{nota}

  \begin{lem}
    Soient $\mathcal{M}$ et $\mathcal{N}$ deux interprétations de $\mathcal{L}$, et $\varphi$ un morphisme de $\mathcal{M}$ dans $\mathcal{N}$. Alors pour tout terme $t$ et environnement $e$, on a \[
    \varphi(\Val_{\mathcal{M}}(t, e)) = \Val_{\mathcal{N}}(t, \varphi(e))
    .\]
    \qed
  \end{lem}

  \begin{lem}
    Soient $\mathcal{M}$ et $\mathcal{N}$ deux interprétations de $\mathcal{L}$, et $\varphi$ un morphisme \textit{\textbf{injectif}} de $\mathcal{M}$ dans $\mathcal{N}$. Alors pour toute formule atomique $F$ et environnement $e$, on a \[
      \mathcal{M}, e \models F \text{ ssi } \mathcal{N}, \varphi(e) \models F
    \]
  \end{lem}

  \begin{lem}
    Soient $\mathcal{M}$ et $\mathcal{N}$ deux interprétations de $\mathcal{L}$, et $\varphi$ un \textit{\textbf{isomorphisme}}\footnote{On utilise ici la \textit{surjectivité} pour le "$\exists $".} de $\mathcal{M}$ dans $\mathcal{N}$. Alors pour toute formule $F$ et environnement $e$, on a \[
      \mathcal{M}, e \models F \text{ ssi } \mathcal{N}, \varphi(e) \models F
    \]
  \end{lem}

  \begin{crlr}
    Deux interprétations isomorphismes satisfont les mêmes formules closes.
  \end{crlr}

  \begin{exo}
    Les groupes $\mathds{Z} / 4 \mathds{Z}$ et $\mathds{Z} / 2 \mathds{Z} \times \mathds{Z} / 2 \mathds{Z}$ sont-ils isomorphes ?
    Non. En effet, les deux formules 
    \begin{itemize}
      \item $\exists x\: (x \neq e \land x * x \neq e \land x * (x * x) \neq e \land x * (x * (x * x)) = e)$,
      \item $\forall x \: (x*x) = e$
    \end{itemize}
    ne sont pas vraies dans les deux (pour la première, elle est vraie dans $\mathds{Z} / 4 \mathds{Z}$ mais pas dans $(\mathds{Z} / 2 \mathds{Z})^2$ et pour la seconde, c'est l'inverse).
  \end{exo}

  \begin{rmk}
    La réciproque du corollaire est \textit{\textbf{fausse}} : deux interprétations qui satisfont les mêmes formules closes ne sont pas nécessairement isomorphes.
    Par exemple, avec $\mathcal{L} = \{{\le}\}$, les interprétations $\mathds{R}$ et $\mathds{Q}$ satisfont les mêmes formules closes, mais ne sont pas isomorphes.
  \end{rmk}

  \begin{defn}
    Soit $\mathcal{L}$ un langage, $\mathcal{M}$ et $\mathcal{N}$ deux interprétations de $\mathcal{L}$.
    On dit que $\mathcal{N}$ est une \textit{extension} de $\mathcal{M}$ (ou $\mathcal{M}$ est une \textit{sous-interprétation} de $\mathcal{N}$) si les conditions suivants sont satisfaites :
    \begin{itemize}
      \item $|\mathcal{M}| \subseteq |\mathcal{N}|$ ;
      \item pour tout symbole de constante $c$, on a $c_{\mathcal{M}} = c_{\mathcal{N}}$ ;
      \item pour tout symbole de fonction $n$-aire $f$, on a $f_{\mathcal{M}} = f_{\mathcal{N}}\big|_{|\mathcal{M}|^n}$ (donc en particulier $f_{\mathcal{N}}(|\mathcal{M}|^n) \subseteq |\mathcal{M}|$) ;
      \item pour tout symbole de relation $n$-aire $R$, on a $R_{\mathcal{M}} = R_{\mathcal{N}} \cap |\mathcal{M}|^n$.
    \end{itemize}
  \end{defn}

  \begin{prop}
    Soient $\mathcal{M}$ et $\mathcal{N}$ deux interprétations de $\mathcal{L}$. Alors $\mathcal{M}$ est isomorphe à une sous-interprétation $\mathcal{M}'$ de $\mathcal{N}$ si et seulement si, il existe un morphisme injectif de $\mathcal{M}$ dans $\mathcal{N}$.
  \end{prop}

  \begin{exm}[Construction de $\mathds{Z}$ à partir de $\mathds{N}$]
    On pose la relation $(p, q) \sim (p', q')$ si  $p + q' = p' + q$.
    C'est une relation d'équivalence sur  $\mathds{N}^2$.
    On pose $\mathds{Z} := \mathds{N}^2 / {\sim}$ (il y a un isomorphisme $\mathds{N}^2 / {\sim} \to \mathds{Z}$ par $(p,q) \mapsto p - q$).
    Est-ce qu'on a $\mathds{N} \subseteq \mathds{N}^2 / {\sim}$ ?
    D'un point de vue ensembliste, non.
    Mais, généralement, l'inclusion signifie avoir un morphisme injectif de $\mathds{N}$ dans $\mathds{N}^2 / {\sim}$.
  \end{exm}

  \begin{defn}
    Une \textit{théorie} est un ensemble (fini ou pas) de formules closes.
    Les éléments de la théorie sont appelés \textit{axiomes}.
  \end{defn}

  \begin{exm}
    La \textit{théorie des groupes} est
    \begin{align*}
      T_\mathrm{groupe} := \big\{
        &\forall x\: (x * e = e * x = x),\\
        &\forall x\: (x * x^{-1} = e \land x^{-1} * x = e),\\
        &\forall x\: \forall y\: \forall z\: (x * (y*z) = (x*y) *z)
      \big\} 
    \end{align*}
    dans le langage $\mathcal{L}_\mathrm{g}$.
  \end{exm}

  \begin{exm}
    La \textit{théorie des ensembles infinis} est 
    \begin{align*}
      T_\mathrm{ens\ infinis} := \big\{ &\exists x\: (x = x),\\
                               & \exists x\: \exists y\: (x \neq  y),\\
                               & \exists x \: \exists y \: \exists z\: (x \neq y \land y \neq z \land z \neq x)\\
                               &\ldots
      \big\}
    \end{align*}
    dans le langage $\mathcal{L}_\mathrm{ens}$.
  \end{exm}

  \begin{defn}[Sémantique]
    \begin{itemize}
      \item Une interprétation $\mathcal{M}$ \textit{satisfait} $T$ (ou $\mathcal{M}$ est un \textit{modèle} de $T$), noté $\mathcal{M} \models T$, si $\mathcal{M}$ satisfait toutes les formules de $T$.
      \item Une théorie $T$ est \textit{contradictoire} s'il n'existe pas de modèle de $T$. Sinon, on dit qu'elle est \textit{non-contradictoire}, ou \textit{satisfiable}, ou \textit{satisfaisable}.
    \end{itemize}
  \end{defn}

  \begin{exm}
    Les deux théories précédentes, $T_\mathrm{groupes}$ et $T_\mathrm{ens\ infinis}$, sont non-contradictoires.
  \end{exm}

  \begin{defn}[Syntaxique]
    Soit $T$ une théorie.
    \begin{itemize}
      \item Soit $A$ une formule. On note $T \vdash A$ s'il existe un sous-ensemble fini $T'$ tel que $T' \subseteq T$ et $T' \vdash A$.
      \item On dit que $T$ est \textit{consistante} si $T \nvdash \bot$, sinon $T$ est \textit{inconsistante}.
      \item On dit que $T$ est \textit{complète} ("\textit{axiome-complète}") si $T$ est consistante et, pour toute formule $F \in \mathcal{F}$, on a $T \vdash F$ ou on a $T \vdash \lnot F$.
    \end{itemize}
  \end{defn}

  \begin{exm}
    La théorie des groupes n'est pas complète : par exemple, \[
    F := \forall x \: \forall y \: (x * y = y * x)
    \] est parfois vraie, parfois fausse, cela dépend du groupe considéré.
  \end{exm}

  \begin{exm}
    La théorie \[
      T = \mathbf{Th}(\mathds{N}) := \{\text{les formules $F$ vraies dans $\mathds{N}$}\}
    \] est complète mais pas pratique.

    De par le théorème d'\textit{incomplétude de Gödel} (c'est un sens différent du "complet" défini avant), on montre qu'on ne peut pas avoir de \textit{joli} ensemble d'axiomes pour $\mathds{N}$.
  \end{exm}

  \begin{prop} \label{prop:fol-ou-ssi-ou}
    Soit $T$ une théorie complète.
    \begin{enumerate}
      \item Soit $A$ une formule close. On a $T \vdash \lnot A$ ssi $T \nvdash A$. \label{chap2-prop5-1}
      \item Soient $A$ et $B$ des formules closes.
        On a $T \vdash A \lor B$ ssi $T \vdash A$ ou $T \vdash B$.
    \end{enumerate}
  \end{prop}

  \begin{prv}
    \begin{itemize}
      \item Si $T \vdash \lnot A$ et $T \vdash  A$, alors il existe $T',T'' \subseteq_\mathrm{fini} T$ tels que $T' \vdash \lnot A$ et $T'' \vdash A$.
        On a donc $T' \cup T'' \vdash \bot$ par :
        \[
        \begin{prooftree}
          \hypo{T' \vdash \lnot A}
          \infer 1[\mathsf{aff}]{T' \cup T'' \vdash \lnot A}
          \hypo{T'' \vdash A}
          \infer 1[\mathsf{aff}]{T' \cup T'' \vdash A}
          \infer 2[\lnot_\mathsf{e}] {T' \cup T'' \vdash \bot}
        \end{prooftree}
        \]
        On en conclut que $T \vdash \bot$, absurde car $T$ supposée complète donc consistante.
        On a donc $T \vdash \lnot A$ implique $T \nvdash A$.
        
        Réciproquement, si $T \nvdash A$ et $T\nvdash \lnot A$, alors c'est impossible car $T$ est complète.
        On a donc  $T \nvdash A$ implique $T \vdash \lnot A$.
      \item Si $T \vdash A$ ou $T \vdash B$, alors par la règle $\lor_\mathsf{i}^\mathsf{g}$ ou $\lor_\mathsf{i}^\mathsf{d}$, on montre que $T \vdash A \lor B$.

        Réciproquement, si $T \vdash A \lor B$ et $T \nvdash A$ et $T \vdash \nvdash B$ alors, par~\ref{chap2-prop5-1}, on a $T \vdash \lnot A$ et $T \vdash \lnot B$.
        On montre ainsi (en exercice) que $T\vdash \lnot (A \lor B)$ d'où $T \vdash \bot$ par $\lnot_\mathsf{e}$.
        C'est impossible car $T$ est complète donc consistante, d'où $T \vdash A \lor B$ implique $T \vdash A$ ou $T \vdash B$.
    \end{itemize}
  \end{prv}

  \section{Théorème de complétude de Gödel.}

  \begin{thm}[Complétude de Gödel (à double sens)]
    ~\\[-\baselineskip]
    \begin{description}
      \item[Version 1.]
        Soit $T$ une théorie et $F$ une formule close.
        On a $T \vdash F$ ssi $T \models F$.
      \item[Version 2.]
        Une théorie $T$ est consistante (syntaxe) ssi elle est non-contradictoire (sémantique).
    \end{description}
  \end{thm}

  \begin{rmk}
    La version 1 se décompose en deux théorèmes :
    \begin{itemize}
      \item le \textit{théorème de correction} (ce que l'on prouve est vrai)
        \[
        T \vdash F \implies T \models F\;
        ;\]
      \item le \textit{théorème de complétude} (ce qui est vrai est prouvable) \[
        T \models F \implies T\vdash F
        .\] 
    \end{itemize}

    Pour la version 2, on peut aussi la décomposer en deux théorèmes\footnote{On a une négation dans ce théorème, donc ce n'est pas syntaxe implique sémantique pour la correction, mais non sémantique implique non syntaxe.} :
    \begin{itemize}
      \item la \textit{correction}, $T$ non-contradictoire implique $T$ consistante ;
      \item la \textit{complétude}, $T$ consistante implique $T$ non-contradictoire.
    \end{itemize}
    Par contraposée, on a aussi qu'une théorie contradictoire est inconsistante.
  \end{rmk}

  \begin{prop}
    Les deux versions du théorème de correction sont équivalentes.
  \end{prop}
  \begin{prv}
    \begin{itemize}
      \item D'une part, on montre (par contraposée) "non V2 implique non V1".
        Soit $T$ non-contradictoire et inconsistante.
        Il existe un modèle $\mathcal{M}$ tel que $\mathcal{M} \models T$ et $T \vdash \bot$.
        Or, par définition, $\mathcal{M} \not\models \bot$ donc $T \not\models \bot$.
      \item D'autre part, on montre "V2 implique V1".
        Soit $T$ et $F$ tels que $T \vdash F$.
        Ainsi, $T \cup \lnot F \vdash \bot$, d'où $T \cup \{\lnot F\}$ est inconsistante, et d'où, par la version 2 de la correction, on a que $T \cup \{\lnot F\}$ contradictoire, donc on n'a pas de modèle.
        On a alors que, touts les modèles de $T$ sont des modèles de $F$, autrement dit $T \models F$.
    \end{itemize}
  \end{prv}

  \begin{prop}
    Les deux versions du théorème de complétude (sens unique) sont équivalentes.
  \end{prop}
  \begin{prv}
    \begin{itemize}
      \item Soit $T$ contradictoire. Elle n'a pas de modèle.
        Ainsi, on a $T \models \bot$ d'où $T \vdash \bot$ par la version 1, elle est donc inconsistante.
      \item Soit $T \models F$. Considérons $T \cup \{\lnot F\}$ : cette théorie n'a pas de modèle, donc est contradictoire, donc est inconsistante, et on a donc que $T \cup \{\lnot F\} \vdash \bot$ d'où $T \vdash F$ par $\bot_\mathsf{e}$.
    \end{itemize}
  \end{prv}

  \begin{rmk}[\textbf{Attention !}]
    On utilise "$\models$" dans deux sens.
    \begin{itemize}
      \item Dans le sens $\textsl{modèle} \models \textsl{formule}$, on dit qu'une formule est vraie dans un modèle, c'est le sens des mathématiques classiques.
      \item Dans le sens $\textsl{théorie} \models \textsl{formule}$, on dit qu'une formule est vraie dans tous les modèles de la théorie, c'est un sens des mathématiques plus inhabituel.
    \end{itemize}
  \end{rmk}

  \subsection{Preuve du théorème de correction.}

  \begin{exo}
    Montrer que le lemme ci-dessous implique la version 1 de la correction.
  \end{exo}

  \begin{lem}
    Soient $T$ une théorie, $\mathcal{M}$ un modèle et $F$ une formule close.
    Si $\mathcal{M} \models T$ et $T \vdash F$ alors $\mathcal{M} \models F$.
  \end{lem}

  \begin{prv}
    Comme d'habitude, pour montrer quelque chose sur les formules closes, on commence par les formules et même les termes.
    On commence par montrer que la substitution dans les termes a un sens sémantique.

    \begin{lem}
      Soient $t$ et $u$ des termes et $e$ un environnement.
      Soient $v := t[x:=u]$ et $e' := e[x := \Val(u, e)]$.
      Alors,  $\Val(v, e) = \Val(t, e')$.
    \end{lem}
    \begin{prv}
      En exercice.
    \end{prv}

    \begin{lem}
      Soit $A$ une formule, $t$ un terme, et $e$ un environnement.
      Si $e' := e[x := \Val(t, e)]$ alors  $\mathcal{M}, e \models A[x := t]$ ssi $\mathcal{M}, e' \models A$.
    \end{lem}
    \begin{prv}
      En exercice.
    \end{prv}

    On termine la preuve en montrant la proposition ci-dessous.
  \end{prv}

  Montrons cette proposition plus forte que le lemme.

  \begin{prop}
    Soient $\Gamma$ un ensemble de formules et $A$ une formule.
    Soit $\mathcal{M}$ une interprétation et soit $e$ un environnement.
    Si $\mathcal{M}, e \models \Gamma$, et $\Gamma \models A$ alors  $\mathcal{M}, e \models A$.
  \end{prop}
  \begin{prv}
    Par induction sur la preuve de $\Gamma \vdash A$, on montre la proposition précédente.

    \begin{itemize}
      \item Cas inductif $\to_\mathsf{i}$.
        On sait que $A$ est de la forme $B \to C$, et on montre que de $\Gamma, B \vdash C$ on montre $\Gamma \vdash B \to C$.
        Soient $\mathcal{M}$ et $e$ tels que $\mathcal{M}, e \models \Gamma$.
        Montrons que $\mathcal{M}, e \models B \to C$.
        Il faut donc montrer que si $\mathcal{M}, e \models B$ alors $\mathcal{M}, e \models C$.
        Si $\mathcal{M}, e \models B$ alors $\mathcal{M}, e \models \Gamma \cup \{B\}$.
        Or, comme $\Gamma, B \vdash C$ alors par hypothèse d'induction, on a que $\mathcal{M}, e \models C$.
      \item Cas inductif $\forall_\mathsf{e}$.
        Si $A$ est de la forme $B[x := t]$, alors de $\Gamma \vdash \forall x \: B$, on en déduit que $\Gamma \vdash B[x := t]$.
        Soit $\mathcal{M}, e \models \Gamma$ et $a := \Val(t, e)$. Par hypothèse de récurrence, on a que  $\mathcal{M}, e \models \forall x \: B$ donc $\mathcal{M}, e[x := a] \models B$ et d'après le lemme précédent, on a que $\mathcal{M}, e \models B[x := t]$.
      \item Les autre cas inductifs sont laissé en exercices.
      \item Cas de base $\mathsf{ax}$.
        Si $A \in \Gamma$ et $\mathcal{M}, e \models \Gamma$ alors $\mathcal{M}, e \models A$.
      \item Cas de base $=_\mathsf{i}$.
        On a, pour tout $\mathcal{M}, e$ que $\mathcal{M}, e \models t = t$.
    \end{itemize}
    \vspace{-\baselineskip}
  \end{prv}
  Cette proposition permet de conclure la preuve du lemme précédent.

  \subsection{Preuve du théorème de complétude.}

  On va montrer la version 2, en \textbf{\textit{trois étapes}}.
  Soit $T$ une théorie consistante sur le langage $\mathcal{L}$.
  \begin{enumerate}
    \item On enrichit le langage $\mathcal{L}$ en $\mathcal{L}'$  avec des constantes, appelées \textit{témoins de Henkin}, et qui nous donnerons les éléments de notre ensemble de base : les termes.
    \item Pour définir complètement le modèle, on complète la théorie $T$ en une théorie $\mathrm{Th}$ sur $\mathcal{L}'$.
    \item On quotiente pour avoir la vraie égalité dans le modèle.
  \end{enumerate}

  Cette construction est assez similaire à la définition de $\mathds{C}$ comme le quotient $\mathds{R}[X] / (X^2 + 1)$.

  \begin{prop}
    On peut étendre $\mathcal{L}$ en $\mathcal{L}'$ et $T$ en $T'$ consistante telle que, pour toute formule $F(x)$ de $\mathcal{L}'$, ayant pour seule variable libre $x$, il existe un symbole de constante $c_F$ de $\mathcal{L}'$ telle que l'on ait $T' \vdash \exists x\: F(x) \to F(c_F)$, d'où le nom de témoin.
  \end{prop}

  \begin{prv}
    On fait la construction "par le bas" :
    \begin{itemize}
      \item $\mathcal{L}_0 = \mathcal{L}$ ;
      \item $T_0 = T$ ;
      \item $\mathcal{L}_{n+1} = \mathcal{L}_n \cup \{c_F  \mid F \text{ formule à une variable libre de } \mathcal{L}_n\} $ ;
      \item $T_{n+1} = T_n \cup \{\exists x \: F \to F(c_F)  \mid F \text{ formule de } \mathcal{L}_n\}$ ;
      \item et enfin $\mathcal{L}' = \bigcup_{n \in \mathds{N}} \mathcal{L}_n$ et $T' = \bigcup_{n \in \mathds{N}} T_n$.
    \end{itemize}

    On commence par montrer quelques lemmes.

    \begin{lem}
      \label{lem:fol-var-libre-un-symb}
      Soient $\Gamma$ un ensemble de formules et $A$ une formule.
      Soit $c$ un symbole de constante qui n'apparait ni dans~$\Gamma$ ni dans $A$.
      Si $\Gamma \vdash  A[x := c]$ alors $\Gamma \vdash \forall x\: A$.
    \end{lem}
    \begin{prv}
      Idée de la preuve. On peut supposer que $x$ n'apparait pas dans $\Gamma$, ni dans la preuve de $\Gamma \vdash A[x := c]$, sinon on renomme $x$ en $y$ dans l'énoncé du lemme.
      Alors, de la preuve de $\Gamma \vdash A[x := c]$, on peut déduire une preuve de $\Gamma \vdash A(x)$ en replaçant $c$ par $x$. Avec la règle $\forall_\mathsf{i}$, on en conclut que $\Gamma \vdash \forall x \: A$.
    \end{prv}

    \begin{lem}
      Pour toute formule $F$ à une variable libre $x$ sur le langage $\mathcal{L}'$,
      \[
        T' \vdash \exists x \: F(x) \to F(c_F)
      .\]
    \end{lem}
    \begin{prv}
      La formule $F$ a un nombre fini de constantes (car c'est un mot fini), donc $F$ est une formule sur $\mathcal{L}_n$ pour un certain $n \in \mathds{N}$, donc $(\exists x \: F(x) \to F(c_F)) \in T_{n+1} \subseteq T'$.
    \end{prv}

    Il nous reste à montrer que la théorie $T'$ est consistante.

    Il suffit de montrer que tous les $T_n$ sont consistantes.
    En effet, si $T'$ est non-consistante, il existe un ensemble fini~$T'' \subseteq T'$ et $T'' \vdash \bot$. Comme $T''$ fini, il existe un certain $n \in \mathds{N}$ tel que $T'' \subseteq T_n$ et donc $T_n \vdash \bot$.

    On montre par récurrence sur $n$ que $T_n$ est consistante.
    \begin{itemize}
      \item On a $T_0 = T$ qui est consistante par hypothèse.
      \item Supposons $T_n$ consistante et que $T_{n+1} \vdash \bot$.
        Alors, il existe des formules à une variable libre $F_1, \ldots, F_k$ écrites sur $\mathcal{L}_n$ et 
        \[
        T_n \cup \mleft\{\,\exists x \: F_i \to F_i(c_{F_i}) \;\middle|\; 1 \le i \le k\,\mright\} \vdash \bot
        .\]
        Ainsi (exercice) \[
        T_n \vdash \Big( \bigwedge_{1 \le i \le k} \big(\exists x \: F_i \to F_i(c_{F_i})\big) \Big) \to \bot
        .\]
        Les $c_{F_i}$ ne sont pas dans $T_n$ d'où, d'après le lemme~\ref{lem:fol-var-libre-un-symb}, que \[
        T_n \vdash \forall y_1 \: \forall y_2 \: \ldots \forall y_n \: \Big( \bigwedge_{1 \le i \le k} \big(\exists x \: F_i \to F_i(y_i)\big) \Big) \to \bot
        .\]
        On peut montrer que (théorème logique)
        \[
          (\star) \quad\quad \vdash \forall y \: (A(y) \to \bot) \leftrightarrow (\exists y \: A(y) \to \bot)
        ,\]
        d'où
        \[
        T_n \vdash \Big(\exists y_1 \: \exists y_2 \: \ldots \exists y_n \: \bigwedge_{1 \le i \le k} \big(\exists x \: F_i \to F_i(y_i)\big) \Big) \to \bot
        .\]
        On a aussi \[
          (\star\star) \quad\quad
          \vdash \exists y_1 \: \exists y_2 \: \big( A(y_1) \land A(y_2)\big) \leftrightarrow
          \big(\exists y_1 \: A(y_1)\big) \land \big(\exists y_2 \: A(y_2)\big)
        ,\]
        et pour $y$ non libre dans $A$, on a \[
        \vdash \exists y \: (A \to B) \leftrightarrow (A \to \exists y \: B)
        .\]
        On a donc \[
        T_n \vdash \Big( \bigwedge_{1 \le i \le k} (\exists x \: F_i(x) \to \exists y_i \: F_i(y_i)) \Big) \to \bot
        .\]
        Or, \[
          (\star{\star}\star) \quad\quad \vdash \bigwedge_{1 \le i \le k} (\exists x \: F_i(x) \to \exists y_i \: F_i(y_i))
        .\]
        On a donc $T_n \vdash \bot$, ce qui contredit l'hypothèse, d'où $T_{n+1}$ consistante.

        En exercice, on pourra montrer les théorèmes logiques $(\star)$,  $(\star\star)$, et  $(\star{\star}\star)$.
    \end{itemize}
  \end{prv}


  Ensuite, on veut compléter $T'$ en préservant le résultat de la proposition précédente.
  On cherche $\mathrm{Th}$ (axiome-)complète telle que $T' \subseteq \mathrm{Th}$ et pour toute formule à une variable libre $F$ de $\mathcal{L}'$, on a \[
  \mathrm{Th} \vdash \exists x \: F \to F(c_F)
  .\]

  Faisons le cas dénombrable (sinon, lemme de Zorn) : supposons $\mathcal{L}'$ au plus dénombrable.
  Soit $(F_n)_{n \in \mathds{N}}$ une énumération des formules closes de $\mathcal{L}'$.
  On définit par récurrence
  \begin{itemize}
    \item $K_0 := T'$ ;
    \item si $K_n$ est complète, alors $K_{n+1} := K_n$ ;
    \item si $K_n$ n'est pas complet, alors soit le plus petit $p \in \mathds{N}$ tel que l'on ait $K_n \nvdash F_p$ et $K_n \nvdash \lnot F_p$, et on pose $K_{n+1} := K_n \cup \{F_p\}$.
  \end{itemize}

  \begin{lem}
    On pose $\mathrm{Th} := \bigcup_{n \in \mathds{N}} T_n$.
    La théorie $\mathrm{Th}$ a les propriétés voulues.
  \end{lem}

  \begin{prv}
    \begin{enumerate}
      \item On a $T' \subseteq \mathrm{Th}$.
      \item La théorie $\mathrm{Th}$ est consistante.
        En effet, il suffit de montrer que tous les $K_n$ le sont (par les mêmes argument que la preuve précédente).
        Montrons le par récurrence.
        \begin{itemize}
          \item La théorie $K_0 = T'$ est consistante par hypothèse.
          \item Si $K_{n+1} = K_n$ alors $K_{n+1}$ est consistante par hypothèse de récurrence.
          \item Si $K_{n+1} = K_n \cup \{F_p\}$, et si $K_n, F_p \vdash \bot$, 
            alors par la règle $\lnot_\mathsf{i}$, on a $K_n \vdash \lnot F_p$, ce qui est faux.
            Ainsi $K_{n+1}$ est consistante.
        \end{itemize}
        On en conclut que $\mathrm{Th}$ est consistante.
      \item La théorie $\mathrm{Th}$ est complète.
        Sinon, à chaque étape $K_{n+1} = K_n \cup \{F_{q_n}\}$ et il existe $F_p$ telle que $\mathrm{Th} \nvdash F_p$ et $\mathrm{Th} \nvdash \lnot F_p$.
        Ainsi, pour tout $n \in \mathds{N}$, $K_n \nvdash F_p$ et $K_n \nvdash \lnot F_p$, d'où pour tout $n \in \mathds{N}$, $p_n \le p$ avec des $p_n$ distincts.
        C'est absurde, il n'y a qu'un nombre fini d'entiers inférieurs à un entier donné.
    \end{enumerate}
  \end{prv}

  On construit un quotient avec "$=$" comme relation d'équivalence, puis on vérifie que les fonctions et relations sont bien définies (ne dépendent pas du représentant choisit, comme pour les groupes quotients).

  Soit $\mathcal{E}$ l'ensemble des termes clos de $\mathcal{L}'$, qui n'est pas vide car il contient les termes $c_{x = x}$ (avec la définition de $c_F$ ci-avant).
  On définit sur $\mathcal{E}$ une relation $\sim$, où $t \sim t'$ ssi $\mathrm{Th} \vdash t = t'$.

  \begin{exo}
    Montrer que $\sim$ est une relation d'équivalence.
  \end{exo}

  On pose enfin $|\mathcal{M}| := \mathcal{E} / {\sim}$.
  On notera $\bar{t}$ la casse de $t$.
  On définit l'interprétation des symboles de $\mathcal{L}'$ :
  \begin{itemize}
    \item si $c$ est une constante, alors $c_{\mathcal{M}} := \bar{c}$ ;
    \item si $f$ est un symbole de fonctions d'arité $n$, \[
        f_{\mathcal{M}}(\bar{t}_1, \ldots, \bar{t}_n) := \overline{f(t_1, \ldots, t_n)}
      .\]
  \end{itemize}

  \begin{lem}
    La définition de dépend pas des représentants choisis, c'est-à-dire 
    si $\bar{u}_1 = \bar{t}_1, \ldots, \bar{u}_n = \bar{t}_n$ alors \[
      \overline{f(t_1, \ldots, t_n)} = \overline{f(u_1, \ldots, u_n)}
    .\] 
  \end{lem}
  \begin{prv}
    \begin{itemize}
      \item On a $\mathrm{Th} \vdash t_i = u_i$ pour tout $i$ par hypothèse
      \item donc avec $=_\mathsf{i}$, on a $\mathrm{Th} \vdash f(t_1, \ldots, t_n) = f(t_1, \ldots, t_n)$
      \item donc avec $=_\mathsf{e}$, on a $\mathrm{Th} \vdash f(u_1, \ldots, t_n) = f(t_1, \ldots, t_n)$
      \item \ldots\textit{etc}\ldots
      \item donc avec $=_\mathsf{e}$, on a $\mathrm{Th} \vdash f(u_1, \ldots, u_n) = f(t_1, \ldots, t_n)$
    \end{itemize}
  \end{prv}

  [\textit{suite de la définition de l'interprétation}]

  \begin{itemize}
    \item si $R$ est un symbole de relation d'arité $n$, on définit 
      \[
        (\bar{t}_1, \ldots, \bar{t}_n) \in R_{\mathcal{M}} \text{ ssi }
        \mathrm{Th} \vdash  R(t_1, \ldots, t_n)
      .\] 
  \end{itemize}

  \begin{exo}
    Montrer que cette définition de dépend pas des représentants choisis.
  \end{exo}

  \begin{lem}
    Soit $F$ une formule à $n$ variables libres et $t_1 ,\ldots, t_n$ des termes clos.
    Alors, $\mathcal{M} \models F[\bar{t}_1, \ldots, \bar{t}_n]$ ssi $\mathrm{Th} \vdash F[t_1, \ldots, t_n]$,
    où l'on interprète la formule à paramètre dans l'environnement $e$ avec $e(y_i) = \bar{t}_i$ alors $\mathcal{M}, e \models F(y_1, \ldots, y_n)$.
  \end{lem}
  \begin{prv}
    Par induction sur $F$ en supposant que $F$ n'utilise que $\lnot$, $\lor$,  $\exists$ comme connecteurs.
    En effet, on a pour toute formule $G$, il existe $F$ qui n'utilise que $\lnot$,  $\lor$,  $\exists$ et  $\vdash F \leftrightarrow G$, ce qui permet de conclure directement pour $G$ si le résultat est vrai sur $F$.

    \begin{itemize}
      \item  Pour $F = \bot$, alors on a $\mathrm{Th} \nvdash \bot$ car $\mathrm{Th}$ consistante et $\mathcal{M} \models \bot$ par définition.
      \item Pour $F = R(u_1, \ldots, u_m)$, où les $u_i$ sont des termes non nécessairement clos 
        et où $u_1, \ldots, u_m$ sont des termes à $n$ variables $x_1, \ldots, x_n$.
        On pose \[
          F[t_1, \ldots, t_n] := R(\underbrace{u_1(t_1, \ldots, t_n)}_{v_1}, \ldots, \underbrace{u_m(t_1, \ldots, t_n)}_{v_m})
        \] 
        où l'on définit $v_i := u_i(t_1, \ldots, t_n)$ qui est clos car les $t_i$ sont clos.
        On veut montrer que \[
          \mathcal{M} \models \underbrace{F[\bar{t}_1, \ldots, \bar{t}_n]}_{R(\bar{v}_1, \ldots, \bar{v}_m)} \text{ ssi }
          \mathrm{Th} \vdash \underbrace{F[t_1, \ldots, t_n]}_{R(v_1, \ldots, v_m)}
        .\]
        Or, on a l'équivalence $\mathcal{M} \models R(\bar{v}_1, \ldots, \bar{v}_m)$ ssi $(\bar{v}_1, \ldots, \bar{v}_m) \in R_{\mathcal{M}}$ ssi $\mathrm{Th} \vdash R(v_1, \ldots, v_m)$.
      \item Pour $F = F_1 \lor F_2$, et  $t_1, \ldots, t_n$ sont des termes clos, on veut montrer que 
        \begin{align*}
          &\mathcal{M} \models F_1[\bar{t}_1, \ldots, \bar{t}_n] \lor  F_2[\bar{t}_1, \ldots, \bar{t}_n]\\
          \text{ ssi }&
          \mathrm{Th} \vdash F_1[t_1, \ldots, t_n] \lor  F_2[t_1, \ldots, t_n]
        .\end{align*}
        Or,
        \begin{align*}
          &\mathcal{M} \models F_1[\bar{t}_1, \ldots, \bar{t}_n] \lor  F_2[\bar{t}_1, \ldots, \bar{t}_n] \\
          \text{ ssi }& \mathcal{M} \models F_1[\bar{t}_1, \ldots, \bar{t}_n] \text{ ou } \mathcal{M} \models F_2[\bar{t}_1, \ldots, \bar{t}_n]\\
          \text{ ssi }& \mathrm{Th} \vdash F_1[t_1, \ldots, t_n] \text{ ou } \mathrm{Th} \vdash F_2[t_1, \ldots, t_n]
        \end{align*}
        par hypothèse.
        Ainsi,
        \begin{itemize}
          \item avec $\lor_\mathsf{i}^\mathsf{g}$ et $\lor_\mathsf{i}^\mathsf{d}$, on a que $\mathrm{Th} \vdash F_1[t_1, \ldots, t_n] \lor F_2[t_1, \ldots, t_n]$ ;
          \item réciproquement, on utilise le lemme~\ref{prop:fol-ou-ssi-ou} car $\mathrm{Th}$ est complète.
        \end{itemize}
      \item Pour $F = \lnot G$, en exercice.
      \item Si  $F = \exists x \: G$ et $t_1, \ldots, t_n$ des termes clos, on a 
        \begin{itemize}
          \item on a $\mathcal{M} \models \exists x \: G[\bar{t}_1, \ldots, \bar{t}_n, x]$ 
          \item ssi il existe $t \in \mathcal{E}$ tel que $\mathcal{M} \models G[\bar{t}_1, \ldots, \bar{t}_n, \bar{t}]$
          \item ssi il existe $t \in \mathcal{E}$ tel que $\mathrm{Th} \vdash G(t_1, \ldots, t_n,t)$
        \end{itemize}
        et donc $\mathrm{Th} \vdash \exists x \: G(t_1, \ldots, t_n, x)$ avec $\exists_\mathsf{i}$.
        Réciproquement, si $\mathrm{Th} \vdash \exists x \: G(t_1, \ldots, t_n, x)$ alors $\mathrm{Th} \vdash G(t_1, \ldots, t_n, c_{G(t_1, \ldots, t_n, x)})$, donc il existe un terme $t$ et $\mathrm{Th} \vdash G(t_1, \ldots, t_n, t)$.
    \end{itemize}
  \end{prv}


  \begin{lem}
    On a $\mathcal{M} \models \mathrm{Th}$ (et donc $\mathcal{M} \models T$).
  \end{lem}

  \begin{prv}
    On montre que, pour toute formule $F$ de $\mathrm{Th}$, on a que $\mathcal{M} \models F$.
    Pour cela, on utilise le lemme précédent : si $F$ est close, alors
    \[
    \mathcal{M} \models F \text{ ssi } \mathrm{Th} \vdash F
    .\]
  \end{prv}

  \subsection{Compacité.}

  \begin{thm}[Compacité (sémantique)]
    Une théorie $T$ et contradictoire ssi elle est finiment contradictoire, \textit{i.e.} il existe $T' \subseteq_\mathrm{fini} T$ telle que $T'$ est contradictoire.
  \end{thm}
  \begin{prv}
    Soit $T$ contradictoire.
    On utilise le théorème de complétude.
    Ainsi $T$ est inconsistante.
    Il existe donc $T' \subseteq_\mathrm{fini} T$ avec $T'$ inconsistante par le théorème de compacité syntaxique ci-dessous (qui est trivialement vrai).
    On applique de nouveau le théorème de complétude pour en déduire que $T'$ est contradictoire.
  \end{prv}

  \begin{thm}[Compacité (syntaxique)]
    Une théorie $T$ est inconsistante ssi elle est finiment inconsistante.
  \end{thm}
  \begin{prv}
    Ceci est évident car une preuve est nécessairement finie.
  \end{prv}

  \vspace{-2\baselineskip}

  \[
  \begin{tikzcd}[column sep=100, row sep=60]
    \begin{array}{c}
      \text{vérité syntaxique}\\
      \text{dans une théorie finie}
    \end{array}
    \arrow[swap]{d}{\begin{array}{r}\text{compacité}\\\text{syntaxique}\end{array}}
    \arrow[bend right=5,swap]{r}{\text{théorème de correction}}&
    \arrow[bend right=5,swap]{l}{\text{théorème de complétude}}
    \begin{array}{c}
      \text{vérité sémantique}\\
      \text{dans une théorie finie}
    \end{array}
    \arrow{d}{\begin{array}{l}\text{compacité}\\\text{sémantique}\end{array}}\\
    \arrow{u}{}
    \begin{array}{c}
      \text{vérité syntaxique}\\
      \text{dans une théorie infinie}
    \end{array}
    \arrow[bend right=5,swap]{r}{\text{théorème de correction}}&
    \arrow[bend right=5,swap]{l}{\text{théorème de complétude}}
    \begin{array}{c}
      \text{vérité sémantique}\\
      \text{dans une théorie infinie}
    \end{array}
    \arrow{u}{}
  \end{tikzcd}
  \]

  Dans la suite de cette sous-section, on étudie des applications du théorème de compacité.

  \begin{thm}
    Si une théorie $T$ a des modèles finis arbitrairement grands, alors elle a un modèle infini.
    \qed
  \end{thm}

  \begin{crlr}
    Il n'y a pas de théorie des groupes finis \textit{i.e.} un ensemble d'axiomes dont les modèles sont exactement les groupes finis.
  \end{crlr}

  \begin{thm}[Löwenheim-Skolem]
    Soit $T$ une théorie dans un langage $\mathcal{L}$ et $\kappa$ un cardinal et $\kappa \ge \operatorname{card} \mathcal{L}$ et $\kappa \ge \aleph_0$.\footnote{Ici, $\aleph_0$ est le cardinal de $\mathds{N}$, on dit donc que $\kappa$ est infini.}
    Si $T$ a un modèle infini, alors $T$ a un modèle de cardinal $\kappa$.
  \end{thm}

  \begin{exm}
    \begin{itemize}
      \item Avec $T = \mathbf{Th}(\mathds{N})$, on a $\kappa = \operatorname{card} \mathds{R}$.
      \item Avec $T = \mathbf{ZFC}$, on a $\kappa = \aleph_0 = \operatorname{card} \mathds{N}$.
    \end{itemize}
  \end{exm}
\end{document}
