\documentclass[./main]{subfiles}

\begin{document}
  \chapter{Groupe symétrique}
  \minitoc

  \section{Exercice 1.}
  \begin{enonce}
    Soit $\sigma = \begin{pmatrix} 1 & 2 & 3 & 4 & 5 & 6 & 7 & 8 & 9\\ 4 & 6 & 9 & 7 & 2 & 5 & 8 & 1 & 3 \end{pmatrix} \in \mathfrak{S}_9$.
    Déterminer sa décomposition canonique en produit de cycles disjoints, son ordre, sa signature, une décomposition en produit de transposition ainsi que $\sigma^{100}$.
  \end{enonce}
  On a $\sigma = \begin{pmatrix} 1 & 4 & 7 & 8 \end{pmatrix} \begin{pmatrix} 2 & 6 & 5 \end{pmatrix}\begin{pmatrix} 3 & 9 \end{pmatrix}$.
  Son ordre est le PPCM des ordres précédent, c'est donc $12$.
  Sa signature est $(-1) \times 1 \times (-1) = 1$.
  On décompose en produit de transposition chaque cycle et on conclut.
  On calcule \[
  \sigma^{100} = \begin{pmatrix} 1 & 4 & 7 & 8 \end{pmatrix}^{100} \begin{pmatrix} 2 & 6 & 5 \end{pmatrix}^{100}\begin{pmatrix} 3 & 9 \end{pmatrix}^{100}
  ,\] car les cycles à supports disjoints commutent, et donc \[
  \sigma^{100} = \begin{pmatrix} 2 & 6 & 5 \end{pmatrix} 
  .\] 

  \section{Exercice 2. \textit{Générateurs de $\mathfrak{A}_n$}}
  \begin{enonce}
    Soit $n \ge 3$.
    \begin{enumerate}
      \item Rappeler pourquoi $\mathfrak{A}_n$ est engendré par les $3$-cycles.
      \item Démontrer que $\mathfrak{A}_n$ est engendré par les carrés d'éléments de $\mathfrak{S}_n$.
        Est-ce que tout élément de $\mathfrak{A}_n$ est un carré dans $\mathfrak{S}_n$ ?
      \item Démontrer que pour $n \ge 5$, $\mathfrak{A}_n$ est engendré par les bitranspositions.
      \item Démontrer que $\mathfrak{A}_n$ est engendré par les $3$-cycles de la forme~$(1\;2\;i)$ pour $i \in \llbracket 3,n\rrbracket$.
      \item En déduire que si $n \ge 5$ est impair, alors $\mathfrak{A}_n$ est engendré par les permutations $(1\;2\;3)$ et $(3\;4\;\cdots\; n)$ et que si $n \ge 4$ est pair, alors $\mathfrak{A}_n$ est engendré par $(1\;2\;3)$ et $(1\;2)(3\;4\;\cdots\; n)$.
    \end{enumerate}
  \end{enonce}
  \begin{enumerate}
    \item On utilise le fait que tout $\sigma \in \mathfrak{A}_n$ se décompose comme produit d'un nombre pair de transpositions.
      Puis, on utilise les égalités 
      \begin{itemize}
        \item $(i\;j)(i\;k) = (i\;j\;k)$,
        \item $(i\;j)(i\;j) = \id$,
        \item $(i\;j)(k\;\ell) = (i\;\ell\;k)(i\;j\;k)$,
      \end{itemize}
      pour déterminer un produit de $3$-cycles égal à $\sigma$.
    \item On utilise la question précédente.
      Soit $(a\;b\;c)$ un $3$-cycle.
      On a alors $(a\;b\;c)^4 = (a\;b\;c)$, et donc $\sigma = (a\;b\;c)^2$.
      Ceci permet d'en déduire que les carrés de permutations engendrent $\mathfrak{A}_n$.
  \end{enumerate}

  \section{Exercice 3.}
  \begin{enonce}
    Soit $n \le 5$.
  \end{enonce}
\end{document}
