\documentclass{../../notes}

\title{DM n°4 -- FDI}
\author{Hugo \scshape Salou}

\usepackage{multicol}

\def\algorithmicrequire{\textbf{Entrée}}
\def\algorithmicensure{\textbf{Sortie}}
\def\algorithmicend{\textbf{fin}}
\def\algorithmicif{\textbf{si}}
\def\algorithmicthen{\textbf{alors}}
\def\algorithmicelse{\textbf{sinon}}
\def\algorithmicelsif{\algorithmicelse\ \algorithmicif}
\def\algorithmicendif{\algorithmicend\ \algorithmicif}
\def\algorithmicfor{\textbf{pour}}
\def\algorithmicprocedure{\textbf{Procédure}}
\def\algorithmicfunction{\textbf{Fonction}}
\def\algorithmicforall{\textbf{pour tout}}
\def\algorithmicdo{\textbf{faire}}
\def\algorithmicendfor{\algorithmicend\ \algorithmicfor}
\def\algorithmicwhile{\textbf{tant que}}
\def\algorithmicendwhile{\algorithmicend\ \algorithmicwhile}
\def\algorithmicloop{\textbf{répéter indéfiniment}}
\def\algorithmicendloop{\algorithmicend\ \algorithmicloop}
\def\algorithmicrepeat{\textbf{répéter}}
\def\algorithmicuntil{\textbf{jusqu'à que}}
\def\algorithmicprint{\textbf{afficher}}
\def\algorithmicreturn{\textbf{retourner}}
\def\algorithmictrue{\textsc{vrai}}
\def\algorithmicfalse{\textsc{faux}}

\let\Sortie\Ensure
\let\Entree\Require
\floatname{algorithm}{\color{deepblue}Algorithme}

\begin{document}
  \maketitle

  Dans la première solution, on n'utilise pas de quine mais on instaure une propriété très utilise : "pour toute lettre du programme, elle apparait exactement 10 fois". Pour réaliser cela, on cache beaucoup de caractères dans un commentaire (comme par exemple \texttt{f} ou \texttt{[}).

  \begin{lstlisting}[language=Python,caption=Solution sans quine au DM 4,breaklines=true,postbreak=\mbox{\textcolor{deepblue}{$\hookrightarrow$}\space}]
import sys; print(10 if sys.argv[1] in "\"\\\n#ev.famit ;p(:n0[rl)1s]gyo" else 0) #eeeeeeevvvvvvvv........ffffffffaaaaaaaammmmmmmmiiiiitttttttppppppp\\\\\\((((((((:::::::::"""""""nnnnnn0000000[[[[[[[[rrrrrr;;;;;;;;llllllll))))))))1111111ssss]]]]]]]]ggggggggyyyyyyyoooooooo
#
#
#
#
#
#
#
#
#
  \end{lstlisting}
  C'est moche, pas toujours généralisable, mais ça répond au DM.

  J'ai quand même proposé une seconde solution à ce DM n'utilisant pas cette technique mais utilisant une quine, c'est à dire un programme qui s'écrit lui même.
  La solution est encore plus concise mais utilise des "\textit{string format}" (à la \texttt{printf} en C).

  \begin{lstlisting}[language=Python,caption=Solution avec quine au DM 4, breaklines=true, postbreak=\mbox{\textcolor{deepblue}{$\hookrightarrow$}\space}]
x="x=%c%s%c;y=x%c(chr(34),x,chr(34),chr(37),chr(10));import sys;print(y.count(sys.argv[1]))%c";y=x%(chr(34),x,chr(34),chr(37),chr(10));import sys;print(y.count(sys.argv[1]))
  \end{lstlisting}

  L'idée est simple, dans \texttt{x}, on écrit le code source.
  Il contient trois parties, le début de la définition de \texttt{x} (\textit{i.e.} "\texttt{x = }"), la chaîne \texttt{x} elle-même (où l'on utilise ici le "\texttt{\%s}" pour interpoler la valeur de \texttt{x} dedans), et enfin le reste du code (la partie qui compte le nombre de caractères, et qui l'affiche en console).
  Les caractères problématiques, comme les guillemets, les retours à la ligne (parce que le \textit{backslash}), le symbole pour-cent, \textit{etc}, sont interpolés aussi dans la chaîne avec~"\texttt{\%c}" et la fonction \texttt{ord} qui prend un entier et renvoie le caractère ASCII à l'indice donné dans la table.
\end{document}

