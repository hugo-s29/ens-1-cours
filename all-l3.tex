\documentclass{notes}

\usepackage{subfiles}
\usepackage{polynom}
\usepackage{pdflscape}

\ifSubfilesClassLoaded{%
  \renewcommand\thesection{\arabic{section}}
  \renewcommand\thepart{\Roman{part}}
  \renewcommand\thechapter{\phantom}
  \renewcommand\theequation{\arabic{section}.\arabic{equation}}
}{}

\title{Notes de cours de L3}
\author{Prises par Hugo \textsc{Salou}}

\newcommand\colorunderline[2]{
  {\color{#1}\underline{{\color{black} #2}}}
}

\DeclareFontFamilySubstitution{OT2}{cmr}{wncyr}
\newcommand{\textcyr}[1]{%
  \begin{hyphenrules}{nohyphenation}%
    \fontencoding{OT2}\selectfont #1%
  \end{hyphenrules}%
}

%%%%%%%%%%%%%%%% THPROG

\newcommand\qeq{\ensuremath{\mathrel{\overset ? =}}} % Question Equal =^?
\newcommand\fresh{\textcyr{I}}

\newcommand\ifte[3]{\ensuremath{\mathtt{if}\ #1\ \mathtt{then}\ #2\ \mathtt{else}\ #3}}
\newcommand\ceq{\ensuremath{\mathrel{\texttt{:=}}}}
\newcommand\col{\ensuremath{\mathrel{\texttt{;}}}}
\newcommand\while[2]{\ensuremath{\mathtt{while}\ #1\ \mathtt{do}\ #2}}
\newcommand\ptr[1]{\ensuremath{\text{\texttt{[$#1$]}}}}

%%%%%%%%%%%%%%%% LOGIQUE
\newcommand{\parens}[1]{\verb|(|\ensuremath{#1}\verb|)|}
\newcommand{\Val}{\ensuremath{\mathcal{V}\hspace{-3.5pt}a\hspace{-0.5pt}\ell}}

\newcommand\repr[1]{\tikz[baseline=-2.5pt] \node[draw, rounded rectangle, inner xsep=0pt, inner ysep=2pt, line width=0.08mm] {\ensuremath{#1}};}

\renewcommand\succ{\mathop{\repr{\boldsymbol{\mathsf{S}}}}}
\newcommand\zero{\mathop{\repr{0}}}

\newcommand\peano[1]{\hyperref[peano-#1]{\textsf{#1}}}
\newcommand\zf[1]{\hyperref[ZF#1]{\textsf{ZF\,#1}}}
\newcommand\ac[1]{\hyperref[AC#1]{\textsf{AC\,#1}}}
\newcommand\pairaxiom{\hyperref[pair-axiom]{Axiome de la paire}}
\newcommand\zorn{\hyperref[zorn]{Zorn}}
\newcommand\zermelo{\hyperref[zermelo]{Zermelo}}


%%%%%%%%%%%%%%%% PROFON
\newcommand\fouine{%
  \textsf{%
  \textcolor{deepblue}  {f}%
  \textcolor{deeppurple}{o}%
  \textcolor{deepblue}  {u}%
  \textcolor{deeppurple}{i}%
  \textcolor{deepblue}  {n}%
  \textcolor{deeppurple}{e}%
  }%
}

\newcommand\pieuvre{%
  \textsf{%
  \textcolor{deepblue}  {p}%
  \textcolor{deeppurple}{i}%
  \textcolor{deepblue}  {e}%
  \textcolor{deeppurple}{u}%
  \textcolor{deepblue}  {v}%
  \textcolor{deeppurple}{r}%
  \textcolor{deepblue}  {e}%
  }%
}


\newcommand{\twodigits}[1]{\ifnum#1<10 0#1\else #1\fi}

\newcommand\thprog[1]{\hyperref[thprog-chap\twodigits{#1}]{\textbf{THPROG}~[Chapitre~#1]}}

\let\rpar\rightrightarrows

\renewcommand\mathtt[1]{\ensuremath{\texttt{#1}}}

\begin{document}
  \maketitle

  \dominitoc
  \pagebreak
  \addstarredchapter{Table des matières.}
  \tableofcontents

  \chapter*{Préface.} \addstarredchapter{Préface.}

  Dans ce document, vous trouverez les notes de cours que j'ai pu prendre en \LaTeX\ durant cette année de L3.
  Ce document rassemble donc les notes des cours suivants :
  \begin{itemize}
    \item \textit{Théorie de la Programmation} (THPROG), Daniel \textsc{Hirschkoff} ;
    \item \textit{Logique} (LOG), Natacha \textsc{Portier} ;
    \item \textit{Projet fonctionnel} (PROFON), Daniel \textsc{Hirschkoff}.
  \end{itemize}

  Sur mon site (\url{http://167.99.84.84/}\footnote{Accès direct à la page web pour l'année "L3 ENS" : \url{http://167.99.84.84/ens1/}}\showfootnote), il y a plusieurs documents :
  \begin{itemize}
    \item les notes des cours ci-dessus en PDF individuel (et même en PDF "par chapitre") ;
    \item les notes de cours d'\textit{Algorithmique 1} (écrites avec \textit{Typst}, donc pas inclue dans ce document) ;
    \item les notes de cours de \textit{Mesure et intégration} (écrites à la main et scannées, donc pas inclue dans ce document) ;
    \item les TDs (Algorithmique 2, Projet fonctionnel, Probabilités, Logique, Algèbre 1, et Théorie des catégories) ;
    \item les DMs (Algorithmique 2, Probabilités, Logique, Algèbre 1, et Fondements de l'informatique).
  \end{itemize}

  Si vous voyez des erreurs dans les notes de cours (formatage, orthographe, grammaire, \textit{etc}), prévenez-moi !
  Ces notes sont disponibles sur \href{https://gitlab.aliens-lyon.fr/hsalou/cours-l3}{GitLab} et sur \href{https://github.com/hugo-s29/ens-1-cours}{GitHub} si vous voulez corriger directement.
  Sinon, prévenez-moi via Discord (\texttt{hugos29}) ou par email (\href{mailto:y.hugo.s29@gmail.com}{\texttt{y.hugo.s29@gmail.com}}).

  Amusez-vous bien avec ce document de \pageref*{LastPage} pages !

  \hfill Hugo \textsc{Salou}

  \part{Théorie de la programmation.}

  \subfile{thprog/chap00/chap00.tex}
  \subfile{thprog/chap01/chap01.tex}
  \subfile{thprog/chap02/chap02.tex}
  \subfile{thprog/chap03/chap03.tex}
  \subfile{thprog/chap04/chap04.tex}
  \subfile{thprog/chap05/chap05.tex}
  \subfile{thprog/chap06/chap06.tex}
  \subfile{thprog/chap07/chap07.tex}
  \subfile{thprog/chap08/chap08.tex}
  \subfile{thprog/chap09/chap09.tex}
  \subfile{thprog/chap10/chap10.tex}

  \part{Logique.}

  \subfile{log/chap00.tex}
  \subfile{log/chap01.tex}
  \subfile{log/chap02.tex}
  \subfile{log/chap03.tex}
  \subfile{log/chap04.tex}
  \subfile{log/chap05.tex}

  \part{Projet fonctionnel.}

  \subfile{profon/chap00.tex}
  \subfile{profon/chap01.tex}
  \subfile{profon/chap02.tex}
  \subfile{profon/chap03.tex}
  \subfile{profon/chap04.tex}
  \subfile{profon/chap05.tex}
\end{document}

