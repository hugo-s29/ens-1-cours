\documentclass{../../notes}

\title{DM n°1 -- Logique}
\author{Hugo \scshape Salou}

\usepackage{pdflscape}
\usepackage{multicol}
\usepackage{fitbox}

\newlength\stextwidth
\newcommand\makesamewidth[3][c]{%
  \settowidth{\stextwidth}{#2}%
  \makebox[\stextwidth][#1]{#3}%
}

\newcounter{questioncounter}
\setcounter{questioncounter}{0}

\newcommand\question[1]{
  \vspace{1em}
  \refstepcounter{questioncounter}
  \makesamewidth[r]{\textbf{Q99.}}{\textbf{Q\thequestioncounter.}}~
  \parbox[t]{11cm}{#1}
  \vspace{1em}
}

\newcommand\questionrem[1]{
  \makesamewidth[r]{\textbf{Q99.}}{}~
  \parbox[t]{11cm}{#1}
  \vspace{1em}
}

\newcommand\questionlandscape[1]{
  \vspace{1em}
  \refstepcounter{questioncounter}
  \makesamewidth[r]{\textbf{Q99.}}{\textbf{Q\thequestioncounter.}}~
  \parbox[t]{17cm}{#1}
  \vspace{1em}
}

\fitboxset{maxwidth=0.93\textwidth, minwidth=\fitboxnatwidth}

\begin{document}
  \maketitle

  \chapter{Complétude du calcul propositionnel}

  Avant de commencer dans l'exercice, fixons quelques notations, et quelques précisions.

  On utilise le système de règles classiques habituel, mais où l'on ajoute deux règles : le \textit{tiers exclu} $\mathsf{te}$ et la \textit{réécriture} $\mathsf{rewrite}$.

  \[
  \begin{prooftree}
    \infer 0[\mathsf{te}]{\Gamma \vdash F \lor \lnot F}
  \end{prooftree}
  \quad\quad
  \begin{prooftree}
    \hypo{\vdash F \leftrightarrow G}
    \hypo{\Gamma \vdash G}
    \infer 2[\mathsf{rewrite}]{\Gamma \vdash F}
  \end{prooftree}
  .\]

  On peut se l'autoriser car ces deux règles sont dérivables dans le système de règles classiques habituel.
  Il suffirait donc de remplacer les utilisations de $\mathsf{te}$ et $\mathsf{rewrite}$ par les morceaux d'arbres ci-dessous.

  {
    \scriptsize
    \[
    \begin{prooftree}
      \infer 0[\mathsf{ax}]{\lnot (P \lor \lnot P), P \vdash P}
      \infer 1[\lor_\mathsf{i}^\mathsf{g}]{\lnot (P \lor \lnot P), P \vdash P \lor \lnot P}
      \infer 0[\mathsf{ax}]{\lnot (P \lor \lnot P), P \vdash \lnot (P \lor \lnot P)}
      \infer 2[\lnot_\mathsf{e}]{\lnot (P \lor \lnot P), P \vdash \bot}
      \infer 1[\lnot_\mathsf{i}]{\lnot (P \lor \lnot P) \vdash \lnot P}
      \infer 1[\lor_\mathsf{i}^\mathsf{d}]{\lnot (P \lor \lnot P) \vdash P \lor \lnot P}
      \infer 0[\mathsf{ax}]{\lnot (P \lor \lnot P) \vdash \lnot (P \lor \lnot P)}
      \infer 2[\lnot_\mathsf{e}] {\lnot (P \lor \lnot P) \vdash}
      \infer 1[\bot_\mathsf{c}] {\vdash P \lor \lnot P}
      \infer 1[\mathsf{aff}] {\Gamma \vdash P \lor \lnot P}
    \end{prooftree}
    \]
  }

  \[
  \begin{prooftree}
    \hypo{\vdash F \leftrightarrow G}
    \infer 1[\land_\mathsf{e}^\mathsf{d}] { \vdash G \to F}
    \infer 1[\mathsf{aff}]{\Gamma \vdash G \to F}
    \hypo{\Gamma \vdash G}
    \infer 2[\to_\mathsf{e}] {\Gamma \vdash F}
  \end{prooftree}
  \]

  On s'autorise également les résultats du TD~2, et notamment de l'exercice~5 (ainsi que des résultats très similaires) :
  \begin{enumerate}
    \item pour tout $P, Q \in \mathcal{F}$, $\vdash (P \lor Q) \leftrightarrow (Q \lor P)$ ; \label{rule:lor-comm}
    \item pour tout $P, Q \in \mathcal{F}$, $\vdash (P \land Q) \leftrightarrow (Q \land P)$ ; \label{rule:land-comm}
    \item pour tout $P, Q, R \in \mathcal{F}$, $\vdash (P \lor Q) \lor R \leftrightarrow P \lor (Q \lor R)$ ; \label{rule:lor-assoc}
    \item pour tout $P, Q, R \in \mathcal{F}$, $\vdash (P \land Q) \land R \leftrightarrow P \land (Q \land R)$ ; \label{rule:land-assoc}
    \item pour tout $P, Q, R \in \mathcal{F}$, $\vdash (P \land Q) \lor R \leftrightarrow (P \lor R) \land (Q \lor R)$ ; \label{rule:de-morgan-and-or-r}
    \item pour tout $P, Q, R \in \mathcal{F}$, $\vdash (P \lor Q) \land R \leftrightarrow (P \land R) \lor (Q \land R)$ ; \label{rule:de-morgan-or-and-r}
    \item pour tout $P, Q, R \in \mathcal{F}$, $\vdash P \land (Q \lor R) \leftrightarrow (P \land Q) \lor (P \land R)$ ; \label{rule:de-morgan-and-or-l}
    \item pour tout $P, Q, R \in \mathcal{F}$, $\vdash P \lor (Q \land R) \leftrightarrow (P \lor Q) \land (P \lor R)$. \label{rule:de-morgan-or-and-l}
  \end{enumerate}
  Ces résultats sont
  \begin{itemize}
    \item pour~\ref{rule:lor-comm}--\ref{rule:land-comm}, la "commutativité" de $\lor$ et $\land$ ;
    \item pour~\ref{rule:lor-assoc}--\ref{rule:land-assoc}, l'"associativité" de $\lor$ et $\land$ ;
    \item pour \ref{rule:de-morgan-and-or-r}--\ref{rule:de-morgan-or-and-l}, les "lois de De Morgan".
  \end{itemize}
  Les termes sont entre guillemets car $\lor$ et  $\land$ ne sont vraiment pas commutatif et associatifs (les formules $P \lor Q$ et  $Q \lor P$ sont différentes), mais on peut considérer qu'ils le sont en utilisant la règle  $\mathsf{rewrite}$ définie précédemment.
  C'est ce que nous ferrons pour le reste de l'exercice.

  Ces considérations de réécriture sont nécessaires : dans l'énoncé on note $\bigwedge E$ où $E$ est un ensemble de formules, et pour qu'il n'y ait pas d'ambigüité, il est nécessaire que $\land$ soit commutatif et associatif "à réécriture près".

  Quitte à rajouter des notations, on notera $\mathds{B}^\mathcal{V}$ l'ensemble des valuations sur $\mathcal{V}$, où $\mathds{B} = \{0,1\}$.
  Pour une valuation $v$ sur $\mathcal{V} \setminus \{X_n\}$, on notera $v[X_n \mapsto b] := v'$ la valuation sur $\mathcal{V}$ définie par $v'(X_n) = b$ et par $v'\big|_{\mathcal{V} \setminus \{X_n\} } = v$.

  Enfin, on notera $\Phi(\mathcal{V})$ la formule (définie à réécriture près) \[
    \Phi(\mathcal{V}) := \bigvee_{v \in \mathds{B}^\mathcal{V}} \varphi(v)
  .\]

  \question{
    On procède par récurrence sur $n$ pour montrer $\vdash \Phi(\mathcal{V})$. \label{q1}
    \begin{itemize}
      \item Pour $\mathcal{V} = \{X_1\}$, on a deux valuations sur $\mathcal{V}$ : \[
            \mathds{B}^\mathcal{V} = \{X_1 \mapsto 0, X_1 \mapsto 1\} 
          ,\]
          et donc $\Phi(\{X_1\}) = X_1 \lor \lnot X_1$.
          Montrons $\vdash \Phi(\mathcal{V})$ :
          \[
          \begin{prooftree}
            \infer 0[\mathsf{te}] {X_1 \lor \lnot X_1}
          \end{prooftree}
          .\]
        \item Pour $\mathcal{V} = \{X_1, \ldots, X_{n+1}\}$, on pose \[
            \mathcal{V} = \underbrace{\{X_1, \ldots, X_n\}}_{=: \mathcal{V}'} \sqcup \{X_{n+1}\} 
          ,\] 
          et par hypothèse de récurrence, on a $\vdash \Phi(\mathcal{V}')$.
          On a la décomposition suivante :
          \begin{align*}
            \mathds{B}^\mathcal{V} = \quad &\mleft\{\,v'[X_{n+1} \mapsto 1]  \;\middle|\; v' \in \mathds{B}^{\mathcal{V}'} \,\mright\}\\
            {}\sqcup {} & \mleft\{\,v' [X_{n+1} \mapsto 0] \;\middle|\; v' \in \mathds{B}^{\mathcal{V}'}\,\mright\}
          .\end{align*}
          D'où, \[
            \Phi(\mathcal{V}) = \big(\Phi(\mathcal{V}') \lor X_{n+1}\big) \land \big(\Phi(\mathcal{V}') \lor \lnot X_{n+1}\big)
          .\]
          Montrons $\vdash \Phi(\mathcal{V})$ :
          \[
          \begin{prooftree}
            \hypo{\text{[résultat~\ref{rule:de-morgan-or-and-l}]}}
            \hypo{\vdash \Phi(\mathcal{V}')}
            \infer 0[\mathsf{te}]{\vdash X_{n+1} \lor \lnot X_{n+1}}
            \infer 2[\land_\mathsf{i}]{\vdash \Phi(\mathcal{V}') \land (X_{n+1} \lor \lnot X_{n+1})}
            \infer 2[\mathsf{rewrite}]{\vdash \underbrace{\big(\Phi(\mathcal{V}') \lor X_{n+1}\big) \land \big(\Phi(\mathcal{V}') \lor \lnot X_{n+1}\big)}_{\Phi(\mathcal{V})}}
          \end{prooftree}
          .\]
    \end{itemize}
  }
  \questionrem{
    D'où l'hérédité.
    On en conclut que le résultat est vrai pour tout ensemble fini $\mathcal{V}$.
  }

  \question{ \label{q2}
    On procède par induction sur $F \in \mathcal{F}$ pour montrer que toute valuation $v$ sur $\mathcal{V}$, si $v(F) = 1$ alors  $\varphi(v) \vdash F$ et si $v(F) = 0$ alors  $\varphi(v) \vdash \lnot F$.
    On a \textit{5} cas.
  }
  \begin{itemize}
    \item On se place dans le cas $F = X_i$ pour un certain indice~$i \in \llbracket 1,n\rrbracket$. Soit $v \in \mathcal{V}$. On note \[
      \varphi'(v) := \varphi(v\big|_{\mathcal{V} \setminus \{X_i\} })
      .\] 
      \begin{itemize}
        \item Si $v(X_i) = 1$ alors, on a la preuve :

          \[
            \hspace{-7em}
        \begin{prooftree}
          \infer 0[\mathsf{ax}]{\varphi(v), X_i \land \tilde{\varphi}(v) \vdash X_i}
          \infer 1[\land_\mathsf{e}^\mathsf{g}]{\varphi(v), X_i \land \tilde{\varphi}(v) \vdash X_i}
          \infer 1[\to_\mathsf{i}]{\varphi(v) \vdash X_i \land \tilde{\varphi}(v) \to X_i}
          \hypo{\text{[résultats~\ref{rule:land-assoc}/\ref{rule:land-comm}]}}
          \infer 0[\mathsf{ax}]{\varphi(v) \vdash \varphi(v)}
          \infer 2[\mathsf{rewrite}^\star]{\varphi(v) \vdash X_i \land \tilde{\varphi}(v)}
          \infer 2[\to_\mathsf{e}]{\varphi(v) \vdash X_i}
        \end{prooftree}
        ,\]
        où lorsque l'on a écrit "$\mathsf{rewrite}^\star$", on applique la règle $\mathsf{rewrite}$ $2i-1$ fois avec l'associativité ($i$ fois) et la commutativité ($i-1$ fois) en alternance.
      \item Si $v(X_i) = 0$ alors on a la preuve très similaire à la précédente :
          \[
            \hspace{-7em}
        \begin{prooftree}
          \infer 0[\mathsf{ax}]{\varphi(v), \lnot X_i \land \tilde{\varphi}(v) \vdash \lnot X_i}
          \infer 1[\land_\mathsf{e}^\mathsf{g}]{\varphi(v), \lnot X_i \land \tilde{\varphi}(v) \vdash \lnot X_i}
          \infer 1[\to_\mathsf{i}]{\varphi(v) \vdash \lnot X_i \land \tilde{\varphi}(v) \to \lnot X_i}
          \hypo{\text{[résultats~\ref{rule:land-assoc}/\ref{rule:land-comm}]}}
          \infer 0[\mathsf{ax}]{\varphi(v) \vdash \varphi(v)}
          \infer 2[\mathsf{rewrite}^\star]{\varphi(v) \vdash \lnot X_i \land \tilde{\varphi}(v)}
          \infer 2[\to_\mathsf{e}]{\varphi(v) \vdash \lnot X_i}
        \end{prooftree}
        .\]
      \end{itemize}
    \item Dans le cas $F = G \lor H$, soit  $v$ une valuation sur $\mathcal{V}$.
      \begin{itemize}
        \item Si $v(F) = 1$, alors $v(G) = 1$ ou  $v(H) = 1$.
          \begin{itemize}
            \item Si $v(G) = 1$, alors $\varphi(v)\vdash G$ par hypothèse d'induction, et donc 
              \[
              \begin{prooftree}
                \hypo{\varphi(v) \vdash G}
                \infer 1[\lor_\mathsf{i}^\mathsf{g}]{\varphi(v) \vdash G \lor H}
              \end{prooftree}
              .\] 
            \item Si $v(H) = 1$, alors $\varphi(v)\vdash H$ par hypothèse d'induction, et donc 
              \[
              \begin{prooftree}
                \hypo{\varphi(v) \vdash H}
                \infer 1[\lor_\mathsf{i}^\mathsf{d}]{\varphi(v) \vdash G \lor H}
              \end{prooftree}
              .\]
          \end{itemize}
        \item Sinon, $v(F) = 0$ et alors $v(G) = 0$ et $v(H) = 0$, d'où, par hypothèse d'induction, $\varphi(v) \vdash \lnot G$ et $\varphi(v) \vdash \lnot H$.
          On construit la preuve :

          \fitbox{$
            \hspace{-10em}
          \begin{prooftree}
            \infer 0[\mathsf{ax}]{\varphi(v), G \lor H \vdash G \lor H}
            \infer 0[\mathsf{ax}]{\varphi(v), G \lor H, G \vdash G}
            \hypo{\varphi(v) \vdash \lnot G}
            \infer 1[\mathsf{aff}]{\varphi(v), G \lor H, G \vdash \lnot G}
            \infer 2[\lnot_\mathsf{e}]{\varphi(v), G \lor H, G \vdash \bot}
            \infer 0[\mathsf{ax}]{\varphi(v), G \lor H, H \vdash H}
            \hypo{\varphi(v) \vdash \lnot H}
            \infer 1[\mathsf{aff}]{\varphi(v), G \lor H, H \vdash \lnot H}
            \infer 2[\lnot_\mathsf{e}]{\varphi(v), G \lor H, H \vdash \bot}
            \infer 3[\lor_\mathsf{e}]{\varphi(v), G \lor H \vdash \bot}
            \infer 1[\lnot_\mathsf{i}]{\varphi(v) \vdash \lnot (G \lor H)}
          \end{prooftree}
          .$
          }
      \end{itemize}
      \item Dans le cas $F = G \land H$, soit  $v$ une valuation sur $\mathcal{V}$.
        \begin{itemize}
          \item Si $v(F) = 1$ alors  $v(G) = v(H) = 1$ et donc, par hypothèse d'induction, $\varphi(v) \vdash G$ et $\varphi(v) \vdash H$.
            On construit la preuve :
            \[
            \begin{prooftree}
              \hypo{\varphi(v) \vdash G}
              \hypo{\varphi(v) \vdash H}
              \infer 2[\land_\mathsf{i}]{\varphi(v) \vdash G \land H}
            \end{prooftree}
            .\]
          \item Si $v(F) = 0$ alors  $v(G) = 0$ ou  $v(H) = 0$.
             \begin{itemize}
              \item Si $v(G) = 0$, on a donc  $\varphi(v) \vdash \lnot G$ par hypothèse d'induction, 
                et on construit la preuve :
                \[
                \begin{prooftree}
                  \infer 0[\mathsf{ax}]{\varphi(v), G \land H \vdash G \land H}
                  \infer 1[\land_\mathsf{e}^\mathsf{g}]{\varphi(v), G \land H \vdash G}
                  \hypo{\varphi(v) \vdash \lnot G}
                  \infer 1[\mathsf{aff}]{\varphi(v), G \land H \vdash \lnot G}
                  \infer 2[\lnot_\mathsf{e}]{\varphi(v), G \land H \vdash \bot}
                  \infer 1[\lnot_\mathsf{i}]{\varphi(v) \vdash \lnot (G \land H)}
                \end{prooftree}
                .\] 
              \item Si $v(H) = 0$, on a donc  $\varphi(v) \vdash \lnot H$ par hypothèse d'induction, 
                et on construit la preuve :
                \[
                \begin{prooftree}
                  \infer 0[\mathsf{ax}]{\varphi(v), G \land H \vdash G \land H}
                  \infer 1[\land_\mathsf{e}^\mathsf{d}]{\varphi(v), G \land H \vdash H}
                  \hypo{\varphi(v) \vdash \lnot H}
                  \infer 1[\mathsf{aff}]{\varphi(v), G \land H \vdash \lnot H}
                  \infer 2[\lnot_\mathsf{e}]{\varphi(v), G \land H \vdash \bot}
                  \infer 1[\lnot_\mathsf{i}]{\varphi(v) \vdash \lnot (G \land H)}
                \end{prooftree}
                .\] 
            \end{itemize}
        \end{itemize}
      \item Dans le cas $F = G \to H$, soit  $v$ une valuation sur $\mathcal{V}$.
        \begin{itemize}
          \item Si $v(F) = 1$ alors  $v(G) = 0$ ou  $v(H) = 1$.
             \begin{itemize}
              \item Si $v(G) = 0$ alors, par hypothèse d'induction,  on a $\varphi(v) \vdash \lnot G$.
                On construit la preuve :
                \[
                \begin{prooftree}
                  \infer 0[\mathsf{ax}]{\varphi(v), G \vdash G}
                  \hypo{\varphi(v) \vdash \lnot G}
                  \infer 1[\mathsf{aff}]{\varphi(v), G \vdash \lnot G}
                  \infer 2[\lnot_\mathsf{e}]{\varphi(v), G \vdash \bot}
                  \infer 1[\bot_\mathsf{i}]{\varphi(v), G \vdash H}
                  \infer 1[\to_\mathsf{i}]{\varphi(v) \vdash G \to H}
                \end{prooftree}
                .\]
                On utilise ici la variante $\bot_\mathsf{i}$ de l'absurdité classique $\bot_\mathsf{c}$ qui est facilement dérivable (c'est juste une application de $\bot_\mathsf{c}$ puis de $\mathsf{aff}$).
              \item Si $v(H) = 1$ alors, par hypothèse d'induction, on a  $\varphi(v) \vdash H$.
                On construit la preuve :
                \[
                \begin{prooftree}
                  \hypo{\varphi(v) \vdash H}
                  \infer 1[\mathsf{aff}]{\varphi(v), G \vdash H}
                  \infer 1[\to_\mathsf{i}]{\varphi(v) \vdash G \to H}
                \end{prooftree}
                .\]
            \end{itemize}
          \item Sinon, $v(F) = 0$, et on a donc  $v(G) = 1$ et  $v(H) = 0$, d'où, par hypothèse d'induction,
            on a $\varphi(v) \vdash G$ et $\varphi(v) \vdash \lnot H$.
            On construit la preuve 
            \[
              \hspace{-7em}
            \begin{prooftree}
              \infer 0[\mathsf{ax}]{\varphi(v), G \to H \vdash G \to H}
              \hypo{\varphi(v)\vdash G}
              \infer 1[\mathsf{aff}]{\varphi(v), G \to H \vdash G}
              \infer 2[\to_\mathsf{e}]{\varphi(v), G \to H \vdash H}
              \hypo{\varphi(v) \vdash \lnot H}
              \infer 1[\mathsf{aff}]{\varphi(v), G \to H \vdash \lnot H}
              \infer 2[\lnot_\mathsf{e}]{\varphi(v), G \to H \vdash \bot}
              \infer 1[\to_\mathsf{i}]{\varphi(v) \vdash \lnot(G \to H)}
            \end{prooftree}
            .\]
        \end{itemize}
      \item Dernier cas : si $F = \lnot G$, soit  $v$ une valuation sur $\mathcal{V}$.
        \begin{itemize}
          \item Si $v(F) = 1$ alors  $v(G) = 0$ et donc, par hypothèse d'induction,  $\varphi(v) \vdash \lnot G$.
            Or, $F = \lnot G$, on a donc une preuve de $\varphi(v) \vdash F$.
          \item Si $v(F) = 0$ alors   $v(G) = 1$ et donc, par hypothèse d'induction,  $\varphi(v) \vdash G$.
            On construit la preuve :
            \[
            \begin{prooftree}
              \hypo{\varphi(v) \vdash G}
              \infer 1[\mathsf{aff}]{\varphi(v), \lnot G \vdash G}
              \infer 0[\mathsf{ax}]{\varphi(v), \lnot G \vdash \lnot G}
              \infer 2[\lnot_\mathsf{e}]{\varphi(v), \lnot G \vdash \bot}
              \infer 1[\lnot_\mathsf{i}]{\varphi(v) \vdash \lnot \lnot G}
            \end{prooftree}
            .\] 
        \end{itemize}
  \end{itemize}
  \questionrem{
    Ceci conclut l'induction.
  }

  \question{
    On écrit \label{q3}\[
    \Phi(\mathcal{V}) := \bigvee_{i=2}^m \varphi(v_i)
    \]
    et on procède par récurrence sur $m$.
  }
  \begin{itemize}
    \item Cas de base ($m = 2$) : on considère $\varphi(v_1) \lor \varphi(v_2)$.
      Comme $F$ est une tautologie, $v_i(F) = 1$ et donc, par Q\ref{q2}, on a $\varphi(v_i) \vdash F$.
      On construit la preuve :
      
      \fitbox{
      $\begin{prooftree}
        \infer 0[\mathsf{ax}]{\varphi(v_1) \lor \varphi(v_2) \vdash \varphi(v_1) \lor \varphi(v_2)}
        \hypo{\varphi(v_1) \vdash F}
        \infer 1[\mathsf{aff}]{\varphi(v_1) \lor \varphi(v_2), \varphi(v_1) \vdash F}
        \hypo{\varphi(v_2) \vdash F}
        \infer 1[\mathsf{aff}]{\varphi(v_1) \lor \varphi(v_2), \varphi(v_2) \vdash F}
        \infer 3[\lor_\mathsf{e}]{\varphi(v_1) \lor \varphi(v_2) \vdash F}
      \end{prooftree}$
      }
    \item Hérédité : on considère $\big(\bigwedge_{i=2}^m \varphi(v_i)\big) \lor \varphi(v_{m+1})$.
      Comme $F$ est une tautologie, $v_{m+1}(F) = 1$ et donc, par Q\ref{q2}, on a $\varphi(v_{m+1}) \vdash F$.
      De plus, par hypothèse de récurrence, on a $\bigvee_{i=2}^m \varphi(v_i) \vdash F$.
      On construit la preuve :

      \fitbox{
      $\begin{prooftree}
        \infer 0[\mathsf{ax}]{\big(\bigwedge_{i=2}^m \varphi(v_i)\big) \lor \varphi(v_{m+1}) \vdash \big(\bigwedge_{i=2}^m \varphi(v_i)\big) \lor \varphi(v_{m+1})}
        \hypo{\big(\bigwedge_{i=2}^m \varphi(v_i)\big) \vdash F}
        \infer 1[\mathsf{aff}]{\big(\bigwedge_{i=2}^m \varphi(v_i)\big) \lor \varphi(v_{m+1}), \big(\bigwedge_{i=2}^m \varphi(v_i)\big) \vdash F}
        \hypo{\varphi(v_{m+1}) \vdash F}
        \infer 1[\mathsf{aff}]{\big(\bigwedge_{i=2}^m \varphi(v_i)\big) \lor \varphi(v_{m+1}), \varphi(v_{m+1}) \vdash F}
        \infer 3[\lor_\mathsf{e}]{\big(\bigwedge_{i=2}^m \varphi(v_i)\big) \lor \varphi(v_{m+1}) \vdash F}
      \end{prooftree}$
      }
  \end{itemize}
  \questionrem{
    Ceci conclut la récurrence.
  }

  \question{
    L'énoncé du théorème de complétude est : si $\models F$ alors on a que $\vdash F$.
    Si l'on a $\models F$, alors $F$ est une tautologie, on peut donc appliquer~Q\ref{q3}.
    On applique aussi Q\ref{q1}.
    Construisons la preuve :
    \[
      \begin{prooftree}
        \hypo{\text{Q\ref{q3}}}
        \infer[rule style=no rule] 1{\Phi(\mathcal{V}) \vdash F}
        \infer 1[\to_\mathsf{i}]{\vdash \Phi(V) \to F}
        \hypo{\text{Q\ref{q1}}}
        \infer[rule style=no rule] 1{\vdash \Phi(\mathcal{V})}
        \infer 2[\to_\mathsf{e}]{\vdash F}
      \end{prooftree}
    .\]
    D'où le théorème de complétude.
  }
\end{document}
