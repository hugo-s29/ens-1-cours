\documentclass[../main]{subfiles}

\begin{document}
  \chapter{Réécriture.} \label{thprog-chap10}
  \minitoc

  \begin{defn}
    Soit $\to$ une relation binaire sur un ensemble $E$. Le~$2$-uplet $(E, \to)$ est un \textit{SRA}, pour \textit{système de réécriture abstraite}.

    Soit $x_0 \in E$. Une \textit{divergence} issue de $x_0$ est une suite $(x_i)_{i \in \mathds{N}}$ telle que, pour tout $i$, on a $x_i \to x_{i+1}$.

    La relation $\to$ est \textit{terminante} ou \textit{termine} si et seulement si, quel que soit $x \in E$, il n'y a pas de divergence issue de $x$.

    La relation $\to$ diverge s'il existe une divergence.
  \end{defn}

  \begin{exm}
    En général, une relation réflexive est divergente.
  \end{exm}

  \begin{thm}
    Une relation $(E, \to)$ est terminante si et seulement si elle satisfait  le \textit{principe d'induction bien fondée (PIBF)} suivant :
    \begin{quote}
      Pour tout prédicat $\mathcal{P}$ sur $E$, si pour tout $x \in E$ \[
        \big[\forall y \in E, x \to y \text{ implique } \mathcal{P}(y)\big] \text{ implique } \mathcal{P}(x)
      \] alors, pour tout $x \in E$, $\mathcal{P}(x)$.
    \end{quote}

    En particulier, dans le principe d'induction bien fondée, on demande que les feuilles (les éléments sans successeurs) vérifient le prédicat.
  \end{thm}
  \begin{prv}
    \begin{itemize}
      \item "PIBF $\implies$ terminaison".
        Montrons que, quel que soit $x \in E$, \[
        \mathcal{P}(x) \; : \; \text{"il n'y a pas de divergence issue de $x$"}
        .\]
        Soit $\mathrm{Next}(x) = \{y \in E  \mid x \to y\}$.
        On suppose que, pour tout $y \in \mathrm{Next}(x)$, on a $\mathcal{P}(y)$.
        On en déduit $\mathcal{P}(x)$ car, sinon, une divergence ne passerait pas par $y \in \mathrm{Next}(x)$.
        Par le principe d'induction bien fondée, on en déduit \[\forall x \in E, \mathcal{P}(x),\] autrement dit, la relation $\to$ termine.
      \item "$\lnot$PIBF  $\implies$ diverge", par contraposée.
        On suppose qu'il existe un prédicat $\mathcal{P}$ tel que, \[
        \forall x, (\forall y, x \to y \text{ implique } \mathcal{P}(y)) \text{ implique }\mathcal{P}(x)
        ,\] et que l'on n'ait pas, $\forall x \in E, \mathcal{P}(x)$ autrement dit qu'il existe $x_0 \in E$ tel que $\lnot \mathcal{P}(x)$.

        Intéressons-nous à $\mathrm{Next}(x_0) = \{ y \in E \mid x_0 \to y\}$.
        Si, pour tout $y \in \mathrm{Next}(x_0)$ on a $\mathcal{P}(y)$ alors par hypothèse $\mathcal{P}(x_0)$, ce qui est impossible.
        Ainsi, il existe $x_1 \in \mathrm{Next}(x_0)$ tel que~$\lnot \mathcal{P}(x_1)$.
        On itère ce raisonnement, ceci crée notre divergence.
    \end{itemize}
  \end{prv}

  \begin{rmk}
    L'induction bien fondée s'appelle aussi l'induction~\textit{noethérienne}, en référence à Emmy Noether, mathématicienne allemande du IX--Xème siècle.
  \end{rmk}

  Une application de ce principe d'induction est le \textit{lemme de König}.
  \begin{defn}
    \begin{itemize}
      \item Un arbre est \textit{fini} s'il a un nombre fini de nœuds (\textit{infini} sinon).
      \item Un arbre est à \textit{branchement fini} si tout nœud a un nombre fini d'enfants immédiats.
      \item Une branche est \textit{infinie} si elle contient un nombre infini de nœuds.
    \end{itemize}
  \end{defn}

  \begin{lem}[Lemme de König]
    Si un arbre est à branchement fini est infini alors il contient une branche infinie.
  \end{lem}

  \begin{prv}
    On considère $E$ l'ensemble des nœuds de l'arbre, et on définit la relation $\to$ par : on a  $x\to y$ si $y$ est enfant immédiat de~$x$.
    On montre qu'un arbre à branchement fini sans branche infinie (\textit{i.e.} la relation $\to$ termine) est fini.
    On choisit la propriété~$\mathcal{P}(x)$ : "le sous-arbre enraciné en $x$ est fini."

    Montrons que, quel que soit $x$, $\mathcal{P}(x)$ et pour ce faire, utilisons le principe d'induction bien fondée puisque la relation $\to$ termine.
    On doit montrer que, si $\forall y \in \mathrm{Next}(x), \mathcal{P}(y)$ implique $\mathcal{P}(x)$.
    Ceci est vrai car l'embranchement est fini.
  \end{prv}


  \section{Liens avec les définitions inductives.}


  On considère $E$ l'ensemble inductif défini par la grammaire suivante :
  \[
  t ::= \mathtt{F}  \mid \mathtt{N}(t_1, k, t_2)
  .\]
  C'est aussi le plus petit point fixe de l'opérateur $f$ associé (par le théorème de Knaster--Tarski).

  On définit la relation $\to$ binaire sur $E$ par : on a $x \to y$ si et seulement si on a $x = \mathtt{N}(y,k,z)$ ou $x = \mathtt{N}(z,k,y)$.

  On sait que la relation $\to$ termine. En effet, l'ensemble des arbres finis est un point fixe de la fonction $f$, donc $E$ ne contient que des arbres finis.

  Le principe d'induction bien fondée nous dit que, pour $\mathcal{P}$ un prédicat sur $E$, pour montrer $\forall x, \mathcal{P}(x)$, il suffit de montrer que, quel que soit~$x$, si $(\forall y, x \to y \text{ implique } \mathcal{P}(y))$ alors $\mathcal{P}(x)$.
  Autrement dit, il suffit de montrer que $\mathcal{P}(\mathtt{E})$ puis de montrer que, si $\mathcal{P}(t_1)$ et $\mathcal{P}(t_2)$ alors on a que~$\mathcal{P}(\mathtt{N}(t_1, k, t_2))$.

  On retrouve le principe d'induction usuel.

  Ce même raisonnement, on peut le réaliser quel que soit l'ensemble inductif, car la relation de "sous-élément" termine toujours puisque il n'y a que des éléments finis dans l'ensemble inductif.

  \section{Établir la terminaison.}

  \begin{thm}
    Soient $(B, >)$ un SRA terminant, et $(A, \to)$ un SRA.
    Soit $\varphi : A \to B$ un \textit{plongement}, c'est à dire une application vérifiant \[
    \forall a, a' \in A, \quad\quad a \to a' \text{ implique } \varphi(a) > \varphi(a')
    .\]
    Alors, la relation $\to$ termine.
  \end{thm}

  \begin{thm}
    Soient $(A, \to_A)$ et $(B, \to_B)$ deux SRA.

    Le \textit{produit lexicographique} de $(A, \to_A)$ et $(B, \to_B)$ est le SRA, que l'on notera $(A \times B, \to_{A \times B})$, défini par \[
      (a, b) \to_{A \times B} (a', b') \text{ ssi } \begin{cases}
        (1) \; a \to_A a' \text{ (et $b'$ quelconque) }\\[-0.2em]
        \quad\quad\quad\text{ou}\\[-0.2em]
        (2)\; a = a' \text{ et } b \to_B b'
      \end{cases}
    .\]

    Alors, les relations $(A, \to_A)$ et $(B, \to_B)$ terminent si et seulement si la relation $(A \times B, \to_{A \times B})$ termine.
  \end{thm}

  \begin{prv}
    \begin{itemize}
      \item "$\implies$".
        Supposons qu'il existe une divergence pour $(A \times B, \to_{A \times B})$ :
        \[
          (a_0, b_0) \to_{A \times B} (a_1, b_1) \to_{A \times B} (a_2, b_2) \to_{A \times B} \cdots
        .\] 

        Dans cette divergence,
        \begin{itemize}
          \item soit on a utilisé $(1)$ une infinité de fois, et alors en projetant sur la première composante et en ne conservant que les fois où l'on utilise $(1)$, on obtient une divergence  $\to_A$ ;
          \item soit on a utilisé $(1)$ un nombre fini de fois, et alors à partir d'un certain rang $N$, pour tout $i \ge N$, on a l'égalité $a_i = a_N$, et donc on obtient une divergence pour $\to_B$ : \[
          b_N \to_B b_{N+1} \to_B b_{N+2} \to \cdots 
          .\]
        \end{itemize}
      \item "$\impliedby$".
        On montre que, si on a une divergence pour $\to_A$ alors on a une divergence pour $\to_{A \times B}$ (on utilise $(1)$ une infinité de fois) ; puis que si on a une divergence pour $\to_B$ alors on a une divergence pour $\to_{A \times B}$ (on utilise $(2)$ une infinité de fois).
    \end{itemize}
  \end{prv}

  \section{Application à l'algorithme d'unification.}

  On note $(\mathcal{P}, \sigma) \to (\mathcal{P}', \sigma')$ la relation définie par l'algorithme d'unification (on néglige le cas où $(\mathcal{P}, \sigma) \to \bot$).

  On note $|\mathcal{P}|$ la somme des tailles (vues comme des arbres) des contraintes de $\mathcal{P}$ et $|\mathsf{Vars}\ \mathcal{P}|$ le nombre de variables.

  On définit $\varphi : (\mathcal{P}, \sigma) \mapsto (|\mathsf{Vars}\ \mathcal{P}|, |\mathcal{P}|)$.

  Rappelons la définition de la relation $\to$ dans l'algorithme d'unification :
  \begin{enumerate}
    \item\label{chap11-unif-1} $ \begin{array}{c}
        \\
        (\{f(t_1,\ldots,t_k) \qeq f(u_1,\ldots, u_n) \sqcup \mathcal{P}, \sigma\}) \to\\
        \quad\quad\quad\quad(\{t_1 \qeq u_1, \ldots, t_k \qeq u_k\} \cup \mathcal{P}, \sigma) \quad ;
      \end{array} $
    \item \label{chap11-unif-2} $(\{f(t_1,\ldots,t_k) \qeq g(u_1,\ldots, u_n) \sqcup \mathcal{P}, \sigma\}) \to \bot$ si $f \neq g$ ;
    \item \label{chap11-unif-3} $(\{X\qeq t\} \sqcup \mathcal{P}, \sigma) \to (\mathcal{P}[\sfrac{t}{X}], [\sfrac{t}{X}] \circ \sigma)$ où $X \not\in \mathsf{Vars}(t)$ ;
    \item \label{chap11-unif-4} $(\{X \qeq t\}\sqcup \mathcal{P}, \sigma) \to \bot$ si $X \in \mathsf{Vars}(t)$ et $t \neq X$ ;
    \item \label{chap11-unif-5} $(\{X \qeq X\}\sqcup \mathcal{P}, \sigma) \to (\mathcal{P}, \sigma)$.
  \end{enumerate}

  Appliquons le plongement pour montrer que $\to$ termine.
  On s'appuie sur le fait que le produit $(\mathds{N}, >) \times (\mathds{N}, >)$ est terminant (produit lexicographique).

  Dans~\ref{chap11-unif-1}, $|\mathsf{Vars}\ \mathcal{P}|$ ne change pas et $|\mathcal{P}|$ diminue.
  Puis dans~\ref{chap11-unif-3}, $|\mathsf{Vars}\ \mathcal{P}|$ diminue.
  Et dans~\ref{chap11-unif-5}, on a $|\mathsf{Vars}\ \mathcal{P}|$ qui décroit ou ne change pas, mais~$|\mathcal{P}|$ diminue.
  Dans les autres cas, on arrive, soit sur $\bot$.

  On en conclut que l'algorithme d'unification termine.


  \section{Multiensembles.}

  \begin{defn}
    Soit $A$ un ensemble. Un \textit{multiensemble} sur $A$ est la donnée d'une fonction $M : A \to \mathds{N}$.
    Un multiensemble $M$ est \textit{fini} si $\{a \in A  \mid M(a) > 0\}$ est fini.

    Le multiensemble vide, noté $\emptyset$, vaut $a \mapsto 0$.

    Pour deux multiensembles $M_1$ et $M_2$ sur $A$, on définit 
    \begin{itemize}
      \item  $(M_1 \cup M_2)(a) = M_1(a) + M_2(a)$ ;
      \item $(M_1 \ominus M_2)(a) = M_1(a) \ominus M_2(a)$ où l'on a $(n + k)\ominus n = k$ mais $n \ominus (n + k) = 0$.
    \end{itemize}

    On note $M_1 \subseteq M_2$ si, pour tout $a \in A$, on a $M_1(a) \le M_2(a)$.

    La \textit{taille} de $M$ est $|M| = \sum_{a \in A} M(a)$.

    On note $x \in M$ dès lors que $x \in A$ et que $M(x) > 0$.
  \end{defn}

  \begin{exm}
    Si on lit $\{1, 1, 1, 2, 3, 4, 3, 5\}$ comme un multiensemble $M$, on obtient que $M(1) = 3$, et $M(2) = 1$, et $M(3) = 2$, et $M(4) = 1$, et $M(5) = 1$, et finalement pour tout autre entier  $n$, $M(n) = 0$.
  \end{exm}

  \begin{defn}[Extension multiensemble.]
    Soit $(A, >)$ un SRA.
    On lui associe une relation notée  $>_\mathrm{mul}$ définie sur les multiensembles \textit{finis} sur $A$ en définissant~$M >_\mathrm{mul} N$ si et seulement s'il existe $X,Y$ deux multiensembles sur $A$ tels que 
    \begin{itemize}
      \item $\emptyset \neq X \subseteq M$ ;
      \item $N = (M \ominus X) \cup Y$\footnote{C'est ici la soustraction usuelle : il n'y a pas de soustraction qui "pose problème".}
      \item $\forall  y \in Y, \exists x \in X, x > y$.
    \end{itemize}
  \end{defn}

  Les multiensembles $X$ et $Y$ sont les "témoins" de $M >_\mathrm{mul} N$.

  \begin{exm}
    Dans $(\mathds{N}, >)$, on a \[
      \{1, 2, \underbrace{5}_X\} >_\mathrm{mul} \{1, 2, \underbrace{4, 4, 4, 4, 3, 3, 3, 3, 3}_Y\} 
    .\]
  \end{exm}

  \begin{thm}
    La relation $>$ termine si et seulement si $>_\mathrm{mul}$ termine.
  \end{thm}
  \begin{prv}
    \begin{itemize}
      \item "$\impliedby$". Une divergence de $>$ induit une divergence de  $>_\mathrm{mul}$.
      \item "$\implies$".
        On se donne une divergence pour $>_\mathrm{mul}$ :
        \[
        M_0 >_\mathrm{mul} M_1 >_\mathrm{mul} M_2 >_\mathrm{mul} \cdots
        ,\] et on montre que $>$ diverge
        À chaque $M_i >_\mathrm{mul} M_{i+1}$ correspondent $X_i$ et $Y_i$ suivant la définition de $>_\mathrm{mul}$.

        On sait qu'il y a une infinité de $i$ tel que $Y_i \neq \emptyset$.
        En effet, si au bout d'un moment $Y_i$ est toujours vide alors 
        $|M_i|$ décroit strictement, impossible.

        Représentons cela sur un arbre.

        \begin{figure}[H]
          \centering
          \begin{tikzpicture}
            \node[draw, rounded rectangle] (1) {racine};
            \node[draw, rounded rectangle,below left=0.5cm and 1cm of 1] (2a) {};
            \node[draw, rounded rectangle,below left=0.5cm and 0cm of 1, fill=black] (2b) {};
            \node[draw, rounded rectangle,below right=0.5cm and 0cm of 1] (2c) {};
            \node[draw, rounded rectangle,below right=0.5cm and 1cm of 1, fill=black] (2d) {};
            \draw (1) -- (2a);
            \draw (1) -- (2b);
            \draw (1) -- (2c);
            \draw (1) -- (2d);
            \node[draw, rounded rectangle,below left=0.5cm and 0.25cm of 2b] (3a) {};
            \node[draw, rounded rectangle,below right=0.5cm and 0.25cm of 2b] (3b) {};
            \node[draw, rounded rectangle,below=0.5cm of 2d] (3c) {};
            \draw (2b) -- (3a);
            \draw (2b) -- (3b);
            \draw (2d) -- (3c);
            \node[draw,deepred,rectangle,fit=(2a) (2b) (2c) (2d)] (M0) {};
            \node[deepred,right=0.5cm of 2d] {$M_0$};
            \draw[deepblue] plot [smooth cycle, tension=0.7] coordinates {(-1.6,-0.6) (-0.3, -1.4) (0.5, -0.6) (1.6, -1.5) (1.6, -2.1) (0.5, -1.3) (-0.3, -2.1) (-1.6, -1.2)};
            \node[deepblue,right=0.5cm of 3c] {$M_1$};
          \end{tikzpicture}
        \end{figure}

        On itère le parcours en obtenant un arbre à branchement fini, qui est infini (observation du dessin) donc par le lemme de König il a une branche infinie.
        Par construction, un enfant de $a$ correspond à $a > a'$, d'où divergence pour $>$.
    \end{itemize}
  \end{prv}

  \begin{thm}[Récursion bien fondée]
    On appelle \textit{fonction} de~$A_1$ dans $A_2$ la donnée d'une relation fonctionnelle totale incluse dans~$A_1 \times A_2$.
    On note $f : A_1 \to A_2$.

    On se donne $(A, >)$ un SRA \textbf{terminant}.

    Pour $a \in A$, on note 
    \begin{itemize}
      \item $\mathrm{Pred}(a) := \{a'  \mid a > a'\}$ ;\footnote{On le notait $\mathrm{Next}$ avant, mais le successeur pour $>$ est un prédécesseur pour $<$ (ce qui est plus usuel).}
      \item $\mathrm{Pred}^+(a) := \{a'  \mid a >^+ a'\}$ ;
      \item $\mathrm{Pred}^\star(a) := \{a'  \mid a >^\star a'\} = \mathrm{Pred}^+(a) \cup \{a\}$.\footnote{On rappelle que l'on note $\mathcal{R}^+$ et respectivement $\mathcal{R}^\star$ la clôture transitive, et respectivement la clôture réflexive et transitive.}
    \end{itemize}

    Pour $f : A \to B$ et $A' \subseteq A$, on note $f \upharpoonright A' := \{(a, f(a))  \mid a \in A'\}$.

    On se donne une fonction $F$ telle que, pour tout $a \in A$, et tout~$h \in \mathrm{Pred}(a) \to B$, on a $F(a,h) \in B$.
    Alors, il existe une unique fonction ${\color{deepblue} f} : A \to B$ telle que \[
      \forall a \in A, {\color{deepblue} f}(a) = F(a, {\color{deepblue} f} \upharpoonright(\mathrm{Pred}(a)))
    .\] 
  \end{thm}
  \begin{prv}
    \begin{description}
      \item[Unicité.]
        Soient $f, g$ telles que, pour tout $a \in A$, on ait 
        \begin{itemize}
          \item $f(a) = F(a, f \upharpoonright \mathrm{Pred}(a))$ ;
          \item $g(a) = F(a, g \upharpoonright \mathrm{Pred}(a))$.
        \end{itemize}
        Montrons que $\forall a \in A, f(a) = g(a)$ par induction bien fondée (car $>$ termine).

        Soit $a \in A$. On suppose $\forall b \in \mathrm{Pred}(b), f(b) = g(b)$ l'hypothèse d'induction.
        Alors, $f \upharpoonright \mathrm{Pred}(a) = g \upharpoonright \mathrm{Pred}(a)$, et donc $f(a) = g(a)$.
      \item[Existence.]
        On montre, par induction bien fondée, que $\mathcal{P}(a)$, la propriété ci-dessous, est vraie quel que soit $a \in A$ :
        \[
        \exists  f_a : \mathrm{Pred}^\star(a) \to B, 
        \forall b \in \mathrm{Pred}^\star(a), f_a(b) = F(b, f \upharpoonright \mathrm{Pred}(b))
        .\] 

        Soit $a \in A$. On suppose $\forall b \in \mathrm{Pred}(a), \mathcal{P}(b)$ ($f_b$ existe).
        Posons la relation $h := \bigcup_{b \in \mathrm{Pred}(a)} f_b$.
        La relation $h$ est 
        \begin{itemize}
          \item fonctionnelle (\textit{c.f.} preuve d'unicité) ;
          \item définie sur $\mathrm{Pred}^+(a)$.
        \end{itemize}
        On conclut la preuve en posant \[
          f_a := h \cup \{(a, F(a, h))\} 
        .\]
    \end{description}
  \end{prv}

  \begin{exm}
    Pour définir une fonction $\mathtt{length} : \mathsf{nlist} \to \mathsf{nat}$.
    La relation $>$ sur $\mathsf{nlist}$ où $k :: \ell > \ell$ est une relation bien fondée.
    On pose la fonction $F(\ell, h)$ par :
    \begin{lstlisting}[language=caml,caption={Définition récursive bien fondée de $\mathtt{length}$}]
      let $F$ $\ell$ $h$ = match $\ell$ with
      | [] -> 0
      | x :: xs -> 1 + $h$(xs)
    \end{lstlisting}
    où l'on a ici $\mathtt{xs} \in \mathrm{Pred}(\texttt{x :: xs})$.
  \end{exm}

  \section{Confluence.}

  \begin{defn}
    Soit $(A, \to)$ un SRA. On dit que $\to$ est \textit{confluente} en $t \in A$ si, dès que $t \to^\star u_1$ et $t \to^\star u_2$ il existe $t'$ tel que $u_1 \to^\star t'$ et $u_2 \to^\star t'$.
    \[
    \begin{tikzcd}
      & t \arrow[near end]{dr}{\star} \arrow[near end,swap]{dl}{\star}\\
      u_1 \arrow[near end,swap,dashed]{dr}{\star} && u_2 \arrow[near end,dashed]{dl}{\star}\\
      & t'
    \end{tikzcd}
    .\]

    Les flèches en pointillés représentent l'existence.

    On dit que $\to$ est \textit{confluente} si $\to $ est confluente en tout $a \in A$.

    La propriété du diamant correspond au diagramme ci-dessous :
    \[
    \begin{tikzcd}
      & t \arrow[near end]{dr}{} \arrow[near end,swap]{dl}{}\\
      u_1 \arrow[near end,swap,dashed]{dr}{} && u_2 \arrow[near end,dashed]{dl}{}\\
      & t'
    \end{tikzcd}
    ,\]
    c'est-à-dire si $t \to u_1$ et $t\to u_2$ alors il existe $t'$ tel que $u_1 \to t'$ et $u_2 \to t'$.

    La propriété de \textit{confluence locale} correspond au diagramme ci-dessous :
    \[
    \begin{tikzcd}
      & t \arrow[near end]{dr}{} \arrow[near end,swap]{dl}{}\\
      u_1 \arrow[near end,swap,dashed]{dr}{\star} && u_2 \arrow[near end,dashed]{dl}{\star}\\
      & t'
    \end{tikzcd}
    ,\]
    c'est-à-dire si $t \to u_1$ et $t\to u_2$ alors il existe $t'$ tel que $u_1 \to\star t'$ et $u_2 \to\star t'$.
  \end{defn}

  \begin{lem}[Lemme de Newman]
    Soit $(A, \to)$ terminante et localement confluente.
    Alors, $(A, \to)$ confluente.
  \end{lem}
  \begin{prv}
    On montre que, quel que soit $t \in A$, que $\to$ est confluente en $t$ par induction bien fondée.
    On suppose que quel que  soit $t ''$ tel que $t \to t'$, alors la relation $\to$ est confluente en $t'$ ;
    Montrons la confluence en $t$.
    Soit $t\to^\star u_1$ et $t \to^\star u_2$.
    Si $t = t_1$ ou $t = u_2$, c'est immédiat.
    On suppose donc 
    \[
    \begin{tikzcd}[row sep={40,between origins}, column sep={40,between origins}]
      && \mathclap{t} \arrow{dr}{} \arrow{dl}{}\\
      &\arrow[near end,swap]{dl}{\star} \arrow[deepblue,near end,swap]{dr}{\star}& {\color{deepblue} \mathclap{(1)}}&\arrow[deepblue,near end]{dl}{\star} \arrow[near end]{dr}{\star} \arrow[bend left,near end, dotted]{ddll}{\star}\\
      \mathclap{u_1} \arrow[swap,deeppurple,near end]{dr}{\star} & {\color{deeppurple} \mathclap{(2)}} & \arrow[deeppurple,near end]{dl}{\star} & {\color{deepgreen} \mathclap{(3)}} & \mathclap{u_2} \arrow[deepgreen,near end]{ddll}{\star}\\
      &\arrow[swap,deepgreen,near end]{dr}{\star}\\
      &&\mathclap{t'}
    \end{tikzcd}
    ,\] 
    où l'on utilise 
    \begin{itemize}
      \item {\color{deepblue}(1)} par confluence locale ;
      \item {\color{deeppurple}(2)} par hypothèse d'induction ;
      \item {\color{deepgreen}(3)} par hypothèse d'induction.
    \end{itemize}
  \end{prv}

  \begin{rmk}
    L'hypothèse de relation terminante est cruciale.
    En effet, la relation ci-dessous est un contre exemple : la relation~$\to$ est non terminante, localement confluente mais pas confluente.
    \[
    \begin{tikzcd}
      \bullet&\arrow{l}{}\arrow[bend right]{r} \bullet&\arrow{r}{}\arrow[bend right]{l}{} \bullet&\bullet
    \end{tikzcd}
    .\]
  \end{rmk}

  \section{Système de réécriture de mots.}

  Les systèmes de réécriture de mots sont parmi les systèmes de réécriture les plus simples.
  On peut définir un système de réécriture de termes, où "terme" correspond à "terme" dans la partie Typage, mais on ne les étudiera pas dans ce cours.

  \begin{defn}[\textit{c.f.} cours de FDI]
    On se donne $\Sigma$ un ensemble de lettres.
    On note :
    \begin{itemize}
      \item $\Sigma^\star$ l'ensemble des mots sur $\Sigma$,
      \item $\varepsilon$ le mot vide,
      \item $uv$ la concaténation de $u$ et $v$ (avec $u,v \in \Sigma^\star$).
    \end{itemize}
  \end{defn}

  \begin{defn}
    Un \textit{SRM} (\textit{système de réécriture de mots}) sur $\Sigma$ est donné par un ensemble $\mathcal{R}$ de couples de mots sur $\Sigma$ noté généralement  \[
    \mathcal{R} = \{(\ell,r)  \mid \ell \neq \varepsilon\} 
    .\]

    Le SRM $\mathcal{R}$ détermine une relation binaire sur $\Sigma^\star$ définie par $u \to_\mathcal{R} v$
    dès lors qu'il existe $(x,y) \in (\Sigma^\star)^2$ et $(\ell,r) \in \mathcal{R}$ tels que l'on ait $u = x \ell y$ et $v = x r y$.
  \end{defn}

  \begin{exm}
    Sur $\Sigma = \{a,b,c\}$ et $\mathcal{R}_0$ donné par
    \begin{align*}
      ab & \to ac \\
      ccc & \to bb \\
      aa &\to a
    ,\end{align*}
    autrement dit \[
    \mathcal{R}_0 = \{(ab, ac), (c c c, b b), (aa, a)\} 
    ,\] 
    et alors 
    \[
    \begin{tikzcd}
      aabab \arrow[swap,near end]{r}{\mathcal{R}} \arrow[swap,near end]{dr}{\mathcal{R}} \arrow[swap,near end]{d}{\mathcal{R}} & aacab\\
      aabac & ab ab
    \end{tikzcd}
    .\]

    La relation $\to_{\mathcal{R}_0}$ est-elle terminante ?
    Oui, il suffit de réaliser un plongement sur le produit lexicographique donné par $\varphi : u \mapsto (|u|, \#_b(u))$, où $\#_b(u)$ est le nombre de  $b$ dans $u$ et $|u|$ est la taille de $u$.
  \end{exm}

  \subsection{Étude de la confluence locale dans les SRM.}

  Soient $(\ell_1, r_1), (\ell_2, r_2) \in \mathcal{R}$ tels que $t \to_{\mathcal{R}} u_1$ avec $(\ell_1, r_1)$ et $t \to_\mathcal{R}$ avec $(\ell_2, r_2)$.

  \begin{description}
    \item[\textit{1er cas} : indépendance.]
      On a la propriété du diamant.
      \[
      \begin{tikzcd}
        &
          \begin{tikzpicture}[scale=0.8]
            \node (1) at (0,0) {};
            \node (2) at (3,0) {};
            \node (l) at (1,0) {};
            \node (r) at (2,0) {};
            \draw (1) -- (2);
            \draw[ultra thick,deeppurple] (l.west) -- (l.east);
            \node[deeppurple,below=0.05cm of l] {$\ell_1$};
            \node[deepgreen,below=0.05cm of r] {$\ell_2$};
            \node[above=0.05cm of l] {\phantom{$\ell_1$}};
            \node[above=0.05cm of r] {\phantom{$\ell_2$}};
            \draw[ultra thick,deepgreen] (r.west) -- (r.east);
          \end{tikzpicture}
         \arrow[deeppurple]{dl}{} \arrow[deepgreen]{dr}{} & \\
          \begin{tikzpicture}[scale=0.8]
            \node (1) at (0,0) {};
            \node (2) at (3,0) {};
            \node (l) at (1,0) {};
            \node (r) at (2,0) {};
            \draw (1) -- (2);
            \draw[ultra thick,deeppurple] (l.west) -- (l.east);
            \node[deeppurple,below=0.05cm of l] {$r_1$};
            \node[deepgreen,below=0.05cm of r] {$\ell_2$};
            \node[above=0.05cm of l] {\phantom{$\ell_1$}};
            \node[above=0.05cm of r] {\phantom{$\ell_2$}};
            \draw[ultra thick,deepgreen] (r.west) -- (r.east);
          \end{tikzpicture}
        \arrow[deepgreen]{dr}{} & & 
          \begin{tikzpicture}[scale=0.8]
            \node (1) at (0,0) {};
            \node (2) at (3,0) {};
            \node (l) at (1,0) {};
            \node (r) at (2,0) {};
            \draw (1) -- (2);
            \draw[ultra thick,deeppurple] (l.west) -- (l.east);
            \node[deeppurple,below=0.05cm of l] {$\ell_1$};
            \node[deepgreen,below=0.05cm of r] {$r_2$};
            \node[above=0.05cm of l] {\phantom{$\ell_1$}};
            \node[above=0.05cm of r] {\phantom{$\ell_2$}};
            \draw[ultra thick,deepgreen] (r.west) -- (r.east);
          \end{tikzpicture}
        \arrow[deeppurple]{dl}{}\\
        & 
          \begin{tikzpicture}[scale=0.8]
            \node (1) at (0,0) {};
            \node (2) at (3,0) {};
            \node (l) at (1,0) {};
            \node (r) at (2,0) {};
            \draw (1) -- (2);
            \draw[ultra thick,deeppurple] (l.west) -- (l.east);
            \node[deeppurple,below=0.05cm of l] {$r_1$};
            \node[deepgreen,below=0.05cm of r] {$r_2$};
            \node[above=0.05cm of l] {\phantom{$\ell_1$}};
            \node[above=0.05cm of r] {\phantom{$\ell_2$}};
            \draw[ultra thick,deepgreen] (r.west) -- (r.east);
          \end{tikzpicture}
      \end{tikzcd}
      .\] 
    \item[\textit{2ème cas} : inclusion.]
      S'il existe $(\ell,\ell')$ tel que $\ell_1 = \ell \ell_2 \ell'$,
      alors, on a que $\ell_1 \to_\mathcal{R} r_1$ et $\ell_2 \to_\mathcal{R} \ell r_2 \ell'$.

      Peut-on confluer ? Ce n'est pas évident.

    \item[\textit{3ème cas} : overlap.]
      S'il existe $(\ell,\ell', \ell'')$ tel que $\ell_1 = \ell \ell'$ et $\ell_2 = \ell' \ell''$, 
      alors $t \to r_1 \ell''$ et $t \to \ell r_2$.

      Peut-on confluer ? Ce n'est pas évident.
  \end{description}

  \begin{defn}
    Soit $\mathcal{R}$ un SRM sur $\Sigma$.
    Soient  $(\ell_1, r_1), (\ell_2, r_2) \in \mathcal{R}$.
    Supposons qu'il existe des mots $z, v_1, v_2$ tels que  
    \begin{itemize}
      \item $|z| < |\ell_1|$ ;
      \item $\ell_1 v_1 = z \ell_2 v_2$ ;
      \item $\varepsilon \in \{v_1, v_2\}$.
    \end{itemize}
    On dit alors que $\{r_1 v_1, z r_2 v_2\}$ est une paire critique de $\mathcal{R}$.
  \end{defn}

  \begin{exm}
    Avec $\Sigma = \{a,b,c\}$ avec le SRM $\mathcal{R}$ défini par \[
    \begin{array}{crcl}
      (1)&aba&\to&abc\\
      (2)&ab&\to&ba
    \end{array}
    ,\]
    on se demande si 
    \begin{itemize}
      \item $\to_\mathcal{R}$ termine ? Oui avec  le nombre de $a$ et le nombre d'inversions $a$--$b$.
      \item quelles sont les paires critiques ?
        On procède cas par cas :
        \begin{itemize}
          \item $(1)$ avec $(2)$, on a $aba \to abc$ et $aba \to baa$ donc $\{abc, baa\}$ est critique ;
          \item $(1)$ avec $(1)$, on a $ababa \to abcba$ et $ababa \to ababc$ donc $\{abcba, ababc\}$ est critique ;
          \item $(2)$ avec $(2)$, il n'y a pas de paires critiques.
        \end{itemize}
    \end{itemize}
  \end{exm}


  Pourquoi s'intéresser aux paires critiques ? Et bien, cela prend son sens grâce au théorème ci-dessous.
  \begin{thm}
    La relation $\to_\mathcal{R}$ est localement confluente si et seulement si toutes ses paires critiques sont \textit{joignables}, c'est-à-dire que, pour $\{u_1,u_2\}$ critique, il existe $t'$ tel que $u_1 \to^\star t'$ et~$u_2 \to^\star t'$.
  \end{thm}
\end{document}
