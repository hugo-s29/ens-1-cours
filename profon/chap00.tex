\documentclass[./main]{subfiles}

\begin{document}
  \chapter*{Introduction.}\addstarredchapter{Introduction.}

  Le coups se décompose en deux parties :
  \begin{itemize}
    \item une partie \textit{pratique} ($\sim 2/3$ du cours) avec de la programmation OCaml en binôme (\fouine, puis \pieuvre) ;
    \item une partie \textit{théorique} : avec du $\lambda$-calcul, typage et lien avec la logique.
  \end{itemize}

  Ce document contiendra les notes de cours pour la partie théorique.
  D'autres documents pour la partie projet sont en ligne sur mon site (transformation CPS, et optimisations associées à la $\beta$-réduction).

  Parfois, des références aux chapitres du cours de théorie de la programmation seront fait, ce sont des liens cliquables qui mènent vers le PDF du chapitre en question, par exemple :
  \begin{center}
    \thprog 0.
  \end{center}
\end{document}
