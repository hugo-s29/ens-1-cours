\documentclass{../../td}

\title{TD n°2\\\itshape Théorie des catégories}
\author{Hugo \textsc{Salou}\\ Dept. Informatique}

\renewcommand\thesection{\arabic{chapter}.\Alph{section}}
\renewcommand\thepart{\Roman{part}}
\renewcommand\thechapter{\phantom}
\renewcommand\theequation{\arabic{chapter}.\arabic{equation}}
\usetikzlibrary{decorations.pathmorphing}

\let\bm\boldsymbol
\makeatletter
\newcommand\showfootnote{%
  \tfn@tablefootnoteprintout% 
  \gdef\tfn@fnt{0}% 
}
\makeatother

\declaretheorem[style=thmredbox, name=Lemme, numbered=no]{lemm}

\newcommand\pullback{\arrow[very near start,description,phantom]{dr}{\lrcorner}}
\newcommand\pushout{\arrow[very near start,description,phantom]{ul}{\ulcorner}}

\newcommand\id{\ensuremath{\operatorname{id}}}

\begin{document}
  \maketitle
  
  {
    \small
    \tableofcontents
  }


  \part{Un peu de chasse au diagramme.}
  \chapter{Exercice 1.}

  \begin{slshape}
    \color{deepblue}
    \begin{enumerate}
      \item Soient $X, Y$ des objets dans une catégorie tels que $X \times Y$ existe.
        Montrer que $Y \times X$ existe et est canoniquement isomorphe à $X \times Y$.
        Énoncer l'énoncé dual.
      \item Soient $X,Y,Z$ des objets tels que  $X \times Y$ et $Y \times Z$ existent.
        Montrer que si $X \times (Y \times Z)$ existe alors $(X \times Y) \times Z$ existe et est canoniquement isomorphe à $X \times (Y \times Z)$.
        Énoncer l'énoncé dual.
    \end{enumerate}
  \end{slshape}

  \begin{enumerate}
    \item Le produit $X \times Y$ est en réalité un triplet $(A, \pi_X, \pi_Y)$.
      Construisons le produit $Y \times X$ comme le triplet $(A, \pi_Y, \pi_X)$.
      Ces deux produits sont clairement isomorphes (il suffit d'inverser le rôle de $\pi_X$ et de $\pi_Y$).
      Montrons que l'on a bien construit $Y \times X$ comme un produit.

      Soit $B$ un objet de la catégorie considérée.
      Soient également deux morphismes $u : B \to X$ et $v : B \to Y$.

      \[
    \begin{tikzcd}
      B \arrow[dashed]{dr}{} \arrow[bend left]{drr}{u}\arrow[bend right]{rdd}{v}\\
      & {Y\times X} \arrow{r}{\pi_X}\arrow{d}{\pi_Y} & X \\
      & Y
    \end{tikzcd}
    \quad\quad
    \begin{tikzcd}
      B \arrow[dashed]{dr}{f} \arrow[bend left]{drr}{v}\arrow[bend right]{rdd}{u}\\
      & {X \times Y} \arrow{r}{\pi_Y}\arrow{d}{\pi_X} & Y \\
      & X
    \end{tikzcd}
      \]
      Pour montrer qu'il existe un unique morphisme $f : B \to Y \times X$ faisant commuter le diagramme de gauche (ci-dessus), il suffit de considérer la propriété universelle de $X \times Y$, qui assure l'unicité d'un morphisme $f : B \to X \times Y$ faisant commuter le diagramme de droite (ci-avant).

      L'énoncé dual est : si $X \sqcup Y$ existe, alors $Y \sqcup X$ aussi et il est canoniquement isomorphe à $X \sqcup Y$.

    \item Supposons que $X \times (Y \times Z)$ existe.
      Soit ainsi un tel élément, sous la forme d'un triplet, $(X \times (Y \times Z), \pi_X, \pi_{Y \times Z})$ (il n'est pas forcément unique, mais il l'est à isomorphisme près).
      Soient de plus $(Y \times Z, \pi_Y, \pi_Z)$ et $(X \times Y, \rho_X, \rho_Y)$.

      \[
      \begin{tikzcd}
        X \times (Y \times Z) \arrow{r}{\pi_{Y \times Z}}\arrow{d}{\pi_X} & Y \times Z\arrow{d}{\pi_Y}\arrow{r}{\pi_Z} & Z\\
        X & Y
      \end{tikzcd}
      .\]

      On définit le produit $(X \times Y) \times Z$ comme le triplet \[
        (X \times (Y \times Z), \rho_{X \times Y}, \rho_Z)
      .\]
      Construisons les morphismes $\rho_{X \times Y}$ et $\rho_Z$.
       \begin{itemize}
        \item D'une part, pour $\rho_{X \times Y} : (X \times Y) \times Z \to X \times Y$, on applique la propriété universelle du produit $X \times Y$, comme indiqué ci-dessous.
          \[
          \begin{tikzcd}
            X \times (Y \times Z) \arrow[bend left]{rrd}{\pi_Y \circ \pi_{Y \times Z}}
            \arrow[bend right]{rdd}{\pi_X} \arrow[dashed]{dr}{\rho_{X \times Y}}\\
            & X \times Y \arrow{r}{\rho_Y} \arrow{d}{\rho_X} & Y\\
            & X
          \end{tikzcd}
          .\]
        \item D'autre part, pour $\rho_Z : (X \times Y) \times Z \to Z$, on le définit comme $\pi_Z \circ \pi_{Y \times Z}$.
          \[
          \begin{tikzcd}
            X \times (Y \times Z) \arrow{r}{\pi_{Y \times Z}}\arrow[bend left]{rr}{\rho_{Z}} & Y \times Z \arrow{r}{\pi_Z} & Z
          \end{tikzcd}
          .\] 
      \end{itemize}
      Le diagramme suivant représente ainsi le produit construit.
      \[
      \begin{tikzcd}
        (X \times Y) \times Z\arrow{r}{\rho_Z}\arrow{d}{\rho_{X \times Y}} & Z\\
        X \times Y\arrow{r}{\rho_Y}\arrow{d}{\rho_X} & Y\\
         X
      \end{tikzcd}
      .\]
      Il ne reste qu'à montrer que $(X \times Y) \times Z$ vérifie la propriété universelle.
      Soient $u : A \to X \times Y$ et $z : A \to Z$.
      On pose ensuite $x = u \circ \rho_X : A \to X$ et $y = u \circ \rho_Y : A \to Y$.
      \[
      \begin{tikzcd}
        A \arrow[bend left]{drrr}{z} \arrow[bend right]{ddr}{y}\arrow[bend right=60]{ddddrr}{x} \arrow[dashed]{drr}{f}\\
        &&(X \times Y) \times Z\arrow{r}{\rho_Z}\arrow{d}{\rho_{X \times Y}} & Z\\
        &Y&X \times Y\arrow{l}{\rho_Y}\arrow{dd}{\rho_X}\\\\
        && X
      \end{tikzcd}
      \]
      On procède en deux temps.
      \begin{itemize}
        \item D'une part, par la propriété universelle de $Y \times Z$, il existe un unique morphisme $v : A \to Y \times Z$ faisant commuter le diagramme ci-dessous.
          \[
          \begin{tikzcd}
            A \arrow[bend left]{drr}{z} \arrow[bend right]{ddr}{y} \arrow[dashed]{dr}{v}\\
            & Y \times Z \arrow{d}{\pi_Y}\arrow{r}{\pi_Z} & Z\\
            & Y 
          \end{tikzcd}
          \] 
        \item D'autre part, par la propriété universelle de $X \times (Y \times Z)$, il existe un unique morphisme $f : A \to X \times (Y \times Z)$ faisant commuter le diagramme ci-dessous.
          \[
          \begin{tikzcd}
            A \arrow[bend left]{drr}{v} \arrow[bend right]{ddr}{x} \arrow[dashed]{dr}{f}\\
            & X \times (Y \times Z) \arrow{d}{\pi_X}\arrow{r}{\pi_{Y \times Z}} & Y \times Z\\
            & X 
          \end{tikzcd}
          .\]
      \end{itemize}
      Par construction de $(X \times Y) \times Z$, le morphisme $f$ est donc aussi l'unique (à isomorphisme près) de la forme $f : A \to (X \times Y) \times Z$.

      L'énoncé dual est : si $X \sqcup Y$, $Y \sqcup Z$ et $X \sqcup (Y \sqcup Z)$ existent alors, $(X \sqcup Y) \sqcup Z$ existe et est canoniquement isomorphe au coproduit $X \sqcup (Y \sqcup Z)$.
  \end{enumerate}

  \chapter{Exercice 2.}
  \begin{slshape}
    \color{deepblue}
    Soient $X$, $Y$ et $S$ des objets d'une catégorie telle que $X \times_S Y$ existe.
    Montrer que $Y \times_S X$ existe et est canoniquement isomorphe à $X \times_S Y$.
    Énoncer l'énoncé dual.
  \end{slshape}

  Soit $X \times_S Y$ muni de ses deux projections $\pi_X$, $\pi_Y$ et des deux morphismes~$u : X \to S$ et $v : Y \to S$.
  On définit $Y \times_S X$ comme le quintuplet $(X \times_S Y, \pi_Y, \pi_X, v, u)$.
  Ils sont canoniquement isomorphes.
  On sait, par construction (il suffit d'inverser la place de $X$ et $Y$ pour retrouver exactement le diagramme d'un produit fibré), que le diagramme suivant commute.
  \[
  \begin{tikzcd}
    Y \times_S X \arrow{r}{\pi_X}\arrow{d}{\pi_Y} & X\arrow{d}{u}\\
    Y \arrow{r}{v} & S
  \end{tikzcd}
  \]
  Montrons enfin qu'il respecte la propriété universelle pour les produits fibrés.
  Soit $A$ un objet de la catégorie considérée et soient deux morphismes $x: A \to X$ et~$y: A \to Y$.
  \[
  \begin{tikzcd}
    A\arrow[dashed]{dr}{f} \arrow[bend left]{drr}{x} \arrow[bend right]{ddr}{y}\\
    & Y \times_S X \arrow{r}{\pi_X}\arrow{d}{\pi_Y} & X\arrow{d}{u}\\
    &Y \arrow{r}{v} & S
  \end{tikzcd}
  \quad\quad
  \begin{tikzcd}
    A\arrow[dashed]{dr}{f} \arrow[bend left]{drr}{y} \arrow[bend right]{ddr}{x}\\
    & X \times_S Y \arrow{r}{\pi_Y}\arrow{d}{\pi_X}\pullback & Y\arrow{d}{v}\\
    &X \arrow{r}{u} & S
  \end{tikzcd}
  \]
  Par la propriété universelle de $X \times_S Y$, il existe un unique morphisme $f$ faisant commuter le diagramme à droite.
  On remarque que $f$ fait aussi commuter le diagramme de gauche.
  De plus, $f$ est l'unique morphisme à faire commuter le diagramme de gauche (car sinon, on pourrait le retranscrire sur le diagramme de gauche, et obtiendrait une contradiction avec la propriété universelle).

  L'énoncé dual est :
  si $X +_S Y$ existe alors  $Y +_S X$ existe et est canoniquement isomorphe à  $X +_S Y$.

  \chapter{Exercice 3.}

  \begin{slshape}
    \color{deepblue}
    Soit $\mathbf{C}$ une catégorie possédant un objet final $\bm{\mathit{1}}$.
    Montrer que, pour tout objet $A$ de $\mathbf{C}$, le produit de $\bm{\mathit{1}}$ et $A$ est $A$.
  \end{slshape}

  Soit $A$ un objet de $\mathbf{C}$.
  Comme on ne suppose pas que $\bm{\mathit{1}} \times A$ existe, on le construit.
  On construit $\bm{\mathit{1}} \times A$ le produit, que l'on défini comme le triplet $(A, \pi_A, \pi_{\bm{\mathit{1}}})$ où $\pi_A = \mathrm{id}_A : A \to A$ et $\pi_{\bm{\mathit{1}}} : A \to \bm{\mathit{1}}$ est l'unique élément de $\mathrm{Hom}(A, \bm{\mathit{1}})$.

  Il reste deux propriétés à démontrer : que $\bm{\mathit{1}} \times A$ défini ainsi vérifie la propriété universelle ; puis que le produit est unique à isomorphisme près.
  \begin{itemize}
    \item Soit $B$ un objet de $\mathbf{C}$ et soient deux morphismes $u : B \to A$ et~$v : B \to \bm{\mathit{1}}$.
      Posons $f : B \to \bm{\mathit{1}} \times A$ comme étant égal à $u$.
      On vérifie que le diagramme suivant commute.
      \[
      \begin{tikzcd}
        B \arrow[bend left]{drr}{u} \arrow[bend right]{ddr}{v} \arrow[dashed]{dr}{f}\\
        & \bm{\mathit{1}} \times A \arrow{r}{\pi_A} \arrow{d}{\pi_{\bm{\mathit{1}}}} & A\\
        & \bm{\mathit{1}}
      \end{tikzcd}
      .\] 
      En effet, on a $\mathrm{id}_A \circ f = f = u$ ($\star$) par la définition de $\circ$ dans $\mathbf{C}$ ; puis, $\pi_{\bm{\mathit{1}}} \circ f = v$ par l'unicité de l'élément dans $\mathrm{Hom}(B, \bm{\mathit{1}})$.
      L'unicité de $f$ découle de ($\star$).

      D'où, $\bm{\mathit{1}} \times A$ vérifie la propriété universelle.
    \item En toute généralité, l'unicité du produit de deux objets (sous réserve d'existence et à isomorphisme près) découle de l'argument ci-après.
      Soient $A$ et $B$ deux objets de $\mathbf{C}$.
      Supposons que $\Pi$ et~$\Gamma$ sont deux objets de $\mathbf{C}$ vérifiant les propriétés d'un produit de $A$ et $B$.
      On suppose $\Pi$ et $\Gamma$ munis des morphismes $(\pi_A, \pi_B)$ et $(\gamma_A, \gamma_B)$ respectivement.
      Ainsi, les deux diagrammes ci-après commutent :
      \[
      \begin{tikzcd}
        \Pi \arrow[dashed]{dr}{f}\arrow[bend left]{drr}{\pi_A}\arrow[bend right]{ddr}{\pi_B} \\
        & \Gamma \arrow{r}{\gamma_A} \arrow{d}{\gamma_B} & A\\
        & B
      \end{tikzcd}
      \quad\quad
      \begin{tikzcd}
        \Gamma \arrow[dashed]{dr}{g}\arrow[bend left]{drr}{\gamma_A}\arrow[bend right]{ddr}{\gamma_B} \\
        & \Pi \arrow{r}{\pi_A} \arrow{d}{\pi_B} & A\\
        & B
      \end{tikzcd}
      ,\]
      où les morphismes $f$ et $g$ existent et sont uniques d'après la propriété universelle pour chacun des produits.

      Ainsi, $f \circ g : \Pi \to \Pi$ est l'identité $\mathrm{id}_\Pi$.
      En effet, par l'absurde, supposons que $f \circ g$ est différent de l'identité.
      Alors, $f \circ g \circ f : \Gamma \to \Gamma$ est différent de $f$.
      Mais, la propriété universelle implique l'unicité de $f$.
      Absurde !

      De même, $g \circ f : \Gamma \to \Gamma$ est l'identité $\mathrm{id}_\Gamma$.

      On en déduit l'unicité, à isomorphisme près, du produit (et par dualité, du coproduit).
  \end{itemize}

  \chapter{Exercice 4.}

  \begin{slshape}
    \color{deepblue}
    Démontrer la réciproque du \textit{Pasting Lemma} (rappelé ci-après).
  \end{slshape}

  \begin{lemm}[\textit{Pasting Lemma}]
    Supposons que le diagramme donné ci-dessous commute :
    \[
    \begin{tikzcd}
      A \arrow{r}{f}\arrow{d}{\alpha} & B\arrow{r}{g}\arrow{d}{\beta}\pullback & C\arrow{d}{\gamma}\\
      D \arrow{r}{f'} & E\arrow{r}{g'} & F
    \end{tikzcd}
    ,\] 
    où le carré à droite est un \textit{pullback}.
    Alors, le carré gauche est un \textit{pullback} si, et seulement si, le rectangle l'est aussi.
  \end{lemm}

  Supposons que le rectangle \[
  \begin{tikzcd}
    A \arrow{rr}{g\circ f} \arrow{d}{\alpha}\pullback && C \arrow{d}{\gamma}\\
    D \arrow{rr}{g'\circ f'} &~& F
  \end{tikzcd}
  \] 
  soit un \textit{pullback}. Montrons que le carré de gauche est un \textit{pullback}.
  Pour cela, soit $X$ un objet de $\mathbf{C}$ et soient $\psi_B : X \to B$ et $\psi_D : X \to D$ deux morphismes tels que $\beta \circ \psi_B = f' \circ \psi_D$.
  On cherche à montrer qu'il existe un unique morphisme~$\phi$ faisant commuter le diagramme ci-dessous.
  \[
    \begin{tikzcd}
      X \arrow[dashed]{dr}{\phi} \arrow[bend left]{drr}{\psi_B} \arrow[bend right]{ddr}{\psi_D}\\
      &A \arrow{r}{f}\arrow{d}{\alpha} & B\arrow{r}{g}\arrow{d}{\beta}\pullback & C\arrow{d}{\gamma}\\
      &D \arrow{r}{f'} & E\arrow{r}{g'} & F
    \end{tikzcd}
  .\]

  Pour cela, on procède en deux temps.
  On définit $\psi_C := g \circ \psi_B : X \to C$, qui fait commuter le diagramme ci-dessous.
  \begin{equation}\label{ex4-diag1}
    \begin{tikzcd}
      X \arrow[dashed]{dr}{\phi} \arrow[bend left]{drr}{\psi_B} \arrow[bend right]{ddr}{\psi_D} \arrow[bend left=40]{drrr}{\psi_C}\\
      &A \arrow{r}{f}\arrow{d}{\alpha} & B\arrow{r}{g}\arrow{d}{\beta}\pullback & C\arrow{d}{\gamma}\\
      &D \arrow{r}{f'} & E\arrow{r}{g'} & F
    \end{tikzcd}
  \end{equation}
  Or, par la propriété universelle du \textit{pullback} ci-dessous, \[
    \begin{tikzcd}
      X \arrow[dashed]{dr}{\phi} \arrow[bend left]{drrr}{\psi_C} \arrow[bend right]{ddr}{\psi_D}\\
      &A \arrow{rr}{g\circ f} \arrow{d}{\alpha}\pullback && C \arrow{d}{\gamma}\\
      &D \arrow{rr}{g'\circ f'} &~& F
    \end{tikzcd}
  .\]
  on a l'existence d'un morphisme $\phi : X \to A$ dans $\mathbf{C}$, faisant commuter le diagramme ci-dessus, donc le diagramme~\ref{ex4-diag1} aussi.
  De plus, un tel morphisme $\phi$ est unique par la propriété universelle du \textit{pullback}~"$A\:C\:D\:F$".

  D'où la réciproque.


  \chapter{Exercice 5.}

  \begin{slshape}
    \color{deepblue}
    \begin{enumerate}
      \item Soit $f : X \to Y$ un morphisme.
        Montrer que $f$ est un monomorphisme si, et seulement si, le diagramme suivant est un~\textit{pullback} :
        \begin{equation}
        \begin{tikzcd}
          X \arrow{r}{\mathrm{id}_X} \arrow{d}{\mathrm{id}_X} & X\arrow{d}{f}\\
          X \arrow{r}{f} & Y
        \end{tikzcd}
        \label{ex5-diag1}
        \end{equation}
      \item Montrer que si $F$ préserve les \textit{pullbacks} alors $F$ préserve les monomorphismes.
    \end{enumerate}
  \end{slshape}

  \begin{enumerate}
    \item On procède en deux temps.\label{ex5-Q1}

      Soit $A$ un objet muni de deux morphismes $u_0 : A \to X$ et $u_1 : A \to X$.
      Le diagramme \ref{ex5-diag1} est un \textit{pullback} si, et seulement si, il existe un unique morphisme $u : A \to X$ tel que le diagramme \[
      \begin{tikzcd}
        A \arrow[bend left]{drr}{u_0} \arrow[bend right]{ddr}{u_1} \arrow[dashed]{dr}{u}\\
        & X \arrow{r}{\mathrm{id}_X} \arrow{d}{\mathrm{id}_X} & X\arrow{d}{f}\\
        & X \arrow{r}{f} & Y
      \end{tikzcd}
      \]
      commute.
      Ceci est vrai, si, et seulement si $u$ est unique, et si l'égalité $f \circ u_0 = f \circ u_1$ est vérifiée.

      Si $f \circ u_0 = f\circ u_1$ et $f$ est un monomorphisme, alors $u_0 = u_1$ donc l'unicité de $u$ et donc la propriété universelle du \textit{pullback}.

      Si le diagramme \ref{ex5-diag1} est un \textit{pullback}, alors $u$ est unique. Donc, si $u_0 \circ f = u_1 \circ f$, on a donc $u_0 = u_1$ par unicité.
      D'où, $f$ est un monomorphisme.

      D'où l'équivalence.
    \item Soient $\mathbf{C}$ et $\mathbf{D}$ deux catégories et $F : \mathbf{C} \to \mathbf{D}$ un foncteur qui préserve les \textit{pullbacks}.
      Montrons que $F$ préserve les monomorphismes.
      Soit $f : X \to Y$ un monomorphisme de $\mathbf{C}$.

      D'après la question~\ref{ex5-Q1}, on peut considérer le \textit{pullback} de $\mathbf{C}$ ci-dessous :
      \[
        \begin{tikzcd}
          X \arrow{r}{\mathrm{id}_X} \arrow{d}{\mathrm{id}_X}\pullback & X\arrow{d}{f}\\
          X \arrow{r}{f} & Y
        \end{tikzcd}
      .\] 
      Comme $F$ préserve les \textit{pullbacks}, on obtient le \textit{pullback} ci-dessous dans $\mathbf{D}$ :
      \[
        \begin{tikzcd}
          F(X) \arrow{r}{\mathrm{id}_{F(X)}} \arrow{d}{\mathrm{id}_{F(X)}}\pullback & F(X)\arrow{d}{F(f)}\\
          F(X) \arrow{r}{F(f)} & F(Y)
        \end{tikzcd}
      \]
      car $F(\mathrm{id}_X) = \mathrm{id}_{F(X)}$

      Montrons que $F(f) : F(X) \to F(Y)$ est un monomorphisme.
      Par la question~\ref{ex5-Q1}, on sait que $F(f)$ est un monomorphisme si, et seulement si le diagramme 
      \[
        \begin{tikzcd}
          F(X) \arrow{r}{\mathrm{id}_{F(X)}} \arrow{d}{\mathrm{id}_{F(X)}}\pullback & F(X)\arrow{d}{F(f)}\\
          F(X) \arrow{r}{F(f)} & F(Y)
        \end{tikzcd}
      \]
      est un \textit{pullback}.
      On en conclut que $F(f)$ est un monomorphisme.
  \end{enumerate}

  \chapter{Exercice 6.}

  \begin{slshape}
    \color{deepblue}
    Soient $f : X \to S$ et $u : Y \to S$ deux morphismes d'une catégorie $\mathbf{C}$, et supposons que le produit fibré $X \times_S Y$ existe. Montrer que si $u$ est un monomorphisme alors $f^\star u$ aussi :
    \[
    \begin{tikzcd}
      X \times_S Y \arrow{r}{} \arrow{d}{f^\star u} & Y \arrow[tail]{d}{u}\\
      X \arrow{r}{f} & S
    \end{tikzcd}
    .\] 

    Énoncer et montrer l'énoncé dual.
  \end{slshape}

  Supposons $u$ un monomorphisme. Par la suite, on notera $\pi_Y$ la projection $X \times_S Y \to Y$.

  Soient $A$ un objet de $\mathbf{C}$ et $x,y : A \to X \times_S Y$ deux morphismes de $\mathbf{C}$.
  Supposons que $f^\star u \circ x = f^\star u \circ y$.

  \[
  \begin{tikzcd}
    A \arrow[bend left]{dr}{x} \arrow[bend right,swap]{dr}{y}\\
    &X \times_S Y \arrow{r}{\pi_Y} \arrow{d}{f^\star u} & Y \arrow[tail]{d}{u}\\
    &X \arrow{r}{f} & S
  \end{tikzcd}
  .\] 

  Premièrement, on a $f \circ f^\star u \circ x = f \circ f^\star u \circ y$,
  et, car le diagramme commute, $u \circ \pi_Y \circ x = u \circ \pi_Y \circ y$,
  mais, comme $u$ est un monomorphisme, on a que $\pi_Y \circ x = \pi_Y \circ y$.

  Deuxièmement, on pose, d'une part, le morphisme $\phi_X := f^\star \circ x = f^\star \circ y$ puis, d'autre part, $\phi_Y := \pi_Y \circ x = \pi_Y \circ y$.

  Troisièmement, on applique la propriété universelle du produit fibré.

  \[
  \begin{tikzcd}
    A \arrow[bend left]{dr}{x} \arrow[bend right,swap]{dr}{y} \arrow[bend right=60,swap]{ddr}{\phi_X} \arrow[bend left=60]{drr}{\phi_Y} \arrow[dashed]{dr}{}\\
    &X \times_S Y \arrow{r}{\pi_Y} \arrow{d}{f^\star u} & Y \arrow[tail]{d}{u}\\
    &X \arrow{r}{f} & S
  \end{tikzcd}
  .\] 

  On obtient l'unicité du morphisme $A \to X \times_S Y$ faisant commuter le diagramme précédent.
  Et, par égalité des définitions de $\phi_X$ et $\phi_Y$ (entre celle utilisant $x$ et celle utilisant $y$), on a que $x = y$.

  On en conclut que $f^\star u$ est un monomorphisme.

  L'énoncé dual est : 
  \begin{quote}
    Soient $f : S \to X$ et $u : S \to Y$ deux morphismes et supposons que la somme amalgamée $X +_S Y$ existe.
    Montrons que si  $u$ est un épimorphisme, alors $f^\star u$ l'est aussi :
    \[
    \begin{tikzcd}
      S \arrow{d}{f} \arrow[two heads]{r}{u} & Y \arrow{d}{\iota_Y}\\
      X \arrow{r}{f^\star u} & X +_S Y\pushout
    \end{tikzcd}
    .\]
  \end{quote}
  On peut le démontrer aisément par dualité : le dual d'une somme amalgamée est un produit fibré, et le dual d'un épimorphisme est un monomorphisme.
  La propriété duale de la propriété universelle de la somme amalgamée est exactement la propriété universelle du produit fibré.

  \chapter{Exercice 7.}

  \begin{slshape}
    \color{deepblue}
    Soit $\mathbf{C}$ une catégorie localement petite et complète (\textit{i.e.} possédant toutes les petites limites, autrement dit, tout diagramme $F : \mathbf{J} \to  \mathbf{C}$ avec $\mathbf{J}$ petite\footnote{On appelle un tel diagramme un "petit diagramme".}\showfootnote, possède une limite) et $A \in \mathrm{ob}(\mathbf{C})$.
    Montrer que le foncteur $h_A := \mathrm{Hom}(A, \cdot) : \mathbf{C} \to \mathbf{Ens}$ est continu.
    En déduire que le foncteur $h^A := \mathrm{Hom}(\cdot, A)$ transforme colimite en limite.
    Montrer qu'alors, dans la catégorie des $\mathds{k}$-espaces vectoriels (ou plus généralement des $R$-modules), on a \[
    \Big( \bigoplus_{i \in I} E_i\Big)^\star = \prod_{i \in I} E_i^\star
    ,\] 
    où $\bigoplus E_i$ désigne la somme ou le coproduit des $E_i$.
  \end{slshape}

  \begin{enumerate}
    \item Montrons que $h_A := \mathrm{Hom}(A, \cdot): \mathbf{C} \to \mathbf{Ens}$ est continu.
      Ce que l'on veut démontrer, c'est que $h_A$ transforme limite en limite.

      Considérons un diagramme $F : \mathbf{J} \to \mathbf{C}$ donc on suppose que la limite $\lim F$ existe, et que cet objet est muni, pour $X \in \mathrm{ob}(\mathbf{J})$, d'un morphisme $\phi_X : \lim F \to X$.
      \begin{equation}
        \begin{tikzcd}
          & \lim F \arrow[swap]{dl}{\phi_X} \arrow{dr}{\phi_Y}\\
          F(X) \arrow{rr}{F(v)} && F(Y)
        \end{tikzcd}
        \label{eq:ex7-diag1}
      \end{equation}

      Considérons $A$ un objet quelconque de $\mathbf{C}$.

      Posons $L = \mathrm{Hom}(A, \lim F)$ muni des morphismes \begin{align*}
        \psi_X = \psi_X \circ -: \mathrm{Hom}(A, \lim F) &\longrightarrow \mathrm{Hom}(A, F(X)) \\
        u : A \to \lim F &\longmapsto \phi_X \circ u : A \to D(X)
      .\end{align*}
      Montrons que $L$ est un cône limite.
      \begin{itemize}
        \item Commençons par montrer que le diagramme \[
          \begin{tikzcd}
            & L \arrow[swap]{dl}{\psi_X} \arrow{dr}{\psi_Y}\\
            \mathrm{Hom}(A, D(X)) \arrow{rr}{F(v)\circ -} && \mathrm{Hom}(A, D(Y))
          \end{tikzcd}
          \] 
          commute, où $v : X \to Y$.
          On a :
          \[
            ((F(v)\circ -) \circ (\phi_X \circ -))(u) = 
            F(v)\circ \phi_X \circ u = \phi_Y \circ u = (\phi_Y \circ -)(u)
          ,\] quel que soit le morphisme $u$, car le diagramme~\ref{eq:ex7-diag1} commute.
          D'où, le diagramme précédent commute, et ceci démontrer que~$L$ est (la pointe d') un cône.
        \item Montrons que le cône $L$ est un cône limite en montrant qu'il vérifie la propriété universelle.
          Considérons un autre cône $W \in \mathbf{Ens}$ muni, pour $X$ un objet de $\mathbf{J}$, d'un morphisme $\mu_X : W \to \mathrm{Hom}(A, D(X))$.

          Considérons $w \in W$ quelconque.
          Alors, pour $X \in \mathrm{ob}(\mathbf{J})$, on a un morphisme $\mu_X(w) : A \to F(X)$.
          On obtient donc un cône car \[
            F(v)\circ \mu_X(w) = ((F(v)\circ -) \circ \mu_X)(w) = \mu_Y(w)
          ,\] car le cône en $W$ commute.
          Ceci implique, par propriété universelle de $\lim F$,
          il existe un unique morphisme de la forme  $f_w : A \to \lim F$, tel que le diagramme \[
          \begin{tikzcd}
            & A \arrow[dashed]{d}{f_w} \arrow[bend right,swap]{ddl}{\mu_X(w)} \arrow[bend left]{ddr}{\mu_Y(w)}\\
            & \lim F \arrow[swap]{dl}{\phi_X} \arrow{dr}{\phi_Y}\\
            X \arrow{rr}{v} & & Y
          \end{tikzcd}
          \]
          commute.
          
          En réalisant cette construction pour tout $w \in W$, on peut définir $f : W \to \mathrm{Hom}(A, \lim F)$ par $f(w) := f_w$.
          De plus, le diagramme suivant commute :
          \[
          \begin{tikzcd}
            & W \arrow[dashed]{d}{f} \arrow[bend right,swap]{ddl}{\mu_X} \arrow[bend left]{ddr}{\mu_Y}\\
            & L \arrow[swap]{dl}{\psi_X} \arrow{dr}{\psi_Y}\\
            \mathrm{Hom}(A, F(X)) \arrow{rr}{F(v)\circ-} & & \mathrm{Hom}(A, F(Y))
          \end{tikzcd}
          .\]
          En effet, ceci est assuré car $\psi_X \circ f = \mu_X$ (il suffit de ré-écrire l'égalité pour tout $w \in W$ et on retrouve le la commutativité du diagramme précédent) quel que soit $X \in \mathrm{ob}(\mathbf{J})$.
      \end{itemize}

      On en conclut que le morphisme $h_A$ envoie une limite sur une limite, d'où sa continuité.

    \item Montrons que $h^A := \mathrm{Hom}(\cdot, A)$ transforme colimite en limite. \label{ex7-p2}
      Par dualité, le morphisme $h^A : \mathbf{C}^\mathrm{op} \to \mathbf{Ens}$ envoie une limite de~$\mathbf{C}^\mathrm{op}$ sur une limite.
      Mais, une limite de $\mathbf{C}^\mathrm{op}$ est une colimite de $\mathbf{C}$.
      On a donc exactement ce que l'on voulait démontrer.

    \item Montrons que $\big(\!\bigoplus E_i\big)^\star = \prod E_i^\star$ dans la catégorie des  $\mathds{k}$-espaces vectoriels.
      Considérons le diagramme discret $F$ suivant \[
        \begin{tikzcd}
          E_{i_1} \arrow[loop above]{}{\mathrm{id}_{E_{i_1}}} &
          E_{i_2} \arrow[loop above]{}{\mathrm{id}_{E_{i_2}}} &
          E_{i_3} \arrow[loop above]{}{\mathrm{id}_{E_{i_3}}} &
          E_{i_4} \arrow[loop above]{}{\mathrm{id}_{E_{i_4}}} & \cdots 
        \end{tikzcd}
      .\]
      On sait que $\mathds{k}$ est un $\mathds{k}$-espace vectoriel. On applique donc le morphisme $h^\mathds{k}$.
      Et on a donc \[
        \Big(\bigoplus_{i \in I} E_i\Big)^\star = h^\mathds{k}(\operatorname{colim} F) \underset {\text{Q\ref{ex7-p2}}} = \lim h^\mathds{k} \circ F = \prod_{i \in I} E_i^\star
      ,\] car $E^\star = h^\mathds{k}(E) = \mathrm{Hom}(E, \mathds{k})$ quel que soit le $\mathds{k}$-espace vectoriel~$E$.
  \end{enumerate}


  \chapter{Exercice 8.}

  \begin{slshape}
    \color{deepblue}
    Soit $F : \mathbf{J} \to \mathbf{C}$ un diagramme, et supposons que $\mathbf{J}$ possède un objet initial $\bm{\mathit{1}}$. Montrer que $\lim F$ existe et calculer cette limite.
  \end{slshape}

  Commençons par montrer que la limite $\lim F$ existe et de la calculer.
  Pour cela, on définit cette objet, et on montre que c'est bien une limite.

  Pour chaque objet $A \in \mathrm{ob}(\mathbf{J})$, il existe un unique morphisme de la forme $h_A : \bm{\mathit{1}} \to A$ dans $\mathbf{J}$, car $\bm{\mathit{1}}$ est un objet initial dans $\mathbf{J}$.

  On pose $L = F(\bm{\mathit{1}})$ muni des morphismes $\psi_A := F(h_A) : L \to F(A)$, pour $A \in \mathrm{ob}(\mathbf{J})$. On obtient le diagramme
  \[
    \forall A,B \in \mathrm{ob}(\mathbf{J}), \quad\quad
  \begin{tikzcd}
    & L \arrow[swap]{dl}{\psi_A} \arrow{dr}{\psi_B}\\
    F(A) \arrow{rr}{F(f)} && F(B)
  \end{tikzcd},
  \]
  qui commute par unicité du morphisme $\bm{\mathit{1}} \to B$. En effet, par morphisme, on a l'égalité~$F(f) \circ \psi_A = F(f \circ h_A)$ avec $f\circ h_A : \bm{\mathit{1}} \to B$, d'où $f\circ h_A = h_B$.

  Montrons que $L$ muni des morphismes $\psi_A$ est bien une limite.
  Considérons $X$ muni des morphismes $\chi_A : X \to A$, pour $A \in \mathrm{ob}(\mathbf{J})$.
  Supposons que $F(f)\circ \chi_A = \chi_B$ pour $f : A \to B$, quel que soient les objets~$A$ et $B$ de $\mathbf{J}$. 

  En particulier, pour $A = \bm{\mathit{1}}$, on a \[
    \begin{tikzcd}
      & X \arrow[bend right,swap]{ddl}{\chi_{\bm{\mathit{1}}}} \arrow[bend left]{ddr}{\chi_B} \arrow[dashed]{d}{u}\\
      & L\arrow[swap]{dl}{\psi_{\bm{\mathit{1}}}} \arrow{dr}{\psi_B}\\
      L\arrow{rr}{F(h_B)} & & B
    \end{tikzcd}
  ,\]
  ce qui nous assure l'unicité de $u := \chi_{\bm{\mathit{1}}} : X \to L$.
  En effet, pour tout objet $B \in \mathrm{ob}(\mathbf{J})$, on a $\chi_B = F(h_B) \circ \chi_{\bm{\mathit{1}}}$.
  Ainsi, la famille $(\chi_B)_{B \in \mathrm{ob}(\mathbf{J})}$ est entièrement déterminée par $\chi_{\bm{\mathit{1}}}$, d'où l'unicité.

  On en déduit que, comme $L$ vérifie la propriété universelle, on a \[
    F(\bm{\mathit{1}}) =: L = \lim F
  .\] 



  \part{Constructions des limites inductives et projectives dans les ensembles.}
  \chapter{Exercice 9.}\label{exo-9}

  \begin{slshape}
    \color{deepblue}
    Soit $(I, \le )$ un ensemble ordonné filtrant, et pour tout $i \in I$, fixons $E_i$ un ensemble, et notons $f_{i,j} : E_j \to E_i$ l'unique application $E_j \to E_i$ si $i \le j$ (on vient juste de se donner un système projectif).
    Considérons \[
    L := \mleft\{\, (x_i)_{i \in I} \in \prod_{i \in I} E_i \;\middle|\; \forall i \le j, f_{i,j}(x_j) = x_i\,\mright\} 
    .\] 
    Montrer que $L$ muni des projections est la limite projective du diagramme $I \to \mathbf{Ens}$.
  \end{slshape}


  

  On considère le diagramme ci-dessous, où l'on suppose $\varphi_i = f_{i,j}\circ \varphi_j$.
  \begin{equation}
    \forall i\le j, \quad\quad
    \begin{tikzcd}
      & Y \arrow[dashed]{d}{\Phi} \arrow[bend right,swap]{ddl}{\varphi_j} \arrow[bend left]{ddr}{\varphi_i}\\
      & L \arrow[swap]{dl}{\pi_j} \arrow{dr}{\pi_i}\\
      E_j \arrow{rr}{f_{i,j}} & & E_i
    \end{tikzcd}
    \label{ex5-diag1}
  \end{equation}
  Afin de montrer que $L$ muni des projections est la limite projective du diagramme $I \to \mathbf{Ens}$, on procède en trois temps.
  \begin{enumerate}
    \item On construit les projections $\pi_i$ pour $i \in I$.
      Pour $i \in I$ et pour une famille $x = (x_{i'})_{i' \in I} \in L$, on pose $\pi_i(x) = x_i$.
      On construit bien des projections de $L$ dans $E_i$, pour $i \in I$.

    \item Montrons que le diagramme~\ref{ex5-diag1} commute (où l'on néglige le morphisme $\Phi$ pour l'instant).
      Pour cela, il suffit que l'on ait l'égalité $f_{i,j} \circ \pi_j = \pi_i$ quels que soient $i \le j$.
      Ainsi, on doit montrer que, quel que soit $(x_i)_{i \in I} \in L$, on ait $f_{i,j}(x_j) = x_i$.
      C'est vrai car $(x_i)_{i \in I}$ est un élément de $L$.
    \item Montrons qu'il existe un unique morphisme $u : Y \to L$ tel que le diagramme~\ref{ex5-diag1} commute.
      On procède par analyse-syntèse.
      \begin{itemize}
        \item \textbf{\textsl{\color{deepblue}Analyse.}}
          Supposons avoir un morphisme $\Phi : Y \to L$ qui fait commuter le diagramme~\ref{ex5-diag1}.
          Alors, $\varphi_j = \pi_j \circ \Phi$ et donc, pour tout $y \in Y$, on a : $[u(y)]_j = \varphi_j(y)$, quel que soit $j \in I$.
          On note ici $[\;\cdot\;]_j = \pi_j$ pour insister sur le fait que, en parcourant tous les $j \in I$, on construit $\Phi(y)$ coordonnée par coordonnée.
        \item \textbf{\textsl{\color{deepblue}Synthèse.}}
          On définit l'application
          \begin{align*}
            \Phi: Y &\longrightarrow L \\
            y &\longmapsto \big(\varphi_i(y)\big)_{i \in I}
          .\end{align*}
          On constate que cette définition fait commuter le diagramme.
          En effet, on a bien $\Phi(y) \in L$ quel que soit $y \in Y$ car $f_{i,j}(\varphi_j(y)) = \varphi_i(y)$ par hypothèse.
      \end{itemize}
  \end{enumerate}

  Par ce que l'on a démontré, on définit bien une limite projective et \[
  L = \varprojlim_{i \in I} X_i
  .\]

  \chapter{Exercice 10.}

  \begin{slshape}
    \color{deepblue}
    On reprend exactement les mêmes notations que l'exercice précédent, mais cette fois-ci on suppose que tous les $E_i$ sont des anneaux et que les $f_{i,j}$ sont des morphismes d'anneaux, montrer qu'alors $L$ possède une structure d'anneau telle que toutes les projections sont des morphismes d'anneaux.
  \end{slshape}

  Comme dans l'exercice~\hyperref[exo-9]{9}, on définit \[
    L := \mleft\{\, (x_i)_{i \in I} \in \prod_{i \in I} E_i \;\middle|\; \forall i \le j, f_{i,j}(x_j) = x_i\,\mright\} 
  ,\]
  muni des projections \begin{align*}
    \pi_i: L &\longrightarrow E_i \\
    (x_j)_{j \in I} &\longmapsto x_i
  .\end{align*}

  Avant de commencer, remarquons que $L$ est non-vide.
  En effet, pour tout $i \in I$, $E_i$ contient l'élément $1_{E_i}$, et $f_{i,j}(1_{E_i}) = 1_{E_j}$.
  Ainsi, $(1_{E_i})_{i \in I}$ est un élément de $L$.

  Montrons que $L$ a une structure d'anneau.
  Pour cela, on peut montrer que $L$ est un sous-anneau de $E := \prod_{i \in I} E_i$.\footnote{Un peu plus de détail : les opérations $+$ et $\cdot$ sont réalisées composantes par composantes (mais elles peuvent être différentes pour $i \neq j$). Il est clair que les propriétés d'anneaux des $E_i$ permettent d'en déduire que $E$ est un anneau.}\showfootnote\ 
  Démontrer cela demande de montrer trois propriétés.
  \begin{enumerate}
    \item Montrons que $L$ est un sous-groupe de $(E, +)$.
      Comme $L$ est non-vide, il ne reste qu'à démontrer que $x - y \in L$ quels que soient $x, y \in L$.
      Soient $x = (x_i)_{i \in I}$ et $y = (y_i)_{i \in I}$ deux éléments de $L$.
      Alors, par définition des opérations, \[
        z := x - y = (x_i -_i y_i) _{i \in I} =: (z_i)_{i \in I}
      \]
      Pour tout $i \le j$, on a \[
        f_{i,j}(z_i) = f_{i,j}(x_i -_i y_i) = f_{i,j}(x_i) -_j f_{i,j}(y_i) = x_j -_j y_j = z_j
      .\]
      Ainsi, on a bien $x - y \in L$.
      D'où, $L$ est un sous-groupe de $(E, +)$.

    \item Montrons que $1_E \in L$.
      En fait, on l'a déjà démontré précédemment (\textit{c.f.} la partie de la preuve pour $L \neq \emptyset$, au début de l'exercice).
      En effet, $1_E = (1_{E_i})_{i \in I}$ est un élément de $L$.


    \item Montrons que $\cdot$ est une loi de composition interne sur $L$.
      Montrons que, pour $x,y$ deux éléments de $L$, l'élément $z := x \cdot y$ est bien un élément de $L$.
      Pour tout $i \le j$, on a :
      \[
      f_{i,j}(z_i) = f_{i,j}(x_i \cdot_i y_i) = f_{i,j}(x_i) \cdot_j f_{i,j}(y_i) = x_j \cdot_j y_j = z_j
      ,\] d'où $z = x\cdot y \in L$.
  \end{enumerate}

  D'où, $L$ est un sous-anneau de $E$, donc en particulier, c'est un anneau.

  Montrons que les projections $\pi_i : L \to E_i$ sont des morphismes d'anneaux.
  Pour cela, on a trois propriétés à démontrer.
  \begin{enumerate}
    \item On a $\pi_i(x + y) = \pi_i(x) +_i \pi_i(y)$, par définition de l'opération $+$ comme addition composante par composante.
    \item On a $\pi_i(x \cdot  y) = \pi_i(x) \cdot_i \pi_i(y)$, par définition de l'opération $\cdot$ comme multiplication composante par composante.
    \item On a $\pi_i(1_E) = 1_{E_i}$ par définition de $1_E = (1_{E_i})_{i \in I}$.
  \end{enumerate}
  Ainsi les projections sont bien des morphismes d'anneaux.

  \chapter{Exercice 11.}

  \begin{slshape}
    \color{deepblue}

    \begin{enumerate}
      \item Soit $\mathbf{C}$ la sous-catégorie de $\mathbf{Ens}$ dont les objets sont des singletons indexés par des entiers naturels, et où il y a une (et une seule flèche) $\{n\} \to  \{m\}$ si, et seulement si, $n$ divise $m$.
        Soit $\mathbf{J}$ le diagramme :\footnote{On identifie ici le foncteur d'inclusion $\mathbf{J} \to \mathbf{C}$ avec $\mathbf{J}$.}
        \showfootnote
        \[
        \begin{tikzcd}
          \{6\}\arrow[loop above] & & 
          \{9\}\arrow[loop above]
        \end{tikzcd}
        .\] 
        Calculer la limite de $J$ dans $\mathbf{C}$. En voyant maintenant $J$ comme un diagramme de $\mathbf{Ens}$, calculer à nouveau cette limite. Que remarquez-vous ?
      \item Soit $\mathbf{Ann}$ la catégorie des anneaux (commutatifs unitaires) et~$\mathbf{Ann_f}$ la sous-catégorie pleine des anneaux finis.
        Soit $p$ un nombre premier et $J$ le système projectif dans $\mathbf{Ann_f}$ où les objets sont les $\mathds{Z} / p^n \mathds{Z}$ avec $n \in \mathds{N}$ et les flèches sont données par :
        \begin{align*}
          \mathds{Z} / p^n \mathds{Z} &\longrightarrow \mathds{Z} / p^m \mathds{Z} \\
          x \pmod{p^n} &\longmapsto x \pmod{p^m}
        \end{align*}
        dès que $n \ge m$.
        Ce diagramme admet-il une limite dans $\mathbf{Ann_f}$ ? Et dans $\mathbf{Ann}$ ?
    \end{enumerate}
  \end{slshape}

  \begin{enumerate}
    \item Posons $L_\mathbf{C} := \{3\}$ et montrons que $L_\mathbf{C}$ est bien la limite de $J$ dans $\mathbf{C}$.
      \begin{itemize}
        \item D'une part, le diagramme \[
            \begin{tikzcd}
              & \{3\} \arrow{dr}{} \arrow[swap]{dl}{}\\
              \{6\} && \{9\}
            \end{tikzcd}
          \]
          commute toujours : en effet, il n'y a pas de flèches entre les objets~$\{6\}$ et $\{9\}$ car $6$ et $9$ ne sont pas comparables pour la relation de divisibilité "$\mid$".
        \item D'autre part, si $\psi_6 : \{a\}  \to \{6\}$ et $\psi_9 : \{a\}  \to \{9\}$ sont deux morphismes et $\{a\}$ est un objet de $\mathbf{C}$, alors $a \mid 6$ et aussi $a \mid 9$.
          C'est donc un diviseur commun de $6$ et $9$ et par définition du PGCD, $a  \mid \mathrm{pgcd}(6,9) = 3$.
          Il existe donc une flèche $\{a\} \to \{3\}$ qui fait commuter le diagramme (ceci est assuré car "$ \mid $" est une relation d'ordre sur $\mathds{N}$).
          \[
            \begin{tikzcd}
              & \{a\} \arrow[bend left]{ddr}{} \arrow[swap,bend right]{ddl}{} \arrow[dashed]{d}{}\\
              & \{3\} \arrow{dr}{} \arrow[swap]{dl}{}\\
              \{6\} && \{9\}
            \end{tikzcd}
          .\]
      \end{itemize}
      Ensuite, on pose $L_\mathbf{Ens} := \{0\}$.
      Montrons que $L_\mathbf{Ens}$ est la limite de $J$ dans $\mathbf{Ens}$.
      \begin{itemize}
        \item D'une part, le diagramme \[
            \begin{tikzcd}
              & \{0\} \arrow{dr}{v} \arrow[swap]{dl}{u}\\
              \{6\} \arrow{rr}{f} && \{9\}
            \end{tikzcd}
          \]
          commute toujours : on a $0 \mapsto^u 6 \mapsto^f 9 = 0\mapsto^v 9$.
        \item D'autre part, si $\psi_6 : A  \to \{6\}$ et $\psi_9 : A  \to \{9\}$ sont deux morphismes et $A$ est un ensemble, alors $\psi_6 : a \mapsto 6$ et $\psi_9 : a \mapsto 9$.
          Ainsi, en posant $f : A \to \{0\}, a \mapsto 0$ et c'est l'unique morphisme de $A \to \{0\}$,
          et le diagramme suivant commute :
          \[
            \begin{tikzcd}
              & A \arrow[bend left]{ddr}{\psi_9} \arrow[swap,bend right]{ddl}{\psi_6} \arrow[dashed]{d}{f}\\
              & \{0\}  \arrow{dr}{f} \arrow[swap]{dl}{u}\\
              \{6\} \arrow{rr}{f} && \{9\}
            \end{tikzcd}
          .\]
      \end{itemize}

      En en conclut que, dans $\mathbf{C}$ la limite de $J$ est $\{3\} $ alors que dans~$\mathbf{Ens}$, la limite est $\{0\}$ (dans la première, c'est le singleton PGCD, dans la seconde, c'est un singleton sans contrainte sur l'élément).
      Ainsi, une limite dans $\mathbf{C}$ est une limite dans $\mathbf{Ens}$ mais la réciproque est fausse.
    \item Pour la suite, on note $\mathds{Z}_p$ l'ensemble des \textit{nombres $p$-adiques}.
      Montrons que le diagramme n'admet pas de limite dans $\mathbf{Ann_f}$ mais que sa limite vaut $\mathds{Z}_p$ dans $\mathbf{Ann}$.
      \begin{itemize}
        \item Pour montrer que $\mathds{Z}_p$ est limite du diagramme dans $\mathbf{Ann}$, nous devons montrer deux propriétés.
          \begin{itemize}
            \item Le diagramme \[
                \begin{tikzcd}
                  & \mathds{Z}_p \arrow{dr}{\pi_m} \arrow[swap]{dl}{\pi_n}\\
                  \mathds{Z}/p^n\mathds{Z} \arrow{rr}{f_{n,m}} && \mathds{Z}/p^m\mathds{Z}
                \end{tikzcd}
              ,\]
              commute où l'on note $f_{n,m}$ la flèche de $\mathds{Z} / p^n \mathds{Z}$ vers l'anneau $\mathds{Z} / p^m \mathds{Z}$ dès lors que $n \ge m$, et où l'on pose les morphismes
              \begin{align*}
                \pi_n: \quad\quad\mathds{Z}_p &\longrightarrow \mathds{Z}/p^n\mathds{Z} \\
                \sum_{i \in \mathds{N}} a_i p^i &\longmapsto \sum_{i < n} a_i p^i
              .\end{align*}
              En effet, on a l'égalité \[
                \sum_{i < n} a_i p^i \pmod{p^m} = \sum_{i < m} a_i p^i \pmod{p^m}
              ,\] et donc $\pi_m = f_{n,m}\circ\pi_n$.
            \item De plus, il existe un unique morphisme $u : A \to \mathds{Z}_p$ tel que le diagramme \[
              \begin{tikzcd}
                & A \arrow[bend left]{ddr}{\psi_m} \arrow[swap,bend right]{ddl}{\psi_n} \arrow[dashed]{d}{u}\\
                & \mathds{Z}_p  \arrow{dr}{\pi_m} \arrow[swap]{dl}{\pi_n}\\
                \mathds{Z}/p^n \mathds{Z} \arrow{rr}{f} && \mathds{Z}/p^m \mathds{Z}
              \end{tikzcd}
              \] commute où $(\psi_n : A \to \mathds{Z} / p^n \mathds{Z})_{n \in \mathds{N}}$ est un cône.
              Remarquons que : si $x \in A$ est, alors
              \begin{itemize}
                \item $\psi_1(x) =: a_0 \in \mathds{Z}/p\mathds{Z}$ ;
                \item $(\psi_2(x) - a_0) / p =: a_1 \in \mathds{Z} / p \mathds{Z}$ ;
                \item $(\psi_3(x) - a_1 p - a_0) / p^2 =: a_2 \in \mathds{Z}/p \mathds{Z}$ ;
                \item \textit{etc}.
              \end{itemize}
              Ainsi, les valeurs de $\psi_i(x)$ définissent un unique nombre $p$-adique $u(x) := \sum_{n \in \mathds{N}}a_n p^n$.
              On en déduit donc qu'il existe un unique morphisme $u$ faisant commuter le diagramme précédent, et que \begin{align*}
                u: A &\longrightarrow \mathds{Z}_d \\
                x &\longmapsto \sum_{n \in \mathds{N}} a_n p^n \quad\quad \text{où} \quad a_n := \frac{\psi_n - \sum_{i < n} a_i p^i}{p^n}
              .\end{align*}
          \end{itemize}
        \item Par l'absurde, supposons que le diagramme admet une limite $A$ dans $\mathbf{Ann_f}$, et qu'elle est munie de la famille de morphismes~$(\psi_n : A \to \mathds{Z}/p^n \mathds{Z})_{n \in \mathds{N}}$ tel que le cône en $A$ commute.
          Soit $N \in \mathds{N}$ tel que $p^N > \# A$, qui existe car $A$ est fini.
      \end{itemize}
  \end{enumerate}

  \chapter{Exercice 12.}
  \begin{slshape}
    \color{deepblue}

    Soit $(I, \le)$ un ensemble ordonné filtrant, et pour tout $i \in I$, fixons~$E_i$ un ensemble, avec $f_{i,j} : E_i \to E_j$ l'unique morphisme de $I$ de la forme~$E_i \to E_j$ si~$i \le j$, et $f_{i,i} = \id_{E_i}$ (on vient juste de se donner un système inductif). Soit~$C' = \{(x,i)  \mid i \in I, x \in E_i\}$.
    Autrement dit, les éléments de $C'$ sont les couples $(x,i)$ où  $i \in I$ est un indice qui contrôle l'ensemble dans lequel $x$ appartient.
    Considérons sur $C'$ la relation \[
      (x, i) \sim (y, j) \quad \iff \quad\exists k \ge i,j,\; f_{i,k}(x) = f_{j,k}(y)
    .\]
    \begin{enumerate}
      \item Montrer que c'est une relation d'équivalence.
      \item Montrer que  $C := C' / \sim$ est la colimite du diagramme $I \to \mathbf{Ens}$, \textit{i.e.} que $C = \varinjlim_{i \in I} E_i$.
      \item On suppose ici que pour tout $i \le j$, $E_i \subseteq E_j$ et $f_{i,j}$ est l'inclusion de $E_i$ dans $E_j$. Montrer que $C \cong \bigcup_{i \in  I} E_i$.
      \item Soit $X$ un espace topologique, et $x \in X$ un point fixe.
        On considère la catégorie posétale $\mathbf{I}$ des voisinages ouverts de $x$, et le diagramme $C : \mathbf{I} \to \mathbf{Ann}$ qui, à un ouvert $U$, associe l'anneau des fonctions continues sur $U$. Calculer la limite de ce diagramme.
    \end{enumerate}
  \end{slshape}

  \begin{enumerate}
    \item Montrons les trois propretés d'une relation d'équivalence.
      \begin{enumerate}
        \item \textbf{\textsl{\color{deepblue}Réflexivité.}}
          Soit $(x, i) \in C'$.
          On a bien $(x,i) \sim (x,i)$.
          En effet, il suffit de choisir $k = i$ et on a bien l'égalité demandée.
        \item \textbf{\textsl{\color{deepblue}Symétrie.}}
          Soient $(x, i), (y, j) \in C'$.
          Si $(x, i) \sim (y,j)$ alors il existe $k \ge i,j$ tel que $f_{i,k}(x) = f_{j,k}(y)$.
          Mais donc $k \ge j,i$ et vérifie $f_{j,k}(y) = f_{i,k}(x)$.
          On a donc $(y,j) \sim (x_i)$.
        \item \textbf{\textsl{\color{deepblue}Transitivité.}}
          Soient $(x, i), (y, j),(z,k) \in C'$.
          Supposons que $(x, i) \overset p\sim (y,j) \overset q\sim (z,k)$.
          On note les indices (le $k$ de la définition) au dessus de la relation.
          Comme $p,q$ sont comparables ($I$ supposé filtrant), leur maximum $r$ existe.
          Alors, on a \[
            f_{i,r}(x) = f_{p,r}(f_{i,p}(x)) = f_{p,r}(f_{j,p}(y)) = f_{j,r}(y)
          ,\] et \[
            f_{k,r}(z) = f_{q,r}(f_{k,q}(x)) = f_{q,r}(f_{j,q}(y)) = f_{j,r}(y)
          .\]
          En effet, ces égalités de composition proviennent de l'unicité d'un morphisme $E_i \to E_r$ et $E_k \to E_r$, comme représenté sur le diagramme ci-après.
          \[
          \begin{tikzcd}
            E_i \arrow{r}{f_{i,p}} \arrow[swap]{dr}{f_{i,r}} & E_p \arrow{d}{f_{p,r}}\\
          & E_r & E_j \arrow[swap,bend right]{ul}{f_{j,p}} \arrow[bend left]{dl}{f_{j,q}} \arrow[dashed]{l}{f_{j,r}}\\
            E_k \arrow[swap]{r}{f_{k,q}} \arrow{ur}{f_{k,r}} & E_q \arrow[swap]{u}{f_{q,r}}\\
          \end{tikzcd}
          .\]
          On en déduit donc que $(x,i) \overset r \sim (z,k)$.
      \end{enumerate}
      Ainsi la relation $\sim$ est une relation d'équivalence.
    \item
      Considérons $C := C' / \sim$ l'ensemble quotient muni des projections  
      \begin{align*}
        \pi_i: E_i &\longrightarrow C \\
        x &\longmapsto [(x,i)]
      ,\end{align*}
      où $[\:\cdot\:] : C' \to C$ associe à un élément de $C'$ sa classe d'équivalence pour la relation $\sim$.


      On doit montrer deux propriétés : on a construit un co-cône, et il vérifie la propriété universelle.
      \begin{enumerate}
        \item Montrons que le diagramme suivant commute :
          \[
          \forall i \le j, \quad \begin{tikzcd}
            E_i \arrow{rr}{f_{i,j}}\arrow{dr}{\pi_i} && E_j \arrow{dl}{\pi_j}\\
            & C
          \end{tikzcd}
          ,\] 
          \textit{i.e.} montrons que $\pi_j \circ f_{i,j} = \pi_i$ pour tout $i \le j$.
          Supposons avoir $i \le j$, et soit $x \in E_i$.
          Alors,
          \begin{itemize}
            \item d'une part, $\pi_i(x) = [(x,i)]$ ;
            \item d'autre part, $\pi_j(f_{i,j}(x)) = [(f_{i,j}(x), j)]$.
          \end{itemize}
          Cependant, on a $(x, i) \overset j \sim (f_{i,j}(x), j)$ avec les notations précédentes.
          D'où l'égalité des classes d'équivalences et donc l'égalité de la composition.
          On en déduit que le diagramme précédent commute.
        \item Soit $A$ un objet et soient $\gamma_i : E_i \to A$ des morphismes.
          Supposons que le diagramme 
          \[
          \forall i \le j, \quad \begin{tikzcd}
            E_i \arrow{rr}{f_{i,j}}\arrow[swap]{dr}{\gamma_i} && E_j \arrow{dl}{\gamma_j}\\
            & A
          \end{tikzcd}
          \]
          commute.
          Montrons qu'alors, il existe un unique morphisme $u : C \to A$ tel que le diagramme suivant commute :
          \[
          \forall i \le j, \quad \begin{tikzcd}
            E_i \arrow[bend right,swap]{rdd}{\gamma_i} \arrow{rr}{f_{i,j}}\arrow[swap]{dr}{\pi_i} && E_j \arrow{dl}{\pi_j} \arrow[bend left]{ldd}{\gamma_j}\\
            & C \arrow[dashed]{d}{u}\\
            & A
          \end{tikzcd}
          .\]

          On procède par analyse-synthèse.
          \begin{itemize}
            \item \textbf{\textsl{\color{deepblue}Analyse.}}
              Supposons avoir un morphisme $u : C \to A$ tel que le diagramme précédent commute.
              Ainsi, si $x \in E_i$, alors $(u \circ \pi_i)(x) = \gamma_i(x)$ et donc $u([x,i]) = \gamma_i(x)$.
            \item \textbf{\textsl{\color{deepblue}Synthèse.}}
              On pose \begin{align*}
                u: C &\longrightarrow A \\
                [x,i] &\longmapsto \gamma_i(x)
              .\end{align*}
              Ce morphisme est bien défini car, si $(x,i) \overset k \sim (y,j)$, alors \[
              \gamma_i(x) = \gamma_k(f_{i,k}(x)) = \gamma_k(f_{j,k}(y)) = \gamma_j(y)
              .\]
          \end{itemize}
          On conclut de l'existence et de l'unicité de $u : C \to A$ faisant commuter le diagramme précédent.
      \end{enumerate}
    \item Avec cette définition des morphismes $f_{i,j}$, on a donc \[
      \forall x,y, \quad\quad (x,i) \sim (y,j) \iff x = y
      .\]
      Ceci implique nécessairement que $C = C' / {\sim} \cong \bigcup_{i \in I} E_i$. En effet, les classes d'équivalences sont \[
        \forall x \in E_i, \quad\quad [(x,i)] = \{(x,j)  \mid j \ge i\} 
      ,\]et on obtient donc l'isomorphisme \begin{align*}
        f: C &\longrightarrow \bigcup_{i \in  I} E_i \\
        [(x,i)] &\longmapsto x
      .\end{align*}
    \item On applique la question précédente ?
  \end{enumerate}

  \chapter{Exercice 13.}

  \begin{slshape}
    \color{deepblue}
    Soit $\mathbf{C}$ une catégorie complète. Soit $\mathbf{J}$ une petite catégorie fixée une fois pour toute. On rappelle que $[\mathbf{J},\mathbf{C}]$ est la catégorie des foncteurs de~$\mathbf{J}$ vers~$\mathbf{C}$, ou autrement dit, des $\mathbf{J}$-diagrammes sur $\mathbf{C}$. Montrer qu'on définit un foncteur $\lim_\mathbf{J} : [\mathbf{J},\mathbf{C}] \to \mathbf{C}$.
  \end{slshape}

  Rappelons que pour la catégorie $[\mathbf{J},\mathbf{C}]$,
  \begin{itemize}
    \item les objets sont les foncteurs $F : \mathbf{J} \to \mathbf{C}$ ;
    \item les morphismes sont les transformations naturelles $\eta : F \Rightarrow G$. 
  \end{itemize}

  Définissons le foncteur $\lim_\mathbf{J}$ sur les objets.
  À un foncteur $F : \mathbf{J} \to \mathbf{C}$, on associe $\lim_\mathbf{J} F$ qui existe (par hypothèse, comme $\mathbf{C}$ complète, toutes les limites de $\mathbf{J}$-diagrammes sur $\mathbf{C}$ existent) et qui est un objet de $\mathbf{C}$.

  Puis, définissons le foncteur $\lim_\mathbf{J}$ sur les objets.
  À une transformation naturelle $\eta : F \Rightarrow G$ (où $F, G : \mathbf{J} \to \mathbf{C}$ sont des foncteurs), on associe le morphisme $\lim_\mathbf{J} \eta : \lim_\mathbf{J} F \to \lim_\mathbf{J} G$.
  Le morphisme $\lim_\mathbf{J} \eta$ est l'unique morphisme faisant commuter le diagramme 
  \[
  \begin{tikzcd}
      & \lim_\mathbf{J} F \arrow[dashed]{d}{\lim_\mathbf{J} \eta} \arrow[bend right,swap]{ddl}{\varphi_A} \arrow[bend left]{ddr}{\varphi_B}\\
      & \lim_\mathbf{J} G \arrow[swap]{dl}{\psi_A} \arrow{dr}{\psi_B}\\
      F(A) \arrow{rr}{F(f)} & & F(B)
  \end{tikzcd}
  ,\] 
  où les $\phi_A$ (\textit{resp}. les $\psi_A$) sont les morphismes associés à la limite $\lim_\mathbf{J} F$ (\textit{resp}. $\lim_\mathbf{J} G$).

  Il ne reste que deux propriétés à vérifier afin de démontrer que $\lim_\mathbf{J}$ est bien un foncteur $[\mathbf{J}, \mathbf{C}] \to \mathbf{C}$.

  \begin{itemize}
    \item On a que $\lim_\mathbf{J} \id_F = \id_{\lim_\mathbf{J} F}$, où $\id_F : F \Rightarrow F$.
      En effet, l'identité fait commuter le diagramme
      \[
      \begin{tikzcd}
          & \lim_\mathbf{J} F \arrow[dashed]{d}{\id_{\lim_\mathbf{J} F}} \arrow[bend right,swap]{ddl}{\psi_A} \arrow[bend left]{ddr}{\psi_B}\\
          & \lim_\mathbf{J} F \arrow[swap]{dl}{\psi_A} \arrow{dr}{\psi_B}\\
          F(A) \arrow{rr}{F(f)} & & F(B)
      \end{tikzcd}
      ,\] 
      et par unicité de ce morphisme (mais aussi par la construction décrite précédemment), on a $\lim_\mathbf{J} \id_F = \id_{\lim_\mathbf{J} F}$.
    \item Pour $\eta : F \Rightarrow G$ et $\varepsilon : G \Rightarrow H$ deux transformations naturelles, on a l'égalité $\lim_\mathbf{J} (\varepsilon \circ \eta) = (\lim_\mathbf{J} \varepsilon)\circ(\lim_\mathbf{J} \eta)$.
      En effet, cela découle de l'unicité du morphisme $\lim_\mathbf{J} F \to \lim_\mathbf{J} H$, comme montré dans le diagramme ci-dessous.
      \[
      \begin{tikzcd}
          & \lim_\mathbf{J} F \arrow[dashed]{d}{\lim_\mathbf{J} \eta} \arrow[bend right=60,swap]{dddl}{\varphi_A} \arrow[bend left=60]{dddr}{\varphi_B} \\
          & \lim_\mathbf{J} G \arrow[dashed]{d}{\lim_\mathbf{J} \varepsilon} \arrow[bend right,swap]{ddl}{\gamma_A} \arrow[bend left]{ddr}{\gamma_B}\\
          & \lim_\mathbf{J} H \arrow[swap]{dl}{\psi_A} \arrow{dr}{\psi_B}\\
          F(A) \arrow{rr}{F(f)} & & F(B)
          \arrow[from=1-2,to=3-2,dashed,bend right=55,crossing over, swap]{dd}{\lim_\mathbf{J}(\varepsilon\circ\eta)}\\
      \end{tikzcd}
      \] 
  \end{itemize}

  On en conclut que l'on a bien définit $\lim_\mathbf{J} : [\mathbf{J}, \mathbf{C}] \to \mathbf{C}$ un foncteur covariant.

\end{document}
