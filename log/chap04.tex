\documentclass[./main]{subfiles}

\begin{document}
  \chapter{La théorie des ensembles.}

  On se place dans la logique du 1er ordre avec $\mathcal{L} = \{{\in}, {=}\}$.
  On se place dans un univers $\mathcal{U}$ non vide, le modèle, dont les éléments sont appelés des \textit{ensembles}.

  Il faudra faire la différence entre les ensembles "naïfs" (les ensembles habituels), et les ensembles "formels" (les éléments de $\mathcal{U}$).

  On a le paradoxe de Russel. On peut l'écrire 
  \begin{quote}
    "On a un barbier qui rase tous les hommes qui ne se rasent pas eux-mêmes. Qui rase le barbier ?".
  \end{quote}
  Si $\mathcal{U}$ est l'ensemble de tous les ensembles, alors \[
  a := \mleft\{\,x \in \mathcal{U} \;\middle|\; x \not\in x\,\mright\}
  \]
  vérifie $a \in a \iff a \not\in a$, \textit{\textbf{paradoxe}}.
  Pour éviter ce paradoxe, on choisit donc de ne pas faire $\mathcal{U}$ un ensemble.

  \section{Les axiomes de la théorie de Zermelo-Fraenkel.}

  \begin{description}
    \item[ZF\,1.] \label{ZF1}\textit{Axiome d'extensionnalité} : deux ensembles sont égaux ssi ils ont les mêmes éléments \[
      \forall x \: \forall y \: \big(\forall z \: (z \in x \leftrightarrow z \in y) \leftrightarrow x = y\big)
      .\]
    \item[\bullet] \label{pair-axiom} \textit{Axiome de la paire}\footnote{On verra plus tard que cet axiome est une conséquence des autres (de \zf 3 et \zf 4).}\showfootnote :
      il existe une paire $\{x,y\}$ pour tout élément $x$ et $y$
      \[
      \forall x \: \forall y \: \exists  z\: \forall t \big(t \in z \leftrightarrow (t = x \lor t = y)\big)
      .\]
  \end{description}

  [\textit{continué plus tard\,\ldots}]

  \begin{rmk}
    Cela nous donne l'existence du \textit{singleton} $\{x\}$ si $x$ est un ensemble. En effet, il suffit de faire la paire $\{x,x\}$ avec l'\pairaxiom.
  \end{rmk}

  \begin{defn}
    Si $a$ et $b$ sont des ensembles, alors $(a,b)$ est l'ensemble  $\{\{a\} , \{a,b\}\}$.
    Ainsi, $(a, a)$ est l'ensemble $\{\{a\} \} $.
  \end{defn}

  \begin{lem}
    Pour tous ensembles $a,b,a', b'$, on a  $(a,b) = (a', b')$ ssi  $a = a'$ et $b = b'$.
  \end{lem}
  \begin{prv}
    En exercice.
  \end{prv}

  \begin{defn}
    On peut construire des $3$-uplets $(a_1, a_2, a_3)$ avec $(a_1, (a_2, a_3))$, et ainsi de suite pour les $n$-uplets.
  \end{defn}

  \begin{nota}
    On utilise les raccourcis 
    \begin{itemize}
      \item $t = \{a\}$ pour $\forall x \: (x \in t \leftrightarrow x = a)$ ;
      \item $t = \{a,b\}$ pour $\forall x \: (x \in t \leftrightarrow (x = a \lor x = b))$ ;
      \item $t \subseteq a$ pour $\forall x \: (z \in t \to z \in a)$.
    \end{itemize}
  \end{nota}

  \begin{description}
    \item[ZF\,3.] \label{ZF3}
      \textit{Axiome des parties} : l'ensemble des parties $\wp(a)$ existe pour tout ensemble $a$
      \[
      \forall a \: \exists b \: \forall t \: (t \in b \leftrightarrow t \subseteq a)
      .\]
    \item[ZF\,2.] \label{ZF2} \textit{Axiome de la réunion} : l'ensemble $y = \bigcup_{z \in x} z$ existe 
      \[
      \forall x \: \exists y \: \forall t (t \in y \leftrightarrow\exists z (t \in z \land z \in x))
      .\]
  \end{description}

  \begin{rmk}
    Comment faire $a \cup b$ ?
    La paire $x = \{a,b\}$ existe par l'\pairaxiom, et $\bigcup_{z \in x} z = a \cup b$ est un ensemble par~\zf 2.
  \end{rmk}

  \begin{description}
    \item[ZF\,4'.]
      \label{ZF4'}
      \textit{Schéma de compréhension} :
      pour toute formule $\varphi(y, v_1, \ldots, v_n)$, on a l'ensemble $x = \mleft\{\,y \in v_{n+1} \;\middle|\; \varphi(y, v_1, \ldots, v_n)\,\mright\}$
      \[
      \forall v_1\: \ldots\: \forall v_n \: \exists x \: \forall y \: \big(y \in x \leftrightarrow (y \in v_{n+1} \land \varphi(y, v_1, \ldots, v_n))\big)
      .\] 

  \end{description}

  \begin{rmk}
    Peut-on faire le paradoxe de Russel ?
    On ne peut pas faire $a := \{z \in \mathcal{U}  \mid z \not\in z \}$ car $\mathcal{U}$ n'est pas un ensemble !
    Et, on ne peut pas avoir de paradoxe avec $b := \{z \in E  \mid z \not\in z\}$, car on a l'ajout de la condition $b \in E$.
  \end{rmk}

  \begin{defn}
    Une \textit{relation fonctionnelle} en $w_0$ est une formule $\varphi(w_1, w_2, a_1, \ldots, w_n)$ \textit{à paramètres} (où les $a_i$ sont dans $\mathcal{U}$) telle que

    \fitbox{$\mathcal{U} \models \forall w_0 \: \forall w_1 \: \forall w_2 \: \big(\varphi(w_0, w_1, a_1, \ldots, a_n) \land \varphi(w_0, w_2, a_1, \ldots, w_n)\to w_1 = w_2\big)$.}
  \end{defn}

  En termes naïfs, c'est une fonction partielle. On garde le terme \textit{fonction} quand le domaine et la collection d'arrivée sont des \textit{ensembles}, autrement dit, des éléments de $\mathcal{U}$.

  \begin{description}
    \item[ZF\,4.] \label{ZF4}
      \textit{Schéma de substitution/de remplacement} :
      "la collection des images par une relation fonctionnelle des éléments d'un ensemble est aussi un ensemble".
      Pour tout $n$-uplet $\bar{a}$, si la formule à paramètres $\varphi(w_0, w_1, \bar{a})$ définit une relation fonctionnelle $f_{\bar{a}}$ en $w_0$ et si $a_0$ est un ensemble alors la collection des images par $f_{\bar{a}}$ des éléments de $a_0$ est un ensemble nommé $a_{n+1}$
  \end{description}

  \begin{gather*}
    \forall a_0 \: \cdots \: \forall a_n \\
\big(
      \forall w_0 \: \forall w_1 \: \forall w_2 
        (\varphi(w_0, w_1, a_1, \ldots, a_n) \land \varphi(w_0, w_2, a_1, \ldots, a_n)) \to w_1 = w_2
      \big)\\
      \vertical\to\\
      \exists a_{n+1} \: \forall a_{n+2} (a_{n+2} \in a_{n+1} \leftrightarrow \exists w_0 \: w_0 \in a_0 \land \varphi(w_0, a_{n+2}, v_1, \ldots, v_n))
  .\end{gather*}

  \begin{thm}
    Si \zf 1, \zf 2, \zf 3 et \zf 4 sont vrais dans  $\mathcal{U}$, il existe (dans $\mathcal{U}$) un et un seul ensemble sans élément, que l'on notera $\emptyset$.
  \end{thm}
  \begin{prv}
    \begin{itemize}
      \item \textit{Unicité} par \zf 1.
      \item  \textit{Existence}.
        On procède par compréhension : l'univers $\mathcal{U}$ est non vide, donc a un élément $x$.
        On considère la formule $\varphi(w_0, w_1) := \bot$ qui est une relation fonctionnelle.
        Par \zf 4 (avec la formule $\varphi$ et l'ensemble  $a_0 := x$) un ensemble $a_{n+1}$ qui est vide.
    \end{itemize}
  \end{prv}

  \begin{prop}
    Si \zf 1, \zf 2, \zf 3 et \zf 4 sont vrais dans  $\mathcal{U}$, alors l'\pairaxiom\ est vrai dans $\mathcal{U}$.
  \end{prop}
  \begin{prv}
    On a $\emptyset$ dans $\mathcal{U}$ et également $\wp(\emptyset) = \{\emptyset\}$ et $\wp(\wp(\emptyset)) = \{\emptyset, \{\emptyset\}\}$ par \zf 3.

    Étant donné deux ensemble $a$ et $b$, on veut montrer que $\{a,b\}$ est un ensemble avec \zf 4 \[
    \varphi(w_0, w_1, a, b) := (w_0 = \emptyset \land w_1 = a) \lor (w_0 = \{\emptyset\} \land w_1 = b)
    ,\]
    où 
    \begin{itemize}
      \item $w_0 = \emptyset$ est un raccourci pour $\forall z \: (z \not\in w_0)$ ;
      \item $w_0 = \{\emptyset\} $ est un raccourci pour $\forall z \: (z \in w_0 \leftrightarrow (\forall t \: t \not\in z))$.
    \end{itemize}
    Ces notations sont compatibles avec celles données précédemment.

    Comme  $\varphi$ est bien une relation fonctionnelle et $\{a,b\}$  est l'image de $ \{\emptyset, \{\emptyset\}\}$.
  \end{prv}

  \begin{prop}
    Si \zf 1, \zf 2, \zf 3 et \zf 4 sont vrais dans  $\mathcal{U}$, alors \zf{4'} est vrai dans $\mathcal{U}$.
  \end{prop}
  \begin{prv}
    On a la formule $\varphi(y, v_1, \ldots, v_n)$ et on veut montrer que 
    \[
    \mathcal{U} \models \forall v_1 \: \cdots \: \forall v_{n+1} \: \exists x \: \forall y \: \big(
      y \in x \leftrightarrow (y \in v_{n+1} \land \varphi(y, v_1, v_n))
    \big)
    .\]
    On considère la formule $\psi(w_0, w_1, \bar{v}) := w_0 = w_1 \land \varphi(w_0, \bar{v})$, qui est bien une relation fonctionnelle en $w_0$.
    La collection \[
    \mleft\{\,x \in v_{n+1} \;\middle|\; \varphi(y, v_1, \ldots, v_n)\,\mright\}
    \] est l'image de $v_{n+1}$ par $\psi$ par \zf 4.
  \end{prv}

  \begin{rmk}
    La réciproque du théorème précédent est fausse ! Les axiomes \zf 4 et \zf{4'} ne sont pas équivalents. On le verra en TD (probablement).
  \end{rmk}

  \begin{prop}
    Le produit ensembliste de deux ensembles est un ensemble.
  \end{prop}

  \begin{prv}
    Soient $v_1$ et $v_2$ deux ensembles.
    On considère \[
    X := v_1 \times v_2 = \mleft\{\,(x,y) \;\middle|\; x \in v_1 \text{ et } y \in v_2 \,\mright\} \text{ (en naïf) }
    .\] 
    La notation $(x,y)$ correspond à l'ensemble $\{\{x\} , \{x,y\}\} \in \wp(\wp(v_1 \cup v_2))$.

    On applique \zf{4'} dans l'ensemble ambiant  $\wp(\wp(v_1 \cup v_2))$, on définit le produit comme la compréhension à l'aide de la formule 
    \[
    \varphi(z, v_1, v_2) := \exists x \: \exists y \: \big(z = \{\{x\} , \{x,y\} \}  \land x \in v_1 \land y \in v_2\big)
    .\]
    C'est bien un élément de $\mathcal{U}$.
  \end{prv}

  \begin{defn}
    Une \textit{fonction} (sous-entendu \textit{totale}) d'un ensemble $a$ dans un ensemble $b$ est un sous-ensemble de $a \times b$ qui vérifie la propriété
    \[
    \varphi(f, a, b) := 
    \begin{pmatrix} 
      f \subseteq a \times b\\
      \land\\
      \forall x \: \forall  y \: \forall y' \: (x, y )\in f \land (x, y') \in f  \to y = y'\\
      \land\\
      \forall x \: x \in a \to \exists y \: y \in b \land (x,y) \in f
    \end{pmatrix}
    .\]
    On identifie ainsi $f$ et son graphe.


    Une \textit{fonction} \textit{partielle} d'un ensemble $a$ dans un ensemble $b$ est un sous-ensemble de $a \times b$ qui vérifie la propriété
    \[
    \varphi(f, a, b) := 
    \begin{pmatrix} 
      f \subseteq a \times b\\
      \land\\
      \forall x \: \forall  y \: \forall y' \: (x, y )\in f \land (x, y') \in f  \to y = y'\\
    \end{pmatrix}
    .\]

    On note $b^a$ la collection des fonctions partielles de $a$ dans $b$.
  \end{defn}

  \begin{prop}
    La collection $b^a$ est un ensemble, \textit{i.e.} si $a$ et $b$ sont dans $\mathcal{U}$ alors $b^a$ aussi.
  \end{prop}
  \begin{prv}
    En exercice.
  \end{prv}

  \begin{rmk}[Réunion indexée]
    Soit $a$ une famille d'ensemble indexée par l'ensemble $I$, \textit{i.e.} $a$ est une fonction de domaine $I$.
    Si $i \in I$, on note $a_i$ pour $a(i)$.
  \end{rmk}
  \begin{prop}
    Si $I$ est un ensemble et $a$ est une fonction de domaine $I$, alors $\bigcup_{i \in  I} a_i$ est un ensemble.
    Autrement dit, si dans $\mathcal{U}$, \zf 1, \zf 2, \zf 3, \zf 4 sont vraies, et que $I$ et $a$ sont dans $\mathcal{U}$, et $a$ est une fonction, alors la collection définie naïvement par $\bigcup_{i \in  I} a_i$ appartient à $\mathcal{U}$.
  \end{prop}
  \begin{prv}
    On pose $b := \{a_i  \mid i \in I\}$. C'est bien un ensemble car $b$ est l'ensemble des images des éléments de $I$ par $a$. On peut écrire $a$ comme relation fonctionnelle :
    \[
    \varphi(w_0, w_1, a) := (w_0, w_1) \in a
    .\]
    On a donc que $b$ est un ensemble avec \zf 4.

    Et, $\bigcup_{i \in  I} a_i = \bigcup_{z \in b} z$ donc on conclut par \zf 2.
  \end{prv}

  \begin{prop}[Propriété d'intersection]
    Si $I$ est un ensemble non vide et $a$ est une fonction de domaine $I$ alors $\bigcap_{i \in  I} a_i$ est un ensemble.
  \end{prop}
  \begin{prv}
    On pose $c := \bigcup_{i \in  I} a_i$ qui est un ensemble par \zf 2.
    On considère \[
    \varphi(x, a, I) := \forall i \: i \in I \to x \in a_i
    .\]
    Par compréhension (\zf{4'}) on construit l'ensemble \[
    \bigcap_{i \in  I} a_i := \{x \in c  \mid \varphi(x, a, I)\} 
    .\]
  \end{prv}

  \begin{prop}
    Si $I$ est un ensemble et $a$ une fonction de domaine $i$ alors $\prod_{i \in I} a_i $ est un ensemble.
  \end{prop}
  \begin{prv}
    La collection $\prod_{i \in I} a_i$ est l'ensemble des fonctions de $I$ dans $\bigcup_{i \in  I} a_i$ telles que $f(i) \in a_i$ pour tout $i$.
  \end{prv}

  \begin{description}
    \item[ZF\,5] \label{ZF5}\textit{Axiome de l'infini} : il existe un ensemble ayant une infinité d'élément
      \[
      \exists x \: (\emptyset \in x \land \forall y \: (y \in x \to y \cup \{y\} \in x ))
      .\]
  \end{description}

  On encode les entiers avec des ensembles :
  \begin{itemize}
    \item $0 \leadsto \emptyset$
    \item $1 \leadsto \{\emptyset\} $
    \item $2 \leadsto \{\emptyset, \{\emptyset\}\}  $
    \item \quad \quad $\vdots$
    \item $n+1 \leadsto n \cup \{n\}$
    \item \quad \quad $\vdots$
  \end{itemize}

  Ainsi, on a bien $n  =\{0, 1, \ldots, n-1\}$.

  \begin{rmk}
    Si on retire $\emptyset \in x$, on peut avoir $x = \emptyset$ qui satisfait la version modifiée de \zf 5.

    Cependant, sans retirer $\emptyset \in x$, on peut quand même avoir un ensemble fini s'il existe un ensemble fini $y$ tel que $y \in y$.
    Ceci est impossible avec l'axiome de bonne fondation.
  \end{rmk}

  \begin{rmk}
    Les français sont les seuls à considérer que l'axiome de bonne fondation ne fait pas partie de la théorie de ZF.
  \end{rmk}

  \section{Ordinaux et induction transfinie.}


  \begin{thm}[Cantor]
    \begin{enumerate}
      \item Soient $A$ et $B$ deux ensembles et supposons qu'il existe des injections $A \to B$ et $B \to A$ alors il existe une bijection $A \to B$.
      \item Il n'existe pas de surjection de $A \to \wp(A)$.
    \end{enumerate}
  \end{thm}
  \begin{prv}
    En TD.
  \end{prv}

  \begin{defn}
    Deux ensembles sont \textit{équipotents} s'il existe une bijection entre-eux.
  \end{defn}

  \begin{defn}
    Soit $A$ un ensemble.
    Un \textit{ordre} (\textit{partiel}, \textit{strict}) sur $A$ est une relation binaire $<$ (donnée par un sous-ensemble de $A \times A$) telle que 
    \begin{enumerate}
      \item \textit{transitivité} : $\forall x \: \forall y \: \forall z \: x < y \to y < z \to x < z$ ;
      \item \textit{anti-réflexif} : $\forall x, x \not< x$.
    \end{enumerate}
  \end{defn}

  \begin{nota}
    On note $x \le y$ pour $x < y$ ou $x = y$.
  \end{nota}

  \begin{exm}
    L'ordre $\subsetneq$ sur $\wp(\mathds{N})$ est partiel.
    Les ordres $<_\mathds{N}$, $<_\mathds{R}$, $<_\mathds{Z}$ sur $\mathds{N}, \mathds{R}, \mathds{Z}$ sont totaux.
  \end{exm}

  \begin{defn}
    Soit $(A, <)$ ordonné. Soient  $a, a' \in A$ et $B \subseteq A$.
    On dit que 
    \begin{itemize}
      \item $a$ est un plus petit élément de $B$ si $a \in B$ et pour tout $b \in B$ et si $b \neq a$ alors $b > a$ ;
      \item $a$ est un élément minimal de $B$ si $a \in B$  et pour tout $b \in B$, $b \not< a$ ;
      \item $a$ est un minorant de $B$ si pour tout $b \in B$, $a \le b$ ;
      \item de la même manière, on définit plus grand élément, élément maximal, majorant ;
      \item $a$ est une borne inférieure de $B$ si $a$ est un plus grand élément de l'exemple des minorants de $B$ ;
      \item $a$ et $a'$ sont incomparables si $a \neq a'$ et $a \not< a'$ et $a' \not< a$ ;
      \item un ordre est bien fondé si toute partie non vide de  $A$ a un élément minimal ;
      \item un bon ordre est un ordre total bien fondé.
    \end{itemize}
  \end{defn}

  \begin{prop}
    Un ordre total est bien fondé ssi il n'existe pas de suite infinie décroissante.
  \end{prop}
  \begin{prv}
    En exercice.
  \end{prv}

  \begin{exm}
    \begin{itemize}
      \item L'ordre $<$ sur $\mathds{N}$ est bien fondé.
      \item L'ordre $<$ sur $\mathds{Z}$ n'est pas bien fondé.
      \item L'ordre $\subsetneq $ sur $\wp_\mathrm{finies}(\mathds{N})$ est bien fondé.
      \item L'ordre $\subsetneq $ sur $\wp(\mathds{N})$ n'est pas bien fondé.
    \end{itemize}
  \end{exm}

  \begin{defn}
    \begin{itemize}
      \item Deux ensembles ordonnés sont \textit{isomorphes} s'il existe une bijection préservant l'ordre de l'un vers l'autre.
        On note $A \simeq B$.
      \item Soit $X$ totalement ordonné. Un sous-ensemble $J \subseteq X$ est un \textit{segment initial} si pour tous $a,b \in X$ avec $a < b$ alors $b \in J$ implique $a \in J$.
      \item Un ensemble $X$ est \textit{transitif} si pour tout $x \in X$ et $y \in x$ alors $y \in X$.
      \item Un ensemble $X$ est un \textit{ordinal} s'il est transitif  et que $\in$ défini un bon ordre sur $X$.
      \item On note $\mathcal{O}$ la \textit{classe des ordinaux}, et on note indifféremment les relations $\in$ et $<$.
    \end{itemize}
  \end{defn}

  \begin{exm}
    Les entiers de Von Neumann sont des ordinaux.
  \end{exm}

  \begin{prop}
    Soient $\alpha$  et $\beta$ des ordinaux. On a les propriétés suivantes :
    \begin{enumerate}
      \item $\emptyset$ est un ordinal ;
      \item si $\alpha \neq \emptyset$ alors $\emptyset \in \alpha$ ;
      \item $\alpha \not \in \alpha$ ;
      \item si $x \in \alpha$ alors $x = \{y \in \alpha  \mid y < x\}$.
      \item si $x \in \alpha$ alors $x$ est un ordinal (on a l'abus de notation $x \in \mathcal{O}$) ;
      \item $\beta \le \alpha$ ssi $\beta = \alpha$ ou  $\beta \in \alpha$ ;
      \item $x = \alpha \cup \{\alpha\}$ est un ordinal noté $\alpha + 1$.
    \end{enumerate}
  \end{prop}

  \begin{prv}
    \begin{enumerate}
      \item C'est vrai.
      \item La relation $\in$ est un bon-ordre sur $\alpha$, soit $\beta$ le plus petit élément.
        Si $\beta \neq \emptyset$ alors il contient au moins un élément $\gamma$ d'où $\gamma < \beta$ et $\gamma \in \alpha$ (par transitivité), \textit{absurde} car $\beta$ minimal.
      \item Le reste sera vu en TD ou en exercice.
    \end{enumerate}
  \end{prv}

  \begin{prop}
    \begin{itemize}
      \item Si $\alpha$ et $\beta$ sont des ordinaux et que l'on a~$\alpha \le \beta \le \alpha + 1$ alors $\beta = \alpha$ ou  $\beta = \alpha + 1$.
      \item Si $X$ est un ensemble non vide d'ordinaux alors $\bigcap_{\alpha \in X} \alpha$ est le plus petit élément de $X$.
    \end{itemize}
    \qed
  \end{prop}

  \begin{defn}
    Soit $\beta$ un ordinal.
    \begin{itemize}
      \item S'il existe $\alpha$ tel que $\beta = \alpha + 1$ alors on dit que $\beta$ est un ordinal successeur ;
      \item Sinon, c'est un ordinal limite.
    \end{itemize}
  \end{defn}

  \begin{exm}
    Quelques ordinaux limites : 
    \begin{gather*}
      \omega\quad\quad\omega \cdot 2\quad\quad\omega \cdot 3 \quad\quad \omega \cdot \omega \\
      \omega^2 + \omega\quad\quad\omega^\omega\quad\quad\omega^{\omega^{\omega^{\iddots^{\omega}}}}.
    \end{gather*}
  \end{exm}

  \begin{lem}
    Soit $X$ un ensemble d'ordinaux.
    Le plus petit élément de $X$ est $\bigcap_{\alpha \in X} \alpha$.
  \end{lem}

  \begin{thm}
    Si $\alpha$ et $\beta$ sont des ordinaux alors une et une seule de ces propriétés est vérifiée :
    \[
      \alpha = \beta
      \quad\quad \quad\quad
      \alpha \in \beta
      \quad\quad \quad\quad
      \alpha \ni \beta
    .\] 
  \end{thm}
  \begin{prv}
    Soit $X = \{\alpha, \beta\}$, on sait que $\alpha \cap \beta$ est le plus petit élément de $X$.
    \begin{itemize}
      \item Si $\alpha \cap \beta = \alpha$ alors $a \subseteq \beta$ donc $\alpha = \beta$ ou $\alpha \in \beta$.
      \item Si $\alpha \cap \beta = \beta$ alors $a \supseteq \beta$ donc $\alpha = \beta$ ou $\alpha \ni \beta$.
    \end{itemize}
  \end{prv}

  \begin{prop}
    Soit $X$ un ensemble d'ordinaux.
    Alors l'ensemble~$b := \bigcup_{\alpha \in X} \alpha$ est un ordinal.
    On le note $b = \sup_{\alpha \in X} \alpha$.
    De plus si $\gamma \in b$ alors il existe un certain $\alpha \in X$ tel que $\gamma \in \alpha$.
  \end{prop}
  \begin{prv}
    En exercice.
  \end{prv}

  \begin{prop}
    Soit $\lambda$ un ordinal non vide. On a :
    \[
      \overbrace{\lambda \text{ est limite}}^{(1)} \quad \iff \quad \overbrace{\lambda = \bigcup_{\alpha \in \lambda} \lambda}^{(2)}
    .\] 
  \end{prop}
  \begin{prv}
    \begin{enumerate}
      \item Par contraposée, si $\lambda$ n'est pas limite, c'est donc un successeur d'un certain ordinal $\beta$ et donc $\lambda = \beta \cup \{\beta\}$.
        On a \[
        \bigcup_{\alpha \in \lambda} \alpha = \beta \cup \bigcup_{\alpha \in \beta} = \beta \neq \lambda
        .\] 
      \item Soit $\lambda$ limite.
        Montrons qu'il n'a pas de plus grand élément~$\beta$.
        Sinon, $\lambda = \beta \cup \{\beta\}$.
        Donc, pour tout $\alpha \in \lambda$ il existe un certain $\gamma \in \lambda$ tel que $\alpha < \gamma$,  \textit{i.e.} $\alpha \in \gamma$.
        On en conclut que $\lambda = \bigcup_{\gamma \in \lambda} \gamma$.
    \end{enumerate}
  \end{prv}

  \begin{thm}[Induction transfinie]
    Soit $\mathcal{P}$ une propriété sur les ordinaux.
    On suppose que :
    \begin{itemize}
      \item $\emptyset$ satisfait $\mathcal{P}$ ;
      \item pour tout ordinal $\alpha$ tel que, pour tout $\beta < \alpha$ satisfait $\mathcal{P}$, alors $\alpha$ satisfait $\mathcal{P}$ :
        \[
        \forall \alpha, \: (\forall \beta < \alpha, \: \mathcal{P}(\beta)) \implies \mathcal{P}(\alpha) \;
        ;\] 
    \end{itemize}
    alors tous les ordinaux satisfont $\mathcal{P}$.
  \end{thm}
  \begin{prv}
    Par l'absurde, soit $\alpha$ ne satisfaisant pas $\mathcal{P}$.
    Soit $\beta$ le plus petit ordinal de $\alpha \cup \{\alpha\}$ ne satisfaisant pas $\mathcal{P}$.
    Tous les ordinaux plus petit que $\beta$ satisfont $\mathcal{P}$, d'où $\mathcal{P}(\alpha)$, \textit{\textbf{absurde}}.
    On en conclut que $\alpha$ n'existe pas.
  \end{prv}

  \begin{rmk}
    En pratique on décompose :
    \begin{itemize}
      \item on montre pour $\emptyset$ ;
      \item on montre pour $\alpha$ successeur ;
      \item on montre pour $\alpha$ limite.
    \end{itemize}
  \end{rmk}

  \begin{defn}
    Un ordinal $\alpha$ est \textit{fini} si $\alpha = \emptyset$ ou si $\alpha$ et sous ses éléments sont des successeurs.
  \end{defn}

  \begin{prop}
    L'ensemble des ordinaux finis $\omega$ est un ordinal. C'est le plus petit ordinal limite.
  \end{prop}
  \begin{prv}
    En exercice.
  \end{prv}

  \begin{lem}
    Soit $f : \alpha \to \alpha'$ une fonction strictement croissante entre deux ordinaux $\alpha$ et $\alpha'$.
    Alors $f(\beta) \ge \beta$ pour tout $\beta \in \alpha$.
    De plus, on a $\alpha' \ge \alpha$.
    Aussi si $f$ est un isomorphisme alors $\alpha = \alpha'$ et  $f$ est l'identité.
  \end{lem}

  \begin{prv}
    \begin{itemize}
      \item Soit $\beta_0$ le plus petit élément tel que $f(\beta_0) < \beta_0$.
        Comme $f$ strictement croissante, on a $f(f(\beta_0)) < f(\beta_0) < \beta_0$ absurde car $\beta_0$ est le plus petit.
      \item Soit $\beta \in \alpha$. On a $f(\beta) \in \alpha'$ et $\beta \le f(\beta)$ donc $\beta \in \alpha'$, donc $\beta \in \alpha'$, d'où $\alpha \subseteq \alpha'$ et donc $\alpha \le \alpha'$.
      \item Si $f$ est un isomorphisme alors $f^{-1}$ est strictement croissante.
        On applique le point précédent à $f^{-1}$, d'où $\alpha = \alpha'$.
      \item Montrons que $f$ est l'identité.
        On sait que, pour tout $\beta \in \alpha$, on a $f_{|\beta}$ strictement croissante de $\beta$ dans $f(\beta)$, d'où $\beta = f(\beta)$ par le point précédent.
        D'où $f$ est l'identité.
    \end{itemize}
  \end{prv}

  \begin{thm}
    Tout ensemble bien ordonné est isomorphe à un ordinal. Cet ordinal ainsi que l'isomorphisme sont uniques.
  \end{thm}
  \begin{prv}
    Cette preuve ressemble à une induction sans en être une.
    On aura le droit d'en faire quand on aura le théorème.

    Si l'isomorphisme existe, il est unique grâce au lemme précédent.
    En effet, s'il y en a deux $f$ et $g$ alors $f \circ g^{-1}$ est un isomorphisme entre deux ordinaux égaux, donc c'est l'identité.

    Notons $\mathcal{P}(x)$ la propriété "il existe un ordinal $\alpha_x$ et un isomorphisme $f_x : S_{\le x} \to \alpha_x$" où $S_{\le x} := \{y \in X \mid y \le x\}$.
    Pour montrer $\mathcal{P}(x)$ pour tout $x \in X$, on pose  
    \[
    Y := \{x \in X  \mid \mathcal{P}(x) \text{ est vraie}\}
    .\] 
    et on montre $Y = X$.

    Supposons $Y \neq X$ et soit $a = \min (X \setminus Y)$.
    \begin{itemize}
      \item Si $Y$ a un plus grand élément $b$, alors il existe un isomorphisme $f_b : S_{\le b} \to \alpha_b$ (car $\mathcal{P}(b)$).
        Or, $S_{\le a} = S_{\le b} \cup \{a\}$ (il faudrait montrer que $Y$ est un segment initial de $X$).
        Et, on construit $f_a : S_{\le a} \to \alpha_b \cup \{\alpha\}_b$ qui est un isomorphisme, donc $a \in Y$, \textit{\textbf{absurde}}.
      \item Si $Y$ n'a pas de plus grand élément, on considère $\alpha := \bigcup_{x \in Y} \alpha_x$ un ordinal.
        Pour tout $x < \alpha$ il existe un isomorphisme de $S_{\le x}$ dans $\alpha_x$.
        Si on prend $x < y < \alpha$ alors  $(f_y)_{|S_{\le x}}$ est un isomorphisme et, par unicité, on a donc que $f_y$ prolonge $f_x$.
        On peut définir un isomorphisme $f_{\le \alpha}$ comme limite des $f_{x}$ pour $x < \alpha$.
        C'est un isomorphisme de $S_{< a}$ dans $\beta := \bigcup_{x < \alpha} \alpha_x$.
        On peut prolonger $f$ en $f_X : S_{\le \alpha} \to \beta + 1$ où $\alpha \mapsto \beta$, d'où $\alpha \le Y$, \textit{\textbf{absurde}}.
    \end{itemize}
  \end{prv}

  \begin{lem}[Définition récursive transfinie des fonctions]
    Soient
    \begin{itemize}
      \item $\alpha$ un ordinal ;
      \item $S$ une collection ;
      \item $\mathcal{F}$ la collection des applications définies sur les ordinaux~$\beta \le \alpha$ et prenant leur valeurs dans $S$ ;
      \item $F$ une relation fonctionnelle de domaine $\mathcal{F}$ à valeur dans $S$.
    \end{itemize}
    Alors il existe une fonction $f$ dans $\mathcal{F}$ (et une unique définie sur~$\alpha$) telle que :
    \[
      (\star) \quad\quad \text{ pour tout } \beta < \alpha \quad\quad f(\beta) = F(f_{|\beta})
    .\] 
  \end{lem}

  \begin{prv}
    \begin{description}
      \item[Unicité.]
        Soient $f$ et $g$ satisfaisant $(\star)$.
        Montrons $\mathcal{P}(\beta)$ : "si $\beta < \alpha$ alors $f(\beta) = g(\beta)$" par induction transfinie.
        \begin{itemize}
          \item On a directement $\mathcal{P}(\emptyset)$ car il y a une unique fonction de $\emptyset \to \emptyset$.
          \item Supposons que $f(\gamma) = g(\gamma)$ pour tout $\gamma < \beta$.
            Alors $f_{|\beta} = g_{|\beta}$ et donc par $(\star)$ on a  $f(\beta) = g(\beta)$.
        \end{itemize}
      \item[Existence.]
        Soit $\tau$ l'ensemble des ordinaux $\gamma \in \alpha$ tels qu'il existe $f_\gamma \in \mathcal{F}$ définie sur $\gamma$ et vérifiant $(\star)$.
        Alors $\tau$ est un segment initial de $\alpha$ donc un ordinal.
        Par unicité si $\gamma < \gamma'$ alors $f_{\gamma'}$ prolonge $f_\gamma$.
        On définit  $f_\tau$ par  $f_\tau(\gamma) := F(f_\gamma)$ si  $\gamma < \tau$.
         \begin{itemize}
          \item Si $\tau \in \alpha$ alors $f_\tau$ prolonge tous les $f_\gamma$ et donc $\tau \in \tau$, \textit{\textbf{impossible}}.
          \item D'où $\tau = \alpha$.
        \end{itemize}
    \end{description}
  \end{prv}

  \begin{exo}
    Généraliser la preuve ci-dessus en replaçant $\alpha$ par la classe de tous les ordinaux $\mathcal{O}$ (et remplacer $\mathcal{F}$ par autre chose).
  \end{exo}

  \begin{prop}
    \begin{enumerate}
      \item La classe $\mathcal{O}$ n'est pas un ensemble.
      \item Il n'existe pas de relation fonctionnelle bijective entre $\mathcal{O}$ et un ensemble $a$.
    \end{enumerate}
  \end{prop}
  \begin{prv}
    \begin{enumerate}
      \item En effet, supposons $\mathcal{O}$ un ensemble. On a que $\mathcal{O}$ est transitif et $\in$ y définit un ordre total, donc $\mathcal{O}$ est un ordinal. D'où $\mathcal{O} \in \mathcal{O}$ ce qui est impossible pour un ordinal.
      \item Sinon $\mathcal{O}$ serait un ensemble.
    \end{enumerate}
  \end{prv}

  \section{Axiome de choix et variantes équivalentes.}

  Les axiomes du choix sont exprimable au premier ordre !

  \begin{description}
    \item[AC\,1.] \label{AC1}
      Le produit d'une famille d'ensembles non vides est non vide.
  \end{description}

  \begin{description}
    \item[AC\,2.] \label{AC2}
      Pour tout ensemble $a$ non vide, il existe une fonction de $\mathcal{P}(a)$ dans $a$ tel que si $x \subseteq a$ est non vide alors $f(x) \in x$.
      (C'est une fonction de choix.)
  \end{description}

  \begin{description}
    \item[AC\,3.] \label{AC3}
      Si $a$ est un ensemble dont tous les éléments sont non vides et deux à deux disjoints alors il existe un ensemble $c$ tel que, pour tout $x \in a$, $x \cap c$ a exactement un élément.
  \end{description}

  \begin{description}
    \item[{\footnotesize(Lemme de)} Zorn.] \label{zorn}
      Tout ensemble non vide partiellement ordonné et inductif admet un élément maximal.
  \end{description}

  On rappelle qu'un ensemble partiellement ordonné $X$ est inductif si~$X \neq \emptyset$ et si tout sous-ensemble $Y \subseteq X$ totalement ordonné admet un majorant dans $X$.

  \begin{description}
    \item[{\footnotesize(Lemme de)} Zermelo.] \label{zermelo}
      Tout ensemble non vide peut être mini d'un bon ordre (\textit{i.e.} un ordre total où toute partie non vide a un plus petit élément).
  \end{description}

  En supposant les axiomes $\text{\zf1}, \ldots, \text{\zf5}$, on va montrer les implications suivantes :
  \[
  \begin{tikzcd}
    \text{\ac 3} \arrow[Rightarrow]{rr}{\ref{subsect:ac3-ac1}} & & \text{\ac 1} \arrow[Rightarrow]{dl}{\ref{subsect:ac1-ac2}} \\
                                                               & \text{\ac 2} \arrow[Rightarrow]{ul}{\ref{subsect:ac2-ac3}} \arrow[bend left,Rightarrow]{dr}{\ref{subsect:ac2-zermelo}} \arrow[Rightarrow]{dl}{\ref{subsect:ac2-zorn}}\\
    \text{\zorn} \arrow[Rightarrow]{uu}{\ref{subsect:zorn-ac3}} && \text{\zermelo} \arrow[bend left,Rightarrow]{ul}{\ref{subsect:zermelo-ac2}}
  \end{tikzcd}
  \]

  \subsection{\ac 1 implique \ac 2.} \label{subsect:ac1-ac2}

  On rappelle que l'ensemble $\prod_{i \in I} a_i$ est l'ensemble de fonctions $f$ de la forme $f : I \to  \bigcup_{i \in I} a_i$ tel que $f(i) \in a_i$ pour tout $i$.

  Soit $a$ non vide. On considère $\prod_{\emptyset \neq x \subseteq a}x $ qui est non vide par \ac1.
  Soit~$f$ un de ces éléments.
  On a, pour tout $\emptyset \neq  x \subseteq a$, que $f(x) \in x$ donc~$f$ est une fonction de choix.

  \subsection{\ac 2 implique \ac 3.}  \label{subsect:ac2-ac3}
  Soit $a$ un ensemble dont les éléments sont non vides et deux à deux disjoints.
  On considère $b = \bigcup_{x \in a} x$ qui est un ensemble.
  Par \ac 2, on a une fonction de choix $f$ sur $\wp(b)$.
  On prend $c = \{f(x)  \mid x \in a\} $. Comme les $x$ sont disjoints, on obtient la propriété recherchée.

  \subsection{\ac 3 implique \ac 1.} \label{subsect:ac3-ac1}
  Soit $X = \prod_{i \in I} a_i$ un produit d'ensemble non vides.
  On considère $A := \mleft\{\,\{i\} \times a_i \;\middle|\; i \in I\,\mright\}$.
  Par \ac 3, il existe $c$ tel que, pour tout $x \in A$, $x \cap c$ a exactement un élément.
  D'où, $c$ peut s'écrire
  \[
  c = \{(i, d_i)  \mid i \in I \text{ et } d_i \in a_i\}
  .\]
  On a donc $c \in \prod_{i \in I} a_i$ (c'est le graphe d'une fonction) et donc $\prod_{i \in I} a_i \neq \emptyset$.

  \subsection{\ac2 implique \zermelo.} \label{subsect:ac2-zermelo}

  Soit $a$ un ensemble non vide.

  \begin{rmk}[Idée]
    L'idée est qu'on utilise $f : \mathcal{P}(a) \to a$ une fonction de choix pour définir l'ordre.
    On peut imaginer définir $x \le y$ ssi $f(\{x,y\}) = x$ (on traite donc $f$ comme la fonction minimum), mais on n'a pas la transitivité.
    Il faut être plus futé.

    On va construire en partant du plus petit une bijection entre $a$ et un ordinal.
    \begin{itemize}
      \item $h(0) := f(a)$ 
      \item $h(1) := f(a \setminus \{h(0)\} )$ 
      \item $h(2) := f(a \setminus \{h(0), h(1)\})$
      \item $\quad\vdots$
      \item $h(\omega) = f(a \setminus \{h(0), h(1), \ldots\} )$ 
      \item $\quad\vdots$
    \end{itemize}
    On s'arrête quand $a \setminus \{h(0), h(1), \ldots\}$ est vide.
  \end{rmk}

  Soit $\theta \not\in a$ (pour "détecter" quand l'ensemble $a \setminus \{h(0), \ldots\}$ est vide).
  Il existe car $a$ est un ensemble donc pas l'univers tout entier.

  On définit \[
  F(\alpha) := \begin{cases}
    f(a \setminus \{F(\beta)  \mid \beta < \alpha\} ) & \text{ si } a \setminus \{F(\beta)  \mid \beta < \alpha\}  \neq \emptyset\\
    \theta & \text{ sinon}
  \end{cases}
  .\]
  D'après le dernier lemme de la section précédente, on peut construire l'application $F$ ainsi et elle est unique.

  S'il n'existe pas d'ordinal $\alpha$ tel que $F(\alpha) = \theta$ alors  $F$ est une injection de $\mathcal{O}$ dans $a$. \textbf{\textit{Absurde}}.
  Il existe donc $\alpha$ tel que $F(\alpha) = \theta$.

  Le sous-ensemble $\{\beta \in \alpha  \mid F(\beta) = \theta\} $ a un plus petit élément $\beta$.
  Montrons que $F_{|\beta}$ est une bijection de $\beta$ dans $\alpha$.
  \begin{itemize}
    \item D'une part, on sait que c'est une injection.
    \item D'autre part, $F(\beta) = \theta$ implique  $\{F(\gamma)  \mid \gamma < \beta\} = a$ donc $F_{|\beta}$ est une surjection.
  \end{itemize}
  On définit le bon ordre $x \prec y$ ssi  $F^{-1}(x) < F^{-1}(y)$.

  \subsection{\zermelo\ implique \ac2.} \label{subsect:zermelo-ac2}

  Soit $a$ non vide.
  Il existe un bon-ordre $<$ sur $a$.
  Soit $\emptyset \neq x \subseteq a$. On définit $f(x) = \min x$, c'est bien une fonction de choix. 

  \subsection{\zorn\ implique \ac3.}  \label{subsect:zorn-ac3}
  Soit $a$ un ensemble dont les éléments sont disjoints et non vides.
  On pose :
  \[
  b := \bigcup_{x \in a} x \quad \text{ et } \quad X := \{c \subseteq b  \mid \forall x \in a, \: |c \cap x| \le 1\} 
  .\] 
  Montrons que l'ensemble $(X, \subsetneq)$ est inductif.

  Si $y \subseteq X$ est totalement ordonné. Montrons que $Y$ a un majorant dans $X$.
  On pose $z = \bigcup_{y \in Y} y$ qui majore $Y$. On a bien $z \in X$ (on ne duplique pas les éléments).
  On en conclut que $X$ est inductif.
  Soit $d$ un élément maximal de $X$ (il existe par \zorn).

  \begin{itemize}
    \item S'il existe $x \in a$ tel que $x \cap d = \emptyset$ alors prenons $u \in x$ et posons $d_1 := d \cup \{u\}$.
      D'où $d_1 \in X$ et $d \subsetneq d_1$ donc $d$ non maximal, \textit{\textbf{absurde}}.
    \item Pour tout $x \in a$, on a $|d \cap x| = 1$, d'où $d$ est l'ensemble recherché (appelé $c$ dans \ac 3).
  \end{itemize}

  \subsection{\ac 2 implique \zorn.} \label{subsect:ac2-zorn}

  Soit $a$ un ensemble inductif.

  \begin{rmk}[Idée]
    On construit une chaîne dans $a$ (\textit{i.e.} un ensemble totalement ordonné) de taille maximale (c'est ici qu'on utilise la fonction de choix de \ac 2).
    Elle a un majorant car $a$ est inductif.
    Ce majorant va être l'élément maximal.
  \end{rmk}

  Soit $f : \wp(a) \to a$ une fonction de choix donnée par \ac 2.
  Si $x \subseteq a$ on appelle majorant strict de $x$ un $y \in a$ tel que $z < y$ pour tout~$y \in x$.

  \begin{rmk}[Idée -- suite]
    \begin{itemize}
      \item On part de $\emptyset$ : tout élément de $a$ a un majorant strict de $\emptyset$ en choisissant $a_1 = f(a)$.
      \item Soit $X_1$ l'ensemble des majorants de $\{a_1\}$. On pose $a_2 := f(X_1)$.
      \item Soit $X_2$ l'ensemble des majorants de $\{a_1,a_2\}$. On pose $a_3 := f(X_2)$.
    \end{itemize}
  \end{rmk}

  Formellement, soit $C := \{x \subseteq a  \mid x \text{ a un majorant strict dans } a \} $.
  On a $\emptyset \in C$. On définit 
  \begin{align*}
   m : C &\longrightarrow a \\
    x &\longmapsto f(\{y \in a  \mid y \text{ est un majorant strict de $x$ dans } a\} )
  .\end{align*}

  On définit par induction la chaîne maximale (dernier lemme de la section précédente).
  Soit $\theta \not\in a$.
  On définit :
  \[
  F(\alpha) := \begin{cases}
    m(\{F(\beta)  \mid \beta < \alpha\} ) & \text{ si } \{F(\beta)  \mid \beta < \alpha\} \in C\\
    \theta & \text{ sinon}
  \end{cases}
  .\] 

  La fonction $F$ n'est pas une injection de $\mathcal{O}$ dans $a$ donc il existe un ordinal $\alpha$ tel que $F(\alpha) = \theta$.
  Comme $\alpha + 1$ est un ordinal, l'ensemble~$\{\beta \in \alpha + 1  \mid F(\beta) = \theta\}$ a un plus petit élément $\alpha_0$.
  D'où l'ensemble $\{F(\beta)  \mid \beta < \alpha_0\}$ n'a pas de majorant strict mais a un majorant $M$ car $a$ inductif.
  Et, $a$ n'a pas d'élément plus grand que $a$.
  Ainsi $M$ est maximal dans $a$.

  \subsection{Indépendance de ZF et de l'axiome du choix.}

  On a les deux théorèmes suivants (que l'on admet).

  \begin{thm}[Gödel, 1938]
    S'il est cohérent, ZF ne réfute pas l'axiome du choix.
  \end{thm}

  \begin{thm}[Cohen, 1963]
    S'il est cohérent, ZF ne montre pas l'axiome du choix.
  \end{thm}

  Ainsi l'axiome du choix est indépendant de ZF.
  Cependant, il existe des versions plus faibles : axiome du choix dépendant (\textsf{ACD}), axiome du choix dénombrable (\textsf{AC}${}_\omega$).
\end{document}
